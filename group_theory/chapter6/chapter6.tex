\documentclass[../group-theory.tex]{subfiles}

\setcounter{chapter}{5}

\chapter{置换群}

置换群在物理和数学上的重要意义 :
\begin{enumerate}
	\item  置换群描写全同粒子体系的置换对称性
	\item  所有有限群都同构于置换群的子群
	\item  杨算符能明确描写张量指标间的复杂对称性
\end{enumerate}


\section{置换群的一般性质}

\begin{definition}{置换}{}
	$n$ 个客体排列次序的变换称为 {\color{main} 置换};\\
	$n$ 个客体共有 $n!$ 个不同的置换
\end{definition}


\begin{definition}{矩阵描写}{}
	设原来排在第$j$位置的客体,经过置换$R$后排到了第$r_j$ 位置,用$2×n$矩阵来描写这一置换$R$
	\begin{equation}\begin{aligned}
	\label{eq.6.1.1}
	R=
	\begin{pmatrix}
	1&2&3&j&n\\
	r_1&r_2&r_3&r_j&r_n
	\end{pmatrix}
	\end{aligned}\end{equation}
\end{definition}

\begin{example} 置换作用于波函数
	
	\begin{equation}\begin{aligned}
	\label{eq.6.1.2}
	\psi={}&
	\begin{pmatrix}
	\phi_1&\phi_2&\phi_3
	\end{pmatrix}\quad
	R=
	\begin{pmatrix}
	1&2&3\\
	2&3&1
	\end{pmatrix}\\
	&\mathrm{then},\\
	R\psi={}&
	\begin{pmatrix}
	1&2&3\\
	2&3&1
	\end{pmatrix}
	%%%%%%%%%%%%%%%%%%
	\begin{pmatrix}
	\phi_1&\phi_2&\phi_3
	\end{pmatrix}=
	\begin{pmatrix}
	\phi_3&\phi_1&\phi_2
	\end{pmatrix} 
	\end{aligned}\end{equation}
\end{example}


\begin{note}
	对一给定的置换,各列的排列次序无关紧要,
	重要的是每一列上下两个数字间的对应关系
\end{note}


\begin{definition}{置换的乘积}{}
	两个{\color{main}置换的乘积}定义为相继做两次置换\\
	考虑$S$和$R$的乘积$SR$:
	重新排列$R$或$S$的各列,使$R$的第二行和$S$的
	第一行排列一样,
	由$R$的第一行和$S$的第二行组成的$2×n$矩阵即为$SR$
\end{definition}

\begin{example} 置换相乘

	\begin{equation}\begin{aligned}
	\label{eq.6.1.3}
	S={}&
	\begin{pmatrix}
	3&4&5&2&1\\
	2&4&5&1&3
	\end{pmatrix}=
	%%%%%%%%%%%%%%%%%%%
	\begin{pmatrix}
	1&2&3&4&5\\
	3&1&2&4&5
	\end{pmatrix}\\
	%%++++++++++++++++++++++++++++++++++	
	R={}&
	\begin{pmatrix}
	1&2&3&4&5\\
	3&4&5&2&1
	\end{pmatrix}=
	%%%%%%%%%%%%%%%%%%%
	\begin{pmatrix}
	5&4&1&2&3\\
	1&2&3&4&5
	\end{pmatrix}\\
	%%++++++++++++++++++++++++++++++++++	
	SR={}&
	\begin{pmatrix}
	3&4&5&2&1\\
	2&4&5&1&3
	\end{pmatrix}
	%%%%%%%%	
	\begin{pmatrix}
	1&2&3&4&5\\
	3&4&5&2&1
	\end{pmatrix}\\
	{}={}&
	\begin{pmatrix}
	1&2&3&4&5\\
	2&4&5&1&3
	\end{pmatrix}
	\end{aligned}\end{equation}
\end{example}

\begin{note}
	\begin{enumerate}
		\item 
		\label{6.1.permutation}
		$SR$可以理解为把$R$置换的第二行数字作$S$置换,
		或者把$S$置换的第一行数字作$R^{-1}$置换
		
		\item 置换用矩阵来描写,但置换的乘积不服从矩阵乘积规则
	\end{enumerate}
\end{note}

\begin{proposition}{置换群}{}
	$n$个客体的$n!$个置换满足群的四个条件,构成群,
	称为{\color{main}$n$个客体的置换群},
	记作{\color{main}$S_n$}
\end{proposition}

\begin{note}
	\begin{enumerate}
		
		\item 把置换的上下两行交换得到的置换是逆置换
		
		\item $n$个客体中$m$个客体的所有变换构成置换群$S_m$,
		显然$S_m$是$S_n$的子群。
		
		\item 置换群的子群链:$S_n\supset S_{n-1}\supset S_{n-2}\supset \cdots \supset S_1=E$
	\end{enumerate}
\end{note}

\begin{definition}{轮换}{}
	{\color{main}轮换} 是一类特殊的置换:
	$n-l$个客体保持不变,余下的$l$个客体顺序变换,
	形成一个循环;
	$l$称为{\color{main}轮换长度}
\end{definition}

\begin{example} 轮换
	
	\begin{equation}\begin{aligned}
	\begin{pmatrix}
	a_1&a_2&a_l
	\end{pmatrix}=
	%%%%+++++++++++++++++++
	\begin{pmatrix}
	a_1&\cdots&a_{l-1}&a_{l}&b_{1}&b_{n-l}\\
	a_2&\cdots&a_{l}&a_{1}&b_{1}&b_{n-l}
	\end{pmatrix}
	\end{aligned}\end{equation}
\end{example}

\begin{note}
	\begin{enumerate}
		\item 用行矩阵描写轮换时,数字的排列次序不能改变,但可以顺序变换。
		\begin{equation}\begin{aligned}
		\begin{pmatrix}
		a&b&c&\cdots&p&q
		\end{pmatrix}=
		%%%%+++++++++++++++++++
		\begin{pmatrix}
		b&c&\cdots&p&q&a
		\end{pmatrix}=
		%%%%+++++++++++++++++++
		\begin{pmatrix}
		c&\cdots&p&q&a&b
		\end{pmatrix}
		\end{aligned}\end{equation}
		%%%%-------------------------
		\item 长度为$1$的轮换时恒等变换,长度为$2$的轮换称为{\color{main}对换},对换满足
		\begin{equation}\begin{aligned}
		\begin{pmatrix}
		a&b
		\end{pmatrix}={}&
		%%%%+++++++++++++++++++
		\begin{pmatrix}
		b&a
		\end{pmatrix}\\
		%%%%+++++++++++++++++++
		\begin{pmatrix}
		a&b
		\end{pmatrix}
		%%%%+++++++++++++++++++
		\begin{pmatrix}
		a&b
		\end{pmatrix}={}&
		E
		\end{aligned}\end{equation}
		%%%%-------------------------
		\item 
		长度为$l$的轮换,它的$l$次自乘等于恒元,即它的阶数为$l$
		\begin{equation}
		\begin{aligned}
		R={}&
		\begin{pmatrix}
		a_1&a_2&\cdots&a_l
		\end{pmatrix}\\
		%%%%+++++++++++++++++++
		R^l={}&
		E
		\end{aligned}
		\end{equation}
		%%%%-------------------------
		\item 
		两个没有公共客体的轮换,乘积次序可以交换
		\begin{equation}
		\begin{aligned}
		%%%%%+++++++++++++++++
		&\begin{pmatrix}
		a_1&a_2&\cdots&a_l
		\end{pmatrix}
		%%%%%+++++++++++++++++
		\begin{pmatrix}
		b_1&b_2&\cdots&b_m
		\end{pmatrix}={}\\
		%%%%++++++++++++++++++
		%%%%%+++++++++++++++++
		&\begin{pmatrix}
		b_1&b_2&\cdots&b_m
		\end{pmatrix}
		%%%%++++++++++++++++++
		\begin{pmatrix}
		a_1&a_2&\cdots&a_l
		\end{pmatrix}
		%%%%%+++++++++++++++++
		\end{aligned}
		\end{equation}
		%%%%-------------------------
		\item 
		轮换的逆
		\begin{equation}
		\begin{aligned}
		%%%%%+++++++++++++++++
		&\begin{pmatrix}
		a_1&a_2&\cdots&a_{l-1}&a_l
		\end{pmatrix}^{-1}={}\\
		%%%%%+++++++++++++++++
		%%%%%+++++++++++++++++
		&\begin{pmatrix}
		a_l&a_{l-1}&\cdots&a_{3}&a_2&a_1
		\end{pmatrix}
		%%%%%+++++++++++++++++
		\end{aligned}
		\end{equation}
	\end{enumerate}
\end{note}


\begin{proposition}{置换分解}{}
	{
		\color{main}任何一个置换,
		都可以唯一地分解为没有公共客体的轮换乘积
	}
\end{proposition}


\begin{example} 置换分解
	
	%%some comment
	\begin{equation}\begin{aligned}
	%%\label{eq.6.1.2}
	%%%%%+++++++++++++++++++++++---------------------
	\begin{pmatrix}
	1&2&3&4&5\\
	3&4&5&2&1
	\end{pmatrix}={}&
	%%%%%+++++++++++++++++
	%%%%%+++++++++++++++++
	\begin{pmatrix}
	1&3&5
	\end{pmatrix}
	%%%%%+++++++++++++++++
	%%%%%+++++++++++++++++
	\begin{pmatrix}
	2&4
	\end{pmatrix}=
	%%%%%+++++++++++++++++
	%%%%%+++++++++++++++++
	\begin{pmatrix}
	2&4
	\end{pmatrix}
	%%%%%+++++++++++++++++
	%%%%%+++++++++++++++++
	\begin{pmatrix}
	1&3&5
	\end{pmatrix}\\
	%%%%%+++++++++++++++++
	%%%%%+++++++++++++++++++++++---------------------
	\begin{pmatrix}
	1&2&3&4&5\\
	3&1&2&4&5
	\end{pmatrix}={}&
	%%%%%+++++++++++++++++
	%%%%%+++++++++++++++++
	\begin{pmatrix}
	1&3&2
	\end{pmatrix}
	%%%%%+++++++++++++++++
	%%%%%+++++++++++++++++
	\begin{pmatrix}
	4
	\end{pmatrix}
	%%%%%+++++++++++++++++
	%%%%%+++++++++++++++++
	\begin{pmatrix}
	5
	\end{pmatrix}=
	%%%%%+++++++++++++++++
	%%%%%+++++++++++++++++
	\begin{pmatrix}
	1&3&2
	\end{pmatrix}
	%%%%%+++++++++++++++++
	\end{aligned}\end{equation}
\end{example}

\begin{note}
	\begin{enumerate}
		%%%%%+++++++++++++++++++++++---------------------
		\item 
		把一置换分解为没有公共客体的轮换乘积时,各轮换长度的集合,称为该轮换的{\color{main}轮换结构}
		\begin{equation}\begin{aligned}
		%%\label{eq.6.1.2}
		%%%%%+++++++++++++++++++++++---------------------
		R={}&
		\begin{pmatrix}
		1&3&5
		\end{pmatrix}
		%%%%%+++++++++++++++++++++++
		%%%%%+++++++++++++++++++++++---------------------
		\begin{pmatrix}
		2&4
		\end{pmatrix}
		%%%%%+++++++++++++++++++++++
		%%%%%+++++++++++++++++++++++---------------------
		=\begin{pmatrix}
		2&4
		\end{pmatrix}
		%%%%%+++++++++++++++++++++++
		%%%%%+++++++++++++++++++++++---------------------
		\begin{pmatrix}
		1&3&5
		\end{pmatrix}\quad \mathrm{structure~is}(3,2)\\
		%%%%%+++++++++++++++++++++++---------------------
		S={}&
		\begin{pmatrix}
		1&3&2
		\end{pmatrix}
		%%%%%+++++++++++++++++++++++
		%%%%%+++++++++++++++++++++++---------------------
		\begin{pmatrix}
		4&5
		\end{pmatrix}
		%%%%%+++++++++++++++++++++++
		%%%%%+++++++++++++++++++++++---------------------
		\begin{pmatrix}
		1&3&2
		\end{pmatrix}\quad \mathrm{structure~is}(3,1,1)=(3,1^2)\\
		%%%%%+++++++++++++++++++++++
		\end{aligned}\end{equation}
		%%%%%+++++++++++++++++++++++---------------------
		\item 
		把一个正整数$n$分解为若干个正整数$l_i$之和,
		这样的正整数的集合称为$n$的一组{\color{main}配分数}
		\begin{equation}\begin{aligned}
		%%\label{eq.6.1.2}
		%%%%%+++++++++++++++++++++++---------------------
		n=4,~ \mathrm{possible~allocations:}(4),(3,1),(2,2),(2,1^2),(1^4)
		%%%%%+++++++++++++++++++++++
		\end{aligned}\end{equation}
		%%%%%+++++++++++++++++++++++---------------------
		\item
		$n$个客体的任一置换1的轮换结构为
		\begin{equation}\begin{aligned}
		%%\label{eq.6.1.2}
		%%%%%+++++++++++++++++++++++---------------------
		\begin{pmatrix}
		l_1&l_2&\cdots
		\end{pmatrix},\qquad
		\sum_i l_i =n
		%%%%%+++++++++++++++++++++++
		\end{aligned}\end{equation}		
	\end{enumerate}
\end{note}

\begin{proposition}{胶水公式}{}
	\begin{equation}\begin{aligned}
	%%\label{eq.6.1.2}
	%%%%%+++++++++++++++++++++++---------------------
	\begin{pmatrix}
	a&b&\cdots&c&{\color{main}d}
	\end{pmatrix}
	%%%%%+++++++++++++++++++++++
	%%%%%+++++++++++++++++++++++---------------------
	\begin{pmatrix}
	{\color{main}d}&e&\cdots&f
	\end{pmatrix}
	%%%%%+++++++++++++++++++++++
	%%%%%+++++++++++++++++++++++---------------------
	=\begin{pmatrix}
	a&b&\cdots&c&{\color{main}d}&e&\cdots&f
	\end{pmatrix}
	%%%%%+++++++++++++++++++++++
	\end{aligned}\end{equation}
\end{proposition}

\begin{note}
	\begin{enumerate}
		%%%%%+++++++++++++++++++++++---------------------
		\item 
		处理有一个公共客体的轮换乘积:在每个轮换内部,把公共客体顺序移到最右或最左,然后按上式把两个轮换接起来。
		
		%%%%%+++++++++++++++++++++++---------------------
		\item 同理也可以把一个轮换分成两个轮换
	\end{enumerate}
\end{note}

\begin{example} 轮换的合并与截断
	
	\begin{enumerate}
		%%%%%+++++++++++++++++++++++---------------------
		\item 
		%%some comment
		\begin{equation}\begin{aligned}
		%%\label{eq.6.1.2}
		%%%%%+++++++++++++++++++++++---------------------
		&\begin{pmatrix}
		1&2&3
		\end{pmatrix}
		%%%%%+++++++++++++++++++++++
		%%%%%+++++++++++++++++++++++---------------------
		\begin{pmatrix}
		4&2&5&6
		\end{pmatrix}\\
		%%%%%+++++++++++++++++++++++
		%%%%%+++++++++++++++++++++++---------------------
		{}={}&\begin{pmatrix}
		3&1&2
		\end{pmatrix}
		%%%%%+++++++++++++++++++++++
		%%%%%+++++++++++++++++++++++---------------------
		\begin{pmatrix}
		2&5&6&4
		\end{pmatrix}
		%%%%%+++++++++++++++++++++++
		%%%%%+++++++++++++++++++++++---------------------
		=\begin{pmatrix}
		3&1&2&5&6&4
		\end{pmatrix}
		%%%%%+++++++++++++++++++++++
		\end{aligned}\end{equation}
		
		%%%%%+++++++++++++++++++++++---------------------
		\item 
		%%some comment
		\begin{equation}\begin{aligned}
		%%\label{eq.6.1.2}
		%%%%%+++++++++++++++++++++++---------------------
		\begin{pmatrix}
		1&2&3&4&5
		\end{pmatrix}
		%%%%%+++++++++++++++++++++++
		%%%%%+++++++++++++++++++++++---------------------
		={}&\begin{pmatrix}
		1&2
		\end{pmatrix}
		%%%%%+++++++++++++++++++++++
		%%%%%+++++++++++++++++++++++---------------------
		\begin{pmatrix}
		2&3&4&5
		\end{pmatrix}\\
		%%%%%+++++++++++++++++++++++
		%%%%%+++++++++++++++++++++++---------------------
		={}&\begin{pmatrix}
		1&2&3
		\end{pmatrix}
		%%%%%+++++++++++++++++++++++
		%%%%%+++++++++++++++++++++++---------------------
		\begin{pmatrix}
		3&4&5
		\end{pmatrix}
		%%%%%+++++++++++++++++++++++
		%%%%%+++++++++++++++++++++++---------------------
		=\begin{pmatrix}
		1&2&3&4
		\end{pmatrix}
		%%%%%+++++++++++++++++++++++
		%%%%%+++++++++++++++++++++++---------------------
		\begin{pmatrix}
		4&5
		\end{pmatrix}
		%%%%%+++++++++++++++++++++++
		\end{aligned}\end{equation}	
		
		
		%%%%%+++++++++++++++++++++++---------------------
		\item 
		%%多重复轮换截断
		\begin{equation}\begin{aligned}
		%%\label{eq.6.1.2}
		%%%%%+++++++++++++++++++++++---------------------
		&\phantom{{}={}}\begin{pmatrix}
		5&1&2&4
		\end{pmatrix}
		%%%%%+++++++++++++++++++++++
		%%%%%+++++++++++++++++++++++---------------------
		\begin{pmatrix}
		4&3&2&6
		\end{pmatrix}\\
		%%%%%+++++++++++++++++++++++
		%%%%%+++++++++++++++++++++++---------------------
		&{}={}\begin{pmatrix}
		5&1&2
		\end{pmatrix}
		%%%%%+++++++++++++++++++++++
		%%%%%+++++++++++++++++++++++---------------------
		\begin{pmatrix}
		2&4
		\end{pmatrix}
		%%%%%+++++++++++++++++++++++
		%%%%%+++++++++++++++++++++++---------------------
		\begin{pmatrix}
		4&3&2
		\end{pmatrix}
		%%%%%+++++++++++++++++++++++
		%%%%%+++++++++++++++++++++++---------------------
		\begin{pmatrix}
		2&6
		\end{pmatrix}\\
		%%%%%+++++++++++++++++++++++
		%%%%%+++++++++++++++++++++++---------------------
		&{}={}\begin{pmatrix}
		5&1&2
		\end{pmatrix}
		%%%%%+++++++++++++++++++++++
		%%%%%+++++++++++++++++++++++---------------------
		\begin{pmatrix}
		2&4
		\end{pmatrix}
		%%%%%+++++++++++++++++++++++
		%%%%%+++++++++++++++++++++++---------------------
		\begin{pmatrix}
		2&4
		\end{pmatrix}
		%%%%%+++++++++++++++++++++++
		%%%%%+++++++++++++++++++++++---------------------
		\begin{pmatrix}
		4&3
		\end{pmatrix}
		%%%%%+++++++++++++++++++++++
		%%%%%+++++++++++++++++++++++---------------------
		\begin{pmatrix}
		2&6
		\end{pmatrix}\\
		%%%%%+++++++++++++++++++++++
		%%%%%+++++++++++++++++++++++---------------------
		&{}={}\begin{pmatrix}
		5&1&2
		\end{pmatrix}
		%%%%%+++++++++++++++++++++++
		%%%%%+++++++++++++++++++++++---------------------
		\begin{pmatrix}
		2&6
		\end{pmatrix}
		%%%%%+++++++++++++++++++++++
		%%%%%+++++++++++++++++++++++---------------------
		\begin{pmatrix}
		4&3
		\end{pmatrix}\\
		%%%%%+++++++++++++++++++++++
		%%%%%+++++++++++++++++++++++---------------------
		&{}={}\begin{pmatrix}
		5&1&2&6
		\end{pmatrix}
		%%%%%+++++++++++++++++++++++
		%%%%%+++++++++++++++++++++++---------------------
		\begin{pmatrix}
		4&3
		\end{pmatrix}
		%%%%%+++++++++++++++++++++++
		\end{aligned}\end{equation}	
		
		{\color{main}把轮换拆成相邻两个轮换只含一个重复公共客体的形式后再相乘}	
	\end{enumerate}	
\end{example}

\begin{proposition}{置换群的类}{}
	\begin{enumerate}
		\item $R$的共轭元素:$SRS^{-1}$
		
		\item 把$R$置换的上下两行数字同时作$S$置换即得$R$置换的共轭元素$SRS^{-1}$
		
		\item 互相共轭的两个置换有相同的轮换结构
	\end{enumerate}
\end{proposition}


\begin{note} 参考 note-\ref{6.1.permutation}
	
	当$R$是轮换时,
	\begin{equation}\begin{aligned}
	%%\label{eq.6.1.2}
	S
	\begin{pmatrix}a&b&c&\cdots &d\end{pmatrix}
	S^{-1}
	=\begin{pmatrix}S_a&S_b&S_c&\cdots &S_d\end{pmatrix}
	\end{aligned}\end{equation}
	\begin{enumerate}
		
		\item	共轭轮换不改变轮换的长度,只改变轮换设计的客体编号
		
		\item 互相共轭的两置换具有相同的轮换结构
		
		\item 亦可证明,有相同轮换结构的两置换必定互相共轭
		
	\end{enumerate}
\end{note}

\begin{note}
	\begin{enumerate}
		%%%%%+++++++++++++++++++++++---------------------
		\item {\color{main}置换群的类由置换的轮换结构来描写}
		%%%%%+++++++++++++++++++++++---------------------
		\item {\color{main}置换群的类数等于整数$n$分解为不同配分数的数目}
	\end{enumerate}
\end{note}

\begin{theorem}{类的元素数目}{}
	如果群$S_n$的类包含$\nu_1$个$1$循环,$\nu_2$个$2$循环,$\cdots$,$\nu_n$个$n$循环,即它的轮换结构为
	\begin{equation}\begin{aligned}
	%%\label{eq.6.1.2}
	(l)=(1^{\nu_1},2^{\nu_2},\cdots,n^{\nu_n}),
	~1\nu_1+2\nu_2+\cdots+n\nu_n=n
	\end{aligned}\end{equation}
	则该类所包含的元素个数为
	\begin{equation}\begin{aligned}
	%%\label{eq.6.1.2}
	C_l=\frac{n!}
	{
		1^{\nu_1}2^{\nu_2}\cdots n^{\nu_n}
		\nu_1! \nu_2! \cdots \nu_n!
	}
	\end{aligned}\end{equation}
\end{theorem}

%\begin{note}
%	证明略
%	%	%comment
%	%	\begin{equation}\begin{aligned}
%	%	%%\label{eq.6.1.2}
%	%	
%	%	\end{aligned}\end{equation}
%\end{note}

\begin{lemma}{置换群元的奇偶性}{}
	%%some comment
	\begin{enumerate}
		%%%%%+++++++++++++++++++++++---------------------
		\item 
		任何置换都可分解为若干个对换的乘积,分解方式不唯一,但它包含对换个数的奇偶性是确定的\\
		长度为{\color{main}奇数}的轮换可分解为{\color{main}偶数}个对换的乘积--{\color{main}偶置换}\\
		长度为{\color{main}偶数}的轮换可分解为{\color{main}奇数}个对换的乘积--{\color{main}奇置换}
		%%%%%+++++++++++++++++++++++---------------------
		\item 
		两个{\color{main}偶置换}或两个{\color{main}奇置换}的乘积是{\color{main}偶置换}\\
		一个{\color{main}偶置换}和一个{\color{main}奇置换}的乘积是{\color{main}奇置换}\\
		恒元是偶置换
		%%%%%+++++++++++++++++++++++---------------------
		\item 
		$n>1$时候,除了恒等表示,$S_n$至少还有一个一维非恒等表示,称为反对称表示,\\
		置换$R$在在该表示中的值称为它的{\color{main}置换宇称},记作$\delta(R)$
		\begin{equation}\begin{aligned}
		%%\label{eq.6.1.2}
		\delta(R)=
		%%%%%+++++++++++++++++++++++
		\begin{cases}
		{1},\quad &\text{$R$是偶置换 } \\
		%%%%%+++++++++++++++++++++++
		{-1},\quad &\text{$R$是奇置换}
		\end{cases}
		\end{aligned}\end{equation}
	\end{enumerate}
\end{lemma}


\begin{definition}{交变子群}{}
	\begin{enumerate}
		\item 
		置换群中所有偶置换的集合构成指数为$2$的不变子群,
		称为{\color{main} 交变子群 }
		\item
		奇置换的集合是它的陪集
		商群是$c_2$群
	\end{enumerate}
\end{definition}


\begin{lemma}{置换群的生成元}{}
	\begin{enumerate}
		%%%%%+++++++++++++++++++++++---------------------
		\item 
		相邻客体的对换:$P_a=(a~~a+1)$
		%%%%%+++++++++++++++++++++++---------------------
		\item 
		任何置换都可以写成{\color{main}无公共客体}轮换的乘积,任何轮换都可分解为若干对换的乘积。
		%%%%%+++++++++++++++++++++++---------------------
		\item
		任何对换都可以表示为{\color{main}相邻客体}对换的乘积
		%%%%%+++++++++++++++++++++++---------------------
		\item
		任何置换都可以表示为{\color{main}相邻客体对换}的乘积
		%%%%%+++++++++++++++++++++++---------------------
		\item
		引入长度为$n$的轮换$W=(1~~2~~\cdots~~n)$\\
		则:$P_a=W P_{a-1} W^{-1}=W^2 P_{a-2} W^{-2}=\cdots=W^{a-1} P_{1} W^{-(a-1)}$\\
		即:{\color{main}任何相邻客体的对换可由$W$和$P_{1}$生成}
	\end{enumerate}
\end{lemma}

\begin{theorem}{置换群的生成元}{}
	置换群的生成元是$W$和$P_{1}$,置换群的秩为$2$
\end{theorem}

\clearpage

\begin{theorem}{Cayley定理}{}
	任何一个$n$阶有限群都与置换群$S_n$的一个子群同构
\end{theorem}


\begin{corollary}{$n$阶有限群的数目}{}
	%%some comment
	\begin{enumerate}
		%%%%%+++++++++++++++++++++++---------------------
		\item 
		若置换群$S_n$的子群与$n$阶有限群$G$同构,则该子群中的元素除恒等置换外,
		任一置换所包含的无公共客体的轮换的轮换长度相等
		%%%%%+++++++++++++++++++++++---------------------
		\item 
		$S-N$的子群数目是有限的,满足上述性质的不同构的子群的数目更加有限
		%%%%%+++++++++++++++++++++++---------------------
		\item 
		不同构的$n$阶有限群的数目是有限的
	\end{enumerate}
\end{corollary}

%\clearpage

\section{杨图、杨表和杨算符}

\begin{lemma}{置换}{置换群$S_n$的类}
	%%some comment
	\begin{enumerate}
		%%%%%+++++++++++++++++++++++---------------------
		\item 
		置换群$S_n$的类的个数等于$n$分解为不同组配分数的数目,\\
		故置换群不等价不可约表示的个数也等于$n$分解为不同组配分数的数目
		%%%%%+++++++++++++++++++++++---------------------
		\item 
		置换群$S_n$的类由$n$的配分数$(\lambda)=(\lambda_1,\lambda_2,\cdots \lambda_m)$描写,
		不等价不可约表示也可以用配分数来描写,\\
		记作$[\lambda]=[\lambda_1,\lambda_2,\cdots \lambda_m]$,其中
		\begin{equation}
		\lambda_1 \geq \lambda_2 \geq \cdots \geq \lambda_m \geq 0,~~\sum\limits_{j=1}^m\lambda_j=n
		\end{equation}
		不过,由相同配分数描写的类和不等价不可约表示并无任何关系。
	\end{enumerate}
\end{lemma}

\begin{definition}{杨图}{}
	对配分数$[\lambda]=[\lambda_1,\lambda_2,\cdots \lambda_3,]$,
	画$m$行放个图,{\color{main}左边对齐},\\
	第一行含$\lambda_1$个,第二行含$\lambda_2$格,以此类推,\\
	这样的方格图称为配分数$[\lambda]$对应的{\color{main}杨图},简称杨图$[\lambda]$
\end{definition}


\begin{note}
	%%some comment
	\begin{enumerate}
		%%%%%+++++++++++++++++++++++---------------------
		\item 
		杨图中,上面行的格子数不少于下面行的格子数,\\
		左边列的格子数不少于右边列的格子数
		为强调这一规则,称它为{\color{main}正则杨图}。
		%%%%%+++++++++++++++++++++++---------------------
		\item 
		{\color{main}每个杨图都唯一地对应于置换群$S_n$的一个不可约表示,
			不同杨图对应的不可约表示不等价。
		}
		
		\item 
		{\color{main}杨图的大小}:
		从第一行开始逐行比较,格子多的杨图大
	\end{enumerate}
\end{note}



\begin{example} $S_4$ 群的杨图从大到小排列为
	
	%%some comment
	\begin{equation}\begin{aligned}
	%%\label{eq.6.1.2}
	%%%%%+++++++++++++++++++++++---------------------
	\ydiagram{4}&\quad&\ydiagram{3,1}&\quad&\ydiagram{2,2}&\quad&\ydiagram{2,1,1}&\quad&\ydiagram{1,1,1,1}\\
	[4]&\quad&[3,1]&\quad&[2,2]&\quad&[2,1^2]&\quad&[1^4]
	%%%%%+++++++++++++++++++++++
	\end{aligned}\end{equation}
\end{example}


\begin{note}
	%%some comment
	\begin{enumerate}
		%%%%%+++++++++++++++++++++++---------------------
		\item 
		把杨图$[\lambda]$的行和列互换得到的杨图$[\tilde{\lambda}]$
		称为杨图$[\lambda]$的{\color{main}对偶杨图},对应的不可约表示称为对偶表示\\
		例:$S_3$群的杨图$[3]$和$[1^3]$互为对偶杨图\\
		\phantom{例:}$S_4$群的杨图$[4]$和$[1^4]$以及$[3,1]$和$[2,1^2]$分别为对偶杨图
		%%%%%+++++++++++++++++++++++---------------------
		\item 
		若杨图$[\lambda]=[\tilde{\lambda}]$则称为自偶杨图\\
		例:$S_3$群的杨图$[2,1]$为自偶杨图\\
		\phantom{例:}$S_4$群的杨图$[2,2]$为自偶杨图
	\end{enumerate}
\end{note}


\begin{definition}{杨表与正则杨表}{}
	%%some comment
	\begin{enumerate}
		%%%%%+++++++++++++++++++++++---------------------
		\item 
		对于给定的杨表$[\lambda]$,把$1$到$n$的$n$个自然数分别填入杨图的$n$个格子中,
		就得到一个{\color{main}杨表}
		%%%%%+++++++++++++++++++++++---------------------
		\item 
		$n$格的杨图有$n!$个不同的杨表
		%%%%%+++++++++++++++++++++++---------------------
		\item 
		如果在杨表的每一行中,左面的填数小于右边的填数,在每一列中,
		上面的填数小于下面的填数,则此杨表称为{\color{main}正则杨表}
		%%%%%+++++++++++++++++++++++---------------------
		\item 
		{\color{main}正则杨表的大小}:同一杨图对应的正则杨表,
		从第一行开始逐行从左到右比较它们的填数,第一次出现填数不同时,填数大的杨表大
		
		%%%%%+++++++++++++++++++++++---------------------
		例如,杨图1对应的全部正则杨表从小到大排列为
		\ytableausetup{mathmode, boxsize=1.2em}
		\begin{equation}\begin{aligned}
		%%\label{eq.6.1.2}
		%%%%%+++++++++++++++++++++++---------------------
		\begin{ytableau}
		1 & 2 & 3 \\
		4 & 5  \\
		\end{ytableau}&\quad&
		%%%%%+++++++++++++++++++++++
		%%%%%+++++++++++++++++++++++
		\begin{ytableau}
		1 & 2 & 4 \\
		3 & 5   \\
		\end{ytableau}&\quad&
		%%%%%+++++++++++++++++++++++
		%%%%%+++++++++++++++++++++++
		\begin{ytableau}
		1 & 2 & 5 \\
		3 & 4   \\
		\end{ytableau}&\quad&
		%%%%%+++++++++++++++++++++++
		%%%%%+++++++++++++++++++++++
		\begin{ytableau}
		1 & 3 & 4 \\
		2 & 5   \\
		\end{ytableau}&\quad&
		%%%%%+++++++++++++++++++++++
		%%%%%+++++++++++++++++++++++
		\begin{ytableau}
		1 & 3 & 5 \\
		2 & 4   \\
		\end{ytableau}
		%%%%%+++++++++++++++++++++++
		\end{aligned}\end{equation}
	\end{enumerate}
\end{definition}


\begin{theorem}{维数定理}{}
	{\color{main}置换群$S_n$的不可约表示$[\lambda]$的维数,等于杨图$[\lambda]$对应的正则杨表的个数}
\end{theorem}

\begin{note}
	%%some comment
	\begin{enumerate}
		%%%%%+++++++++++++++++++++++---------------------
		\item 
		杨图$[\lambda]$对应的不可约表示的维数(即正则杨表的个数)由钩形规则给出
		%%%%%+++++++++++++++++++++++---------------------
		%%%%%+++++++++++++++++++++++---------------------
		\item 
		杨图中任一格子的钩形数,等于该格子所在行右面的格子数$+$该格子所在列下面的格子再$+1$
		\item 
		杨图$[\lambda]$对应的不可约表示的维数为
		\begin{equation}\begin{aligned}
		%%\label{eq.6.1.2}
		%%%%%+++++++++++++++++++++++---------------------
		{\color{main}d_{[\lambda]}(S_n)=\frac{n!}{\prod\limits_{ij}h_{ij}}}
		%%%%%+++++++++++++++++++++++
		\end{aligned}\end{equation}
		%%%%%+++++++++++++++++++++++---------------------
		\item 
		{\color{main}钩形数杨表}:将杨图$[\lambda]$中每格的钩形数$h_{ij}$填入该杨图,
		得到的杨表称为该杨表的钩形数杨表
	\end{enumerate}
\end{note}


\begin{example}  $S_3$群各个不可约表示的维数
	
	%%some comment
	\begin{equation}\begin{aligned}
	%%\label{eq.6.1.2}
	%%%%%+++++++++++++++++++++++---------------------
	&\quad&\text{不可约表示}&\quad&\text{钩形数杨表}&\quad&\text{不可约表示的维数}\\
	%%%%%%%%%%%%%%%% start row
	&\quad&
	[3]:
	&\quad&
	%%%%%+++++++++++++++++++++++---------------------
	\begin{ytableau}
	3 & 2 & 1 
	\end{ytableau}%%&\quad&
	%%%%%+++++++++++++++++++++++
	&\quad&
	d_{[3]}(S_3)
	=\frac{3!}{3\times 2\times 1}
	=\frac{3\times 2\times 1}{3\times 2\times 1}
	=1 \\
	%%%%%%%%%%%%%%%% end row
	%%%%%%%%%%%%%%%% start row
	&\quad&
	[2,1]:
	&\quad&
	%%%%%+++++++++++++++++++++++---------------------
	\begin{ytableau}
	3 & 1\\
	1 
	\end{ytableau}%%&\quad&
	%%%%%+++++++++++++++++++++++
	&\quad&
	d_{[2,1]}(S_3)
	=\frac{3!}{3\times 1\times 1}
	=\frac{3\times 2\times 1}{3\times 1\times 1}
	=2 \\
	%%%%%%%%%%%%%%%% end row
	%%%%%%%%%%%%%%%% start row
	&\quad&
	[1^3]:
	&\quad&
	%%%%%+++++++++++++++++++++++---------------------
	\begin{ytableau}
	3 \\
	2 \\
	1
	\end{ytableau}%%&\quad&
	%%%%%+++++++++++++++++++++++
	&\quad&
	d_{[1^3]}(S_3)
	=\frac{3!}{3\times 2 \times 1}
	=\frac{3\times 2\times 1}{3\times 2\times 1}
	=1
	%%%%%%%%%%%%%%%% end row
	\end{aligned}\end{equation}
\end{example}


对于给定的杨图$[\lambda]=[\lambda_1,\lambda_2,\cdots,\lambda_m]$,
其对偶杨图记为
$[\tilde{\lambda}]=[\tau_1,\tau_2,\cdots,\tau_{\lambda1}]$;
考虑杨图$[\lambda]$对应的某一正则杨表

\begin{lemma}{横纵置换}{}
	\begin{enumerate}
		
		%%%%%+++++++++++++++++++++++---------------------
		\item 
		保持杨表中同一行数字只在这一行中变动的置换称为{\color{main}横向置换},记作$p$,
		所有横向置换的集合记作$R(\lambda)=\{p|p\in S_n \}$.
		%%%%%+++++++++++++++++++++++*********************
		\begin{enumerate}
			%%%%%+++++++++++++++++++++++*********************
			\item 
			第$i$行$\lambda_i$个数字间的$\lambda_i !$个横向置换构成的集合构成$S_n$群的子群$P_i$
			%%%%%+++++++++++++++++++++++*********************
			\item 
			$m$行的正则杨表共有$m$个这样的子群,
			它们的直乘\footnote{恒元为唯一公共元素,分属不同子群的元素可对易}
			构成$S_n$群$\lambda_1! \lambda_2! \cdots \lambda_m!$阶的子群,
			记为$R(\lambda)=P_1\otimes P_2\otimes \cdots \otimes P_m$
		\end{enumerate}
		%%%%%+++++++++++++++++++++++---------------------
		\item
		保持杨表中同一列数字只在这一列中变动的置换称为1纵向置换,记作1,所有纵向置换的集合记作1.
		%%%%%+++++++++++++++++++++++*********************
		\begin{enumerate}
			%%%%%+++++++++++++++++++++++*********************
			\item 
			第$j$列$\tau_j$个数字间的$\tau_j!$个纵向置换构成的集合构成$S_n$群的子群$Q_j$
			%%%%%+++++++++++++++++++++++*********************
			\item 
			$\lambda_1$列的正则杨表共有$\lambda_1$个这样的子群,
			它们的直乘构成$S_n$群$\tau_1! \tau_2! \cdots \tau_m!$阶的子群,
			记为$C(\lambda)=Q_1\otimes Q_2\otimes \cdots \otimes Q_{\lambda_1}$
		\end{enumerate}	
	\end{enumerate}
\end{lemma}


\begin{lemma}{横算符和纵算符}{}
	%%some comment
	\begin{enumerate}
		%%%%%+++++++++++++++++++++++---------------------
		\item 
		所有横向置换之和称为给定杨表的{\color{main}横算符}
		{
			\color{main}
			\begin{equation}
			\mathcal{P}=\sum\limits_{p\in R(\lambda)}=\prod\limits_i P_i
			\end{equation}
		}
		%%%%%+++++++++++++++++++++++---------------------
		\item 
		所有纵向置换乘以置换宇称后相加,称为给定杨表的纵算符
		{
			\color{main}
			\begin{equation}
			\mathcal{Q}=\sum\limits_{q\in C(\lambda)} \delta(q) q
			\end{equation}
		}
		%%%%%+++++++++++++++++++++++---------------------
		\item
		横算符和纵算符之乘积称为给定杨表的{\color{main}杨算符},
		正则杨表对应的杨算符称为{\color{main}正则杨算符}。
		{
			\color{main}
			\begin{equation}
			\mathcal{Y}
			=\mathcal{P}\mathcal{Q}
			=\sum\limits_{p\in R(\lambda)} \sum\limits_{q\in C(\lambda)}
			\delta(q)\,p\,q
			\end{equation}
		}
		%%%%%+++++++++++++++++++++++---------------------
		\item 
		横向置换、纵向置换、横算符、纵算符、杨算符均为群代数中的矢量
		%%%%%+++++++++++++++++++++++---------------------
		\item 
		横向置换的集合$R(\lambda)$与纵向置换的集合$C(\lambda)$只有一个公共元素恒元,
		故杨算符$\mathcal{Y}$展开式中每一项$p\,q$都是$S_n$群的不同元素,
		因此{\color{main}$\mathcal{Y}\neq 0$}
		%%%%%+++++++++++++++++++++++---------------------
		\item 
		只有在给定杨图和杨表时,才能写出杨算符$\mathcal{Y}$,
		故通常把杨算符$\mathcal{Y}$对应的杨图和杨表,称为杨图$\mathcal{Y}$和杨表$\mathcal{Y}$;
		若单独说$\mathcal{Y}$,则指杨算符本身
	\end{enumerate}
\end{lemma}

\begin{note}
	%%some comment
	\begin{enumerate}
		%%%%%+++++++++++++++++++++++---------------------
		\item 
		给定杨表{\color{main}横算符的写法}:
		先把每一行的横向置换加起来,再把不同行的横向置换之和乘起来
		%%%%%+++++++++++++++++++++++---------------------
		\item 
		给定杨表{\color{main}纵算符的写法}:先把每一列的所有纵向置换乘上各自的置换宇称后加起来,
		再把不同列的纵向置换之代数和乘起来
	\end{enumerate}
\end{note}

\begin{example} $S_3$群各不可约表示杨图对应的正则杨表的杨算符
	
	\ytableausetup{mathmode, boxsize=1em}
	\begin{equation}\begin{aligned}
	%%\label{eq.6.1.2}
	%%%%%+++++++++++++++++++++++---------------------
	\begin{ytableau}
	1 & 2 & 3 \\
	\end{ytableau} &\quad&
	%%%%%+++++++++++++++++++++++
	\mathcal{Y}^{[3]}=E+(1~~2)+(1~~3)+(2~~3)+(1~~2~~3)+(1~~3~~2)\\
	%%%%%+++++++++++++++++++++++
	%%%%%+++++++++++++++++++++++---------------------
	\begin{ytableau}
	1 & 2 \\
	3 
	\end{ytableau} &\quad&
	%%%%%+++++++++++++++++++++++
	\mathcal{Y}^{[2,1]}=\{E+(1~~2)\}+\{E-(1~~3)\}
	=E+(1~~2)-(1~~3)-(1~~3~~2)\\
	%%%%%+++++++++++++++++++++++
	%%%%%+++++++++++++++++++++++---------------------
	\begin{ytableau}
	1 & 3 \\
	2 
	\end{ytableau} &\quad&
	%%%%%+++++++++++++++++++++++
	\mathcal{Y}^{[2,1]}=\{E+(1~~3)\}+\{E-(1~~2)\}
	=E+(1~~3)-(1~~2)-(1~~2~~3)\\
	%%%%%+++++++++++++++++++++++
	%%%%%+++++++++++++++++++++++---------------------
	\begin{ytableau}
	1 & 2& 3 \\
	\end{ytableau} &\quad&
	%%%%%+++++++++++++++++++++++
	\mathcal{Y}^{[1^3]}
	=E-(1~~2)-(1~~3)-(2~~3)+(1~~2~~3)+(1~~3~~2)
	%%%%%+++++++++++++++++++++++
	\end{aligned}\end{equation}
\end{example}


\section{置换群的不可约标准表示}

\begin{theorem}{置换群的原始幂等元}{}
	{\color{main}杨算符$\mathcal{Y}$是置换群群代数$\mathcal{L}(S_n)$本质的原始幂等元},\\
	最小左理想$\mathcal{L}(S_n) \mathcal{Y}$给出$S_n$群的一个不可约表示;\\
	{\color{main}由同一杨图的不同正则杨表给出的表示是等价的,不同杨图给出的表示是不等价的}。
	
\end{theorem}

\begin{note}
	%%some comment
	\begin{enumerate}
		%%%%%+++++++++++++++++++++++---------------------
		\item 
		$(\dfrac{f}{n!})\mathcal{Y}$是置换群的原始幂等元($f$是不可约表示的维数)
		%%%%%+++++++++++++++++++++++---------------------
		\item 
		等价的原始幂等元不一定正交;\\
		不等价的原始幂等元一定正交
		%%%%%+++++++++++++++++++++++---------------------
		\item 
		$n\geq5$时会出现同一杨图的不同正则杨表对应的杨算符可能不正交的情况。
		%%%%%+++++++++++++++++++++++---------------------
		\item 
		一行的杨图对应一维恒等表示
	\end{enumerate}
\end{note}


\begin{example} $S_n$群的杨图$[n]$确定的不可约表示
	
	该杨图只有一个正则杨表
	%%%%%+++++++++++++++++++++++---------------------
	$
	%%\label{eq.6.1.2}
	\begin{ytableau}
	1 & 2 & \cdots & n
	\end{ytableau}%%&\quad&
	%%%%%+++++++++++++++++++++++
	$
	
	该杨表的杨算符为\quad $\mathcal{Y}^{[n]}=\sum\limits_i p_i,\quad p_i \in S_n$
	
	由重排定理,对任意$t \in S_n $,有
	\begin{equation}
	t \mathcal{Y}^{[n]}=t \sum\limits_i p_i = \sum\limits_i p_i=\mathcal{Y}^{[n]}
	\end{equation}
	
	故杨图$[n]$对应$S_n$群的一维恒等表示
\end{example}


\begin{example} $S_n$ 群的杨图 $[1^n]$ 确定的不可约表示
	
	该杨图只有一个正则杨表
	%%%%%+++++++++++++++++++++++---------------------
	$
	%%\label{eq.6.1.2}
	\begin{ytableau}
	1 \\ 2 \\ \cdots \\ n
	\end{ytableau}%%&\quad&
	%%%%%+++++++++++++++++++++++
	$
	
	该杨表的杨算符为\quad $\mathcal{Y}^{[n]}=\sum\limits_i \delta(q_i) q_i,\quad q_i \in S_n$
	
	由重排定理,对任意$t \in S_n $,有
	\begin{equation}
	t \mathcal{Y}^{[n]}=t \sum\limits_i \delta(q_i) q_i = \sum\limits_i \delta(q_i) t\,q_i
	=t \sum\limits_i \delta(t\,q_i) t\,q_i
	=\delta(t) \mathcal{Y}^{[n]}
	\end{equation}
	
	故杨图$[1^n]$对应$S_n$群的一维全反对称表示
\end{example}

%\clearpage

\begin{note}
	%%some comment
	\begin{enumerate}
		%%%%%+++++++++++++++++++++++---------------------
		\item 
		一列的杨图对应一维全反对称表示
		%%%%%+++++++++++++++++++++++---------------------
		\item 
		由于任意对换作用在上面,给出一个负号
	\end{enumerate}
\end{note}

\begin{example} $S_3 群的不可约标准表示$
	杨图1有两个正则杨表,给出两个等价的二维不可约表示。
	以其中一个为例:
	\begin{equation}\begin{aligned}
	%%\label{eq.6.1.2}
	%%%%%+++++++++++++++++++++++---------------------
	%%\label{eq.6.1.2}
	%%%%%+++++++++++++++++++++++---------------------
	\begin{ytableau}
	1 & 2  \\
	3 
	\end{ytableau} &\quad&
	\mathcal{Y}^{[2,1]}=(E+(1~~2))(E-(1~~3))=E+(1~~2)-(1~~3)-(1~~3~~2)
	%%%%%+++++++++++++++++++++++
	\end{aligned}\end{equation}
\end{example}

$\mathcal{L}(S_3)\mathcal{Y}^{[2,1]}$确实是二维的。

令
\begin{equation}
	\begin{cases}
	\psi^1=\mathcal{Y}^{[2,1]} \\
	%%%%%+++++++++++++++++++++++
	\psi^2=(1~~3)\mathcal{Y}^{[2,1]} \\
	\end{cases}
\end{equation}

由此得1群生成元在二维不可约标准表示中的表示矩阵
\begin{equation}\begin{aligned}
%%\label{eq.6.1.2}
%%%%%+++++++++++++++++++++++---------------------
D^{[2,1]}(1~~2)=
\begin{pmatrix}
1&-1\\
0&-1
\end{pmatrix} & \quad &
%%%%%+++++++++++++++++++++++---------------------
D^{[2,1]}(1~~2~~3)=
\begin{pmatrix}
0&-1\\
1&-1
\end{pmatrix} 
%%%%%+++++++++++++++++++++++
\end{aligned}\end{equation}



\section{置换群的不可约正交表示}

\subsection{不可约表示按子群链的分解}

\begin{proposition}{分支律}{}%%\ref{pro:label}

	$S_n$群的不可约表示$[\lambda]$对它的子群$S_{n-1}$来时,
	一般是可约的,从杨图$[\lambda]$中按所有可能的方式去掉一个方格后,
	所剩下的如果仍是正则杨图$[\lambda^\prime]$,
	则$[\lambda^\prime]$就是$[\lambda]$作为$S_{n-1}$群的表示进行约化时可能出现的不可约表示,
	且每个$[\lambda^\prime]$只出现一次。
	
\end{proposition}
	 
\begin{example}
	
	例如,$S_5$群的不可约表示$[2^2,1]$中包含$S_4$群的不可约表示$[2^2]$和$[2,1^2]$各一次。
	\begin{equation}\begin{aligned}
	&\phantom{\text{维数}}&\quad
	&\ydiagram{2,2,1}&\quad
	&\to&\quad
	&\ydiagram{2,2}&\quad
	&\bigoplus&\quad
	&\ydiagram{2,1,1} \\
	%%%%%%%%%%%%%%%%%%
	&\text{维数}&\quad
	&5&\quad
	&=&\quad
	&2&\quad
	&+&\quad
	&3
	\end{aligned}\end{equation}
\end{example}

\begin{definition}{name}{}%%\ref{def:label}
	荷载子群链$S_n \supset S_{n-1} \supset S_{n-2} \supset \cdots
	\supset S_{2}$
	中所有子群的 不可约表示的基 互相正交,
	由它们得到的表示称为置换群的{\color{main}实正交表示}
\end{definition}

\begin{note}
	用正则样表标记正交基:从$S_n$的正则杨表中去掉填$n$的格子,
	仍是正则杨表,标识$S_{n-1}$不可约表示的基;
	依次分解下去,杨表逐步缩小的过程反映出
	置换群表示逐步按子群表示分解的过程,
	也确定了基函数按子群链的分类
\end{note}

\begin{example}
\begin{equation}\begin{aligned}
	%%%%%%%%------------------------
	{\color{main}
		\begin{ytableau}
		1&2\\
		3&4\\
		5
	\end{ytableau}
}&\quad&
	%%%%%%%%------------------------
	\to&\quad&
		{\color{second}
		\begin{ytableau}
		1&2\\
		3&4
		\end{ytableau}
	}&\quad&
	%%%%%%%%------------------------
	\to&\quad&
		{\color{third}
		\begin{ytableau}
	1&2\\
	3
	\end{ytableau}}&\quad&		
	%%%%%%%%------------------------
	\to&\quad&
	\begin{ytableau}
	1&2
	\end{ytableau}&\quad&		
	%%%%%%%%------------------------
	\to&\quad&
	\begin{ytableau}
	1
	\end{ytableau}		
\end{aligned}\end{equation}
\end{example}

\subsection{不可约正交表示的具体形式}

\begin{note} %%some comment
	
	\begin{enumerate}
		%%%%%+++++++++++++++++++++++---------------------
		\item 
		任意置换都可以分解为无公共客体的轮换的乘积,
		任一轮换都可以分解为对换的乘积,
		任一对换都可以分解为相邻客体的对换的乘积
		%%%%%+++++++++++++++++++++++---------------------
		\item 
		只要知道了相邻客体的对换$(k-1~~k)$的表示矩阵,
		就可以由乘法求得$S_n$群的任意元素的表示矩阵
		\item 
		用$\mathcal{Y}^{[\lambda]}_r$表示不可约表示
		$[\lambda]$的第$r$个正则杨表,
		$\ket{\mathcal{Y}^{[\lambda]}}$表示
		荷载正交表示$[\lambda]$的基
		\item 
		${\color{main}k-1\text{到}k\text{的轴距}}$
		正则杨表$\mathcal{Y}^{[\lambda]}_r$中,
		从填$k-1$的格子到填$k$的格子,
		向左或向下数一个方格为$+1$,
		向右或向上数一个方格为$-1$,
		这样数出的代数和$\mu$称为数字$k-1$到$k$的轴距
		\item 
		若$k-1$和$k$不在正则杨表$\mathcal{Y}^{[\lambda]}_r$的同一行或同一列,
		则$(k-1~~k)$把正则杨表$\mathcal{Y}^{[\lambda]}_r$变为正则杨表$\mathcal{Y}^{[\lambda]}_s$
		\begin{equation}
			\mathcal{Y}^{[\lambda]}_s
			=(k-1~~k)\,\mathcal{Y}^{[\lambda]}_r
		\end{equation}
		\item
		对换$(k-1~~k)$在正交表示中的表示矩阵
		\begin{enumerate}

		\item 
		相邻客体的的同一行对称,在同一列反对称
		
		\item 
		当$k-1$和$k$在杨表$\mathcal{Y}^{[\lambda]}_r$的同一行或同一列
		\begin{equation}
			(k-1~~k)\ket{\mathcal{Y}^{[\lambda]}_r}
			=-\mu \ket{\mathcal{Y}^{[\lambda]}_r}
		\end{equation}
		\item 
		当$k-1$和$k$不在杨表$\mathcal{Y}^{[\lambda]}_r$的同一行或同一列
		\begin{equation}
		(k-1~~k)\ket{\mathcal{Y}^{[\lambda]}_r}
		=-\frac{1}{\mu}\ket{\mathcal{Y}^{[\lambda]}_r}+
		\frac{\sqrt{\mu^2-1}}{|\mu|}\ket{\mathcal{Y}^{[\lambda]}_s}
		\end{equation}
	\end{enumerate}
		
	\end{enumerate}
\end{note}


\begin{example}
	$S_3$群的实正交表示
	\begin{enumerate}
		\item
		\begin{equation}	\begin{aligned}
		\ytableausetup{mathmode, boxsize=1.5em}
		(1~~2)\ket{
			\begin{ytableau}
			1&2&3
			\end{ytableau}
		}=
		\ket{
			\begin{ytableau}
			1&2&3
			\end{ytableau}
		}
		\end{aligned}\end{equation}
		%%%%%%%%%%%%%%%%%%
		\item
		\begin{equation}	\begin{aligned}
		\ytableausetup{mathmode, boxsize=1.5em}
		(1~~2)\ket{
			\begin{ytableau}
			1\\2\\3
			\end{ytableau}
		}=-	\ket{
			\begin{ytableau}
			1\\2\\3
			\end{ytableau}
		}
		\end{aligned}\end{equation}
		%%%%%%%%%%%%%%%%%%%%%%%%%%%%%
		\item
		\begin{equation}	\begin{aligned}
		\ytableausetup{mathmode, boxsize=1.5em}
		(1~~2)\ket{
			\begin{ytableau}
			1&2\\
			3
			\end{ytableau}
		}=
		\ket{
			\begin{ytableau}
			1&2\\
			3
			\end{ytableau}
		}
		\end{aligned}\end{equation}
%%%%%%%%%%%%%%%%%%
\item
\begin{equation}	\begin{aligned}
\ytableausetup{mathmode, boxsize=1.5em}
(1~~2)\ket{
	\begin{ytableau}
	1&3\\
	2
	\end{ytableau}
}=-\ket{
	\begin{ytableau}
	1&3\\
	2
	\end{ytableau}
}
\end{aligned}\end{equation}	

%%%%%%%%%%%%%%%%%%
\item
\begin{equation}	\begin{aligned}
\ytableausetup{mathmode, boxsize=1.5em}
(2~~3)\ket{
	\begin{ytableau}
	1&2&3
	\end{ytableau}
}=-\ket{
	\begin{ytableau}
	1&2&3
	\end{ytableau}
}
\end{aligned}\end{equation}	

%%%%%%%%%%%%%%%%%%
\item
\begin{equation}	\begin{aligned}
\ytableausetup{mathmode, boxsize=1.5em}
(2~~3)\ket{
	\begin{ytableau}
	1&2\\
	3
	\end{ytableau}
}=-\frac{1}{2}\ket{
	\begin{ytableau}
	1&2\\
	3
	\end{ytableau}
}
+\frac{\sqrt{3}}{2}\ket{
	\begin{ytableau}
	1&3\\
	2
	\end{ytableau}
}
\end{aligned}\end{equation}	

%%%%%%%%%%%%%%%%%%
\item
\begin{equation}	\begin{aligned}
\ytableausetup{mathmode, boxsize=1.5em}
(2~~3)\ket{
	\begin{ytableau}
	1&3\\
	2
	\end{ytableau}
}=\frac{1}{2}\ket{
	\begin{ytableau}
	1&3\\
	2
	\end{ytableau}
}
+\frac{\sqrt{3}}{2}\ket{
	\begin{ytableau}
	1&2\\
	3
	\end{ytableau}
}
\end{aligned}\end{equation}	

%%%%%%%%%%%%%%%%%%
\item
\begin{equation}	\begin{aligned}
\ytableausetup{mathmode, boxsize=1.5em}
(2~~3)\ket{
	\begin{ytableau}
	1\\2\\3
	\end{ytableau}
}=-\ket{
	\begin{ytableau}
	1\\2\\3
	\end{ytableau}
}
\end{aligned}\end{equation}	

\item
\begin{equation}\begin{aligned}
D^{[3]}(2,3)=
-D^{[1^3]}(2,3)=1\\
D^{[2,1]}(2,3)=
-D^{[1^3]}(2,3)=\frac{1}{2}
\begin{pmatrix}
-1&\sqrt{3}\\
\sqrt{3}&1
\end{pmatrix}
\end{aligned}\end{equation}

\item 
其它元素的表示矩阵可由乘法给出:

\begin{equation}\begin{aligned}
%%
D(1~~3)=D(1~~2)D(2~~3)D(1~~2)\\
%%
D(1~~2~~3)=D(1~~2)D(2~~3)\\
%%
D(1~~3~~2)=D(1~~2)D(1~~3)
%%
\end{aligned}\end{equation}

\item
于是有:
\begin{equation}\begin{aligned}
%%%%%%%%%%%%%
D^{[3]}(1~~3)
=D^{[1^3]}(1~~3)
=1
&\quad&
D^{[2,1]}(1~~3)
=\frac{1}{2}
\begin{pmatrix}
-1&-\sqrt{3}\\
-\sqrt{3}&1
\end{pmatrix}\\
%%%%%%%%%%%%%
D^{[3]}(1~~2~~3)
=D^{[1^3]}(1~~2~~3)
=1
&\quad&
D^{[2,1]}(1~~2~~3)
=\frac{1}{2}
\begin{pmatrix}
-1&\sqrt{3}\\
-\sqrt{3}&-1
\end{pmatrix}\\
%%%%%%%%%%%%%
D^{[3]}(1~~3~~2)
=D^{[1^3]}(1~~3~~2)
=1
&\quad&
D^{[2,1]}(1~~3~~2)
=\frac{1}{2}
\begin{pmatrix}
-1&-\sqrt{3}\\
\sqrt{3}&-1
\end{pmatrix}
%%%%%%%
\end{aligned}\end{equation}

\end{enumerate}
\end{example}

\subsection{不可约表示的基函数}

\begin{enumerate}
	
	\item 
	有了不可约正交表示的表示矩阵,可得投影算符
	\begin{equation}
	\mathcal{P}^{[\lambda]}_{rs}
	=\frac{d_{[\lambda]}}{n!}\sum\limits_{R\in S_n} D^{[\lambda]}_{rs}(R)R
	\end{equation}
	它作用在有置换变换的函数上,可得具有指定对称性$[\lambda]$的波函数
	
	\item
	设$n$粒子系统的波函数为$\phi(1,2,\cdots,n)$,
	其中$1,2,\cdots,n$为粒子的坐标,
	则具有对称性$[\lambda]$的$d_{[\lambda]}$个波函数为($s$固定)
	\begin{equation}
		\psi^{[\lambda]}_{rs}
		=\mathcal{P}^{[\lambda]}_{rs} \, \phi(1,2,\cdots,n)
		=\frac{d_{[\lambda]}}{n!}\sum\limits_{R\in S_n} D^{[\lambda]}_{rs}(R)\,R
		\,\phi(1,2,\cdots,n)
	\end{equation}
\end{enumerate}


\begin{example} 由组态$\phi=uds$构造$S_3$群$[3]$表示的基
	
\begin{equation}
	\mathcal{P}^{[3]}_{11} =
	\frac{1}{6}[E +(132)+(123)+(12)+(23)+(13)] 
\end{equation}
	
\begin{equation}
\psi^{[3]} =\mathcal{P}^{[3]}_{11}=
\frac{1}{6}[uds+dsu+sud+dus+usd+sdu] 
\end{equation}

\begin{note}
	flavor wave function of $\Sigma^\ast$
\end{note}
	
\begin{equation}
%%\label{eq.6.1.2}
%%%%%+++++++++++++++++++++++
\begin{cases}
\mathcal{P}^{[2,1]}_{11} ={}&
\frac{1}{6}[2E -(132)-(123)+2(12)-(23)-(13)]  \\
%%%%%+++++++++++++++++++++++
\mathcal{P}^{[2,1]}_{21} ={}&
\frac{\sqrt{3}}{6}[(132)-(123)+(23)-(13)]  \\
%%%%%+++++++++++++++++++++++
\mathcal{P}^{[2,1]}_{12} ={}&
\frac{\sqrt{3}}{6}[-(132)+(123)+(23)-(13)]  \\
%%%%%+++++++++++++++++++++++
\mathcal{P}^{[2,1]}_{22} ={}&
\frac{1}{6}[2E-(132)-(123)-2(12)+(23)+(13)]  
\end{cases}
\end{equation}

\begin{note}
	$\Sigma^0$ flavor wave function
\end{note}

\begin{equation}
%%\label{eq.6.1.2}
%%%%%+++++++++++++++++++++++ wave function
\begin{cases}
\mathcal{\psi}^{[2,1]}_{11} =\mathcal{P}^{[2,1]}_{11}\phi={}
\frac{1}{6}[2uds-dsu-sud+2dus-usd-sdu]  \\
%%%%%+++++++++++++++++++++++
\mathcal{\psi}^{[2,1]}_{21} =\mathcal{P}^{[2,1]}_{21}\phi={}
\frac{\sqrt{3}}{6}[dsu-sud+usd-sdu]  
\end{cases}
\end{equation}


\begin{note}
	$\Lambda$ flavor wave function
\end{note}

\begin{equation}
%%\label{eq.6.1.2}
%%%%%+++++++++++++++++++++++ wave function
\begin{cases}
%%%%%+++++++++++++++++++++++
\mathcal{\psi}^{[2,1]}_{12} =\mathcal{P}^{[2,1]}_{21}\phi={}
\frac{\sqrt{3}}{6}[-dsu+sud+usd-sdu]  \\
%%%%%+++++++++++++++++++++++
\mathcal{\psi}^{[2,1]}_{22} =\mathcal{P}^{[2,1]}_{22}\phi={}
\frac{1}{6}[2uds-dsu-sud-2dus+usd+sdu]  
\end{cases}
\end{equation}




\end{example}


\section{ 置换群不可约表示的内积和外积 }

\begin{conclusion} %%comment
	\begin{enumerate}
		%%%%%+++++++++++++++++++++++---------------------
		\item 
		置换群不可约表示的直乘称为{\color{main}内积};
		直乘分解的1级数可以按特征标方法计算
		%%%%%+++++++++++++++++++++++---------------------
		\item 
		考虑到置换群的不可约表示是实表示,特征标是实数,有
		\begin{equation}
		\chi^{[\lambda]}(R) \chi^{[\mu]}(R)
		=\sum\limits_\nu a_{\lambda\mu\nu} \chi^{[\nu]}(R)
		\end{equation}
		
		\begin{equation}
		a_{\lambda\mu\nu}=\frac{1}{n!} \sum\limits_{R \in S_n}
		\chi^{[\nu]}(R) \chi^{[\lambda]}(R) \chi^{[\mu]}(R) \chi^{[\nu]}(R)
		\end{equation}
	
		容易看出$a_{\lambda\mu\nu}$对三个指标完全对称

		\item
		一行的杨图对应{\color{main}恒等表示};
		一列的杨图对应{\color{main}反对称表示},
		每个元素在该表示的表示矩阵即为该元素的置换宇称
		\item
		可以证明,互相对偶的杨图对应的表示维数相等,
		每个类在这两个表示中的特征标只相差类中元素的置换宇称
		%%%%%%%%%%%%%%%%%%%%%%%
		\item
		任一杨图对应的表示与反对称表示的直乘等价于改杨图的对偶杨图对应的表示
		\begin{equation}
			[\lambda]\otimes[1^n] \simeq  [\tilde{\lambda}]
		\end{equation}		
		%%%%%%%%%%%%%%%%%%%%%%
		\item
		考虑到$a_{\lambda\mu\nu}$对三个指标对称,可得
		\begin{equation}\begin{aligned}   
		[n] \otimes[\lambda] = [\lambda], \quad 
		[\lambda]\otimes[\tilde{\mu}] \simeq 
		[\tilde{\lambda}] \otimes [\mu] \quad \\
		[\lambda] \otimes [\mu] \simeq 
		[\tilde{\lambda}] \otimes [\tilde{\mu}] \simeq
		[\mu] \otimes [\lambda]
		\end{aligned}\end{equation}	
		\item 
		在$[\lambda]\otimes[\mu]$的分解中,
		{\color{main}出现恒等表示的充要条件是$[\lambda]\simeq [\mu]$,
			出现反对称表示的充要条件是$[\lambda]\simeq [\tilde{\mu}]$},
		且在此条件下,
		{\color{main}恒等表示或反对称表示只出现一次}
	
\end{enumerate}
\end{conclusion}

\begin{example} $S_3$群不可约表示直乘分解的CG系数为
	
	\begin{equation}\begin{aligned}
	%%\label{eq.6.1.2}
	%%%%%+++++++++++++++++++++++---------------------
	[3]\otimes[3]\simeq [1^3]\otimes[1^3]\simeq [3],\quad [3]\otimes[1^3]\simeq[1^3]\\
	[3]\otimes[2,1]\simeq [1^3]\otimes[2,1]\simeq [2,1]\\
	[2,1]\
	%%%%%+++++++++++++++++++++++
	\end{aligned}\end{equation}
\end{example}

