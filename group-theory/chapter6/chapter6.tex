%!TEX root = ..\group-theroy.tex
%!TEX encoding = UTF-8 Unicode

%%%%set the counter of chapter
%%% https://www.jianshu.com/p/9d53cc6a64b8
\setcounter{chapter}{5}

\chapter{置换群}

置换群在物理和数学上的重要意义 :
\begin{enumerate}
\item  置换群描写全同粒子体系的置换对称性
\item  所有有限群都同构于置换群的子群
\item  杨算符能明确描写张量指标间的复杂对称性
\end{enumerate}


\section{置换群的一般性质}

\begin{newdef}[置换]
 $n$ 个客体排列次序的变换称为 {\color{seco}置换};\\
 $n$ 个客体共有 $n!$ 个不同的置换
\end{newdef}


\begin{newdef}[矩阵描写]
 设原来排在第$j$位置的客体,经过置换$R$后排到了第$r_j$ 位置,用$2×n$矩阵来描写这一置换$R$
 \begin{equation}\begin{aligned}
 \label{eq.6.1.1}
R=
\begin{pmatrix}
1&2&3&j&n\\
r_1&r_2&r_3&r_j&r_n
\end{pmatrix}
\end{aligned}\end{equation}
\end{newdef}

\begin{example}[波函数]
	\begin{equation}\begin{aligned}
	\label{eq.6.1.2}
	\psi={}&
	\begin{pmatrix}
	\phi_1&\phi_2&\phi_3
	\end{pmatrix}\quad
	R=
	\begin{pmatrix}
	1&2&3\\
	2&3&1
	\end{pmatrix}\\
	&\mathrm{then},\\
	R\psi={}&
	\begin{pmatrix}
	1&2&3\\
	2&3&1
	\end{pmatrix}
	%%%%%%%%%%%%%%%%%%
	\begin{pmatrix}
	\phi_1&\phi_2&\phi_3
	\end{pmatrix}=
	\begin{pmatrix}
	\phi_3&\phi_1&\phi_2
	\end{pmatrix} 
	\end{aligned}\end{equation}
\end{example}


\begin{note}
对一给定的置换,各列的排列次序无关紧要,
重要的是每一列上下两个数字间的对应关系
\end{note}


\begin{newdef}[置换的乘积]
两个{\color{seco}置换的乘积}定义为相继做两次置换\\
考虑$S$和$R$的乘积$SR$:
重新排列$R$或$S$的各列,使$R$的第二行和$S$的
第一行排列一样,
由$R$的第一行和$S$的第二行组成的$2×n$矩阵即为$SR$
\end{newdef}

\begin{example}[置换相乘]
	\begin{equation}\begin{aligned}
	\label{eq.6.1.3}
	S={}&
	\begin{pmatrix}
	3&4&5&2&1\\
	2&4&5&1&3
	\end{pmatrix}=
	%%%%%%%%%%%%%%%%%%%
	\begin{pmatrix}
	1&2&3&4&5\\
	3&1&2&4&5
	\end{pmatrix}\\
%%++++++++++++++++++++++++++++++++++	
	R={}&
	\begin{pmatrix}
	1&2&3&4&5\\
	3&4&5&2&1
	\end{pmatrix}=
	%%%%%%%%%%%%%%%%%%%
	\begin{pmatrix}
	5&4&1&2&3\\
	1&2&3&4&5
	\end{pmatrix}\\
%%++++++++++++++++++++++++++++++++++	
	SR={}&
	\begin{pmatrix}
	3&4&5&2&1\\
	2&4&5&1&3
	\end{pmatrix}
	%%%%%%%%	
	\begin{pmatrix}
	1&2&3&4&5\\
	3&4&5&2&1
	\end{pmatrix}\\
	{}={}&
	\begin{pmatrix}
	1&2&3&4&5\\
	2&4&5&1&3
	\end{pmatrix}
	\end{aligned}\end{equation}
\end{example}

\begin{note}
\begin{enumerate}
\item 
\label{6.1.permutation}
$SR$可以理解为把$R$置换的第二行数字作$S$置换,
	或者把$S$置换的第一行数字作$R^{-1}$置换
	
\item 置换用矩阵来描写,但置换的乘积不服从矩阵乘积规则
\end{enumerate}
\end{note}

\begin{newprop}[置换群]
$n$个客体的$n!$个置换满足群的四个条件,构成群,
称为{\color{seco}$n$个客体的置换群},
记作{\color{seco}$S_n$}
\end{newprop}

\begin{note}
\begin{enumerate}
	
	\item 把置换的上下两行交换得到的置换是逆置换
		
	\item $n$个客体中$m$个客体的所有变换构成置换群$S_m$,
	显然$S_m$是$S_n$的子群。
	
	\item 置换群的子群链:$S_n\supset S_{n-1}\supset S_{n-2}\supset \cdots \supset S_1=E$
\end{enumerate}
\end{note}

\begin{newdef}[轮换]
{\color{seco}轮换} 是一类特殊的置换:
$n-l$个客体保持不变,余下的$l$个客体顺序变换,
形成一个循环;
$l$称为{\color{seco}轮换长度}
\end{newdef}

\begin{example}[轮换]
	\begin{equation}\begin{aligned}
	\begin{pmatrix}
	a_1&a_2&a_l
	\end{pmatrix}=
	%%%%+++++++++++++++++++
	\begin{pmatrix}
	a_1&\cdots&a_{l-1}&a_{l}&b_{1}&b_{n-l}\\
	a_2&\cdots&a_{l}&a_{1}&b_{1}&b_{n-l}
	\end{pmatrix}
	\end{aligned}\end{equation}
\end{example}

\begin{note}
\begin{enumerate}
\item 用行矩阵描写轮换时,数字的排列次序不能改变,但可以顺序变换。
\begin{equation}\begin{aligned}
\begin{pmatrix}
a&b&c&\cdots&p&q
\end{pmatrix}=
%%%%+++++++++++++++++++
\begin{pmatrix}
b&c&\cdots&p&q&a
\end{pmatrix}=
%%%%+++++++++++++++++++
\begin{pmatrix}
c&\cdots&p&q&a&b
\end{pmatrix}
\end{aligned}\end{equation}
%%%%-------------------------
\item 长度为$1$的轮换时恒等变换,长度为$2$的轮换称为{\color{seco}对换},对换满足
\begin{equation}\begin{aligned}
\begin{pmatrix}
a&b
\end{pmatrix}={}&
%%%%+++++++++++++++++++
\begin{pmatrix}
b&a
\end{pmatrix}\\
%%%%+++++++++++++++++++
\begin{pmatrix}
a&b
\end{pmatrix}
%%%%+++++++++++++++++++
\begin{pmatrix}
a&b
\end{pmatrix}={}&
E
\end{aligned}\end{equation}
%%%%-------------------------
\item 
长度为$l$的轮换,它的$l$次自乘等于恒元,即它的阶数为$l$
\begin{equation}
\begin{aligned}
	R={}&
\begin{pmatrix}
a_1&a_2&\cdots&a_l
\end{pmatrix}\\
%%%%+++++++++++++++++++
R^l={}&
E
\end{aligned}
\end{equation}
%%%%-------------------------
\item 
两个没有公共客体的轮换,乘积次序可以交换
\begin{equation}
\begin{aligned}
%%%%%+++++++++++++++++
&\begin{pmatrix}
a_1&a_2&\cdots&a_l
\end{pmatrix}
%%%%%+++++++++++++++++
\begin{pmatrix}
b_1&b_2&\cdots&b_m
\end{pmatrix}={}\\
%%%%++++++++++++++++++
%%%%%+++++++++++++++++
&\begin{pmatrix}
b_1&b_2&\cdots&b_m
\end{pmatrix}
%%%%++++++++++++++++++
\begin{pmatrix}
a_1&a_2&\cdots&a_l
\end{pmatrix}
%%%%%+++++++++++++++++
\end{aligned}
\end{equation}
%%%%-------------------------
\item 
轮换的逆
\begin{equation}
\begin{aligned}
%%%%%+++++++++++++++++
&\begin{pmatrix}
a_1&a_2&\cdots&a_{l-1}&a_l
\end{pmatrix}^{-1}={}\\
%%%%%+++++++++++++++++
%%%%%+++++++++++++++++
&\begin{pmatrix}
a_l&a_{l-1}&\cdots&a_{3}&a_2&a_1
\end{pmatrix}
%%%%%+++++++++++++++++
\end{aligned}
\end{equation}
\end{enumerate}
\end{note}


\begin{newprop}[置换分解]
{
	\color{seco}任何一个置换,
	都可以唯一地分解为没有公共客体的轮换乘积
}
\end{newprop}


\begin{example}[置换分解]
	%%some comment
\begin{equation}\begin{aligned}
%%\label{eq.6.1.2}
%%%%%+++++++++++++++++++++++---------------------
\begin{pmatrix}
1&2&3&4&5\\
3&4&5&2&1
\end{pmatrix}={}&
%%%%%+++++++++++++++++
%%%%%+++++++++++++++++
\begin{pmatrix}
1&3&5
\end{pmatrix}
%%%%%+++++++++++++++++
%%%%%+++++++++++++++++
\begin{pmatrix}
2&4
\end{pmatrix}=
%%%%%+++++++++++++++++
%%%%%+++++++++++++++++
\begin{pmatrix}
2&4
\end{pmatrix}
%%%%%+++++++++++++++++
%%%%%+++++++++++++++++
\begin{pmatrix}
1&3&5
\end{pmatrix}\\
%%%%%+++++++++++++++++
%%%%%+++++++++++++++++++++++---------------------
\begin{pmatrix}
1&2&3&4&5\\
3&1&2&4&5
\end{pmatrix}={}&
%%%%%+++++++++++++++++
%%%%%+++++++++++++++++
\begin{pmatrix}
1&3&2
\end{pmatrix}
%%%%%+++++++++++++++++
%%%%%+++++++++++++++++
\begin{pmatrix}
4
\end{pmatrix}
%%%%%+++++++++++++++++
%%%%%+++++++++++++++++
\begin{pmatrix}
5
\end{pmatrix}=
%%%%%+++++++++++++++++
%%%%%+++++++++++++++++
\begin{pmatrix}
1&3&2
\end{pmatrix}
%%%%%+++++++++++++++++
\end{aligned}\end{equation}
\end{example}

\begin{note}
	\begin{enumerate}
		%%%%%+++++++++++++++++++++++---------------------
		\item aaa
		把一置换分解为没有公共客体的轮换乘积时,各轮换长度的集合,称为该轮换的{\color{seco}轮换结构}
		\begin{equation}\begin{aligned}
		%%\label{eq.6.1.2}
		%%%%%+++++++++++++++++++++++---------------------
		R={}&
		\begin{pmatrix}
		1&3&5
		\end{pmatrix}
		%%%%%+++++++++++++++++++++++
		%%%%%+++++++++++++++++++++++---------------------
		\begin{pmatrix}
		2&4
		\end{pmatrix}
		%%%%%+++++++++++++++++++++++
		%%%%%+++++++++++++++++++++++---------------------
		=\begin{pmatrix}
		2&4
		\end{pmatrix}
		%%%%%+++++++++++++++++++++++
		%%%%%+++++++++++++++++++++++---------------------
		\begin{pmatrix}
		1&3&5
		\end{pmatrix}\quad \mathrm{structure~is}(3,2)\\
		%%%%%+++++++++++++++++++++++---------------------
		S={}&
		\begin{pmatrix}
		1&3&2
		\end{pmatrix}
		%%%%%+++++++++++++++++++++++
		%%%%%+++++++++++++++++++++++---------------------
		\begin{pmatrix}
		4&5
		\end{pmatrix}
		%%%%%+++++++++++++++++++++++
		%%%%%+++++++++++++++++++++++---------------------
		\begin{pmatrix}
		1&3&2
		\end{pmatrix}\quad \mathrm{structure~is}(3,1,1)=(3,1^2)\\
		%%%%%+++++++++++++++++++++++
		\end{aligned}\end{equation}
		%%%%%+++++++++++++++++++++++---------------------
		\item 
		把一个正整数$n$分解为若干个正整数$l_i$之和,
		这样的正整数的集合称为$n$的一组{\color{seco}配分数}
		\begin{equation}\begin{aligned}
		%%\label{eq.6.1.2}
		%%%%%+++++++++++++++++++++++---------------------
		n=4,~ \mathrm{possible~allocations:}(4),(3,1),(2,2),(2,1^2),(1^4)
		%%%%%+++++++++++++++++++++++
		\end{aligned}\end{equation}
		%%%%%+++++++++++++++++++++++---------------------
		\item
		$n$个客体的任一置换1的轮换结构为
		\begin{equation}\begin{aligned}
		%%\label{eq.6.1.2}
		%%%%%+++++++++++++++++++++++---------------------
		\begin{pmatrix}
		l_1&l_2&\cdots
		\end{pmatrix},\qquad
		\sum_i l_i =n
		%%%%%+++++++++++++++++++++++
		\end{aligned}\end{equation}		
	\end{enumerate}
\end{note}

\begin{newprop}[胶水公式]
	\begin{equation}\begin{aligned}
	%%\label{eq.6.1.2}
	%%%%%+++++++++++++++++++++++---------------------
	\begin{pmatrix}
	a&b&\cdots&c&{\color{seco}d}
	\end{pmatrix}
	%%%%%+++++++++++++++++++++++
	%%%%%+++++++++++++++++++++++---------------------
	\begin{pmatrix}
	{\color{seco}d}&e&\cdots&f
	\end{pmatrix}
	%%%%%+++++++++++++++++++++++
	%%%%%+++++++++++++++++++++++---------------------
	=\begin{pmatrix}
	a&b&\cdots&c&{\color{seco}d}&e&\cdots&f
	\end{pmatrix}
	%%%%%+++++++++++++++++++++++
\end{aligned}\end{equation}
\end{newprop}

\begin{note}
	\begin{enumerate}
		%%%%%+++++++++++++++++++++++---------------------
		\item 
		即,有一个公共客体的轮换乘积:在每个轮换内部,吧公共客体顺序移到最右或最左,然后按上式把两个轮换接起来。
		
		%%%%%+++++++++++++++++++++++---------------------
		\item 同理也可以把一个轮换分成两个轮换
	\end{enumerate}
\end{note}

\begin{example}[轮换的合并与截断]
	\begin{enumerate}
		%%%%%+++++++++++++++++++++++---------------------
		\item 
		%%some comment
		\begin{equation}\begin{aligned}
		%%\label{eq.6.1.2}
		%%%%%+++++++++++++++++++++++---------------------
		\begin{pmatrix}
		1&2&3
		\end{pmatrix}
		%%%%%+++++++++++++++++++++++
		%%%%%+++++++++++++++++++++++---------------------
		\begin{pmatrix}
		4&2&5&6
		\end{pmatrix}
		%%%%%+++++++++++++++++++++++
		%%%%%+++++++++++++++++++++++---------------------
		=\begin{pmatrix}
		3&1&2
		\end{pmatrix}
		%%%%%+++++++++++++++++++++++
		%%%%%+++++++++++++++++++++++---------------------
		\begin{pmatrix}
		2&5&6&4
		\end{pmatrix}
		%%%%%+++++++++++++++++++++++
		%%%%%+++++++++++++++++++++++---------------------
		=\begin{pmatrix}
		3&1&2&5&6&4
		\end{pmatrix}
		%%%%%+++++++++++++++++++++++
		\end{aligned}\end{equation}
		
	%%%%%+++++++++++++++++++++++---------------------
	\item 
	%%some comment
	\begin{equation}\begin{aligned}
	%%\label{eq.6.1.2}
	%%%%%+++++++++++++++++++++++---------------------
	\begin{pmatrix}
	1&2&3&4&5
	\end{pmatrix}
	%%%%%+++++++++++++++++++++++
	%%%%%+++++++++++++++++++++++---------------------
	={}&\begin{pmatrix}
	1&2
	\end{pmatrix}
	%%%%%+++++++++++++++++++++++
	%%%%%+++++++++++++++++++++++---------------------
	\begin{pmatrix}
	2&3&4&5
	\end{pmatrix}\\
	%%%%%+++++++++++++++++++++++
	%%%%%+++++++++++++++++++++++---------------------
	={}&\begin{pmatrix}
	1&2&3
	\end{pmatrix}
	%%%%%+++++++++++++++++++++++
	%%%%%+++++++++++++++++++++++---------------------
	\begin{pmatrix}
	3&4&5
	\end{pmatrix}
	%%%%%+++++++++++++++++++++++
	%%%%%+++++++++++++++++++++++---------------------
	=\begin{pmatrix}
	1&2&3&4
	\end{pmatrix}
	%%%%%+++++++++++++++++++++++
	%%%%%+++++++++++++++++++++++---------------------
	\begin{pmatrix}
	4&5
	\end{pmatrix}
	%%%%%+++++++++++++++++++++++
	\end{aligned}\end{equation}	
	
	
		%%%%%+++++++++++++++++++++++---------------------
	\item 
	%%多重复轮换截断
	\begin{equation}\begin{aligned}
	%%\label{eq.6.1.2}
	%%%%%+++++++++++++++++++++++---------------------
	&\phantom{{}={}}\begin{pmatrix}
	5&1&2&4
	\end{pmatrix}
	%%%%%+++++++++++++++++++++++
	%%%%%+++++++++++++++++++++++---------------------
	\begin{pmatrix}
	4&3&2&6
	\end{pmatrix}\\
	%%%%%+++++++++++++++++++++++
	%%%%%+++++++++++++++++++++++---------------------
	&{}={}\begin{pmatrix}
	5&1&2
	\end{pmatrix}
	%%%%%+++++++++++++++++++++++
	%%%%%+++++++++++++++++++++++---------------------
	\begin{pmatrix}
	2&4
	\end{pmatrix}
	%%%%%+++++++++++++++++++++++
	%%%%%+++++++++++++++++++++++---------------------
	\begin{pmatrix}
	4&3&2
	\end{pmatrix}
	%%%%%+++++++++++++++++++++++
	%%%%%+++++++++++++++++++++++---------------------
	\begin{pmatrix}
	2&6
	\end{pmatrix}\\
	%%%%%+++++++++++++++++++++++
	%%%%%+++++++++++++++++++++++---------------------
	&{}={}\begin{pmatrix}
	5&1&2
	\end{pmatrix}
	%%%%%+++++++++++++++++++++++
	%%%%%+++++++++++++++++++++++---------------------
	\begin{pmatrix}
	2&4
	\end{pmatrix}
	%%%%%+++++++++++++++++++++++
	%%%%%+++++++++++++++++++++++---------------------
	\begin{pmatrix}
	2&4
	\end{pmatrix}
	%%%%%+++++++++++++++++++++++
	%%%%%+++++++++++++++++++++++---------------------
	\begin{pmatrix}
	4&3
	\end{pmatrix}
	%%%%%+++++++++++++++++++++++
	%%%%%+++++++++++++++++++++++---------------------
	\begin{pmatrix}
	2&6
	\end{pmatrix}\\
	%%%%%+++++++++++++++++++++++
	%%%%%+++++++++++++++++++++++---------------------
	&{}={}\begin{pmatrix}
	5&1&2
	\end{pmatrix}
	%%%%%+++++++++++++++++++++++
	%%%%%+++++++++++++++++++++++---------------------
	\begin{pmatrix}
	2&6
	\end{pmatrix}
	%%%%%+++++++++++++++++++++++
	%%%%%+++++++++++++++++++++++---------------------
	\begin{pmatrix}
	4&3
	\end{pmatrix}\\
	%%%%%+++++++++++++++++++++++
	%%%%%+++++++++++++++++++++++---------------------
	&{}={}\begin{pmatrix}
	5&1&2&6
	\end{pmatrix}
	%%%%%+++++++++++++++++++++++
	%%%%%+++++++++++++++++++++++---------------------
	\begin{pmatrix}
	4&3
	\end{pmatrix}
	%%%%%+++++++++++++++++++++++
	\end{aligned}\end{equation}	
	
{\color{seco}把轮换拆成相邻两个轮换只含一个重复公共客体的形式后再相乘}	
	\end{enumerate}	
\end{example}

\begin{newprop}[置换群的类]
	\begin{enumerate}
	\item $R$的共轭元素:$SRS^{-1}$
		
	\item 把$R$置换的上下两行数字同时作$S$置换即得$R$置换的共轭元素$SRS^{-1}$
		
	\item 互相共轭的两个置换有相同的轮换结构
	\end{enumerate}
\end{newprop}


\begin{newproof}
reference:(\ref{6.1.permutation}) ,
当$R$是轮换时,
	\begin{equation}\begin{aligned}
	%%\label{eq.6.1.2}
	S
	\begin{pmatrix}a&b&c&\cdots &d\end{pmatrix}
	S^{-1}
	=\begin{pmatrix}S_a&S_b&S_c&\cdots &S_d\end{pmatrix}
	\end{aligned}\end{equation}
	\begin{enumerate}
			
	\item	共轭轮换不改变轮换的长度,只改变轮换设计的客体编号
	
	\item 互相共轭的两置换具有相同的轮换结构
	
	\item 亦可证明,有相同轮换结构的两置换必定互相共轭
	
	\end{enumerate}
\end{newproof}

\begin{note}
	\begin{enumerate}
		%%%%%+++++++++++++++++++++++---------------------
		\item {\color{seco}置换群的类由置换的轮换结构来描写}
		%%%%%+++++++++++++++++++++++---------------------
		\item {\color{seco}置换群的类数等于整数$n$分解为不同配分数的数目}
	\end{enumerate}
\end{note}

\begin{newthem}[类的元素数目]
	如果群$S_n$的类包含$\nu_1$个$1$循环,$\nu_2$个$2$循环,$\cdots$,$\nu_n$个$n$循环,即它的轮换结构为
	\begin{equation}\begin{aligned}
	%%\label{eq.6.1.2}
	(l)=(1^{\nu_1},2^{\nu_2},\cdots,n^{\nu_n}),
	~1\nu_1+2\nu_2+\cdots+n\nu_n=n
	\end{aligned}\end{equation}
	则该类所包含的元素个数为
	\begin{equation}\begin{aligned}
	%%\label{eq.6.1.2}
	C_l=\frac{n!}
	{
	1^{\nu_1}2^{\nu_2}\cdots n^{\nu_n}
	\nu_1! \nu_2! \cdots \nu_n!
	}
	\end{aligned}\end{equation}
\end{newthem}

\begin{newproof}
	略
%	%comment
%	\begin{equation}\begin{aligned}
%	%%\label{eq.6.1.2}
%	
%	\end{aligned}\end{equation}
\end{newproof}

\begin{newlemma}[置换群元的奇偶性]
	%%some comment
	\begin{enumerate}
		%%%%%+++++++++++++++++++++++---------------------
		\item 
		任何置换都可分解为若干个对换的乘积,分解方式不唯一,但它包含对换个数的奇偶性是确定的\\
		长度为{\color{seco}奇数}的轮换可分解为{\color{seco}偶数}个对换的乘积--{\color{seco}偶置换}\\
		长度为{\color{seco}偶数}的轮换可分解为{\color{seco}奇数}个对换的乘积--{\color{seco}奇置换}
		%%%%%+++++++++++++++++++++++---------------------
		\item 
		两个{\color{seco}偶置换}或两个{\color{seco}奇置换}的乘积是{\color{seco}偶置换}\\
		一个{\color{seco}偶置换}和一个{\color{seco}奇置换}的乘积是{\color{seco}奇置换}\\
		恒元是偶置换
		%%%%%+++++++++++++++++++++++---------------------
		\item 
		$n>1$时候,除了恒等表示,$S_n$至少还有一个一维非恒等表示,称为反对称表示,\\
		置换$R$在在该表示中的值称为它的{\color{seco}置换宇称},记作$\delta(R)$
		\begin{equation}\begin{aligned}
		%%\label{eq.6.1.2}
			\delta(R)=
			%%%%%+++++++++++++++++++++++
			\begin{cases}
			{1},\quad &\text{$R$是偶置换 } \\
			%%%%%+++++++++++++++++++++++
			{-1},\quad &\text{$R$是奇置换}
			\end{cases}
		\end{aligned}\end{equation}
	\end{enumerate}
\end{newlemma}


\begin{newdef}[交变子群]
	\begin{enumerate}
	\item 
	置换群中所有偶置换的集合构成指数为$2$的不变子群,
	称为{\color{seco} 交变子群 }
	\item
	奇置换的集合是它的陪集
	商群是$c_2$群
	\end{enumerate}
\end{newdef}


\begin{newlemma}[置换群的生成元]
	\begin{enumerate}
		%%%%%+++++++++++++++++++++++---------------------
		\item 
		相邻客体的对换:$P_a=(a~~a+1)$
		%%%%%+++++++++++++++++++++++---------------------
		\item 
		任何置换都可以写成{\color{seco}无公共客体}轮换的乘积,任何轮换都可分解为若干对换的乘积。
		%%%%%+++++++++++++++++++++++---------------------
		\item
		任何对换都可以表示为{\color{seco}相邻客体}对换的乘积
		%%%%%+++++++++++++++++++++++---------------------
		\item
		任何置换都可以表示为{\color{seco}相邻客体对换}的乘积
		%%%%%+++++++++++++++++++++++---------------------
		\item
		引入长度为$n$的轮换$W=(1~~2~~\cdots~~n)$\\
		则:$P_a=W P_{a-1} W^{-1}=W^2 P_{a-2} W^{-2}=\cdots=W^{a-1} P_{1} W^{-(a-1)}$\\
		即:{\color{seco}任何相邻客体的对换可由$W$和$P_{1}$生成}
	\end{enumerate}
\end{newlemma}

\begin{newthem}[置换群的生成元]
	置换群的生成元是$W$和$P_{1}$,置换群的秩为$2$
\end{newthem}

\clearpage

\begin{newthem}[Cayley定理]
	任何一个$n$阶有限群都与置换群$S_n$的一个子群同构
\end{newthem}

\begin{newthem}[Cayley定理]
	任何一个$n$阶有限群都与置换群$S_n$的一个子群同构
\end{newthem}



\begin{newcorol}[$n$阶有限群的数目]
	%%some comment
	\begin{enumerate}
		%%%%%+++++++++++++++++++++++---------------------
		\item 
		若置换群$S_n$的子群与$n$阶有限群$G$同构,则该子群中的元素除恒等置换外,
		任一置换所包含的无公共客体的轮换的轮换长度相等
		%%%%%+++++++++++++++++++++++---------------------
		\item 
		$S-N$的子群数目是有限的,满足上述性质的不同构的子群的数目更加有限
		%%%%%+++++++++++++++++++++++---------------------
		\item 
		不同构的$n$阶有限群的数目是有限的
	\end{enumerate}
\end{newcorol}

\clearpage

\section{杨图、杨表和杨算符}

\begin{newlemma}[aa]
	%%some comment
	\begin{enumerate}
		%%%%%+++++++++++++++++++++++---------------------
		\item 
		置换群$S_n$的类的个数等于$n$分解为不同组配分数的数目,\\
		故置换群不等价不可约表示的个数也等于$n$分解为不同组配分数的数目
		%%%%%+++++++++++++++++++++++---------------------
		\item 
		置换群$S_n$的类由$n$的配分数$(\lambda)=(\lambda_1,\lambda_2,\cdots \lambda_m)$描写,
		不等价不可约表示也可以用配分数来描写,\\
		记作$[\lambda]=[\lambda_1,\lambda_2,\cdots \lambda_m]$,其中
		\begin{equation}
			\lambda_1 \geq \lambda_2 \geq \cdots \geq \lambda_m \geq 0,~~\sum\limits_{j=1}^m\lambda_j=n
		\end{equation}
		不过,由相同配分数描写的类和不等价不可约表示并无任何关系。
	\end{enumerate}
\end{newlemma}

\begin{newdef}[杨图]
	对配分数$[\lambda]=[\lambda_1,\lambda_2,\cdots \lambda_3,]$,
	画$m$行放个图,{\color{seco}左边对齐},\\
	第一行含$\lambda_1$个,第二行含$\lambda_2$格,以此类推,\\
	这样的方格图称为配分数$[\lambda]$对应的{\color{seco}杨图},简称杨图$[\lambda]$
\end{newdef}


\begin{note}
	%%some comment
	\begin{enumerate}
		%%%%%+++++++++++++++++++++++---------------------
		\item 
		杨图中,上面行的格子数不少于下面行的格子数,\\
		左边列的格子数不少于右边列的格子数
		为强调这一规则,称它为{\color{seco}正则杨图}。
		%%%%%+++++++++++++++++++++++---------------------
		\item 
		{\color{seco}每个杨图都唯一地对应于置换群$S_n$的一个不可约表示,
			不同杨图对应的不可约表示不等价。
		}
	
		\item 
		{\color{seco}杨图的大小}:
		从第一行开始逐行比较,格子多的杨图大
	\end{enumerate}
\end{note}



\begin{example}[$S_4$ 群的杨图从大到小排列为]
	%%some comment
	\begin{equation}\begin{aligned}
	%%\label{eq.6.1.2}
	%%%%%+++++++++++++++++++++++---------------------
	\ydiagram{4}&\quad&\ydiagram{3,1}&\quad&\ydiagram{2,2}&\quad&\ydiagram{2,1,1}&\quad&\ydiagram{1,1,1,1}\\
	[4]&\quad&[3,1]&\quad&[2,2]&\quad&[2,1^2]&\quad&[1^4]
	%%%%%+++++++++++++++++++++++
	\end{aligned}\end{equation}
\end{example}


\begin{note}
	%%some comment
	\begin{enumerate}
		%%%%%+++++++++++++++++++++++---------------------
		\item 
		把杨图$[\lambda]$的行和列互换得到的杨图$[\tilde{\lambda}]$
		称为杨图$[\lambda]$的{\color{seco}对偶杨图},对应的不可约表示称为对偶表示\\
		例:$S_3$群的杨图$[3]$和$[1^3]$互为对偶杨图\\
		\phantom{例:}$S_4$群的杨图$[4]$和$[1^4]$以及$[3,1]$和$[2,1^2]$分别为对偶杨图
		%%%%%+++++++++++++++++++++++---------------------
		\item 
		若杨图$[\lambda]=[\tilde{\lambda}]$则称为自偶杨图\\
		例:$S_3$群的杨图$[2,1]$为自偶杨图\\
		\phantom{例:}$S_4$群的杨图$[2,2]$为自偶杨图
	\end{enumerate}
\end{note}


\begin{newdef}[杨表]
%%some comment
\begin{enumerate}
	%%%%%+++++++++++++++++++++++---------------------
	\item 
	对于给定的杨表1,把1到1的1个自然数分别填入杨图的1个格子中,就得到一个1杨表
	%%%%%+++++++++++++++++++++++---------------------
	\item 1格的杨图有1个不同的杨表
\end{enumerate}
\end{newdef}










\section{置换群的不可约标准表示}


\section{置换群的不可约正交表示}


\section{置换群不可约表示的内积和外积}














