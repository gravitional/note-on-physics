\documentclass[./main.tex]{subfiles}
\setcounter{chapter}{5}

\chapter{非定域手征微扰理论}

定域手征拉氏量的重整化不同于标准模型的重整化,因为幂数求和规则(power counting)受到破坏,
因此采用红外减除或heavy--baryon重整化方案。
但是光锥坐标系中无法采用这些方案。

非定域模型的基本思想,是把原来在一个点上的相互作用的场算符放在不同的时空点上,
并考虑它们原有的对称性。

因为我们的研究对象是强子的电磁结构,强子满足严格U(1)对称性,
所以把1式中的包含强相互作用和电磁相互作用的拉氏量写成如下形式,

\begin{equation}\begin{aligned}
        \label{eq.2.34}
        %%%%%+++++++++++++++++++++++---------------------
        \mathcal{L}^{(local)}(x)
         & {}=\bar{B}(i \gamma^\mu D_{\mu,x} - M_B) B(x)
        +\frac{C_{B\phi}}{f}
        \left[
            \bar{p}(x) \gamma^{\mu} \gamma^5 B(x) D_{\mu,x} \phi(x) + h.c.
            \right]
            \\ & {}+
            \bar{T}_\mu (x) (
        i \gamma^{\mu\nu\alpha} D_{\alpha,x} - M_T \gamma^{\mu\nu}
        )T_\nu(x)
        \\ & {}+
        \frac{C_{T\phi}}{f} \left[
            \bar{p}(x) \Theta^{\mu\nu} T_\nu(x) D_{\mu,x} \phi(x) + h.c.
            \right]
            \\ & {}+ 
            \frac{i C_{\phi \phi^\dagger}}{2f^2} \bar{p}(x) \gamma^\mu p(x)
        \left[
            \phi(x) (D_{\mu,x \phi})^\dagger(x) - D_{\mu,x} \phi(x) \phi^\dagger(x)
            \right]
            \\ & {}+
            D_{\mu,x} \phi(x) (D_{\mu,x} \phi)^\dagger + \ldots
        %%%%%+++++++++++++++++++++++
    \end{aligned}\end{equation}

其中协变导数定义为,

\begin{equation}\begin{aligned}
        \label{eq.2.35}
        %%%%%+++++++++++++++++++++++---------------------
        D_{\mu,x} B(x)
         &{} = \left[
            \partial_\mu - i e^q_B A_\mu(x)
            \right] B(x), \\
        D_{\mu,x} T^\nu(x)
         &{} = \left[
            \partial_\mu - i e^q_T A_\mu(x)
            \right] T^\nu(x), \\
        D_{\mu,x} \phi(x)
         &{} = \left[
            \partial_\mu - i \eofphi A_\mu(x)
            \right] \phi(x),
    \end{aligned}\end{equation}

in which, $e^q_B$,$e^q_T$和$\eofphi$分别表示八重态重子$B$,
十重态重子$T$和赝标量介子$\phi$的夸克电荷,
比如对质子来说$e^u_p=2 e^d_p=2$,$e^s_p=0$,
$\Sigma^+$超子的夸克电荷$e^u_{\Sigma^+}=2 e^s_{\Sigma^+}=2$,
$e^d_{\Sigma^+}=0$,
\eqref{eq.2.34}中$C_{B\phi}$,$C_{\phi \phi^{\dagger}}$和
$C_{T \phi}$分别代表八重态重子——介子,十重态重子——介子
和双介子——核子相互作用耦合常数列于表中。
从\eqref{eq.2.34}式中我们很容易算出前一节所给出的轻夸克电磁流以及相应的相互作用

<待补充>

\href{http://www.tablesgenerator.com/}{table generator}

由于赝标量介子质量小于八重态和十重态重子质量,我们首先考虑赝标量介子的非定域性。
定域相互作用中赝标量介子--重子相互作用与同一个时空点$x$上。
在非定域情况下,我们将赝标量介子移到$x+a$点上。
考虑对称性后非定域拉氏量可以写成,

\begin{equation}\begin{aligned}
        \label{eq.2.36}
        %%%%%+++++++++++++++++++++++---------------------
        \mathcal{L}^{nonlocal}(x)
         & {}=\bar{B}(x) (i \gamma^\mu D_{\mu,x}-M_B)B(x)
        +\bar{T}_\mu(x)(
        i \gamma^{\mu\nu\alpha} D_{\alpha,x}-M_T \gamma^{\mu\nu}
        )T_\nu(x)                    \\
         & {}+\bar{p}(x)\left[
            \frac{C_{B}\phi}{f} \gamma^\mu \gamma^5 B(x)
            +\frac{C_{T\phi}}{f}\Theta^{\mu\nu} T_\nu(x)
            \right]                  \\
         & {}\times \int d^4 a\int d^4 b \mathcal{G}_\phi^q(x+b,x+a)F(a)F(b) 
         \\ & {}\left[
        \phi(x+a) D_{\mu,x+b\,\phi}^\dagger(x+b)
        -D_{\mu,x+a}\,\phi(x+a)\phi^\dagger(x+b)
        \right]       \\
         & {}+ D_{\mu,x}\phi(x) (D_{\mu,x}\phi)^\dagger(x)+\cdots
        %%%%%+++++++++++++++++++++++---------------------
    \end{aligned}\end{equation}

in which, $\mathcal{G}^q_\phi $ 叫做规范链接函数。
为了保持非定域拉氏量的规范不变性,规范链接函数可以写成,

\begin{equation}\begin{aligned}
        \label{eq.2.37}
        %%%%%+++++++++++++++++++++++---------------------
        \mathcal{G}^q_\phi=\exp \left[
            -i \eofphi \int_x^y d z^\mu A_\mu(z)
            \right]
        %%%%%+++++++++++++++++++++++---------------------
    \end{aligned}\end{equation}

结合规范链接的性质,可验证\eqref{eq.2.36}式满足规范不变性。

除此之外,$F(a)$是时空坐标中的正规化因子,
其具体形式比较随意,但是要满足一些限制条件。
在定域极限下,1(动量空间中的定域极限是截断参数1),
非定域拉氏量回到定域拉氏量。
我们很容易证明定域和非定域拉氏量1和1在以下规范变换下不变

\begin{equation}\begin{aligned}
        %%\label{eq.2.36}
        %%%%%+++++++++++++++++++++++---------------------
        B(x)     & \to B^\prime(x) = B(x) \exp\left[i e^q_B \theta(x) \right] 
        \\ T_\mu(x) & \to 
        T_\mu^\prime(x) =
        T_\mu(x) \exp\left[i e^q_T \theta(x) \right]  
        \\ \phi(x)  & \to 
        \phi^\prime(x) =
        \phi(x) \exp\left[i \eofphi \theta(x) \right]
        %%%%%+++++++++++++++++++++++---------------------
    \end{aligned}\end{equation}

规范不变性要求规范矢量场满足

\begin{equation}\begin{aligned}
        %%\label{eq.2.36}
        %%%%%+++++++++++++++++++++++---------------------
        A^\mu(x)\to A^{\prime \mu}(x) = A^\mu(x) +\partial^\mu \theta(x)
        %%%%%+++++++++++++++++++++++---------------------
    \end{aligned}\end{equation}

in which, $\theta$是任意标量函数。
对规范链接函数做泰勒展开后得到,

\begin{equation}\begin{aligned}
        %%\label{eq.2.36}
        %%%%%+++++++++++++++++++++++---------------------
        \mathcal{G}^q_\phi(x+b,x+a)
         & {}= \exp \left[
            -i \eofphi(a-b)^\mu \int_0^1 dt A_\mu (x+at+b(1-t))
            \right]
        \\  & {}=
        1+ \delta \mathcal{G}^q_\phi + \ord (\eofphi)
        %%%%%+++++++++++++++++++++++---------------------
    \end{aligned}\end{equation}

in which, $\ord (\eofphi)$是高阶相互作用。

\begin{equation}\begin{aligned}
        %%\label{eq.2.36}
        %%%%%+++++++++++++++++++++++---------------------
        \delta \mathcal{G}^q_\phi = -i \eofphi (a-b)^\mu \int_0^1 dt\,
        A_\mu(x+at+b(1-t))
        %%%%%+++++++++++++++++++++++---------------------
    \end{aligned}\end{equation}

以上公式中做了替换$z^\mu\to x^\mu + a^\mu t + b^\mu(1-t)$

$z^\mu\to x^\mu + t (a^\mu-b^\mu) + b^\mu $。
最后将非定域拉氏量$\mathcal{L}^{nonlocal}$写成以下部分之和:
纯强相互作用拉氏量$\mathcal{L}^{nonlocal}_{hadron}$,
电磁相互作用$\mathcal{L}^{nonlocal}_{em}$和
规范链接项$\mathcal{L}^{nonlocal}_{link}$.

《待补充》

最后$\delta \mathcal{G}_\phi^q$项产生额外的电磁相互作用,其表达式为

\begin{equation}\begin{aligned}
        %%\label{eq.2.36}
        %%%%%+++++++++++++++++++++++---------------------
        \mathcal{L}^{nonlocal}_{link}
         & {}= -i \eofphi \bar{p}(x) \left[
            \frac{C_{B\phi}}{f} \gamma^\rho \gamma^5 B(x)
            \frac{C_{T\phi}}{f} \Theta^\rho\nu T_\nu(x)
            \right] \times
        \\  &{} \phantom{\times} 
        \int_0^1 dt \int d^4a F(a) a^\mu \partial_\rho \phi(x+a)
        A_\mu(x+at)+h.c.
        \\  &{}+ 
        \frac{\eofphi \coupleofphi}{2f^2}
        \bar{p}(x) \gamma^\rho p(x) \int_0^1 dt \int d^4a
        \int d^4b F(a) F(b) (a-b)^\mu \times
        \\  &{} \phantom{\times} 
        \left[
            \phi(x+a) \partial_\rho \phi^\dagger(x+b)
            -\partial_\rho \phi(x+a) \phi^\dagger(x+b)
            \right]
        \times A_\mu(x+at+b(1-t))
        %%%%%+++++++++++++++++++++++---------------------
    \end{aligned}\end{equation}

显然这种类型的相互作用正比于粒子之间的相对距离$a$和$b$,
如果介子场是定域场,额外电磁相互作用项消失。
从非定域拉氏量可以得到非定域夸克流$J^{\mu,q}_\mathrm{em}$

\begin{equation}\begin{aligned}
        %%\label{eq.2.36}
        %%%%%+++++++++++++++++++++++---------------------
        J^{\mu}_\mathrm{q,em}
         & {}\equiv\frac{
        \delta \int d^4y\, \mathcal{L}^{(nonlocal)}_{\mathrm{em}} (y)
        }{\delta A_\mu (x)}
        \\  &{} =
        e^q_B \bar{B}(x) \gamma^\mu B(x) +
        e^q_T \bar{T}_\alpha(x) \gamma^{\alpha\nu\mu} T_\nu(x)
        +i \eofphi \left[
            \partial^\mu \phi(x)\phi^\dagger(x)-
            \phi(x)\partial^\mu \phi^\dagger(x)
            \right]
        \\  &{} -
        i \eofphi \int d^4a F(a) \bar{p}(x-a)
        \left[
            \frac{C_{B\phi}}
            {f} \gamma^\mu \gamma^5 B(x-a) +
            \frac{C_{T\phi}}
            {f} \Theta^{\mu\nu} T_\nu(x-a)
            \right]\phi (x) +h.c.
        \\  & - \frac{\eofphi \coupleofphi}
        {2f^2} \int d^4a F(a) \int d^4b F(b) \times
        \\  &{} \phantom{\times}
        \left[
            \bar{p}(x-a) \gamma^\mu p(x-a) \phi(x)\phi^\dagger(x+b-a)
            +\bar{p}(x-b) \gamma^\mu p(x-b) \phi(x+a-b) \phi^\dagger(x)
            \right]
        %%%%%+++++++++++++++++++++++---------------------
    \end{aligned}\end{equation}

采用同样的方法可以得到规范链接流

\begin{equation}\begin{aligned}
        %%\label{eq.2.36}
        %%%%%+++++++++++++++++++++++---------------------
        \frac{
            \delta A_\mu (y)
        }{\delta A_\mu (x)} & {}= \delta(y-x) =\delta(x-y) \\  
        \frac{
            \delta \int d^4y\, A_\mu (y) f(y)
        }{\delta A_\mu (x)} & {}= \int d^4y\, \delta(y-x) f(y) =f(x)
        %%%%%+++++++++++++++++++++++---------------------
    \end{aligned}\end{equation}

apply this rule, we get,

\begin{equation}\begin{aligned}
        %%\label{eq.2.36}
        %%%%%+++++++++++++++++++++++---------------------
        J^{\mu}_\mathrm{q,link}
         & {}\equiv \frac{
        \delta \int d^4y\, \mathcal{L}^{(nonlocal)}_{\mathrm{link}} (y)
        }{\delta A_\mu (x)}
        %%%%%%%%%%%%%%%
        \\  &{} =
        -i\eofphi \int_0^1 dt \int d^4a \, F(a) a^\mu \bar{p}(x-at) \times
        \\  &{}
        \left[
            \frac{C_{B_\phi}}{f}
            \gamma^\rho \gamma^5 B(x-at) +
            \frac{C_{T\phi}}{f}
            \Theta^{\rho\nu} T_\nu (x-at)
            \right]
        \partial_\rho \phi(x+a(1-t))+h.c.
        \\  &{} + \frac{\eofphi\,\coupleofphi}{2f^2}
        \int_0^1 dt \int d^4a F(a) \int d^4b (a-b)^\mu
        \\  &{}
        \bar{p}(x-at-b(1-t))
        \gamma^\rho p(x-at-b(1-t))\times
        \\  &{} \phantom{\times}
        \left[
            \phi(x+(a-b)(1-t)) \partial_\rho \phi^\dagger (x-(a-b)t)
            -\partial_\rho \phi(x+(a-b)(1-t))
            \phi^\dagger(x-(a-b)t)
            \right]
        %%%%%+++++++++++++++++++++++---------------------
    \end{aligned}\end{equation}
