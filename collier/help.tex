\documentclass{ctexart} 
%%++++++++++++++++++++
\usepackage{graphicx}
\usepackage{eso-pic}

\usepackage{xcolor} % Required for specifying colors by name
\definecolor{ocre}{RGB}{243,102,25}
\usepackage{enumerate}
\usepackage{latexsym}
\usepackage{makeidx}
\usepackage[colorlinks,linkcolor=blue]{hyperref}
\usepackage{amsfonts}
\usepackage{amsmath}
\usepackage{amssymb}
\usepackage{hep}
\usepackage{slashed}
\usepackage{bm}
\usepackage{graphicx,graphics,color}
\usepackage{multirow}
\usepackage{simplewick}
\usepackage{axodraw2}
\usepackage{amsfonts}
\usepackage{amsmath}
\usepackage{latexsym}
\usepackage{amssymb}
\usepackage{makeidx}
\usepackage{amsmath}
\usepackage{graphicx}
\usepackage{graphicx,graphics,color}
\usepackage{multirow}
\usepackage{axodraw2}
\usepackage{slashed}
%% \usepackage{axodraw2}
%%++++++++++++++++++++
\DeclareMathOperator{\trace}{tr}
\DeclareMathOperator{\Real}{Re}
\DeclareMathOperator{\Imag}{Im}
\newcommand*{\dif}{\mathop{}\!\mathrm{d}}
\def\chntoday{\the\year~年~\the\month~月~\the\day~日}
%%+++++++++++++++++++++++++++++


\begin{document}

\title{Collier help}
\author{ymy}
\today
\maketitle 

COLLIER is a fortran 单圈--标量和张量的数值积分程序库。
这些积分出现在微扰的相对论性量子场论中。
它具有以下features:

\begin{itemize}
\item 多粒子复杂度 scalar and tensor integrals
\item ultraviolet divergences 的维数正规化
\item soft infrared divergences 的维数正规化(对于非阿贝尔场,也支持 mass regularization)
\item 对于共线质量奇点的维数正规化或者质量正规化
\item 对于不稳定粒子,complex 内线质量完全支持(外动量和virtualities认作是实数)
\item 数值危险区域(小 Gram 或者其他运动学行列式),使用专用的展开处理。
\item 所有基本模块都有两种平行的实现方式,可以用作内部交叉检验
\item 缓存系统--用来加速计算
\end{itemize}

代码提供了量子场论中任意张量和标量积分的数值结果。
对于张量积分,不管协变分解中的系数还是张量元本身都将给出。
Collier 支持 complex 质量,在计算不稳定粒子时会需要。
采用维数正规化处理紫外和红外奇点。
对于 soft 和 共线奇点,有可选用的质量正规化方案。

\section{introduction}

multi-leg one-loop amplitudes 振幅求值的巨大进步,来自于两方面:

\begin{enumerate}
    \item 传统费曼图方法的系统改进
    \item 基于推广的幺正性关系的新理论技术
\end{enumerate}

在第二种方法中,单圈振幅被直接表示成标量积分的组合。
这种向固定标量积分基的直接约化,会引发相空间特定区域的数值问题。
一般可以通过采用四次精度的数值计算克服。

相反,费曼图方法,包括最近的递归方法依赖于张量积分。
此方法允许分解方法自适应于相空间的不同区域,在相当大的程度上,
通过最优选择避免数值不稳定性。
Collier 库提供了计算标量和张量积分的全面工具。
在两种互补的方法中都能应用。

将张量积分约化到一小族基本积分的方法,可以追溯到 Brown and Feynman, Melrose, and Passarino and Veltman。
又经过了数十年的发展,文献[43, 50]  展示的完整方法,是Collier代码的基础。
作为张量分解基础的标量积分首次由t' Hooft and Veltman 进行了系统研究。
已经存在数个计算单圈标量和张量积分的库,比如:

\begin{itemize}
    \item FF,LoopTools,
    \item QCDLoop,OneLoop,
    \item GoLem95C,PJFRY,Package-X
\end{itemize}

这里介绍的Collier库,提供完全的张量积分集合,处理带有复数质量的过程,
并且没有先验的粒子数限制。

Collier 已在用于多个前沿课题的计算。

文章结构:

\begin{itemize}
    \item Section 2: Collier相关约定
    \item Section 3:计算张量积分的方法轮廓
    \item Section 4:Collier 库的内部结构
    \item Section 5: 用法
    \item Section 6:总结
    \item Appendix A:定义 单圈积分的 运动学输入细节
\end{itemize}

\section{Convention}

一致性地使用 Refs. [50, 59] 中的约定。约化张量积分的方法在 Refs. [43, 50] 中有描述,
已经实现4--点函数的结果可以在Ref. [59]中找到。
标量1-,2-,3-点 函数的结果基于Refs. [45, 52]

$D$维空间中,单圈$N$点张量积分的一般形式为:
\begin{equation}
    %% \label{eq:1} 
    \begin{aligned}      
T^{N,\mu_{1},\cdots, \mu_{P}}(p_{1},\cdots,p_{N-1},m_{0},\cdots,m_{N-1})
=\frac{(2\pi\mu)^{4-D}}{i \pi^2}
\int\dif^D q \frac{q^{\mu_{1}}\cdots q^{\mu_{P}}}{N_0 N_1 \cdots N_{N-1}}
    \end{aligned}
\end{equation}











\end{document}

