% \iffalse meta-comment balanced on line 122
% revtex4-1.dtx: package to typeset APS, AIP, OSA, ASA, etc. journal articles
% Copyright (c) 2009 The American Physical Society.
% http://publish.aps.org/revtex4/
% mailto:revtex@aps.org
%
% Disclaimer
%   This file is distributed WITHOUT ANY WARRANTY;
%   without even the implied warranty of
%   MERCHANTABILITY or FITNESS FOR A PARTICULAR PURPOSE.
% License
%   You may distribute this file under the conditions of the 
%   LaTeX Project Public License 1.3 or later 
%   (http://www.latex-project.org/lppl.txt).
% ReadMe
%   For the documentation and more detailed instructions for
%   installation, typeset this document with \LaTeX.
% Maintenance Status
%   This work has the LPPL maintenance status "maintained";
%   Current Maintainer of this work is Arthur Ogawa.
%
% This work consists of the main source file revtex4-1.dtx
% and the derived files
%    revtex4.cls, revtex4.pdf
% Distribution:
%    CTAN:macros/latex/contrib/revtex/
%
% Unpacking:
%    tex revtex4-1.dtx
%
% Documentation:
%    latex revtex4-1.dtx; ...
%
%    Programm calls to get the documentation (example):
%       pdflatex revtex4-1.dtx
%       makeindex -s gind.ist revtex4-1.idx
%       makeindex -s gglo.ist -o revtex4-1.gls revtex4-1.glo
%       pdflatex revtex4-1.dtx
%       makeindex -s gind.ist revtex4-1.idx
%       pdflatex revtex4-1.dtx
%
% Installation:
%    TDS:bibtex/bst/revtex/
%    TDS:doc/latex/revtex/
%    TDS:tex/latex/revtex/
%    TDS:source/latex/revtex/
%
% Thanks, Heiko!
%    This method of letting a single .dtx file serve as both
%    documentation (via latex) and installer (via tex) follows
%    the example of Heiko Oberdiek. Thanks!
%<*ignore>
\begingroup
  \def\x{LaTeX2e}%
\expandafter\endgroup
\ifcase
 0\expandafter\ifx\csname processbatchFile\endcsname\relax\else1\fi\ifx\fmtname\x\else 1\fi
 \relax
\else
 \csname fi\endcsname
%</ignore>
%<*install>
%% This file will generate documentation and runtime files
%% from revtex4-1.dtx when run through TeX.
\input docstrip
\preamble

This file is part of the APS files in the REVTeX 4 distribution.
For the version number, search on the string %FileInfo

Original version by David Carlisle
Modified by Arthur Ogawa (mailto:arthur_ogawa at sbcglobal dot net)

Copyright (c) 2009 The American Physical Society.
http://publish.aps.org/revtex4/
mailto:revtex4@aps.org

See the REVTeX 4 README file for restrictions and more information.

\endpreamble
\askforoverwritefalse
\keepsilent
\generate{%
  \file{revtex4-1.cls}{%
   \from{revtex4-1.dtx}{kernel}%
   \from{ltxutil.dtx}{kernel}%
   \from{ltxfront.dtx}{kernel}%
   \from{ltxgrid.dtx}{kernel}%
   \from{revtex4-1.dtx}{options}%
   \from{ltxutil.dtx}{options}%
   \from{ltxfront.dtx}{options}%
   \from{ltxgrid.dtx}{options}%
   \from{revtex4-1.dtx}{package}%
  }%
  \file{aps4-1.rtx}{\from{revtex4-1.dtx}{aps}}%
  \file{apsrmp4-1.rtx}{\from{revtex4-1.dtx}{rmp}}%
  \file{aps10pt4-1.rtx}{\from{revtex4-1.dtx}{10pt}}%
  \file{aps11pt4-1.rtx}{\from{revtex4-1.dtx}{11pt}}%
  \file{aps12pt4-1.rtx}{\from{revtex4-1.dtx}{12pt}}%
  \file{revsymb4-1.sty}{\from{revtex4-1.dtx}{revsymb}}%
}%
\ifToplevel{
\Msg{***********************************************************}
\Msg{*}
\Msg{* To finish the installation, please move}
\Msg{*    *.cls, *.rtx, and *.sty }
\Msg{* into a directory searched by TeX;}
\Msg{* in a TDS-compliant installation:}
\Msg{* texmf/tex/macros/latex/revtex/.}
\Msg{*}
\Msg{* To produce the documentation,
       run revtex4-1.dtx through LaTeX.}
\Msg{*}
\Msg{* Happy TeXing}
\Msg{***********************************************************}
}
\endbatchfile
%</install>
%<*ignore>
\fi
%</ignore>
% \fi
%
% \GetFileInfo{revtex4-1.dtx}\CheckSum{7177}
%
% \iffalse ltxdoc klootch
%<*package>
%%%  @LaTeX-file{
%%%     filename        = "revtex4-1.dtx",
%%%     version         = "4.1r",
%%%     date            = "2010/07/25",
%%%     time            = "20:33:00 UT-8",
%%%     checksum        = "7177",
%%%     author          = "Arthur Ogawa (mailto:arthur_ogawa at sbcglobal.net),
%%%                        commissioned by the American Physical Society.
%%%                        ",
%%%     copyright       = "Copyright (C) 1999, 2009 Arthur Ogawa,
%%%                        distributed under the terms of the 
%%%                        LaTeX Project Public License, see
%%%                        ftp://ctan.tug.org/macros/latex/base/lppl.txt
%%%                        ",
%%%     address         = "Arthur Ogawa,
%%%                        USA",
%%%     telephone       = "",
%%%     FAX             = "",
%%%     email           = "mailto colon arthur_ogawa at sbcglobal.net",
%%%     codetable       = "ISO/ASCII",
%%%     keywords        = "latex, page grid, main vertical list",
%%%     supported       = "yes",
%%%     abstract        = "package to change page grid, MVL",
%%%     docstring       = "The checksum field above generated by ltxdoc",
%%%  }
%</package>
% \fi
%
% \iffalse ltxdoc klootch
% The following references the \file{00readme.tex} file,
% which contains the frontmatter for the programmer's docs.
% The contents of this file are generated when
% you typeset this file with LaTeX.
% Search on "{filecontents*}{00readme.tex}" to locate it.
% \fi\input{00readme.tex}%
%
% \subsection{Bill of Materials}
%
% Following is a list of the files in this distribution arranged
% according to provenance.
%
% \subsubsection{Primary Source}%
% One single file generates all.
%\begin{verbatim}
%revtex4-1.dtx
%\end{verbatim}
%
% \subsubsection{Generated by \texttt{tex revtex4-1.dtx}}%
% Typesetting ths file under \TeX\ itself runs the installer, 
% which generates the package files.
%\begin{verbatim}
%revtex4-1.cls, revtex4.ins, revtex4.drv, aps4-1.rtx, 
%aps10pt4-1.rtx, aps11pt4-1.rtx, aps12pt4-1.rtx, revsymp.sty
%\end{verbatim}
%
% \subsubsection{Generated by \texttt{pdflatex revtex4-1.dtx}}%
% Typesetting the source file under \LaTeX\
% generates the documentation.
%\begin{verbatim}
%revtex4.pdf, 
%\end{verbatim}
%
% \subsubsection{Auxiliary}%
% The following are auxiliary files generated
% in the course of running \LaTeX:
% \begin{verbatim}
%revtex4.aux revtex4.idx revtex4.ind revtex4.log revtex4.toc
% \end{verbatim}
%
% \section{Code common to all modules}%
%
% The following may look a bit klootchy, but we
% want to require only one place in this file
% where the version number is stated,
% and we also want to ensure that the version
% number is embedded into every generated file.
%
% Now we declare that
% these files can only be used with \LaTeXe.
% An appropriate message is displayed if
% a different \TeX{} format is used.
%    \begin{macrocode}
%<*doc|kernel|aps|rmp|revsymb>
\NeedsTeXFormat{LaTeX2e}[1996/12/01]%
%</doc|kernel|aps|rmp|revsymb>
%<kernel>\ProvidesClass{revtex4-1}
%<aps>\ProvidesFile{aps4-1}
%<rmp>\ProvidesFile{apsrmp4-1}
%<10pt>\ProvidesFile{aps10pt4-1}
%<11pt>\ProvidesFile{aps11pt4-1}
%<12pt>\ProvidesFile{aps12pt4-1}
%<revsymb>\ProvidesPackage{revsymb4-1}
%<*doc>
\ProvidesFile{revtex4-1.dtx}
%</doc>
%<*!package&!options>
%<version>
 [2010/07/25/20:33:00 4.1r (http://publish.aps.org/revtex4/ for documentation)]% \fileversion
%</!package&!options>
%    \end{macrocode}
% The current class name is remembered in \cmd\class@name.
% This is something of a klootch, relying as it does on knowledge of the implementation of \cmd\ProvidesPackage.
%    \begin{macrocode}
%<kernel>\let\class@name\@gtempa
%    \end{macrocode}
%
% \section{The driver module \texttt{driver}}
%
% This module, consisting of the present section,
% typesets the programmer's documentation,
% generating the \file{00readme.txt} and sample document as needed.
%
% Because the only uncommented-out lines of code at the beginning of
% this file constitute the \file{driver} module itself,
% we can simply typeset the \file{.dtx} file directly,
% and there is thus rarely any need to
% generate the ``driver'' {\sc docstrip} module.
% Module delimiters are nonetheless required so that
% this code does not find its way into the other modules.
%
% The \enve{document} command concludes the typesetting run.
%
%    \begin{macrocode}
%<*doc>
%    \end{macrocode}
% The driver uses packages \file{ltxdoc.sty}, \file{ltxdocext.sty}, 
% \file{hyperref.sty}, and whatever font package has been selected.
%    \begin{macrocode}
\documentclass{ltxdoc}
\RequirePackage{ltxdocext}%
\let\url\undefined
\RequirePackage[colorlinks=true,linkcolor=blue]{hyperref}%
\expandafter\ifx\csname package@font\endcsname\@undefined\else
 \expandafter\RequirePackage\expandafter{\csname package@font\endcsname}%
\fi
%    \end{macrocode}
% We ask for the usual indices and glossaries.
%    \begin{macrocode}
\CodelineIndex\EnableCrossrefs % makeindex -s gind.ist revtex4
\RecordChanges % makeindex -s gglo.ist -o revtex4.gls revtex4.glo
%    \end{macrocode}
%
% \subsubsection{Docstrip and info directives}
%    We use so many {\sc docstrip} modules that we set the
%    \texttt{StandardModuleDepth} counter to 1.
%    \begin{macrocode}
\setcounter{StandardModuleDepth}{1}
%    \end{macrocode}
%    The following command retrieves the date and version information
%    from this file.
%    \begin{macrocode}
\expandafter\GetFileInfo\expandafter{\jobname.dtx}%
%    \end{macrocode}
%
% \subsection{The Frontmatter File}
% As promised above, here is the contents of the frontmatter file. 
%    \begin{macrocode}
\begin{filecontents*}{00readme.tex}
\title{%
 The \classname{revtex4-1} document class of the American Physical Society%
 \protect\thanks{Work under hire to The American Physical Society.}%
}%
\author{Arthur Ogawa%
 \protect\thanks{First revision of REV\TeX4.0 (unreleased) by David Carlisle}%
}%
\date{Version \fileversion, dated \filedate}%
\newcommand\revtex{REV\TeX}

\maketitle

This file embodies the implementation of the APS \revtex\ 4.1 document class
for electronic submissions to journals.

The distribution point for this work is
\url{http://publish.aps.org/revtex4/},
which contains fully unpacked, prebuilt runtime files and documentation.

\tableofcontents

\section{Using \protect\revtex}

The file \file{README} has retrieval and installation information.

User documentation is presented separately in \file{auguide.tex}.

The file \file{template.aps} is a boilerplate file. 

\changes{4.0a}{1998/01/16}{Initial version}
\changes{4.0a}{1998/01/31}{Move after process options, so \cs{clearpage} not in scope of twocolumn}
\changes{4.0a}{1998/01/31}{Rearrange the ordering so numerical ones come first. AO: David, what does this mean?}
\changes{4.0a}{1998/01/31}{use font-dependent spacing}
\changes{4.0a}{1998/01/31}{4.0d had twoside option setting twoside switch to false}
\changes{4.0a}{1998/01/31}{Move after process options, so the following test works}
\changes{4.0a}{1998/01/31}{print homepage}
\changes{4.0a}{1998/01/31}{protect against hyperref revtex kludges which are not needed now}
\changes{4.0a}{1998/06/10}{multiple preprint commands}
\changes{4.0a}{1998/06/10}{comma not space between email and homepage}
\changes{4.0a}{1998/06/10}{single space footnotes}
\changes{4.0b}{1999/06/20}{First modifications by Arthur Ogawa (mailto:arthur\_ogawa at sbcglobal dot net)}
\changes{4.0b}{1999/06/20}{Added localization of \cs{figuresname}}
\changes{4.0b}{1999/06/20}{Added localization of \cs{tablesname}}
\changes{4.0b}{1999/06/20}{AO: all code for \protect\classoption{10pt} is in this module.}
\changes{4.0b}{1999/06/20}{AO: all code for \protect\classoption{11pt} is in this module.}
\changes{4.0b}{1999/06/20}{AO: all code for \protect\classoption{12pt} is in this module.}
\changes{4.0b}{1999/06/20}{AO: made aps.rtx part of revtex4.dtx}
\changes{4.0b}{1999/06/20}{AO: remove duplicates}
\changes{4.0b}{1999/06/20}{call \cs{print@floats}}
\changes{4.0b}{1999/06/20}{Defer assignment until \cs{AtBeginDocument} time.}
\changes{4.0b}{1999/06/20}{Defer decision until \cs{AtBeginDocument} time}
\changes{4.0b}{1999/06/20}{Define three separate environments, defer assignment to \cs{AtBeginDocument} time.}
\changes{4.0b}{1999/06/20}{Frank Mittelbach, has stated in \protect\classname{multicol}: ``The kernel command \cs{@footnotetext} should not be modified.'' Thus, I have removed David Carlisle's redefinition of that command. Note, however, that later versions of \protect\classname{multicol} do not require this workaround. Belt and suspenders.}%
\changes{4.0b}{1999/06/20}{Move this ``complex'' option to the front, where it can be overridden by ``simple'' options.}
\changes{4.0b}{1999/06/20}{New option}
\changes{4.0b}{1999/06/20}{One-line caption sets flush left.}
\changes{4.0b}{1999/06/20}{only execute if appropriate}
\changes{4.0b}{1999/06/20}{Processing delayed to \cs{AtBeginDocument} time}
\changes{4.0b}{1999/06/20}{Removed invocation of nonexistent class option \protect\classoption{groupauthors} and all other class options that should only be invoked by the document. (Otherwise precedence of class options does not work.)}
\changes{4.0b}{1999/06/20}{Restore all media size class option of \protect\file{classes.dtx}}
\changes{4.0b}{1999/06/20}{Stack \cs{preprint} args flush right at right margin.}
\changes{4.0c}{1999/11/13}{(AO, 115) If three or more preprints specified, set on single line, with commas.}
\changes{4.0c}{1999/11/13}{(AO, 129) section* within appendix was producing appendixname}
\changes{4.0c}{1999/11/13}{*-form mandates pagebreak}
\changes{4.0c}{1999/11/13}{also spelled ``acknowledgements''.}
\changes{4.0c}{1999/11/13}{Do not put by REVTeX in every page foot}
\changes{4.0c}{1999/11/13}{grid changes via ltxgrid procedures}
\changes{4.0c}{1999/11/13}{grid changes with ltxgrid}
\changes{4.0c}{1999/11/13}{Insert procedure \cs{checkindate}}
\changes{4.0c}{1999/11/13}{Lose compatability mode.}
\changes{4.0c}{1999/11/13}{New ltxgrid-based  code, other bug fixes}
\changes{4.0c}{1999/11/13}{New option ``checkin''}
\changes{4.0c}{1999/11/13}{Prevent an inner footnote from performing twice}
\changes{4.0d}{2000/04/10}{Also alter how lists get indented.}
\changes{4.0d}{2000/04/10}{eprint takes an optional argument, syntactical only in this case.}
\changes{4.0d}{2000/04/10}{New option}
\changes{4.0d}{2000/05/10}{More features and bug fixes: compatability with longtable and array packages. Now certainly incompatible with multicol.}
\changes{4.0d}{2000/05/17}{make longtable trigger the head, too}
\changes{4.0d}{2000/05/18}{But alternative spelling is deprecated.}
\changes{4.0e}{2000/09/20}{New option showkeys}
\changes{4.0e}{2000/11/14}{Bug fixes and minor new features: title block affiliations can have ancillary data, just like authors; clearpage processing revamped, with floats staying in order; widetext ornaments.}
\changes{4.0e}{2000/11/21}{adornments above and below.}
\changes{4.0f}{2001/02/13}{Last bug fixes before release.}
\changes{4.0rc1}{2001/06/17}{Running headers always as if two-sided}
\changes{4.0rc1}{2001/06/18}{grid changes with push and pop}
\changes{4.0rc1}{2001/06/18}{grid changes with push and pop}
\changes{4.0rc4}{2001/07/23}{hyperref is no longer loaded via class option: use a usepackage statement instead}
\changes{4.1a}{2008/01/18}{(AO, 457) Endnotes to be sorted in with numerical citations.}%
\changes{4.1a}{2008/01/18}{(AO, 451) ``Cannot have more than 256 cites in a document''}%
\changes{4.1a}{2008/01/18}{(AO, 457) Endnotes to be sorted in with numerical citations.}%
\changes{4.1a}{2008/01/18}{(AO, 460) ``Proper style is "FIG. 1. ..." (no colon)''}%
\changes{4.1a}{2008/01/18}{(AO, 478) \cs{ds@letterpaper}, so that ``letterpaper really is the default''}%
\changes{4.1a}{2008/01/18}{(AO, 488) Change processing of options to allow an unused option to specify society and journal}%
\changes{4.1a}{2008/01/19}{(AO, 461) Change the csname revtex uses from @dotsep to ltxu@dotsep. The former is understood in mu. (What we wanted was a dimension.)}%
\changes{4.1a}{2008/01/19}{For natbib versions before 8.21, \cs{NAT@sort} was consulted only as natbib was being read in. Now it is fully dynamic.}
\changes{4.1b}{2008/05/29}{The csname substyle@ext is now defined without a dot (.), to be compatible with \LaTeX usage (see @clsextension and @pkgextension).}
\changes{4.1b}{2008/06/01}{(AO) Implement bibnotes through \cs{frontmatter@footnote@produce} instead of \cs{bibnotes@sw}}%
\changes{4.1b}{2008/06/01}{Add option reprint, opposite of preprint, and preferred alternative to twocolumn}
\changes{4.1b}{2008/06/29}{(AO, 455) Be nice to a list within the abstract (assign \cs{@totalleftmargin}).}
\changes{4.1b}{2008/06/30}{(AO) Structure the Abstract using the \texttt{bibliography} environment}
\changes{4.1b}{2008/07/01}{(AO) coordinate \cs{if@twoside} with \cs{twoside@sw}}
\changes{4.1b}{2008/07/01}{(AO) make settings at class time instead of deferring them to later.}
\changes{4.1b}{2008/07/01}{(AO) No longer need to test \cs{chapter} as of \texttt{natbib} version 8.2}
\changes{4.1b}{2008/07/01}{(AO) No longer use \cs{secnumarabic@sw}, instead use \cs{setup@secnums}}
\changes{4.1b}{2008/07/01}{(AO) Provide more diagnostics when \cs{@society} is assigned.}
\changes{4.1b}{2008/07/01}{(AO) provide option longbibliography}
\changes{4.1b}{2008/07/01}{Add \cs{@hangfroms@section}}
\changes{4.1b}{2008/07/01}{Break out \cs{@caption@fignum@sep}}
\changes{4.1b}{2008/07/01}{Class option galley sets \cs{preprintsty@sw} to false}
\changes{4.1b}{2008/07/01}{Code relating to new syntax for frontmatter has been placed in \file{ltxfront.dtx}}
\changes{4.1b}{2008/07/01}{Package textcase is now simply a required package}
\changes{4.1b}{2008/07/01}{Procedures \cs{@parse@class@options@society} and \cs{@parse@class@options@journal} and friends}
\changes{4.1b}{2008/07/01}{Read in all required packages together}
\changes{4.1b}{2008/07/01}{Remove options newabstract and oldabstract}
\changes{4.1b}{2008/08/01}{Section numbering via procedures \cs{secnums@rtx} and \cs{secnums@arabic}.}
\changes{4.1b}{2008/08/04}{As with author formatting, rag the right more, and assign \cs{@totalleftmargin}. Also neutralize \cs{def@after@address}.}%
\changes{4.1b}{2008/08/04}{Rag the right even more: .8\cs{hsize}. Also, assign \cs{@totalleftmargin}.}%
\changes{4.1b}{2008/08/04}{The \texttt{rmp} journal substyle selects \texttt{groupedaddress} by default.}%
\changes{4.1b}{2008/08/04}{Use \cs{setup@hook} to initialize all.}
\changes{4.1c}{2008/08/15}{Document class option longbibliography via \cs{substyle@post}}
\changes{4.1d}{2009/03/27}{Definition of \cs{	@fnsymbol} follows fixltx2e.sty}
\changes{4.1e}{2008/06/29}{(AO, 455) be nice to a list within the abstract}
\changes{4.1f}{2009/07/07}{(AO, 513) Add class option linenumbers: number the lines a la \classname{lineno}}
\changes{4.1f}{2009/07/07}{(AO, 516) Merged references are separated with a semicolon}
\changes{4.1f}{2009/07/10}{(AO, 520) Automatically produce \cs{bibliography} command when needed}%
\changes{4.1f}{2009/07/11}{(AO, 521) Lonely bibliography head}%
\changes{4.1f}{2009/07/11}{(AO, 522) Warn if software is expired}%
\changes{4.1f}{2009/07/15}{(AO, 523) Add class option nomerge, to turn off new natbib 8.3 syntax}
\changes{4.1f}{2009/07/20}{(AO, 524) Makes no sense if citations are superscript numbers and so are footnotes}
\changes{4.1f}{2009/10/05}{(AO, 530) \cs{@fnsymbol}: Failed to import fixltx2e.sty technology. Return to LaTeX core.}
\changes{4.1g}{2009/10/07}{(AO, 525) Remove phantom paragraph above display math that is given in vertical mode}%
\changes{4.1g}{2009/10/07}{(AO, 538) \cs{MakeTextUppercase} inappropriately expands the double backslash}
\changes{4.1h}{2009/10/09}{(AO) Remove expiry code in the release software}%
\changes{4.1i}{2009/10/23}{(AO, 541) Defer assignment of \cs{cite} until after natbib loads}
\changes{4.1j}{2009/10/24}{(AO, 549) Repairing natbib's \cs{BibitemShut} and \cs{bibAnnote}}
\changes{4.1j}{2009/10/25}{(AO, 545) hypertext capabilities off by default; enable with \classoption{hypertext}}
\changes{4.1j}{2009/10/25}{(AO, 552) Repair spacing in \cs{onlinecite}}
\changes{4.1k}{2009/11/06}{(AO, 554) give the \cs{newlabel} command syntax appropriate to the hyperref package}
\changes{4.1n}{2009/11/06}{(AO, 565) restore 4.0 behavior: invoking class option preprint implies class option preprintnumbers}
\changes{4.1n}{2009/11/30}{(AO, 566) restore 4.0 behavior: flush column bottoms}
\changes{4.1n}{2009/12/05}{(AO, 569) Use of \classname{hyperref} interferes with column balancing of last page}%
\changes{4.1n}{2009/12/09}{(AO, 569) execute the after-last-shipout procedures from within the safety of the output routine}%
\changes{4.1n}{2010/01/02}{(AO, 571) Interface \cs{set@footnotewidth} for determining the set width of footnotes}%
\changes{4.1n}{2010/01/02}{(AO, 572) Independent footnote counter for title block. Abstract footnote counter shared with body.}%
\changes{4.1n}{2009/12/13}{(AO, 573) arrange to load \classname{lineno} after any other packages.}%
\changes{4.1n}{2010/01/04}{(AO, 575) the default for journal prstper is longbibliography}%
\changes{4.1n}{2010/01/04}{(AO, 576) In .bst files, remove support for the annote field}%
\changes{4.1n}{2010/01/02}{(AO) fine-tune spacing above and below widetext}%
\changes{4.1n}{2010/01/02}{(AO, 571) class file must set \cs{splittopskip}; fine tune \cs{skip}\cs{footins}; \cs{footnoterule} defined in terms of \cs{skip}\cs{footins}}%
\changes{4.1n}{2010/01/02}{(AO, 572) \cs{@makefntext} and \cs{frontmatter@makefntext} must be defined harmoniously}%
\changes{4.1o}{2010/02/02}{(AO, 575) Automatically incorporate the (Bib\TeX-generated) .bbl into an explicit \env{thebibliography}}%
\changes{4.1o}{2010/02/05}{(AO, 549) Remove patch to natbib, which is now at version 8.31a}
\changes{4.1o}{2010/02/07}{(AO, 578) accommodate the possible space character preceding \cs{BibitemShut}.} 
\changes{4.1o}{2010/02/05}{(AO, 579) Endnote shall comprise their own Bib\TeX\ entry type: @FOOTNOTE.} 
\changes{4.1o}{2010/02/10}{(AO, 580) Provide a document class option to turn off production of eprint field in bibliography.} 
\changes{4.1o}{2010/02/12}{(AO, 580) Control .bst at run time.}%
\changes{4.1o}{2010/02/09}{(AO, 581) Handle case: merged references, with first ending in a stop character.} 
\changes{4.1p}{2010/02/24}{(AO, 583) Provide interface to \classname{ltxgrid} \cs{onecolumn@grid@setup} and \cs{twocolumn@grid@setup}} 
\changes{4.1p}{2010/02/24}{(AO, 584) Per MD, remove trailing space character from each journal abbreviation: it had caused an extraneous space in the .bbl} 
\changes{4.1q}{2010/04/01}{(AO, 586) When .bbl is pasted into the document, prevent automatic bibliography inclusion.}%
\changes{4.1q}{2010/04/13}{(AO, 588) Only write \revtex-specific BibTeX .bib data if the .bst style is set by REVTeX.}%
\changes{4.1r}{2010/06/22}{(AO, 595) Provide \cs{lovname} along with other List of Videos definitions.}%

\end{filecontents*}
%    \end{macrocode}
%
% \subsection{The Document Body}
%
% Here is the document body, containing only a
% \cmd\DocInput\ directive---referring to this very file.
% This very cute self-reference is a common \classname{ltxdoc} idiom.
%    \begin{macrocode}
\begin{document}%
\expandafter\DocInput\expandafter{\jobname.dtx}%
\PrintChanges
\end{document}
%    \end{macrocode}
% And that is the end of the driver for the programmer's documentation.
%    \begin{macrocode}
%</doc>
%    \end{macrocode}
%
% \section{Overview}%
% \revtex\ is a \LaTeXe\ document class, somewhat like a hybrid of 
% the standard \LaTeX\ \classname{book} and \classname{article} classes.
%
% Certain packages are (should be) loaded by this class in any case:
% \classname{amssymb}, \classname{amsmath}, \classname{bm}, \classname{natbib}.
% 
% Certain packages are automatically loaded by this class
% when a corresponding class option has been invoked:
% 
% \begin{tabular}{ll}
% \textbf{\revtex\ option}&\textbf{package}\\
% \classoption{amsfonts}&\classname{amsfonts}\\
% \classoption{amssymb}&\classname{amssymb}\\
% \classoption{aps}&\classname{overcite}\\
% \end{tabular}
%
% Certain other packages are to be loaded by the
% document through explicit use of \cmd\usepackage.
% Some mentioned in the user documentation are
% \classname{graphicx}, \classname{longtable}, \classname{hyperref}, 
% and \classname{bm}.
%
% Certain commonly used packages are known to be incompatible with
% \revtex, among them \classname{multicol} and \classname{cite}.
% If such a package is found to be loaded, \revtex\ issues an error message
% and halts the job.
% Halting might be considered severe punishment for loading an incompatible 
% package, but if we were to proceed, an even weirder error might be encountered
% further down the road.
%
% This document class implements the substyle: a set of mutually
% exclusive class options that, in this case, allow the 
% document class to address multiple societies.
% It also implements a sub-substyle, giving the 
% journal of the given society.
% Any society may create a substyle;
% this file generates one for \classoption{aps}.
%
% FIXME: should always load the \classname{graphicx} package. No, allow user to load whichever graphics package is desired.
% 
% QUERY: since \classname{amsfonts} and \classname{amssymb} extend
% syntax, why not load them in any case?
% 
% Certain processing occurs at the endgame for reading in \revtex, thereby establishing
% precedence for assignments to \LaTeX's (and \revtex's) parameters:
% \begin{enumerate}
% \item
%  Figure out which society is operative and read in the indicated \file{.rtx} file.
% \item
%  Figure out which journal option is operative and execute the indicated journal command. This may lead to reading in a journal substyle \file{.rtx} file.
% \item
%  Figure out which pointsize is operative and execute the indicateded pointsize command. This may lead to reading in a pointsize  \file{.rtx} file.
% \item
%  Execute all of the document's options, in the order declared within the document.
% \item
%  Read in all required packages (like \classname{natbib}, \classname{amsfonts}, \classname{amssymb}), that were determined by class, society, and journal.
% \item
%  The last required package, if existing, is the document's style file, the \file{.rty} file. 
%  Note that the \file{.rty} file can override the assignments of \revtex, society, journal, and required packages, and even load its own packages.
%  It can also, via appendations to \cmd\setup@hook, override the setup code itself.
% \item
%  Execute all of the setup code accumulated. Such code can be queued by \revtex\ itself, by the society, by the journal, or by the pointsize. 
% \item
%  At this point, \revtex\ has completed the process of inputting itself, 
%  and \LaTeX\ will now execute the \cmd\AtEndOfClass\ procedures.
% \end{enumerate}
%
% \revtex\ will have enqueued code to execute at \cmd\AtBeginDocument\ time,
% in two different queues. 
% \cmd\document@inithook\ executes immediately upon encountering the \cmd\begin\env{document} statement, 
% \cmd\class@documenthook\ at the end of all the code enqueued via \cmd\AtBeginDocument. 
% \begin{enumerate}
% \item
%  Install procedures to execute at the very end of the class's
%  \cmd\AtBeginDocument\ processing, such as
%  \begin{enumerate}
%  \item
%   closing out the page grid
%  \item
%   putting out the LastPage label.
%  \item
%   issuing a \cmd\bibliographystyle\ command, based on the value set
%   by the society substyle.
%  \item
%   setting default values for parameters used in the document.
%FIXME: differentiate between class's parameters that can wait until
% they are used in the document, and parameters that are used at \cmd\AtBeginDocument\ time.
%  \end{enumerate}
% \item
%  Install procedures to execute the very last at \cmd\AtEndDocument\ time,
%  such as the \cmd\clearpage\ processing.
% \end{enumerate}
%
% Certain events are optionally scheduled for \cmd\AtBeginDocument\ time:
% \begin{enumerate}
% \item
%  Setting default values for the Booleans and for other procedures used in formatting.
% \item
%  In response to class options options, adjusting parameters and procedures used in formatting.
% \item
%  Implementing the \classoption{eqsecnum} option, if required.
% \item
%  Setting the state engine for data commands.
% \item
%  Memorizing procedures for later use.
% \item
%  Setting type size and area, for use by later calculations.
% \end{enumerate}
%
% Certain events are scheduled for \cmd\class@enddocumenthook\ time:
% \begin{enumerate}
% \item
%  Print out the migrated floats or the end notes, if needed.
% \item
%  Close out the page grid.
% \item
%  Label the last page of the document
% \item
%  (\classname{natbib}) prepare to read in the \file{.aux} file.
% \end{enumerate}
%
% \section{Writing journal-specifc extensions to \revtex}
%
% With this version of \revtex, we introduce a somewhat different
% scheme for adapting \revtex\ to the needs of a specific journal.
%
% To create a journal substyle, you create new class options in 
% \revtex\ for the society, say \classoption{osa}, and any of that society's journals,
% one of which is, say, \classoption{josaa},
% using the code for the APS as a guide.
% In particular, each of your new options should separately define \cmd\@society\
% and \cmd\@journal. That for the former will be the same for
% all options relating to a particular society.
%
% Then, for the society, you create a corresponding \file{.rtx}
% file, in our case \file{osa.rtx}.
% Within that file, you override procedures and parameter assignments as you see fit.
% Ideally they will be generally applicable to all of that society's
% journals (see the file \file{aps.rtx} for a realization of this scheme).
% Also within that file, you include a 
% section of code for each journal, that for josaa looks like:
% \begin{verbatim}
% \@ifx{\@journal\journal@josaa}{%
% <code specific to the josaa>
% }{}%
% \end{verbatim}
% 
% Thus far, the scheme is similar to that used in \revtex\ 3.1.
% However, the new scheme does differ from the old in that 
% the \file{.rtx} file should define no syntactical extensions
% to \revtex.
%
% \section{The \classname{revtex4} Document Class}
% Above, we took
% advantage of 1) the \LaTeX\ definition of \cmd\ProvidesPackage\ and
% 2) that the line of code immediately afterwards follows the \cmd\ProvidesClass\ statement above. 
%    \begin{macrocode}
%<*kernel>
%    \end{macrocode}
%
% Print a banner in the log:
%    \begin{macrocode}
\GenericInfo{}{\space
 Copyright (c) 2009 The American Physical Society.^^J
 mailto:revtex@aps.org^^J
 Licensed under the LPPL:^^Jhttp://www.ctan.org/tex-archive/macros/latex/base/lppl.txt^^J
 Arthur Ogawa <arthur_ogawa at sbcglobal dot net>^^J
 Based on work by David Carlisle <david at dcarlisle.demon.co.uk>%
 \@gobble
}%
%    \end{macrocode}
%
% \subsection{Compatibility Processing}
%
% If the document has \cmd\documentstyle|{revtex4}|,
% then, instead of attempting to run in compatability mode,
% just complain and exit.
% \changes{4.0c}{1999/11/13}{Lose compatability mode.}
%    \begin{macrocode}
\if@compatibility    
  \edef\reserved@a{\errhelp{%
   Change your \string\documentstyle\space statement to
   \string\documentclass\space and rerun.
  }}\reserved@a
  \errmessage{You cannot run \class@name\space in compatability mode}%
  \expandafter\@@end
\fi
%    \end{macrocode}
%
%
% \section{Extensions to the \LaTeX\ Kernel}
%
%    \begin{macrocode}
%</kernel>
%    \end{macrocode}
%
% Here, we incorporate the utility, frontmatter, and page grid packages.
% The \classname{ltxutil}, \classname{ltxfront}, and \classname{ltxgrid} source are distributed with 
% \revtex.
%
% Here begins the \file{options} \textsc{docstrip} module.
%    \begin{macrocode}
%<*options>
%    \end{macrocode}
%
% \subsection{Hooks}
%
% \begin{macro}{\setup@hook}
% The procedure 
% \cmd\setup@hook\
% serves as the vehicle for all code that 
% gives values to the class's parameters once all the 
% society, journal, options, and packages have been processed. 
%
% Arrange for journal substyles to set their own default values.
%    \begin{macrocode}
\let\setup@hook\@empty
%    \end{macrocode}
% \end{macro}
%
% After preamble processing is complete,
% detect whether package \classname{longtable} has been loaded and patch it.
%    \begin{macrocode}
\appdef\document@inithook{%
 \switch@longtable
 \let\LT@makecaption\LT@makecaption@rtx
}%
%    \end{macrocode}
%
% \begin{macro}{\LT@makecaption}
% We override the caption processing method of the \classname{longtable} package:
% space below the caption is created via strut instead of whitespace.
%    \begin{macrocode}
\def\LT@makecaption@rtx#1#2#3{%
  \LT@mcol\LT@cols c{%
    \hbox to\z@{%
     \hss
     \parbox[t]\LTcapwidth{%
      \sbox\@tempboxa{#1{#2: }#3\unskip\nobreak\vrule\@width\z@\@height\z@\@depth .5\baselineskip}%
      \ifdim\wd\@tempboxa>\hsize
        #1{#2: }#3\unskip\nobreak\vrule\@width\z@\@height\z@\@depth .5\baselineskip
      \else
        \hbox to\hsize{\hfil\box\@tempboxa\hfil}%
      \fi
      \endgraf
     }%
     \hss
    }%
  }%
}%
%    \end{macrocode}
% \end{macro}
%
% \begin{macro}{\robust@boldmath}
% Robustify the \cmd\boldmath command. 
% If Team \LaTeX\ (or any package) ever gets around to fixing this problem, 
% we will bow out. (This fix relates to bug \#394.)
%    \begin{macrocode}
\def\protectdef@boldmath{%
 \expandafter\@ifnotrelax\csname boldmath \endcsname{}{%
  \class@info{Robustifying \string\LaTeX's \string\boldmath\space command}%
  \expandafter\let\csname boldmath \endcsname\boldmath
  \expandafter\def\expandafter\boldmath\expandafter{\expandafter\protect\csname boldmath \endcsname}%
 }%
}%
\appdef\document@inithook{%
 \protectdef@boldmath
}%
%    \end{macrocode}
% \end{macro}
%
% \section{Compatability with the geometry package}
%
% The \classname{geometry} package of Hideo Umeki provides a way to 
% specify the metrics of the media and page layout.
% We want to ensure that \revtex\ does not foreclose on the use of 
% this or any other such package, nor to interfere with
% explicit assignments of such metric parameters within the 
% document preamble.
% 
% The \LaTeX\ parameters (resp. \TeX\ primitives) set by \classname{geometry} are:
% \cmd\paperwidth, \cmd\paperheight, 
% \cmd\textwidth, \cmd\textheight, 
% \cmd\topmargin, \cmd\headheight, \cmd\headsep, 
% \cmd\footskip, \cmd\skip\cmd\footins, 
% \cmd\marginparwidth, \cmd\marginparsep, 
% \cmd\oddsidemargin, \cmd\evensidemargin, 
% \cmd\columnsep, 
% \cmd\hoffset, \cmd\voffset, \cmd\mag, 
% \cmd\if@twocolumn, \cmd\if@twoside, \cmd\if@mparswitch, \cmd\if@reversemargin, 
%
% \revtex's assignments can be overridden by 
% any package or other statement in the document preamble, 
% so it should be compatible. 
%
%FIXME: one parameter is rendered semantically void (by \file{ltxgrid.dtx}): 
% \cmd\if@twocolumn. 
%
% \section{Options}\label{sec:options}%
%
% \subsection{Define Booleans Used in Options}%
% The following Booleans are used within the document class to
% allow the document or the substyle to make selections
% of formatting.
% An explicit document class option overrides an assignment made by a substyle.
%
% \subsection{Declare Options}%
%
% \subsubsection{Checkin: for Editorial Use}%
% A document class option declaring that the document is being processed by the 
% editorial staff.
%
% This option should:
% \begin{itemize}
% \item
% put date in footer along with folio
% \item
% Have the effect of selecting the \classoption{preprint} class option.
% \item
% Have the effect of selecting the \classoption{showpacs} class option.
% \item
% specify that when a float is placed h or H, it will be allowed to break over pages.
% (Note: be sure that if the enclosed \env{tabular} has an optional argument, you change it to [v], or remove it entirely.)
% \end{itemize}
% \changes{4.0c}{1999/11/13}{New option ``checkin''}
%    \begin{macrocode}
\DeclareOption{checkin}{%
  \@booleantrue\dateinRH@sw
  \@booleantrue\preprintsty@sw
  \def\@pointsize{12}%
  \@booleantrue\showPACS@sw
  \@booleantrue\showKEYS@sw
  \def\fp@proc@h{\allow@breaking@tables}%
  \def\fp@proc@H{\allow@breaking@tables}%
}%
\@booleanfalse\dateinRH@sw
\def\checkindate{\dateinRH@sw{{\tiny(\today)}}{}}%
\def\allow@breaking@tables{%
 \def\array@default{v}% tabular can break over pages
 \@booleanfalse\floats@sw % table can break over pages
}%
%    \end{macrocode}
%
% \subsubsection{Preprint Style}%
%
% \begin{macro}{\preprintsty@sw}
% The boolean \cmd\preprintsty@sw\ signifies that 
% the document is to be formatted in preprint style.
% \changes{4.0a}{1998/01/31}{Rearrange the ordering so numerical ones come first. AO: David, what does this mean?}
% \changes{4.0a}{1998/06/10}{comma not space between email and homepage}
% \changes{4.0b}{1999/06/20}{Move this ``complex'' option to the front, where it can be overridden by ``simple'' options.}
% \changes{4.1b}{2008/06/01}{Add option reprint, opposite of preprint, and preferred alternative to twocolumn}
%    \begin{macrocode}
\DeclareOption{preprint}{%
 \@booleantrue\preprintsty@sw
 \ExecuteOptions{12pt}%
}%
\DeclareOption{reprint}{%
 \@booleanfalse\preprintsty@sw
 \@booleantrue\twocolumn@sw
 \ExecuteOptions{10pt}%
}%
\DeclareOption{manuscript}{%
 \class@warn{Document class option manuscript is obsolete; use preprint instead}%
 \ExecuteOptions{preprint}%
}%
\@booleanfalse\preprintsty@sw
%    \end{macrocode}
% \end{macro}
%
% \begin{macro}{\showPACS@sw}
% \begin{macro}{\showKEYS@sw}
% If \cmd\showPACS@sw\ is true, print the PACS information in the title block,
% otherwise not.
% Similarly for \cmd\showKEYS@sw\ and the keywords.
% \subsubsection{Showing PACS and keywords}%
% \changes{4.0b}{1999/06/20}{New option}
% \changes{4.0e}{2000/09/20}{New option showkeys}
%    \begin{macrocode}
\DeclareOption{showpacs}{%
  \@booleantrue\showPACS@sw
}%
\DeclareOption{noshowpacs}{%
  \@booleanfalse\showPACS@sw
}%
\DeclareOption{showkeys}{%
  \@booleantrue\showKEYS@sw
}%
\DeclareOption{noshowkeys}{%
  \@booleanfalse\showKEYS@sw
}%
\@booleanfalse\showPACS@sw
\@booleanfalse\showKEYS@sw
%    \end{macrocode}
% \end{macro}
% \end{macro}
%
% \subsubsection{Balance the last page when in two-column page grid}%
%
% \begin{macro}{\balancelastpage@sw}
% \changes{4.1b}{2008/07/01}{(AO) make settings at class time instead of deferring them to later.}
% If we are in a two-column page grid, we may wish to balance the columns
% of the last page. This will be done automatically if the 
% \classoption{twocolumn} document class option is chosen.
% This action will be turned off by the \classoption{nobalancelastpage}.
% A complementary class option, \classoption{balancelastpage} is also provided.
%    \begin{macrocode}
\DeclareOption{balancelastpage}{%
  \@booleantrue\balancelastpage@sw
}%
\DeclareOption{nobalancelastpage}{%
  \@booleanfalse\balancelastpage@sw
}%
\@booleantrue\balancelastpage@sw
%    \end{macrocode}
% \end{macro}
%
% \subsubsection{Showing preprint numbers}%
%
% \begin{macro}{\preprint@sw}
% \changes{4.0d}{2000/04/10}{New option}
% \changes{4.1b}{2008/07/01}{(AO) make settings at class time instead of deferring them to later.}
% \changes{4.1n}{2009/11/06}{(AO, 565) restore 4.0 behavior: default value of \cs{preprint@sw} follows \cs{preprintsty@sw}}
% The boolean \cmd\preprint@sw\ signifies that 
% the preprints (cf. \cmd\preprint) are to be formatted (usually on the title page).
% The option \classoption{preprintnumbers} declares to do so,
% \classoption{nopreprintnumbers} declares not to; 
% the default is to follow \cmd\preprintsty@sw. 
%    \begin{macrocode}
\DeclareOption{nopreprintnumbers}{\@booleanfalse\preprint@sw}%
\DeclareOption{preprintnumbers}{\@booleantrue\preprint@sw}%
\appdef\setup@hook{%
 \@ifxundefined\preprint@sw{\let\preprint@sw\preprintsty@sw}{}%
}%
%    \end{macrocode}
% \end{macro}
%
% \subsubsection{Hypertext Option}%
%
% \changes{4.0a}{1998/01/31}{protect against hyperref revtex kludges which are not needed now}
% \changes{4.0b}{1999/06/20}{Move this ``complex'' option to the front, where it can be overridden by ``simple'' options.}
% \changes{4.0rc4}{2001/07/23}{hyperref is no longer loaded via class option: use a usepackage statement instead}
%    \begin{macrocode}
\DeclareOption{hyperref}{%
%    \end{macrocode}
% The following code had been commented out, it is now truly a comment:
%    \begin{verbatim}
%\AtEndOfClass{%
% \begingroup
%  \edef\@tempa{%
%   \let
%   \noexpand\@clsextension
%   \noexpand\@empty
%   \noexpand\RequirePackage{hyperref}%
%   \def\noexpand
%   \@clsextension{\@clsextension}%
%  }%
%  \expandafter
% \endgroup
% \@tempa
%}%
% \def\@pointsize{10}%
%    \end{verbatim}
%
% If you have a hyper-foo enabled browser you may prefer this format
% which does not print the URL for the home page, but just makes the
% name a link, but by default print it so it works on paper.
% \changes{4.0a}{1998/01/31}{print homepage}
%
%    \begin{macrocode}
 \class@warn{Class option "hyperref" is no longer supported.^^JEmploy \string\usepackage{hyperref} instead}%
}%
%    \end{macrocode}
%
% \subsubsection{Type Size}%
% \changes{4.1b}{2008/07/01}{(AO) make settings at class time instead of deferring them to later.}
% Use \cmd\@pointsize|=10| rather than \cmd\@ptsize|=0| to allow easy extensions
% to 9pt or whatever. 
% Note: the three alternatives are mutually exclusive. 
%
% At this point, the parameter \cmd\@pointsize\ is set to \cmd\undefined:
% a society \emph{must} give it a definition.
%    \begin{macrocode}
\DeclareOption{10pt}{\def\@pointsize{10}}%
\DeclareOption{11pt}{\def\@pointsize{11}}%
\DeclareOption{12pt}{\def\@pointsize{12}}%
\let\@pointsize\@undefined
%    \end{macrocode}
%
% \subsubsection{Media Size}%
%
% \begin{macro}{\paperheight}
% \begin{macro}{\paperwidth}
% \changes{4.0b}{1999/06/20}{Restore all media size class option of \protect\file{classes.dtx}}
%    \begin{macrocode}
\DeclareOption{a4paper}{%
    \setlength\paperheight {297mm}%
    \setlength\paperwidth  {210mm}%
}%
\DeclareOption{a5paper}{%
    \setlength\paperheight {210mm}%
    \setlength\paperwidth  {148mm}%
}%
\DeclareOption{b5paper}{%
    \setlength\paperheight {250mm}%
    \setlength\paperwidth  {176mm}%
}%
\DeclareOption{letterpaper}{%
    \setlength\paperheight {11in}%
    \setlength\paperwidth  {8.5in}%
}%
\DeclareOption{legalpaper}{%
    \setlength\paperheight {14in}%
    \setlength\paperwidth  {8.5in}%
}%
\DeclareOption{executivepaper}{%
    \setlength\paperheight {10.5in}%
    \setlength\paperwidth  {7.25in}%
}%
\DeclareOption{landscape}{%
    \setlength\@tempdima   {\paperheight}%
    \setlength\paperheight {\paperwidth}%
    \setlength\paperwidth  {\@tempdima}%
}%
%    \end{macrocode}
%
% \changes{4.1a}{2008/01/18}{(AO, 478) \cs{ds@letterpaper}, so that ``letterpaper really is the default''}%
% Effectively select \classoption{letterpaper}. 
%    \begin{macrocode}
\ds@letterpaper
%    \end{macrocode}
% \end{macro}
% \end{macro}
%
% \subsubsection{Bibnotes}%
%
% \begin{macro}{\frontmatter@footnote@produce}
% Frontmatter footnotes result from frontmatter commands like
% \cmd\email, \cmd\homepage, \cmd\altaffiliation, and \cmd\thanks.
% The default for \cmd\frontmatter@footnote@produce\ is
% \cmd\frontmatter@footnote@produce@footnote, which formats the frontmatter footnotes
% at the foot of the title page. The \classoption{bibnotes} class option
% defers them to the bibliography.
% \changes{4.1b}{2008/06/01}{(AO) Implement bibnotes through \cs{frontmatter@footnote@produce} instead of \cs{bibnotes@sw}}%
%    \begin{macrocode}
\DeclareOption{bibnotes}{\let\frontmatter@footnote@produce\frontmatter@footnote@produce@endnote}%
\DeclareOption{nobibnotes}{\let\frontmatter@footnote@produce\frontmatter@footnote@produce@footnote}%
\let\frontmatter@footnote@produce\frontmatter@footnote@produce@footnote
\appdef\class@enddocumenthook{\auto@bib}%
%    \end{macrocode}
% \end{macro}
%
% \subsubsection{Footinbib}%
%
% \begin{macro}{\footinbib@sw}
% The boolean \cmd\footinbib@sw\ signifies that 
% text footnotes are to be set in the bibliography, as endnotes.
% 
% The document may set the value one way or the other via
% the following two class options.
%    \begin{macrocode}
\DeclareOption{footinbib}{\@booleantrue\footinbib@sw}
\DeclareOption{nofootinbib}{\@booleanfalse\footinbib@sw}
%    \end{macrocode}
% The default value is \cmd\false@sw, and the society or journal
% may override the default. 
% \changes{4.1b}{2008/07/01}{(AO) make settings at class time instead of deferring them to later.}
%    \begin{macrocode}
\@booleanfalse\footinbib@sw
%    \end{macrocode}
% \end{macro}
%
% \subsubsection{altaffilletter}%
% 
% \begin{macro}{\altaffilletter@sw}
% \changes{4.1b}{2008/07/01}{(AO) make settings at class time instead of deferring them to later.}
% Determine the procedure \cmd\thefootnote\ used in frontmatter: 
% the footnote symbol used in titlepage footnotes.
%     \begin{macrocode}
\DeclareOption{altaffilletter}{\@booleantrue\altaffilletter@sw}%
\DeclareOption{altaffilsymbol}{\@booleanfalse\altaffilletter@sw}%
\@booleanfalse\altaffilletter@sw
%    \end{macrocode}
% \end{macro}
%
% \subsubsection{superbib}%
%
% \begin{macro}{\place@bibnumber}
% \changes{4.1b}{2008/07/01}{(AO) make settings at class time instead of deferring them to later.}
% The procedure \cmd\place@bibnumber\ produces the number
% at the head of the \cmd\bibitem, in the bibliography.
% By default, it has the \cmd\bibnumfmt\ meaning assigned by the \classname{natbib} package.
% It may be overridden by society, journal, or by the document options.
%
%     \begin{macrocode}
\DeclareOption{superbib}{%
 \let\place@bibnumber\place@bibnumber@sup
}%
\def\place@bibnumber{\NATx@bibnumfmt}%
%    \end{macrocode}
% \end{macro}
%
% \begin{macro}{\place@bibnumber@sup}
% \begin{macro}{\place@bibnumber@inl}
% For producing the \cmd\bibitem\ device,
% we define two procedures to select from.
%
% Note that we could have used \classname{natbib}'s \cmd\ifNAT@super\
% switch, but it does not allow for altering the meaning of \cmd\bibnumfmt.
%    \begin{macrocode}
\def\place@bibnumber@sup#1{\textsuperscript{#1}}%
\def\place@bibnumber@inl#1{[#1]}%
%    \end{macrocode}
% \end{macro}
% \end{macro}
%
% \subsubsection{citeautoscript}%
% \begin{macro}{\citeautoscript@sw}
% This class option allows you to automatically accomodate a change from non-superscripted, numbered references
% to superscripted, bunmbered references.
%
% Note: you should always mark up your document with the assumption that references are \emph{not} going to be superscripted.
% Otherwise this option has no hope of working properly.
%
%     \begin{macrocode}
\DeclareOption{citeautoscript}{\@booleantrue\citeautoscript@sw}%
\@booleanfalse\citeautoscript@sw
%    \end{macrocode}
% \end{macro}
%
% \subsubsection{Variants on the Bibliography Style}%
% \changes{4.1b}{2008/07/01}{(AO) provide option longbibliography}
%
% \revtex\ anticipates that a society or journal will provide two related Bib\TeX bibliography style variants,
% designating one as the default.
% A pair of document class options \classoption{longbibliography} and \classoption{nolongbibliography}
% allows the user to explicitly select between them. 
% \begin{macro}{\longbibliography@sw}
%
% \revtex's default for \cmd\longbibliography@sw\ is \cmd\false@sw.
%
%     \begin{macrocode}
\DeclareOption{longbibliography}{\@booleantrue\longbibliography@sw}%
\DeclareOption{nolongbibliography}{\@booleanfalse\longbibliography@sw}%
\@booleanfalse\longbibliography@sw
%    \end{macrocode}
% \end{macro}
%
% \begin{macro}{\eprint@enable@sw}
% \changes{4.1o}{2010/02/10}{(AO, 580) Provide a document class option to turn off production of eprint field in bibliography.} 
%
% The document class options \classoption{eprint} and \classoption{noeprint}
% give the user the ability to turn off production of the \texttt{eprint} field in the bibliography. 
%     \begin{macrocode}
\DeclareOption{eprint}{\@booleantrue\eprint@enable@sw}%
\DeclareOption{noeprint}{\@booleanfalse\eprint@enable@sw}%
\@booleantrue\eprint@enable@sw
%    \end{macrocode}
% \end{macro}
%
% \subsubsection{Simplex/Duplex Pages}%
%
% \begin{macro}{\twoside@sw}
% \changes{4.1b}{2008/07/01}{(AO) coordinate \cs{if@twoside} with \cs{twoside@sw}}
% The flag \cmd\twoside@sw\ signifies that the document is to be 
% formatted for duplex printing.
% After the preamble is processed, we align the value
% of the kernel \cmd\newif\ switch \cmd\if@twoside\ 
% to that of \cmd\twoside@sw. 
% \cmd\if@twoside\ itself is used in the kernel's
% \cmd\cleardoublepage\ and \cmd\@outputpage\ procedures.
%    \begin{macrocode}
\@booleanfalse\twoside@sw
\appdef\document@inithook{%
 \twoside@sw{\@twosidetrue}{\@twosidefalse}%
}%
%    \end{macrocode}
% \changes{4.0a}{1998/01/31}{4.d had twoside option setting twoside switch to false}
% The complementary options \classoption{twoside} and \classoption{oneside}
% assert formatting for duplex or simplex printing, respectively.
% At the same time, we arrange for the selection of the page grid 
% with respect to the marginal column:
% Because \cmd\if@reversemargin\ remains default (false),
% if duplex printing, this column will always be on the (right),
% if simplex printing, it will always be on the (outside).
% QUERY: correct choice?
%FIXME: assign \cmd\if@mparswitch\ later (and protect the assignment, too). 
%    \begin{macrocode}
\DeclareOption{twoside}{\@booleantrue \twoside@sw\@mparswitchfalse}%
\DeclareOption{oneside}{\@booleanfalse\twoside@sw\@mparswitchtrue}%
%    \end{macrocode}
% \end{macro}
%
% \subsubsection{Two-Column Page Grid}%
%
% \begin{macro}{\twocolumn@sw}
% \changes{4.1b}{2008/07/01}{(AO) make settings at class time instead of deferring them to later.}
% \changes{4.1p}{2010/02/24}{(AO, 583) Provide interface to \classname{ltxgrid} \cs{onecolumn@grid@setup} and \cs{twocolumn@grid@setup}} 
% The flag \cmd\twocolumn@sw\ signifies that the document is to be 
% formatted in the two-column page grid.
%
% If no options relating to page grid are invoked by 
% \cmd\AtBeginDcoument\ time, we set default values.
% Up to that point, the class can check if \cmd\twocolumn@sw\
% is \cmd\undefined\ to see if any related options have been invoked.
%
% \classoption{onecolumn} specifies one-column page grid.
% The \classoption{twocolumn} class option employs the standard mechanism for
% changing the column grid: the \classname{ltxgrid} package.
%    \begin{macrocode}
\DeclareOption{onecolumn}{\@booleanfalse\twocolumn@sw}%
\DeclareOption{twocolumn}{\@booleantrue \twocolumn@sw}%
\@booleanfalse\twocolumn@sw
%    \end{macrocode}
% The column grid is determined by the state of the switch \cmd\twocolumn@sw
% and is effected at \cmd\class@documenthook\ time.
% The society or journal file may re-define \cmd\select@column@grid\ 
% to accomodate, e.g., more than two choices for the page grid. 
%
% Note that \cmd\open@column@two\ adds items to the Main Vertical List, 
% so constitutes the true beginning of the document.
%
% Note also that if the selected column grid is a one-column grid, 
% there is nothing to do, because \classname{ltxgrid} has already
% set that up via \cmd\@begindocumenthook.
%    \begin{macrocode}
\def\select@column@grid{%
 \twocolumn@sw{%
  \twocolumn@grid@setup
  \open@twocolumn
 }{%
  \onecolumn@grid@setup
%    \end{macrocode}
%    \begin{verbatim}
% \open@onecolumn
%    \end{verbatim}
%    \begin{macrocode}
 }%
}%
\appdef\class@documenthook{%
 \select@column@grid
}%
%    \end{macrocode}
%
% \end{macro}
%
% \begin{macro}{\clear@document}
% \changes{4.1n}{2009/12/05}{(AO, 569) Use of \classname{hyperref} interferes with column balancing of last page}%
%
% We install into \cmd\class@enddocumenthook\ 
% a trap for the procedure \cmd\clearpage\ that attempts to end the current page.
% This procedure needs to be headpatched with \cmd\close@column\ to end the current page grid. 
%
% This procedure is executed after all typesetting is complete
% i.e., after items like \cmd\printtables, as well as all commands queued up by \cmd\AtEndDocument. 
%
%    \begin{macrocode}
\appdef\setup@hook{%
 \let\clearpage@ltx\clearpage
 \prepdef\clear@document{\let\clearpage\clearpage@ltx\let\clear@document\@empty\close@column}%
 \appdef\class@documenthook{%
  \appdef\class@enddocumenthook{%
   \let\clearpage\clear@document
  }%
 }%
}%
%    \end{macrocode}
%
% \end{macro}
%
% \begin{macro}{\authoryear@sw}
% \changes{4.1a}{2008/01/18}{(AO, 457) Endnotes to be sorted in with numerical citations.}%
% The boolean \cmd\authoryear@sw\ signifies that we are to use author-year citations rather than numerical citations.
% 
% The \classoption{author-year} class option selects ``'author-year'' citations;  
% \classoption{numerical} selectes ``numerical'' citations.
% The former is the default.
%    \begin{macrocode}
\DeclareOption{author-year}{\@booleantrue\authoryear@sw}%
\DeclareOption{numerical}{\@booleanfalse\authoryear@sw}%
\@booleanfalse\authoryear@sw
%    \end{macrocode}
% \end{macro}
%
% \begin{macro}{\galley@sw}
% The boolean \cmd\galley@sw\ signifies that 
% the document is to be formatted in galley style.
%
% Asserting both \cmd\galley@sw\ and \cmd\preprintsty@sw\ 
% may produce strange formatting results, but it is not illegal.
% However, it \emph{is} illegal to assert galley and any 
% twocolumn option.
%    \begin{macrocode}
%    \end{macrocode}
% \classoption{galley} emulates setting the galleys of
% a two-column journal.
% CHANGED: this option should effectively set \cmd\preprintsty@sw\ false.
% NOTE: it makes no sense to assert both galley and twocolumn.
% \changes{4.1b}{2008/07/01}{Class option galley sets \cs{preprintsty@sw} to false}
% 
%    \begin{macrocode}
\DeclareOption{galley}{%
  \ExecuteOptions{onecolumn}%
  \@booleantrue\galley@sw
  \@booleanfalse\preprintsty@sw
  \appdef\setup@hook{%
    \advance\textwidth-\columnsep
    \textwidth.5\textwidth
  }%
}%
\@booleanfalse\galley@sw
%    \end{macrocode}
% \end{macro}
%
% \subsubsection{raggedbottom or flushbottom}%
% \begin{macro}{\raggedcolumn@sw}
% \changes{4.1b}{2008/07/01}{(AO) make settings at class time instead of deferring them to later.}
% The class options \classoption{raggedbottom} and \classoption{flushbottom} determine whether the columns (page)
% are ragged bottom or flush bottom.
% Note that we do not select a default here; that is done by the journal substyle.
%    \begin{macrocode}
\DeclareOption{raggedbottom}{\@booleantrue\raggedcolumn@sw}
\DeclareOption{flushbottom}{\@booleanfalse\raggedcolumn@sw}
\@booleanfalse\raggedcolumn@sw
\appdef\setup@hook{%
 \raggedcolumn@sw{\raggedbottom}{\flushbottom}%
}%
%    \end{macrocode}
% \end{macro}
%
% \subsubsection{tightenlines}%
%
% This class option specifies that standard leading 
% is to be used to set the type.
% If lacking, the leading will be loose.
%
% \begin{macro}{\tightenlines@sw}
% \changes{4.1b}{2008/07/01}{(AO) make settings at class time instead of deferring them to later.}
% The boolean \cmd\tightenlines@sw\ signifies that 
% the leading is to be made standard amount.
% If false, it means that the leading is to be set extra open.
% Has no effect on 10pt size option.
%
%    \begin{macrocode}
\DeclareOption{tightenlines}{\@booleantrue\tightenlines@sw}
\@booleanfalse\tightenlines@sw
%    \end{macrocode}
% \end{macro}
%
% \subsubsection{lengthcheck}%
%
% \begin{macro}{\lengthcheck@sw}
% \changes{4.1b}{2008/07/01}{(AO) make settings at class time instead of deferring them to later.}
% The flag \cmd\lengthcheck@sw\ signifies that 
% the length checking is in effect.
% It is up to the individual journal substyle 
% to alter its formatting accordingly.
%    \begin{macrocode}
\@booleanfalse\lengthcheck@sw
%    \end{macrocode}
% This class option specifies that the formatted document 
% should approach as closely as possible the
% formatting of an actual journal article 
% to facilitate the author's performance of a length check.
%
%FIXME: society or journal may have its own definition of this option.
%    \begin{macrocode}
\DeclareOption{lengthcheck}{%
 \@booleantrue\lengthcheck@sw
 \ExecuteOptions{reprint}%
}%
%    \end{macrocode}
% In addition, if length checking is in effect, we will enable 
% the tally of text length. 
%    \begin{macrocode}
\appdef\setup@hook{%
 \lengthcheck@sw{\@booleantrue\tally@box@size@sw}{}%
}%
%    \end{macrocode}
% \end{macro}
%
% \subsubsection{Draft and Final}%
%
% \begin{macro}{\draft@sw}
% \changes{4.1b}{2008/07/01}{(AO) make settings at class time instead of deferring them to later.}
% The flag \cmd\draft@sw\ signifies that the document is to be 
% formatted in draft mode. 
%
%    \begin{macrocode}
\appdef\setup@hook{%
 \draft@sw{\overfullrule 5\p@}{\overfullrule\z@}%
}%
%    \end{macrocode}
%
% Certain packages may pay attention to the 
% class option \classoption{draft} that sets this Boolean.
%    \begin{macrocode}
\DeclareOption{draft}{\@booleantrue\draft@sw}%
\DeclareOption{final}{\@booleanfalse\draft@sw}%
\@booleanfalse\draft@sw
%    \end{macrocode}
%
% \end{macro}
%
% \subsubsection{eqsecnum}%
%
% \begin{macro}{\eqsecnum@sw}
% The flag \cmd\eqsecnum@sw\ signifies that equations are to be 
% numbered with the section, e.g., ``Eq. (2.13)''.
%    \begin{macrocode}
\appdef\setup@hook{%
 \eqsecnum@sw{%
  \@addtoreset{equation}{section}%
  \def\theequation@prefix{\arabic{section}.}%
 }{}%
}%
%    \end{macrocode}
%
% The \classoption{eqsecnum} class option signifies that equations are
% to be numbered within sections.
%    \begin{macrocode}
\DeclareOption{eqsecnum}{\@booleantrue\eqsecnum@sw}%
\@booleanfalse\eqsecnum@sw
%    \end{macrocode}
% \end{macro}
%
% \subsubsection{secnumarabic}%
% \changes{4.1b}{2008/07/01}{(AO) No longer use \cs{secnumarabic@sw}, instead use \cs{setup@secnums}}
% The \classoption{secnumarabic} class option signifies that sectioning commands are
% to be numbered arabic: 
% the procedure \cmd\secnums@arabic\ is executed as the default. 
% Otherwise, the procedure \cmd\secnums@rtx\ determines things. 
% The society or journal may redefine either procedure,
% and may change the definition of \cmd\setup@secnums\ itself, 
% thereby establishing a different default. 
%    \begin{macrocode}
\appdef\setup@hook{%
 \setup@secnums
}%
\DeclareOption{secnumarabic}{%
 \def\setup@secnums{\secnums@arabic}%
}%
\def\setup@secnums{\secnums@rtx}%
%    \end{macrocode}
% The code that defines \cmd\secnums@rtx\ and \cmd\secnums@arabic\ appears
% in Section~\ref{sec:secnumarabic}.
%
% \begin{macro}{fleqn}
% FIXME: model fleqn after amsfonts. I no longer understand why I said this.
% \file{fleqn.clo} is not a package, so it can simply be \cmd\input. 
%    \begin{macrocode}
\DeclareOption{fleqn}{%
 \input{fleqn.clo}%
}%
%    \end{macrocode}
% \end{macro}
%
% \subsubsection{floats/endfloats}%
%
% \begin{macro}{\floats@sw}
% \begin{macro}{\floatp@sw}
% \changes{4.1b}{2008/07/01}{(AO) make settings at class time instead of deferring them to later.}
% The Boolean \cmd\floats@sw\ signifies that floats are to be floated;
% if false, that floats are to be deferred to the end of the document.
% By default, the former.
% Note that the state of this Boolean is to be changed by 
% the document class in response to user-selected options.
% 
% This boolean and the assignment of its default value is done by
% the \classname{ltxutil} package.
%
% The Boolean \cmd\floatp@sw\ signifies that endfloats are to be set one per page;
% if false, that endfloats are to be set with multiple floats per page permitted.
% By default, the latter.
% Note that the state of this Boolean is to be changed by 
% the document class in response to user-selected options.
% The default is established here.
%
% These options control, via the Boolean \cmd\floats@sw, 
% whether floats are to be migrated to the end of the document.
%    \begin{macrocode}
\DeclareOption{floats}{\@booleantrue\floats@sw\@booleanfalse\floatp@sw}
\DeclareOption{endfloats}{\@booleanfalse\floats@sw\@booleanfalse\floatp@sw}
\DeclareOption{endfloats*}{\@booleanfalse\floats@sw\@booleantrue\floatp@sw}
\@booleantrue\floats@sw
\@booleantrue\floatp@sw
%    \end{macrocode}
% \end{macro}
% \end{macro}
%
% \subsubsection{titlepage/notitlepage}%
% These options control, via \cmd\titlepage@sw, 
% whether the title block is to be set on a separate page.
%
% \begin{macro}{\titlepage@sw}
% \changes{4.1b}{2008/07/01}{(AO) make settings at class time instead of deferring them to later.}
% The flag \cmd\titlepage@sw\ signifies that a forced 
% page break is to follow the title page: the article
% title appears on a page by itself.
%
%    \begin{macrocode}
\DeclareOption{titlepage}{\@booleantrue\titlepage@sw}
\DeclareOption{notitlepage}{\@booleanfalse\titlepage@sw}
\@booleanfalse\titlepage@sw
%    \end{macrocode}
% \end{macro}
%
% \subsubsection{Substyle and Sub-substyle}%
%
% \begin{macro}{\change@society}
% \begin{macro}{\change@journal}
% \changes{4.1b}{2008/07/01}{(AO) Provide more diagnostics when \cs{@society} is assigned.}
% If the society or, resp., journal has already been assigned,
% notify user whether it is being overridden. 
% 
%    \begin{macrocode}
\def\change@society#1{%
 \def\@tempa{#1}%
 \@ifxundefined\@society{%
  \class@info{Selecting society \@tempa}%
  \let\@society\@tempa
 }{%
  \@ifx{\@tempa\@society}{}{%
   \class@warn{Conflicting society \@tempa<>\@society; not selected}%
  }%
 }%
}%
\def\change@journal#1{%
 \def\@tempa{#1}%
 \@ifxundefined\@journal{%
  \class@info{Selecting journal \@tempa}%
  \let\@journal\@tempa
 }{%
  \@ifx{\@tempa\@journal}{}{%
   \class@warn{Conflicting journal \@tempa<>\@journal; not selected}%
  }%
 }%
}%
%    \end{macrocode}
% \end{macro}
% \end{macro}
%
% Here had been the class options relating to the APS.
% Now that all societies are on an equal footing, this code is in the
% respective \file{.rtx} file.
%    \begin{macrocode}
%    \end{macrocode}
%
% \subsubsection{Optical Society of America}%
% Here are the class options relating to the Optical Society of America.
%
% Note: as of 2008, the only OSA module being distributed by ctan is \file{osajnl.rtx}.
% The class options declared here are, I think, unused.
%    \begin{macrocode}
\DeclareOption{osa}{\change@society{osa}\let\@journal\@undefined}%
\DeclareOption{osameet}{\change@society{osa}\def\@journal{osameet}}%
\DeclareOption{opex}{\change@society{osa}\def\@journal{opex}}%
\DeclareOption{tops}{\change@society{osa}\def\@journal{tops}}%
\DeclareOption{josa}{\change@society{osa}\def\@journal{josa}}%
%    \end{macrocode}
%
% \begin{macro}{\rtx@require@packages}
% \changes{4.1b}{2008/07/01}{Read in all required packages together}
% The procedure \cmd\rtx@require@packages\ accumulates all \cmd\RequirePackage\ statements
% in the course of loading the document class. 
% Carrying out these operations at that time is needed:
% \cmd\ProcessOptions\ must be executed first. 
%    \begin{macrocode}
\let\rtx@require@packages\@empty
%    \end{macrocode}
% \end{macro}
%
% \begin{macro}{\MakeUppercase}
% \begin{macro}{\MakeLowercase}
% \changes{4.1b}{2008/07/01}{Package textcase is now simply a required package}
% \changes{4.1b}{2008/07/01}{Read in all required packages together}
% \changes{4.1g}{2009/10/07}{(AO, 538) \cs{MakeTextUppercase} inappropriately expands the double backslash}
% We load the \classname{textcase} package of David Carlisle.
% Now that its bug of long standing has been repaired, we no longer need to doctor it up. 
% And, because its loading has been deferred until \cmd\rtx@require@packages\ time, 
% we no longer override \LaTeX\ here. Instead, the \classname{textcase} package
% will be asked to do that. 
%    \begin{macrocode}
\appdef\rtx@require@packages{%
 \RequirePackage[overload]{textcase}%
}%
%    \end{macrocode}
% The following code used to
% let the \classname{textcase} commands override those of \LaTeX:
%    \begin{verbatim}
%\appdef\setup@hook{%
% \expandafter
% \let\csname MakeUppercase \expandafter\endcsname
%    \csname MakeTextUppercase \endcsname
% \expandafter
% \let\csname MakeLowercase \expandafter\endcsname
%    \csname MakeTextLowercase \endcsname
% }%
%    \end{verbatim}
% \end{macro}
% \end{macro}
%
% \begin{macro}{amsfonts}
% \changes{4.1b}{2008/07/01}{Read in all required packages together}
% The class option \classoption{amsfonts} has the
% same effect as if the document preamble contained
% a \cmd\usepackage\verb${amsfonts}$ statement.
%    \begin{macrocode}
\DeclareOption{amsfonts}{%
  \def\class@amsfonts{\RequirePackage{amsfonts}}%
}%
\DeclareOption{noamsfonts}{%
  \let\class@amsfonts\@empty
}%
\appdef\rtx@require@packages{%
 \@ifxundefined\class@amsfonts{}{\class@amsfonts}%
}%
%    \end{macrocode}
% \end{macro}
%
% \begin{macro}{amssymb}
% \changes{4.1b}{2008/07/01}{Read in all required packages together}
% The class option \classoption{amssymb} has the
% same effect as if the document preamble contained
% a \cmd\usepackage\verb${amssymb}$ statement.
%    \begin{macrocode}
\DeclareOption{amssymb}{%
  \def\class@amssymb{\RequirePackage{amssymb}}%
}%
\DeclareOption{noamssymb}{%
  \let\class@amssymb\@empty
}%
\appdef\rtx@require@packages{%
 \@ifxundefined\class@amssymb{}{\class@amssymb}%
}%
%    \end{macrocode}
% \end{macro}
%
% \begin{macro}{amsmath}
% \changes{4.1b}{2008/07/01}{Read in all required packages together}
% The class option \classoption{amsmath} has the
% same effect as if the document preamble contained
% a \cmd\usepackage\verb${amsmath}$ statement.
%
% We require version 1.2 (datestamped 1997/03/20) or later.
% The \expandafter\cmd\csname ver@amsmath.sty\endcsname, will 
% 
% \LaTeX\ note:
% Certain \LaTeX\ procedures have an arbitrary and pointless restriction that they may be used only within the preamble.
% We get around this by preserving the procedures in private \cmd\csname s.
%
% FIXME note: it is difficult to ensure that an error summary will be printed on the console at the very end,
% but \classname{ltxgrid} allows accomplishing this via an interrupt, put down at \cmd\AtEndDocument\ time.
%
%    \begin{macrocode}
\DeclareOption{amsmath}{%
  \def\class@amsmath{\RequirePackage{amsmath}[\ver@amsmath@prefer]}%
}%
\DeclareOption{noamsmath}{%
  \let\class@amsmath\@empty
}%
\appdef\rtx@require@packages{%
 \preserve@LaTeX
 \@ifxundefined\class@amsmath{}{\class@amsmath}%
 \appdef\class@enddocumenthook{\test@amsmath@ver}%
}%
\appdef\preserve@LaTeX{%
 \let\@ifl@aded@LaTeX\@ifl@aded
 \let\@ifpackageloaded@LaTeX\@ifpackageloaded
 \let\@pkgextension@LaTeX\@pkgextension
 \let\@ifpackagelater@LaTeX\@ifpackagelater
 \let\@ifl@ter@LaTeX\@ifl@ter
 \let\@ifl@t@r@LaTeX\@ifl@t@r
 \let\@parse@version@LaTeX\@parse@version
}%
\appdef\restore@LaTeX{%
 \let\@ifl@aded\@ifl@aded@LaTeX
 \let\@ifpackageloaded\@ifpackageloaded@LaTeX
 \let\@pkgextension\@pkgextension@LaTeX
 \let\@ifpackagelater\@ifpackagelater@LaTeX
 \let\@ifl@ter\@ifl@ter@LaTeX
 \let\@ifl@t@r\@ifl@t@r@LaTeX
 \let\@parse@version\@parse@version@LaTeX
}%
\def\test@amsmath@ver{%
 \begingroup
  \restore@LaTeX
  \@ifpackageloaded{amsmath}{%
   \@ifpackagelater{amsmath}{\ver@amsmath@prefer}{}{%
     \class@warn{%
      You have loaded amsmath, version "\csname ver@amsmath.sty\endcsname",\MessageBreak
      but this class requires version "\ver@amsmath@prefer", or later.\MessageBreak
      Please update your LaTeX installation.
     }%
   }%
  }{%
  }%
 \endgroup
}%
\def\ver@amsmath@prefer{2000/01/15 v2.05 AMS math features}%
%    \end{macrocode}
% \end{macro}
%
% \subsubsection{Presenting Authors and Their Affiliations}%
% \changes{4.1b}{2008/07/01}{Code relating to new syntax for frontmatter has been placed in \file{ltxfront.dtx}}
% Class options for presenting authors and their affiliations are now defined in 
% \file{ltxfront.dtx}.
%
%    \begin{macrocode}
%    \end{macrocode}
%
% \subsubsection{Typeset by \revtex}%
%
% \begin{macro}{\byrevtex@sw}
% \changes{4.1b}{2008/07/01}{(AO) make settings at class time instead of deferring them to later.}
% The flag \cmd\byrevtex@sw\ signifies that the document should
% bear an imprint to the effect that it was formatted by this document class.
%
% The class option \classoption{byrevtex} signifies that you want the 
% ``Typeset by \revtex'' byline to appear on your formatted output.
% By default, no such byline appears.
%    \begin{macrocode}
\DeclareOption{byrevtex}{\@booleantrue\byrevtex@sw}%
\@booleanfalse\byrevtex@sw
%    \end{macrocode}
% \end{macro}
%
% \subsection{Attempt to fix float placement failure}%
% 
% \begin{macro}{\force@deferlist@sw}
% \changes{4.1b}{2008/07/01}{(AO) make settings at class time instead of deferring them to later.}
% \revtex\ uses the \classname{ltxgrid} package, which provides the ability to attempt repairs when 
% \LaTeX's float placement mechanism is about to fail, but that facility is turned off by default.
% Users should invoke the \classoption{floatfix} document class option to enable this \LaTeX\ extension.
% If not, a helpful message is printed in the log, indicating how to work around the difficulty.
%    \begin{macrocode}
\DeclareOption{floatfix}{\@booleantrue\force@deferlist@sw}%
\DeclareOption{nofloatfix}{\@booleanfalse\force@deferlist@sw}%
\@booleanfalse\force@deferlist@sw
%    \end{macrocode}
% \end{macro}
%
% \begin{macro}{\@fltovf}
% \begin{macro}{\@fltstk}
%  The \LaTeX\ kernel error message \cmd\@fltovf\ may now be a bit more helpful to the user;
%  likewise for the \cmd\@fltstk\ message of \classname{ltxgrid}.
%    \begin{macrocode}
\gdef\@fltovf{%
 \@latex@error{%
  Too many unprocessed floats%
  \force@deferlist@sw{}{; try class option [floatfix]}%
 }\@ehb
}%
\def\@fltstk{%
 \@latex@warning{%
  A float is stuck (cannot be placed)%
  \force@deferlist@sw{}{; try class option [floatfix]}%
 }%
}%
%    \end{macrocode}
% \end{macro}
% \end{macro}
%
% \begin{macro}{\ltxgrid@info@sw}
% \begin{macro}{\outputdebug@sw}
% The two options \classoption{ltxgridinfo} and \classoption{outputdebug} turn on informative diagnostics within the 
% package \classname{ltxgrid}. Only people who really want to see this output will select these class options.
% Consult documentation for the \classname{ltxgrid} package to see what output the related switches enable.
%    \begin{macrocode}
\DeclareOption{ltxgridinfo}{%
 \@booleantrue\ltxgrid@info@sw
%\@booleantrue\ltxgrid@foot@info@sw
}%
\DeclareOption{outputdebug}{%
 \@booleantrue\outputdebug@sw
 \@booleantrue\ltxgrid@info@sw
 \@booleantrue\ltxgrid@foot@info@sw
 \traceoutput
}%
%    \end{macrocode}
% \end{macro}
% \end{macro}
%
% \subsection{Option to relax page height}%
% 
% \begin{macro}{\textheight@sw}
% The \classname{ltxgrid} package can set text pages to their natural height or force them to the full 
% text height; the latter is the default.
% If setting the pages with a variable length,
% the running foot will move up or down with the natural length of the text column.
% While I recommend against doing so, this option will turn that switch to the latter setting.
%    \begin{macrocode}
\DeclareOption{raggedfooter}{\@booleanfalse\textheight@sw}%
\DeclareOption{noraggedfooter}{\@booleantrue\textheight@sw}%
%    \end{macrocode}
% \end{macro}
%
% \subsection{Selecting procedure for processing abstract}%
% \changes{4.1b}{2008/07/01}{Remove options newabstract and oldabstract}
% Code defining options \classoption{newabstract} and \classoption{oldabstract} has been removed.
%
% \subsection{Option to turn on diagnostics in the frontmatter}%
% 
% \begin{macro}{\frontmatterverbose@sw}
% \changes{4.1b}{2008/07/01}{Code relating to new syntax for frontmatter has been placed in \file{ltxfront.dtx}}
% A diagnostic option, not for the average enduser, which reveals 
% the workings of the frontmatter.
% This code interfaces to that of \file{ltxfront.dtx}.
%    \begin{macrocode}
\DeclareOption{frontmatterverbose}{\@booleantrue\frontmatterverbose@sw}%
\@booleanfalse\frontmatterverbose@sw
%    \end{macrocode}
% \end{macro}
% 
% \begin{macro}{\linenumbers@sw}
% \changes{4.1f}{2009/07/07}{(AO, 513) Add class option linenumbers: number the lines a la \classname{lineno}}
% \changes{4.1n}{2009/12/13}{(AO, 573) arrange to load \classname{lineno} after any other packages.}%
% An option to number the lines of type in the output in the manner of \classname{lineno}.
%
% At present, we use that very package to implement this functionality.
% This means that users may modify the workings of that package per its documentation (which see).
%
% However, compatibility with \classname{amsmath} requires that \classname{lineno} be loaded afterwards.
% Therefore, we defer loading of this package until after the preamble is completed. 
%    \begin{macrocode}
\DeclareOption{linenumbers}{%
 \appdef
 \class@documenthook{%
  \RequirePackage{lineno}[2005/11/02 v4.41]%
  \linenumbersep4pt\relax
  \linenumbers\relax
 }%
}%
%    \end{macrocode}
% By default, line numbering is off. 
% \end{macro}
% 
% \begin{macro}{\NAT@merge}
% \changes{4.1f}{2009/07/15}{(AO, 523) Add class option nomerge, to turn off new natbib 8.3 syntax}
% Add class option nomerge, to turn off natbib 8.3 syntax for citation key. 
% The default value of REVTeX 4.1 for \cmd\NAT@merge is \cmd\thr@@, which turns on  
% the new syntax along with its semantics. 
% Legacy documents that would be incompatible with the new syntax can be successfully processed 
% with class option \classoption{nomerge}.
%    \begin{macrocode}
\DeclareOption{nomerge}{%
 \appdef\setup@hook{%
  \@ifnum{\NAT@merge>\z@}{\let\NAT@merge\z@}{}%
 }%
}%
%    \end{macrocode}
% \end{macro}
%
% \subsection{Default Option, Society, Journal, and pointsize}%
%
% \changes{4.1a}{2008/01/18}{(AO, 488) Change processing of options to allow an unused option to specify society and journal}%
% This change will not break OSA documents because that society is still built in to revtex4.
%
% \begin{macro}{\@parse@class@options@society}
% \begin{macro}{\@parse@class@options@}
% \changes{4.1b}{2008/07/01}{Procedures \cs{@parse@class@options@society} and \cs{@parse@class@options@journal} and friends}
% The procedure \cmd\@parse@class@options@society\ parses the options passed to this document class 
% for the \cmd\@society. 
% It is like \cmd\ProcessOptions*\ in that it accesses
% \cmd\@ptionlist\{\cmd\@currname.\cmd\@currext\}.
% Any undefined option is considered: if there is a corresponding \file{.rtx} file,
% it will change the society accordingly
% and define a placeholder class option for the society thus found
% (thus preventing a spurious "option not found" message). 
% 
% The procedure \cmd\@parse@class@options@\ parses the document's options
% for any that set the \cmd\csname\ provided. 
%    \begin{macrocode}
\def\@parse@class@options@society{%
 \edef\@tempa{\@ptionlist{\@currname.\@currext}}%
 \expandafter\@for\expandafter\CurrentOption\expandafter:\expandafter=\@tempa\do{%
  \expandafter\@ifnotrelax\csname ds@\CurrentOption\endcsname{}{%
   \IfFileExists{\CurrentOption\substyle@post.\substyle@ext}{%
    \expandafter\change@society\expandafter{\CurrentOption}%
    \expandafter\let\csname ds@\CurrentOption\endcsname\@empty
   }{}%
  }%
 }%
}%
\def\@parse@class@options@#1{%
 \edef\@tempa{\@ptionlist{\@currname.\@currext}}%
 \expandafter\@for\expandafter\CurrentOption\expandafter:\expandafter=\@tempa\do{%
  \expandafter\@ifnotrelax\csname ds@\CurrentOption\endcsname{%
   \begingroup\csname ds@\CurrentOption\endcsname
    \@ifxundefined#1{%
     \endgroup
    }{%
     \expandafter\endgroup\expandafter\def\expandafter#1\expandafter{#1}%
    }%
  }{}%
 }%
}%
\def\@parse@class@options@journal{%
 \edef\@tempa{\@ptionlist{\@currname.\@currext}}%
 \expandafter\@for\expandafter\CurrentOption\expandafter:\expandafter=\@tempa\do{%
  \expandafter\@ifnotrelax\csname ds@\CurrentOption\endcsname{%
   \begingroup
    \csname ds@\CurrentOption\endcsname
    \@ifxundefined\@journal{%
     \endgroup
    }{%
     \expandafter\endgroup\expandafter\def\expandafter\@journal\expandafter{\@journal}%
    }%
  }{}%
 }%
}%
\def\@parse@class@options{%
 \edef\@tempa{\@ptionlist{\@currname.\@currext}}%
 \expandafter\@for\expandafter\CurrentOption\expandafter:\expandafter=\@tempa\do{%
  \expandafter\@ifnotrelax\csname ds@\CurrentOption\endcsname{%
   \begingroup
    \csname ds@\CurrentOption\endcsname
    \@ifxundefined\@pointsize{%
     \endgroup
    }{%
     \expandafter\endgroup\expandafter\def\expandafter\@pointsize\expandafter{\@pointsize}%
    }%
  }{%
   \IfFileExists{\CurrentOption\substyle@post.\substyle@ext}{%
    \expandafter\change@society\expandafter{\CurrentOption}%
    \expandafter\let\csname ds@\CurrentOption\endcsname\@empty
   }{}%
  }%
 }%
}%
%    \end{macrocode}
% \end{macro}
% \end{macro}
%
% \changes{4.1j}{2009/10/25}{(AO, 545) hypertext capabilities off by default; enable with \classoption{hypertext}}
% The class option \classoption{hypertext} enables the built-in hypertext capabilities,
% which coincide with those of \classname{custom-bib}-generated BIBTeX styles
% using the guard code \texttt{hypertext}. 
%
% Note that APS has these capabilities turned off by default;
% Loading the hyperref package turns them on.
%    \begin{macrocode}
\DeclareOption{hypertext}{\hypertext@enable@ltx}%
\appdef\document@inithook{\@ifpackageloaded{hyperref}{\hypertext@enable@ltx}{}}%
%    \end{macrocode}
%
% The default handling for a document class option depends upon whether the \cmd\@society\ is 
% defined. 
%
% If not, then hunt for a \file{.rtx} file with that name.
% If it exists, then we will take this option as the name of the society, 
% otherwise, declare the option as not used.
% 
% (This behavior is similar to the \LaTeX2.09 handling, where
% one looked for a \file{.sty} file, except that 
% in this case, we must provide for journal substyles that may be 
% defined in the society file, or have their own journal substyle file.)
%
% At the point where the class file is finished loading, 
% we then read in the society file. 
% That file can define further class options, such as the journal substyle.
%
% For users, this will mean that they can specify the society and journal simply 
% by specifying first the former and then the latter among their document class options.
% The society \emph{must} have a corresponding \file{.rtx}.
%
% \changes{4.1b}{2008/05/29}{The csname substyle@ext is now defined without a dot (.), to be compatible with \LaTeX usage (see @clsextension and @pkgextension).}
%
%    \begin{macrocode}
%</options>
%    \end{macrocode}
%
%    \begin{macrocode}
%<*package>
%    \end{macrocode}
%
% \begin{macro}{\@process@society}
% \begin{macro}{\@process@journal}
% \begin{macro}{\@process@pointsize}
%    \begin{macrocode}
\def\substyle@post{4-1}%
\def\substyle@ext{rtx}%
\DeclareOption*{\OptionNotUsed}%
%    \end{macrocode}
% A society substyle may define its own options, via \cmd\DeclareOption. 
%
% At the end of this document class, we process the society file,
% using \file{aps.rtx} if none has been specified in the document.
%    \begin{macrocode}
\def\@process@society#1{%
 \@ifxundefined\@society{%
  \class@warn{No Society specified, using default society #1}%
  \def\@society{#1}\let\@journal\@undefined
 }{}%
 \expandafter\input\expandafter{\@society\substyle@post.\substyle@ext}%
}%
%    \end{macrocode}
%
% A society substyle can encompass any number of journal substyles;
% we use the following procedure to invoke the proper one.
%    \begin{macrocode}
\def\@process@journal#1{%
 \@ifxundefined\@journal{%
  \class@warn{No journal specified, using default #1}%
  \def\@journal{#1}%
 }{}%
 \expandafter\expandafter
 \expandafter\rtx@do@substyle
 \expandafter\expandafter
 \expandafter{\expandafter\@society\@journal}%
}%
\def\rtx@do@substyle#1{%
 \InputIfFileExists{#1\substyle@post.\substyle@ext}{}{\csname rtx@#1\endcsname}%
}%
%    \end{macrocode}
% 
% Document class options \classoption{10pt}, \classoption{11pt}, and \classoption{12pt} 
% are implemented by \revtex\ itself and determine \cmd\@pointsize. 
% These provide formatting settings appropriate to the society's journals. 
%
% If not specified by the document, a value \cmd\@pointsize@default\ is used. 
% This default can be set by the journal. Here, the society sets its default. 
%    \begin{macrocode}
\def\@process@pointsize#1{%
 \@ifxundefined\@pointsize{%
  \def\@pointsize{#1}%
  \class@warn{No type size specified, using default \@pointsize}%
 }{}%
 \expandafter\expandafter
 \expandafter\rtx@do@substyle
 \expandafter\expandafter
 \expandafter{\expandafter\@society\@pointsize pt}%
}%
%    \end{macrocode}
% \end{macro}
% \end{macro}
% \end{macro}
%
% \subsection{Class-Asserted Options}%
% \changes{4.0b}{1999/06/20}{Removed invocation of nonexistent class option \protect\classoption{groupauthors} and all other class options that should only be invoked by the document. (Otherwise precedence of class options does not work.)}
% Here we establish the default document class options.
% Those of the document itself will override these.
%
% \section{Procedures Dependent Upon Options}
%
% Here we introduce \classname{classes.dtx} definitions for 
% the page styles that people will expect to be able to use.
% \begin{macro}{\ps@headings}
% \begin{macro}{\ps@myheadings}
%    \begin{macrocode}
  \def\ps@headings{%
      \let\@oddfoot\@empty\let\@evenfoot\@empty
      \def\@evenhead{\thepage\hfil\slshape\leftmark}%
      \def\@oddhead{{\slshape\rightmark}\hfil\thepage}%
      \let\@mkboth\markboth
    \def\sectionmark##1{%
      \markboth {\MakeUppercase{%
        \ifnum \c@secnumdepth >\z@
          \thesection\quad
        \fi
        ##1}}{}}%
    \def\subsectionmark##1{%
      \markright {%
        \ifnum \c@secnumdepth >\@ne
          \thesubsection\quad
        \fi
        ##1}}}%
\def\ps@myheadings{%
    \let\@oddfoot\@empty\let\@evenfoot\@empty
    \def\@evenhead{\thepage\hfil\slshape\leftmark}%
    \def\@oddhead{{\slshape\rightmark}\hfil\thepage}%
    \let\@mkboth\@gobbletwo
    \let\sectionmark\@gobble
    \let\subsectionmark\@gobble
    }%
%    \end{macrocode}
% \end{macro}
% \end{macro}
%
% \begin{macro}{\ps@article}
% \begin{macro}{\ps@article@final}
% \begin{macro}{\ps@preprint}
% \changes{4.0a}{1998/01/31}{Move after process options, so the following test works}
% \changes{4.0b}{1999/06/20}{Defer decision until \cs{AtBeginDocument} time}
% \changes{4.0c}{1999/11/13}{Insert procedure \cs{checkindate}}
% \changes{4.0rc1}{2001/06/17}{Running headers always as if two-sided}
%    \begin{macrocode}
\def\ps@article{%
    \def\@evenhead{\let\\\heading@cr\thepage\quad\checkindate\hfil{\leftmark}}%
    \def\@oddhead{\let\\\heading@cr{\rightmark}\hfil\checkindate\quad\thepage}%
    \def\@oddfoot{}%
    \def\@evenfoot{}%
    \let\@mkboth\markboth
  \let\sectionmark\@gobble
  \let\subsectionmark\@gobble
}%
\def\ps@article@final{%
    \def\@evenhead{\let\\\heading@cr\thepage\quad\checkindate\hfil{\leftmark}}%
    \def\@oddhead{\let\\\heading@cr{\rightmark}\hfil\checkindate\quad\thepage}%
    \def\@oddfoot{}%
    \def\@evenfoot{}%
    \let\@mkboth\markboth
    \def\sectionmark##1{%
      \markboth{%
       \MakeTextUppercase{%
        \@ifnum{\c@secnumdepth >\z@}{\thesection\hskip 1em\relax}{}%
         ##1%
       }%
       }{}%
    }%
    \def\subsectionmark##1{%
      \markright {%
        \@ifnum{\c@secnumdepth >\@ne}{\thesubsection\hskip 1em\relax}{}%
         ##1%
      }%
    }%
}%
%    \end{macrocode}
%
%    \begin{macrocode}
\def\heading@cr{\unskip\space\ignorespaces}%
%    \end{macrocode}
%
% \changes{4.0c}{1999/11/13}{Do not put by REVTeX in every page foot}
% \changes{4.0c}{1999/11/13}{Insert procedure \cs{checkindate}}
%    \begin{macrocode}
\def\ps@preprint{%
  \def\@oddfoot{\hfil\thepage\quad\checkindate\hfil}%
  \def\@evenfoot{\hfil\thepage\quad\checkindate\hfil}%
  \def\@oddhead{}%
  \def\@evenhead{}%
  \let\@mkboth\@gobbletwo
  \let\sectionmark\@gobble
  \let\subsectionmark\@gobble
}%
%    \end{macrocode}
%
%    \begin{macrocode}
\let\@oddhead\@empty
\let\@evenhead\@empty
\let\@oddfoot\@empty
\let\@evenfoot\@empty
%    \end{macrocode}
% \end{macro}
% \end{macro}
% \end{macro}
%
% \begin{macro}{\lastpage@putlabel}
% Support the default meaning of \cmd\@endpage.
% \changes{4.0a}{1998/01/31}{Move after process options, so \cs{clearpage} not in scope of twocolumn}
% \changes{4.1k}{2009/11/06}{(AO, 554) give the \cs{newlabel} command syntax appropriate to the hyperref package}
% Name of this macro (and the \cmd\label\ key) taken from 
% \url{CTAN:/macros/latex/contrib/other/lastpage}%
% with code optimised slightly.
%    \begin{macrocode}
\def\lastpage@putlabel{%
 \if@filesw
  \begingroup
    \advance\c@page\m@ne
    \immediate\write\@auxout{\string\newlabel{LastPage}{{}{\thepage}{}{}{}}}%
  \endgroup
 \fi
}%
%    \end{macrocode}
% Install a procedure into document endgame processing that 
% labels the last page of the document.
% This is done just before the \file{.aux} file
% is closed, and does not require a \cmd\shipout, because it writes
% directly to the \file{.aux} file.
% Note that we assume no further \cmd\shipout s will be done past this point.
% \changes{4.1n}{2009/12/09}{(AO, 569) execute the after-last-shipout procedures from within the safety of the output routine}%
%    \begin{macrocode}
\appdef\clear@document{%
 \do@output@cclv{%
  \lastpage@putlabel
  \tally@box@size@sw{\total@text}{}%
 }%
}%
\providecommand\write@column@totals{}%
%    \end{macrocode}
% \end{macro}
%
%
% \section{Required Packages}
% \url{CTAN:macros/latex/contrib/other/misc/url.sty}%
%    \begin{macrocode}
\appdef\rtx@require@packages{%
 \RequirePackage{url}%
}%
%    \end{macrocode}
%
% \section{Incompatible Packages}
%
% We wait until after the preamble is processed, then check for any packages 
% that might have been loaded which we know to be incompatible with \revtex.
%
% The \classname{multicol} package is incompatible with \classname{ltxgrid}, which replaces it.
% The \classname{cite} package is incompatible with \classname{natbib}, which replaces its functionality.
% The functionality of the \classname{mcite} package is provided by \classname{natbib}. 
%    \begin{macrocode}
\appdef\document@inithook{%
 \incompatible@package{cite}%
 \incompatible@package{mcite}%
 \incompatible@package{multicol}%
}%
%    \end{macrocode}
%
% \section{Society- and Journal-Specific Code}
%
% \begin{macro}{\@journal}
% Journal test helper, used as
% \begin{verbatim}
%\@ifx{\@journal\journal@pra}{%
%  <journal-specific setup>
%\fi
% \end{verbatim}
% Journal code might like to further specify (if as yet undefined) or distinguish on
% the following Booleans.
%
% Note: the journal substyle code should only alter the value of one of these Booleans if
% the Boolean is \cmd\undefined. This convention is what makes the document's
% options take precedence over the values set by the journal.
% 
% FIXME: make this table an exhaustive listing of all the parameters set by
% the class options.
%
% \begin{tabular}{ll}
% \cmd\@pointsize      &(10|11|12), depending on the type size\\
% \cmd\footinbib@sw    &true if footnotes are to be formatted in the bibliography\\
% \cmd\preprintsty@sw  &true for preprint and hyperpreprint\\
% \cmd\eqsecnum@sw     &true means that equations are numbered within sections\\
% \cmd\groupauthors@sw &true means authors listed separately for each address\\
% \cmd\preprint@sw     &true means to produce the preprint numbers as part of the title block\\
% \cmd\showPACS@sw     &true means to produce the PACS as part of the title block\\
% \cmd\showKEYS@sw     &true means to produce the keywords as part of the title block\\
% \cmd\@affils@sw      &true means each affiliation is printed, for each author\\
% \cmd\runinaddress@sw &true means author addresses are printed run-in\\
% \cmd\draft@sw        &true implies that PACS will be printed\\
% \cmd\tightenlines@sw &true if preprint single spaced\\
% \cmd\lengthcheck@sw  &true if length checking is in effect\\
% \cmd\byrevtex@sw     &true means to announce ``typeset by \revtex''\\
% \cmd\titlepage@sw    &true for title is to be set on a separate page\\
% \cmd\twocolumn@sw    &true if two-column page grid\\
% \cmd\twocolumn@sw    &true if we are to automatically balance the columns of the last page\\
% \cmd\twoside@sw      &true means to format pages for duplex printing\\
% \cmd\floats@sw       &false means floats are migrated to end of document\\
% \cmd\floatp@sw       &true means endfloats are set one to a page\\
% \cmd\class@amsfonts  &if \cmd\@empty, means that \classname{amsfonts} will \emph{not} be loaded\\
% \cmd\class@amssymb   &if \cmd\@empty, means that \classname{amssymb} will \emph{not} be loaded\\
% \cmd\frontmatter@footnote &if \cmd\undefined, means that the default (\cmd\footnote) will be used\\
% \cmd\place@bibnumber &if \cmd\undefined, means that the default (inline) will be used\\
% \end{tabular}
%
% Note: if \cmd\twocolumn@sw\ and \cmd\preprintsty@sw\ are both false,
% then `galley' style is in effect. The \classoption{galley} option invokes
% \classoption{onecolumn}, but does not affect the \cmd\preprintsty@sw.
%
% Note: \cmd\paperwidth\ and \cmd\paperheight\ are not integrated into this scheme,
% and should be selected by the document alone.
% \end{macro}
%
% \section{Body}
%
% \subsection{counters}
% The following definitions are probably identical to those in 
% \file{classes.dtx}%
%    \begin{macrocode}
\def\labelenumi{\theenumi.}
\def\theenumi{\arabic{enumi}}
%    \end{macrocode}
%
%    \begin{macrocode}
\def\labelenumii{(\theenumii)}
\def\theenumii{\alph{enumii}}
\def\p@enumii{\theenumi}
%    \end{macrocode}
%
%    \begin{macrocode}
\def\labelenumiii{\theenumiii.}
\def\theenumiii{\roman{enumiii}}
\def\p@enumiii{\theenumi(\theenumii)}
%    \end{macrocode}
%
%    \begin{macrocode}
\def\labelenumiv{\theenumiv.}
\def\theenumiv{\Alph{enumiv}}
\def\p@enumiv{\p@enumiii\theenumiii}
%    \end{macrocode}
%
%    \begin{macrocode}
\def\labelitemi{\textbullet}
\def\labelitemii{\normalfont\bfseries\textendash}
\def\labelitemiii{\textasteriskcentered}
\def\labelitemiv{\textperiodcentered}
%    \end{macrocode}
%
%    \begin{macrocode}
\pagenumbering{arabic}
%    \end{macrocode}
%
% \subsection{float parameters}
%
% from the old aps.sty. (DPC: same as article I think)
% AO: here, \LaTeX's standard classes fail very poorly
% (the price of backward compatability): the values for
% \cmd\floatpagefraction\ and \cmd\dblfloatpagefraction\
% need to be raised to avoid creating extremely short
% float pages.
%    \begin{macrocode}
\setcounter{topnumber}{2}
\def\topfraction{.9}
\setcounter{bottomnumber}{1}
\def\bottomfraction{.9}
\setcounter{totalnumber}{3}
\def\textfraction{.1}
\def\floatpagefraction{.9}
\setcounter{dbltopnumber}{2}
\def\dbltopfraction{.9}
\def\dblfloatpagefraction{.9}
%    \end{macrocode}
%
% \subsection{List Environments}
%
%    \begin{macrocode}
\newenvironment{verse}{%
  \let\\=\@centercr
  \list{}{%
    \itemsep\z@ \itemindent -1.5em\listparindent \itemindent
    \rightmargin\leftmargin\advance\leftmargin 1.5em}\item[]%
}{%
 \endlist
}%
%    \end{macrocode}
%
%    \begin{macrocode}
\newenvironment{quotation}{%
  \list{}{%
    \listparindent 1.5em
    \itemindent\listparindent
    \rightmargin\leftmargin \parsep \z@ \@plus\p@}\item[]%
}{%
 \endlist
}%
%    \end{macrocode}
%
%    \begin{macrocode}
\newenvironment{quote}{%
  \list{}{%
  \rightmargin\leftmargin}\item[]%
}{%
 \endlist
}%
%    \end{macrocode}
%
% \changes{4.1b}{2008/06/30}{(AO) Structure the Abstract using the \texttt{bibliography} environment}
%    \begin{macrocode}
\def\descriptionlabel#1{%
  \hspace\labelsep \normalfont\bfseries #1\unskip:%
}%
%    \end{macrocode}
%
%    \begin{macrocode}
\newenvironment{description}{%
 \list{}{%
    \labelwidth\z@ \itemindent-\leftmargin
    \let\makelabel\descriptionlabel
 }%
}{%
 \endlist
}%
%    \end{macrocode}
%
% \subsection{Sectioning Commands\label{sec:secnumarabic}}
%
% \subsubsection{Sectioning Commands and Their Productions}%
% The following counters are defined by LaTeX's standard document classes.
% We do likewise, then assign flag values to the productions, awaiting
% overrides.
%    \begin{macrocode}
\newcounter{part}%
\let\thepart\@undefined
\newcounter{section}%
\let\thesection\@undefined
\newcounter{subsection}[section]%
\let\thesubsection\@undefined
\newcounter{subsubsection}[subsection]%
\let\thesubsubsection\@undefined
\newcounter{paragraph}[subsubsection]%
\let\theparagraph\@undefined
\newcounter{subparagraph}[paragraph]%
\let\thesubparagraph\@undefined
%    \end{macrocode}
% The procedure invoked by \cmd\setup@secnums\ provides meanings for these productions. 
% 
% \begin{macro}{\secnums@rtx}
% \begin{macro}{\secnums@arabic}
% These two procedures define the meanings of each of the productions
% of the counters of the sectioning commands,
% but only if nothing else has defined it.
%    \begin{macrocode}
\def\secnums@rtx{%
 \@ifxundefined\thepart{%
  \def\thepart{\Roman{part}}%
 }{}%
 \@ifxundefined\thesection{%
  \def\thesection       {\Roman{section}}%
  \def\p@section        {}%
 }{}%
 \@ifxundefined\thesubsection{%
  \def\thesubsection    {\Alph{subsection}}%
  \def\p@subsection     {\thesection\,}%
 }{}%
 \@ifxundefined\thesubsubsection{%
  \def\thesubsubsection {\arabic{subsubsection}}%
  \def\p@subsubsection  {\thesection\,\thesubsection\,}%
 }{}%
 \@ifxundefined\theparagraph{%
  \def\theparagraph     {\alph{paragraph}}%
  \def\p@paragraph      {\thesection\,\thesubsection\,\thesubsubsection\,}%
 }{}%
 \@ifxundefined\thesubparagraph{%
  \def\thesubparagraph  {\arabic{subparagraph}}%
  \def\p@subparagraph   {\thesection\,\thesubsection\,\thesubsubsection\,\theparagraph\,}%
 }{}%
}%
\def\secnums@arabic{%
 \@ifxundefined\thepart{%
  \def\thepart          {\Roman{part}}%
 }{}%
 \@ifxundefined\thesection{%
  \def\thesection       {\Roman{section}}%
  \def\p@section        {}%
 }{}%
 \@ifxundefined\thesubsection{%
  \def\thesubsection    {\thesection.\arabic{subsection}}%
  \def\p@subsection     {}%
 }{}%
 \@ifxundefined\thesubsubsection{%
  \def\thesubsubsection {\thesubsection.\arabic{subsubsection}}%
  \def\p@subsubsection  {}%
 }{}%
 \@ifxundefined\theparagraph{%
  \def\theparagraph     {\thesubsubsection.\arabic{paragraph}}%
  \def\p@paragraph      {}%
 }{}%
 \@ifxundefined\thesubparagraph{%
  \def\thesubparagraph  {\theparagraph.\arabic{subparagraph}}%
  \def\p@subparagraph   {}%
 }{}%
}%
%    \end{macrocode}
% \end{macro}
% \end{macro}
%
% \subsubsection{The Acknowledgments Environment}%
% This user-level markup produces a head introducing 
% the acknowledgments, and acts as a wrapper for the text.
% In this implementation, it is an unnumbered section,
% but appears within the toc.
%
% For compatiability's sake, we implement it under the alternative
% spelling \env{acknowledgements}.
% \changes{4.0c}{1999/11/13}{also spelled ``acknowledgements''.}
% \changes{4.0d}{2000/05/18}{But alternative spelling is deprecated.}
%    \begin{macrocode}
\newenvironment{acknowledgments}{%
 \acknowledgments@sw{%
  \expandafter\section\expandafter*\expandafter{\acknowledgmentsname}%
 }{%
  \par
  \phantomsection
  \addcontentsline{toc}{section}{\protect\numberline{}\acknowledgmentsname}%
 }%
}{%
 \par
}%
\@booleantrue\acknowledgments@sw
\newenvironment{acknowledgements}{%
 \replace@environment{acknowledgements}{acknowledgments}%
}{%
 \endacknowledgments
}%
%    \end{macrocode}
%
% \subsubsection{Part Opener}%
% section setup copied verbatim from revtex3 aps/osa.
% Does not explicitly depend on pointsize options.
%    \begin{macrocode}
\def\part{\par
   \addvspace{4ex}%
   \@afterindentfalse
   \secdef\@part\@spart}%
%    \end{macrocode}
%
%    \begin{macrocode}
\def\@part[#1]#2{%
 \@ifnum{\c@secnumdepth >\m@ne}{%
        \refstepcounter{part}%
        \addcontentsline{toc}{part}{\thepart\hspace{1em}#1}%
 }{%
      \addcontentsline{toc}{part}{#1}%
 }%
 \begingroup
    \parindent \z@ \raggedright
    \interlinepenalty\@M
    \@ifnum{\c@secnumdepth >\m@ne}{%
      \Large \bf \partname~\thepart%
      \par\nobreak
    }{}%
    \huge \bf
    #2%
    \markboth{}{}\par
 \endgroup
   \nobreak
   \vskip 3ex
   \@afterheading
}%
%    \end{macrocode}
%
%    \begin{macrocode}
\def\@spart#1{{\parindent \z@ \raggedright
    \interlinepenalty\@M
    \huge \bf
    #1\par}
    \nobreak
    \vskip 3ex
    \@afterheading}
%    \end{macrocode}
%
% \subsubsection{Stacked Heads}%
% Here are the class default definitions for sectioning commands.
% A society or a journal substyle will likely override these definitions.
%
% In doing so, you can customize the formatting for a particular level by defining,
% e.g., \cmd\@hangfrom@section\ or \cmd\@subsectioncntformat.
%    \begin{macrocode}
\def\section{%
  \@startsection
    {section}%
    {1}%
    {\z@}%
    {0.8cm \@plus1ex \@minus .2ex}%
    {0.5cm}%
    {\normalfont\small\bfseries}%
}%
%    \end{macrocode}
%
%    \begin{macrocode}
\def\subsection{%
  \@startsection
    {subsection}%
    {2}%
    {\z@}%
    {.8cm \@plus1ex \@minus .2ex}%
    {.5cm}%
    {\normalfont\small\bfseries}%
}%
%    \end{macrocode}
%
%    \begin{macrocode}
\def\subsubsection{%
  \@startsection
    {subsubsection}%
    {3}%
    {\z@}%
    {.8cm \@plus1ex \@minus .2ex}%
    {.5cm}%
    {\normalfont\small\itshape}%
}%
%    \end{macrocode}
%
% \subsubsection{Runin Heads}%
%    \begin{macrocode}
\def\paragraph{%
  \@startsection
    {paragraph}%
    {4}%
    {\parindent}%
    {\z@}%
    {-1em}%
    {\normalfont\normalsize\itshape}%
}%
%    \end{macrocode}
%
%    \begin{macrocode}
\def\subparagraph{%
  \@startsection
    {subparagraph}%
    {5}%
    {\parindent}%
    {3.25ex \@plus1ex \@minus .2ex}%
    {-1em}%
    {\normalfont\normalsize\bfseries}%
}%
%    \end{macrocode}
%
%
% \subsection{Math}
%
% \begin{macro}{\theequation}
% We change the production of the equation counter so that 
% we can accomodate the \classoption{eqsecnum} option.
%    \begin{macrocode}
\def\theequation{%
 \theequation@prefix\arabic{equation}%
}%
\def\theequation@prefix{}%
%    \end{macrocode}
% \end{macro}
%
%
% \subsection{Type Size-Dependent Settings}
%
% \subsection{All Point Sizes}
%    \begin{macrocode}
\setcounter{secnumdepth}{4}
%    \end{macrocode}
%
%    \begin{macrocode}
\lineskip 1pt
\normallineskip 1pt
\def\baselinestretch{1}%
%    \end{macrocode}
%
%    \begin{macrocode}
\@lowpenalty   51
\@medpenalty  151
\@highpenalty 301
%    \end{macrocode}
%
%    \begin{macrocode}
\@beginparpenalty -\@lowpenalty
\@endparpenalty   -\@lowpenalty
\@itempenalty     -\@lowpenalty
%    \end{macrocode}
%
%    \begin{macrocode}
\arraycolsep 3pt
\tabcolsep 2pt
\arrayrulewidth .4pt
\doublerulesep 2pt
%    \end{macrocode}
%
%    \begin{macrocode}
\skip\@mpfootins = 0pt
%    \end{macrocode}
%
%    \begin{macrocode}
\fboxsep  = 3.0pt
\fboxrule = 0.4pt
%    \end{macrocode}
%
% \subsection{Figures}
% \changes{4.0b}{1999/06/20}{Processing delayed to \cs{AtBeginDocument} time}
%
% \begin{environment}{figure}
% We define the \env{figure} environment.
% Later, we will horse around with its meaning
% in order to accomodate \cmd\floats@sw.
%    \begin{macrocode}
\newenvironment{figure}
               {\@float{figure}}
               {\end@float}
\newenvironment{figure*}
               {\@dblfloat{figure}}
               {\end@dblfloat}
%    \end{macrocode}
% \end{environment}
%
%    \begin{macrocode}
\def\listoffigures{\print@toc{lof}}%
\def\l@figure{\@dottedtocline{1}{1.5em}{2.3em}}%
%    \end{macrocode}
%
% \begin{macro}{\@makecaption}
% If caption is one line long, to be centered; if lines turn, then set justified.
%    \begin{macrocode}
\newlength\abovecaptionskip
\newlength\belowcaptionskip
\setlength\abovecaptionskip{10\p@}
\setlength\belowcaptionskip{2\p@}
%    \end{macrocode}
%
% \changes{4.0b}{1999/06/20}{One-line caption sets flush left.}
% \changes{4.0c}{1999/11/13}{Prevent an inner footnote from performing twice}
% \changes{4.1a}{2008/01/18}{(AO, 460) ``Proper style is "FIG. 1. ..." (no colon)''}%
% There is a hook \cmd\@caption@fignum@sep\ for determining 
% the separator following the float number, e.g., ``Fig.1''.
% Formerly, we had defined it to be ``: '', now the colon has been replace by a period (full stop). 
%    \begin{macrocode}
\long\def\@makecaption#1#2{%
  \par
% \nobreak
  \vskip\abovecaptionskip
  \begingroup
   \small\rmfamily
   \sbox\@tempboxa{%
    \let\\\heading@cr
    \@make@capt@title{#1}{#2}%
   }%
   \@ifdim{\wd\@tempboxa >\hsize}{%
    \begingroup
     \samepage
     \flushing
     \let\footnote\@footnotemark@gobble
     \@make@capt@title{#1}{#2}\par
    \endgroup
   }{%
     \global \@minipagefalse
     \hb@xt@\hsize{\hfil\unhbox\@tempboxa\hfil}%
   }%
  \endgroup
  \vskip\belowcaptionskip
}%
\def\@make@capt@title#1#2{%
 \@ifx@empty\float@link{\@firstofone}{\expandafter\href\expandafter{\float@link}}%
  {#1}\@caption@fignum@sep#2%
}%
\def\@footnotemark@gobble{%
 \@footnotemark
 \@ifnextchar[{\@gobble@opt@i}{\@gobble}%
}%
\def\@gobble@opt@i[#1]#2{}%
\def\@mpmakefntext#1{%
 \flushing
 \parindent=1em
 \noindent
 \hb@xt@1em{\hss\@makefnmark}%
 #1%
}%
\def\@caption@fignum@sep{. }%
\def\setfloatlink{\def\float@link}%
\let\float@link\@empty
%    \end{macrocode}
% \end{macro}
%
% \begin{macro}{\thefigure}
% The figure counter and float placement defaults.
%    \begin{macrocode}
\newcounter{figure}
\renewcommand \thefigure {\@arabic\c@figure}
%    \end{macrocode}
% Note that we give the `!' modifier by default.
% This is an effort to avoid the syndrome wherein
% a deferred float finds itself unqualified for 
% placement, thereby getting carried until \cmd\clearpage.
%    \begin{macrocode}
\def\fps@figure{tbp}
\def\ftype@figure{1}
\def\ext@figure{lof}
\def\fnum@figure{\figurename~\thefigure}
%    \end{macrocode}
% We allocate a box register for use in tallying the column inches of floats of this type.
%    \begin{macrocode}
\expandafter\newbox\csname fbox@\ftype@figure\endcsname
\expandafter\setbox\csname fbox@\ftype@figure\endcsname\hbox{}%
%    \end{macrocode}
% \end{macro}
%
% \subsubsection{Deferring \env{figure} Floats}%
%
% We determine if figures are to float or be deferred until \cmd\printfigures\ time. 
% If so, we open the stream that will receive the deferred document portions. 
%    \begin{macrocode}
\appdef\class@documenthook{%
 \do@if@floats{figure}{.fgx}%
}%
\appdef\class@enddocumenthook{%
 \printfigures\relax
}%
%    \end{macrocode}
% 
% \begin{macro}{\printfigures}
% The user-level command \cmd\printfigures\ determines
% where the figures are to appear in a document in which
% \cmd\floats@sw\ is false.
% If the user invokes the \classoption{endfloats} class option and fails to
% insert a \cmd\printfigures\ command, 
% the figures will be printed at the end of the document.
% If the command is given, but floats are not being deferred, 
% it amounts to a no-op.
% \changes{4.0b}{1999/06/20}{call \cs{print@floats}}
% \changes{4.0c}{1999/11/13}{*-form mandates pagebreak}
%    \begin{macrocode}
\newcommand\printfigures{%
 \@ifstar{\true@sw}{\floatp@sw{\true@sw}{\false@sw}}%
 {%
  \print@float{figure}{\oneapage}%
 }{%
  \print@float{figure}{}%
 }%
}%
%    \end{macrocode}
% \end{macro}
%
%
% \begin{macro}{\@xfloat@prep}
% We patch into the procedure \cmd\@xfloat@prep. 
% This patch applies to all floats (not \env{figure} alone)
% and makes the type center.
%    \begin{macrocode}
\appdef\@xfloat@prep{%
 \appdef\@parboxrestore{\centering}%
%\let\@makefnmark\@makefnmark@latex
}%
%    \end{macrocode}
% \end{macro}
%
%
% \subsection{Tables}
% \changes{4.0b}{1999/06/20}{Processing delayed to \cs{AtBeginDocument} time}
%
% DPC: More or less taken from \file{revtex2} \file{aps.sty},
% but using dcolumn for decimal alignment.
% \begin{environment}{table}
% We define the \env{table} environment.
% Later, we will horse around with its meaning
% in order to accomodate \cmd\floats@sw.
%    \begin{macrocode}
\newenvironment{table}
               {\@float{table}}
               {\end@float}
\newenvironment{table*}
               {\@dblfloat{table}}
               {\end@dblfloat}
%    \end{macrocode}
% \end{environment}
%
% \begin{macro}{\thetable}
% Table counter and default float placement declarations.
%    \begin{macrocode}
\newcounter{table}
\renewcommand\thetable{\@Roman\c@table}
%    \end{macrocode}
% Note that we give the `!' modifier by default.
% This is an effort to avoid the syndrome wherein
% a deferred float finds itself unqualified for 
% placement, thereby getting carried until \cmd\clearpage.
%    \begin{macrocode}
\def\fps@table{tbp}
\def\ftype@table{2}
\def\ext@table{lot}
\def\fnum@table{\tablename~\thetable}
%    \end{macrocode}
% We allocate a box register for use in tallying the column inches of floats of this type.
%    \begin{macrocode}
\expandafter\newbox\csname fbox@\ftype@table\endcsname
\expandafter\setbox\csname fbox@\ftype@table\endcsname\hbox{}%
%    \end{macrocode}
% \end{macro}
%
%    \begin{macrocode}
\def\listoftables{\print@toc{lot}}%
\let\l@table\l@figure
%    \end{macrocode}
%
% \begin{macro}{\table@hook}
% \begin{macro}{\squeezetable}
% Assign a meaning to the hook installed into float processing.
%
% By default floats are \cmd\small. The \cmd\squeezetable\ declaration
% makes them smaller (\cmd\scriptsize). In general you can locally
% redefine \cmd\table@hook\ to be whatever you like.
% (DPC: |\Huge\color{magenta}|\ldots?)
%    \begin{macrocode}
\def\table@hook{\small}%
\def\squeezetable{\def\table@hook{\scriptsize}}%
\appdef\@floatboxreset{\table@hook}%
%    \end{macrocode}
% \end{macro}
% \end{macro}
%
% \subsubsection{Deferring \env{table} Floats}%
%
% After all packages are loaded, we decide
% if tables will float or will be deferred until \cmd\printtables\ time.
%
% We also deal with the possibility of \env{longtable} environments.
% 
%    \begin{macrocode}
\def\set@table@environments{%
 \floats@sw{}{%
  \let@environment{longtable@float}{longtable}%
  \let@environment{longtable}{longtable@write}%
  \let@environment{longtable*@float}{longtable*}%
  \let@environment{longtable*}{longtable*@write}%
  \let@environment{turnpage@float}{turnpage}%
  \let@environment{turnpage}{turnpage@write}%
 }%
 \do@if@floats{table}{.tbx}%
}%
\appdef\document@inithook{%
 \set@table@environments
}%
\appdef\class@enddocumenthook{%
 \printtables\relax
}%
\newenvironment{longtable@write}{%
 \write@@float{longtable}{table}%
}{%
 \endwrite@float
}%
\newenvironment{longtable*@write}{%
 \write@@float{longtable*}{table}%
}{%
 \endwrite@float
}%
\newenvironment{turnpage@write}{%
 \immediate\write\tablewrite{\string\begin{turnpage}}%
}{%
 \immediate\write\tablewrite{\string\end{turnpage}}%
}%
%    \end{macrocode}
%
% \begin{macro}{\printtables}
% The user-level command \cmd\printtables\ determines
% where the tables are to appear in a document in which
% \cmd\floats@sw\ is false.
% If the user invokes the \classoption{nofloats} and fails to
% insert a \cmd\printtables\ command, 
% the tables will be printed at the end of the document.
% If the command is given, but floats are not being deferred, 
% it amounts to a no-op.
% \changes{4.0b}{1999/06/20}{only execute if appropriate}
% \changes{4.0c}{1999/11/13}{*-form mandates pagebreak}
% \changes{4.0d}{2000/05/17}{make longtable trigger the head, too}
%    \begin{macrocode}
\newcommand\printtables{%
 \begingroup
  \let@environment{longtable}{longtable@float}%
  \let@environment{longtable*}{longtable*@float}%
  \let@environment{turnpage}{turnpage@anchored}%
  \prepdef\longtable{\trigger@float@par}%
  \expandafter\prepdef\csname longtable*\endcsname{\trigger@float@par}%
  \expandafter\prepdef\csname table@floats\endcsname{%
   \onecolumngrid@push
  }%
  \expandafter\appdef\csname endtable@floats\endcsname{%
   \onecolumngrid@pop
  }%
  \@ifstar{\true@sw}{\floatp@sw{\true@sw}{\false@sw}}%
  {%
   \print@float{table}{\oneapage}%
  }{%
   \print@float{table}{}%
  }%
 \endgroup
}%
\newenvironment{turnpage@anchored}{%
 \onecolumngrid@push
 \setbox\z@\vbox to\textwidth\bgroup
  \columnwidth\textheight
}{%
  \vfil
 \egroup
 \rotatebox{90}{\box\z@}%
 \onecolumngrid@pop
}%
%    \end{macrocode}
% \end{macro}
%
%
% \subsection{Videos}
%\changes{4.1r}{2010/06/22}{(AO, 595) Provide \cs{lovname} along with other List of Videos definitions.}%
%
% \begin{environment}{video}
% We define the \env{video} environment analogously to the 
% \env{figure} and \env{table} environments; it is intended to contain a video.
%
%    \begin{macrocode}
\newenvironment{video}
 {\@float{video}}
 {\end@float}%
\newenvironment{video*}
 {\@dblfloat{video}}
 {\end@dblfloat}%
%    \end{macrocode}
% \end{environment}
%
% \begin{macro}{\thevideo}
% The video counter, float placement defaults, strings.
%    \begin{macrocode}
\newcounter{video}
\renewcommand \thevideo {\@arabic\c@video}
%    \end{macrocode}
%
% File extension and localizable strings.
%    \begin{macrocode}
\def\ext@video{lov}%
\def\fname@video{Video}%
\def\lovname{List of Videos}%
%    \end{macrocode}
%
% Float type and default placement. 
%    \begin{macrocode}
\def\fps@video{tbp}%
\def\ftype@video{4}%
\def\fnum@video{\fname@video~\thevideo}%
\appdef\document@inithook{%
 \@ifxundefined\c@float@type{}{%
  \global\setcounter{float@type}{8}%
 }%
}%
%    \end{macrocode}
% We allocate a box register for use in tallying the column inches of floats of this type.
%    \begin{macrocode}
\expandafter\newbox\csname fbox@\ftype@video\endcsname
\expandafter\setbox\csname fbox@\ftype@video\endcsname\hbox{}%
%    \end{macrocode}
% The documentation for the \classname{hyperref} package, \file{hyperref.dtx} states:
% ``classes or package which introduce new elements need to define 
%   an equivalent  \verb+\theH<name>+  for every  \verb+\the<name>+''
% We do accordingly here.
%    \begin{macrocode}
\let\theHvideo\thevideo
%    \end{macrocode}
% But \file{hyperref.dtx} goes on to say, 
% ``We do make a trap to make \verb+\theH<name>+ be the same as \verb+\arabic{<name>}+,
% if \verb+\theH<name>+ is not defined\ldots''
% However, it's not doing that right now (as of 6.77u), 
% and I cannot find any such code in there anymore.
% \end{macro}
%
%    \begin{macrocode}
\def\listofvideos{\print@toc{lov}}%
\let\l@video\l@figure
%    \end{macrocode}
%
% \subsubsection{Deferring \env{video} Floats}%
%
% We determine if videos are to float or be deferred until \cmd\printvideos\ time. 
% If so, we open the stream that will receive the deferred document portions. 
%    \begin{macrocode}
\appdef\class@documenthook{%
 \do@if@floats{video}{.vdx}%
}%
\appdef\class@enddocumenthook{%
 \printvideos\relax
}%
%    \end{macrocode}
% 
% \begin{macro}{\printvideos}
% The user-level command \cmd\printvideos\ determines
% where the videos are to appear in a document in which
% \cmd\floats@sw\ is false.
% If the user invokes the \classoption{endfloats} class option and fails to
% insert a \cmd\printvideos\ command, 
% the videos will be printed at the end of the document.
% If the command is given, but floats are not being deferred, 
% it amounts to a no-op.
%    \begin{macrocode}
\newcommand\printvideos{%
 \@ifstar{\true@sw}{\floatp@sw{\true@sw}{\false@sw}}%
 {%
  \print@float{video}{\oneapage}%
 }{%
  \print@float{video}{}%
 }%
}%
%    \end{macrocode}
% \end{macro}
%
%
% \section{Tabular}
% Every APS tabular has a double (Scotch) rule above and below.
% The column specifier ``d'' is implemented using the \classname{dcolumn}
% package, if available.
% FIXME: always load dcolumn!
% \begin{macro}{\tabular@hook}
% \begin{macro}{\endtabular@hook}
% \begin{environment}{ruledtabular}
%    \begin{macrocode}
\def\endtabular@hook{}%
%    \end{macrocode}
%    \begin{verbatim}
%\RequirePackage{dcolumn}%
%    \end{verbatim}
%    \begin{macrocode}
\appdef\document@inithook{%
 \@ifpackageloaded{dcolumn}{%
  \expandafter\@ifnotrelax\csname NC@find@d\endcsname{}{%
   \newcolumntype{d}{D{.}{.}{-1}}%
  }%
 }{}%
}%
\def\toprule{\hline\hline}%
\def\colrule{\hline}%
\def\botrule{\hline\hline}%
\newenvironment{ruledtabular}{%
 \def\array@default{v}%
 \appdef\tabular@hook{\def\@halignto{to\hsize}}%
 \let\tableft@skip@default\tableft@skip
 \let\tableft@skip\tableft@skip@float
 \let\tabmid@skip@default\tabmid@skip
 \let\tabmid@skip\tabmid@skip@float
 \let\tabright@skip@default\tabright@skip
 \let\tabright@skip\tabright@skip@float
 \let\array@row@pre@default\array@row@pre
 \let\array@row@pre\array@row@pre@float
 \let\array@row@pst@default\array@row@pst
 \let\array@row@pst\array@row@pst@float
 \appdef\array@row@rst{%
  \let\array@row@pre\array@row@pre@default
  \let\array@row@pst\array@row@pst@default
  \let\tableft@skip\tableft@skip@default
  \let\tabmid@skip\tabmid@skip@default
  \let\tabright@skip\tabright@skip@default
  \appdef\tabular@hook{\let\@halignto\@empty}%
 }%
}{%
}%
%    \end{macrocode}
% \end{environment}
% \end{macro}
% \end{macro}
%
% \section{Footnote Text}
%
% \changes{4.0a}{1998/06/10}{single space footnotes}
% \changes{4.0b}{1999/06/20}{Frank Mittelbach, has stated in \protect\classname{multicol}: ``The kernel command \cs{@footnotetext} should not be modified.'' Thus, I have removed David Carlisle's redefinition of that command. Note, however, that later versions of \protect\classname{multicol} do not require this workaround. Belt and suspenders.}%
%
% \begin{macro}{\@makefntext}
% \begin{macro}{\@makefnmark}
% We customize the presentation of the footnote mark: it will not be italic.
%    \begin{macrocode}
\def\@makefntext#1{%
  \def\baselinestretch{1}%
  \parindent1em%
  \noindent
  \hb@xt@1.8em{%
   \hss\@makefnmark
  }%
  #1%
  \par
}%
\def\@makefnmark{%
 \hbox{%
  \@textsuperscript{%
   \normalfont\@thefnmark
  }%
 }%
}%
%    \end{macrocode}
% \end{macro}
% \end{macro}
%
%
% \subsection{Citations, Bibliography, Endnotes}
%
% \subsubsection{Bibliography}
%
% Load Patrick Daly's \classname{natbib} package,
% \url{ftp://ctan.tug.org/macros/latex/contrib/supported/natbib}
%
% Note that \classname{natbib} assumes that it loads
% over a document class, such as the \classname{article} class, 
% that has already defined \env{thebibliography} and \cmd\@listi.
%
% Note also that \classname{natbib} also installs a command
% \cmd\NAT@set@cites\ into \cmd\AtBeginDocument\ which presumes
% that the proper \cmd\bibpunct\ command has been issued.
% 
% Note that the macro \cmd\NAT@sort\ controls 
% whether citations are left alone (\cmd\NAT@sort=0), sorted (\cmd\NAT@sort=1), or sorted and compressed (\cmd\NAT@sort=2).
% Since we give \classname{natbib} the \classoption{sort\&compress} option,
% if you prefer \classoption{sort}, you need only \cmd\let \cmd\NAT@sort\ to be \cmd\@ne.
% However, if you prefer the effect of having neither \classoption{sort} nor \classoption{sort\&compress},
% you must \cmd\let \cmd\NAT@sort\ to be \cmd\z@ and you must also define 
% \cmd\let\cmd\NAT@cmprs\ to be \cmd\z@.
%
% As of version 8.2, \classname{natbib} now no longer binds at the point where it is read in.
% This means that we can freely change \cmd\NAT@sort, \cmd\NAT@cmprs, and the new
% \cmd\NAT@merge. Henceforth, we require that this later version be used.
%
% For other \classname{natbib} customizations, you may proceed as if you were going to use
% the \file{natbib.cfg} file: anything that you can modify by this means is fair game.
% Once \revtex\ is finished loading, you can assert any definitions for \classname{natbib} that you wish.
%
% \begin{macro}{\rev@citet}
% \begin{macro}{\rev@citealp}
% \begin{macro}{\rev@citealpnum}
% We define variants on \classname{natbib}'s commands \cmd\citet, \cmd\citealp, and \cmd\citealpnum. 
% \cmd\rev@citealpnum\ uses a numerical citation.
% \cmd\rev@citealp\ and \cmd\rev@citealpnum\ are the aliases of \cmd\onlinecite,
% \cmd\rev@citet\ that of \cmd\textcite.
% 
% In each case, we invoke \cmd\rtx@swap@citea\ to effect different productions between 
% multiple arguments to the \cmd\cite\ command.
%
% \cmd\rev@citealpnum\ provides textual citations where superscript citations are the default.
% These should be accessible via the \cmd\citet\ command.
%
% Therefore we remember how to do a numerical citation
% even when the superscript citation has been selected.
%    \begin{macrocode}
\expandafter\DeclareRobustCommand
\expandafter\rev@citet
\expandafter{%
 \expandafter\begingroup
  \expandafter\rtx@swap@citea
  \expandafter\g@bblefirsttoken
              \csname citet \endcsname
}%
\expandafter\DeclareRobustCommand
\expandafter\rev@citealp
\expandafter{%
 \expandafter\begingroup
  \expandafter\rtx@swap@citea
  \expandafter\g@bblefirsttoken
              \csname citealp \endcsname
}%
\expandafter\DeclareRobustCommand
\expandafter\rev@citealpnum
\expandafter{%
 \expandafter\begingroup
  \expandafter\rtx@swap@citenum
  \expandafter\g@bblefirsttoken
              \csname citealp \endcsname
}%
\def\rtx@swap@citenum{%
  \rtx@swap@citea
  \let\@cite\NAT@citenum
  \let\NAT@mbox\mbox
  \let\citeyear\NAT@citeyear
  \let\NAT@space\NAT@spacechar
}%
\def\g@bblefirsttoken{%
 \expandafter\true@sw
 \expandafter\@empty
}%
%    \end{macrocode}
% \end{macro}
% \end{macro}
% \end{macro}
%
% \begin{macro}{\rtx@citesuper}
% We prepare to redefine \classname{natbib}'s procedure \cmd\NAT@citesuper,
% which is executed when setting a superscript citation.
% The \cmd\hspace\ is removed: in any case, it
% should really be \cmd\hspace*, to prevent an unwanted pagebreak.
%    \begin{macrocode}
\newcommand\rtx@citesuper[3]{%
 \ifNAT@swa
  \leavevmode
  \unskip
% \hspace{1\p@}%
  \textsuperscript{\normalfont#1}%
  \if*#3*\else\ (#3)\fi
 \else
  #1%
 \fi
 \endgroup
}%
%    \end{macrocode}
% \end{macro}
%
% \begin{macro}{\@makefnmark@cite}
% We define a procedure that will set a footnote mark the same way that
% a citation is set. If footnotes are put in the bibliography with
% \cmd\footinbib@sw, then the corresponding mark should look the same
% as the result of a \cmd\cite. This is how we do it.
%    \begin{macrocode}
\def\@makefnmark@cite{\begingroup\NAT@swatrue\@cite{{\@thefnmark}}{}{}}%
%    \end{macrocode}
% \end{macro}
%
% \begin{macro}{\rtx@bibsection}
% Prepare to override \classname{natbib}'s definition of \cmd\bibsection.
%    \begin{macrocode}
\def\rtx@bibsection{%
 \@ifx@empty\refname{%
  \par
 }{%
% \let\@hangfroms@section\@hang@froms
  \expandafter\section\expandafter*\expandafter{\refname}%
  \@nobreaktrue
 }%
}%
%    \end{macrocode}
% \end{macro}
%
% \begin{macro}{\rtx@swap@citea}
% \begin{macro}{\rtx@def@citea}
% \begin{macro}{\rtx@def@citea@close}
% \begin{macro}{\rtx@def@citea@box}
% The procedures |\rtx@def@citea|, |\rtx@def@citea@close|, and |\rtx@def@citea@box| 
% can take over the management of \classname{natbib}'s |\@citea| macro to effect more sophisticated
% behavior of the punctuation between textual citations. The switch is performed by \cmd\rtx@swap@citea.
%
% \changes{4.1j}{2009/10/25}{(AO, 552) Repair spacing in \cs{onlinecite}}
% In these procedures, we use \cmd\count@\ to count the number of arguments of the \cmd\cite\ command,
% and we use \cmd\c@NAT@ctr\ to keep track of which argument we are processing.
% The latter counter is created by \classname{natbib} and used there solely in bibliography processing,
% where it keeps track of the reference number. 
% We take over its use in these macros, but only locally; therefore these procedures should work properly, even within the bibliography.
%FIXME: check whether this is true!
%
% Because we are using a scratch counter \cmd\count@, we are vulnerable to other \TeX\ programmers
% who patch in to \classname{natbib}'s processing and who might use that counter at the same time we are doing so.
% This is a potential source of trouble for us. 
%FIXME: store the value of \cmd\count@\ in a private \cmd\csname!
%
% Note that \cmd\rtx@def@citea\ begins the same as \cmd\NAT@def@citea, which it replaces, 
% then makes further decisions based on the values of the counters. 
%
% Note also that, in \classname{natbib}, the replacement part of \cmd\NAT@def@citea@close\ could be rewritten as
% \cmd\NAT@def@citea\cmd\prepdef\cmd\@citea\verb+{+\cmd\NAT@@close\verb+}+, which 
% would them obviate the need for us to override its meaning. 
%
% Note, too, the effect of \cmd\rtx@def@citea@box, which replaces \cmd\NAT@def@citea@box,
% is almost the same as the latter, except the entire \cmd\@citea\ is given as the argument
% of \cmd\NAT@mbox.
%
% Finally, bear in mind that the English (and some American editors) do not place a comma
% before the ``and''; our procedures do (but they could be rewritten with that convention). 
%    \begin{macrocode}
\def\rtx@swap@citea{%
 \let\NAT@def@citea\rtx@def@citea
 \let\NAT@def@citea@close\rtx@def@citea@close
 \let\NAT@def@citea@box\rtx@def@citea@box
}%
\def\rtx@def@citea{%
 \def\@citea{\NAT@separator\NAT@space}%
 \advance\c@NAT@ctr\@ne
 \@ifnum{\count@>\tw@}{%
  \@ifnum{\c@NAT@ctr=\count@}{\appdef\@citea{\NAT@conj\NAT@space}}{}%
 }{%
  \def\@citea{\NAT@space\NAT@conj\NAT@space}%
 }%
}%
\def\rtx@def@citea@close{%
 \rtx@def@citea
 \prepdef\@citea{\NAT@@close}%
}%
\def\rtx@def@citea@box{%
 \rtx@def@citea@close
 \expandafter\def\expandafter\@citea\expandafter{\expandafter\NAT@mbox\expandafter{\@citea}}%
}%
\def\NAT@conj{and}%
%    \end{macrocode}
%
% \end{macro}
% \end{macro}
% \end{macro}
% \end{macro}
%
% \begin{macro}{\BibitemShut}
% \begin{macro}{\bibAnnote}
% \changes{4.1j}{2009/10/24}{(AO, 549) Repairing natbib's \cs{BibitemShut} and \cs{bibAnnote}}
% \changes{4.1o}{2010/02/05}{(AO, 549) Remove patch to natbib, which is now at version 8.31a}
% \changes{4.1o}{2010/02/07}{(AO, 578) accommodate the possible space character preceding \cs{BibitemShut}.} 
% We remember a temporary patch to \classname{natbib}'s definition of \cmd\BibitemShut.
%    \begin{macrocode}
\def\NAT@BibitemShut#1{%
 \def\@bibstop{#1}%
 \let\bibitem@Stop\bibitemStop
 \let\bibitem@NoStop\bibitemNoStop
 \@ifx{\bibitemShut\relax}{\let\@bibitemShut\@empty}{%
  \expandafter\def\expandafter\@bibitemShut\expandafter{\bibitemShut}%
 }%
}%
%    \end{macrocode}
% The following is a bug fix to \classname{natbib} version 8.31b.
%    \begin{macrocode}
\def\BibitemShut@ltx#1{%
 \unskip
 \def\@bibstop{#1}%
 \let\bibitem@Stop\bibitemStop
 \let\bibitem@NoStop\bibitemNoStop
 \@ifx{\bibitemShut\relax}{\let\@bibitemShut\@empty}{%
  \expandafter\def\expandafter\@bibitemShut\expandafter{\bibitemShut}%
 }%
}%
%    \end{macrocode}
%
%    \begin{verbatim}
%\providecommand{\bibAnnote}[3]{%
%  \BibitemShut{#1}%
%  \def\@tempa{#3}\@ifx{\@tempa\@empty}{}{%
%   \begin{quotation}\noindent
%    \textsc{Key:}\ #2\\\textsc{Annotation:}\ \@tempa
%   \end{quotation}%
%  }%
%  \ignorespaces
%}%
%\def\@bibitemShut{}%
%    \end{verbatim}
% \end{macro}
% \end{macro}
%
%
%    \begin{macrocode}
\newenvironment{thebibliography}{}{}%
\let\@listi\@empty
\appdef\rtx@require@packages{%
 \RequirePackage[sort&compress]{natbib}[2009/11/07 8.31a (PWD, AO)]%
 \let@environment{NAT@thebibliography}{thebibliography}%
 \let@environment{thebibliography}{rtx@thebibliography}%
 \let\bibliographystyle@latex\bibliographystyle
 \let\NAT@citesuper\rtx@citesuper
%    \end{macrocode}
%
% \begin{macro}{\NAT@bibsetnum}
% \begin{macro}{\NAT@bibsetup}
% \begin{macro}{\bibpreamble}
% \begin{macro}{\newblock}
% \begin{macro}{\bibnumfmt}
% \begin{macro}{\NAT@merge}
% We define the sectioning command to use when starting the bibliography
% and gently coax \classname{natbib} into using
% the formatting procedures that \emph{we} want it to use.
%
% This way of setting up \env{thebibliography} 
% automatically sets the label width based on
% the largest number used within the bibliography.
% This scheme will not work properly using the
% author/year style of bib entry, though.
%
% We define \cmd\bibnumfmt\ to be \cmd\place@bibnumber, which is a macro
% managed by \revtex. If the document defines \cmd\bibnumfmt, then that
% definition will be used instead, which is what the \classname{natbib}
% package gives as its programming interface.
%
% We set \cmd\NAT@merge\ to \cmd\tw@, which turns on \classname{natbib}'s
% \classname{mcite} capabilities. This is the default setting. 
% If numerical citations are not to be used, then \cmd\NAT@merge\ 
% should be set to \cmd\@ne\ (syntax is still enabled, but semantics are turned off). 
%    \begin{macrocode}
\let\bibsection\rtx@bibsection
\let\NATx@bibsetnum\NAT@bibsetnum
\def\NAT@bibsetnum#1{%
 \setlength{\topsep}{\z@}%
 \NATx@bibsetnum{\ref{LastBibItem}}%
}%
\let\NATx@bibsetup\NAT@bibsetup
\def\NAT@bibsetup{%
 \setlength{\labelwidth}{\z@}%
 \setlength{\labelsep}{\z@}%
 \setlength{\itemindent}{\z@}%
 \setlength{\listparindent}{\z@}%
 \setlength{\topsep}{\z@}%
 \setlength{\parsep}{\z@}%
 \NATx@bibsetup
}%
\let\bibpreamble\@empty
\def\newblock{\ }%
\let\NATx@bibnumfmt\bibnumfmt
\def\bibnumfmt{\place@bibnumber}%
\let\NAT@merge\thr@@
\let\NAT@citeyear\citeyear
\let\onlinecite\rev@citealp
\let\textcite\rev@citet
%    \end{macrocode}
% The following is needed until \classname{natbib} is at 8.31b.
%    \begin{macrocode}
\@ifx{\BibitemShut\NAT@BibitemShut}{%
 \class@info{Repairing natbib's \string\BibitemShut}%
 \let\BibitemShut\BibitemShut@ltx
}{}%
%    \end{macrocode}
% \end{macro}
% \end{macro}
% \end{macro}
% \end{macro}
% \end{macro}
% \end{macro}
%
% \begin{macro}{\bibliographystyle}
% \changes{4.1q}{2010/04/01}{(AO, 586) When .bbl is pasted into the document, prevent automatic bibliography inclusion.}%
% \changes{4.1q}{2010/04/13}{(AO, 588) Only write \revtex-specific BibTeX .bib data if the .bst style is set by REVTeX.}%
% We arrange for the selection of bibliography style
% to occur either due to the document's explicit
% \cmd\bibliographystyle\ statement or
% via the journal substyle.
%
% Note that \revtex\ is incompatible with 
% any package that patches \cmd\bibliographystyle.
% Since \classname{natbib} does this, we need a fix.
%
% The Boolean \cmd\bibliographystyle@sw\ signifies that the document  contains explicit
% \cmd\bibliographystyle\ markup. If, on the contrary, the bibliography style is set by
% the the society or the journal, then no explicit \cmd\bibliographystyle\ command appears in the document instance.
% In this case \cmd\bibliographystyle@sw\ will be \cmd\false@sw.
%    \begin{macrocode}
\let\bibliographystyle@latex\bibliographystyle
\def\bibliographystyle{\@booleantrue\bibliographystyle@sw\def\@bibstyle}%
\@booleanfalse\bibliographystyle@sw
%    \end{macrocode}
% \end{macro}
%
% \changes{4.1j}{2009/11/03}{(AO, 551) Punctuation at end of reference when optional arguments to the cite key are given.}
% The following had been bug fixes to \classname{natbib} version 8.31a.
%    \begin{verbatim}
%\def\bibitemStop{\@bibitemShut}%
%\def\NAT@bibitem@cont{%
% \let\bibitem@Stop\bibitemContinue
% \let\bibitem@NoStop\bibitemContinue
%}%
%    \end{verbatim}
% The following are alterations to \classname{natbib} version 8.31a to 
% accommodate the possible space character preceding \cmd\BibitemShut,
% and to handle the case of merged references, where the first ends with a stop character. 
% \changes{4.1o}{2010/02/07}{(AO, 578) accommodate the possible space character preceding \cs{BibitemShut}.} 
% \changes{4.1o}{2010/02/09}{(AO, 581) Handle case: merged references, with first ending in a stop character.} 
%    \begin{macrocode}
\def\NAT@bibitem@cont{%
 \let\bibitem@Stop\bibitemContinue@Stop
 \let\bibitem@NoStop\bibitemContinue
}%
%    \end{macrocode}
%    \begin{macrocode}
\def\bibitemNoStop{%
 \@ifx@empty\@bibitemShut{.\spacefactor\@mmm\space}{\@bibitemShut}%
}%
%    \end{macrocode}
%    \begin{macrocode}
\def\bibitemContinue{%
 \@ifx@empty\@bibitemShut{;\spacefactor\@mmm\space}{\@bibitemShut}%
}%
%    \end{macrocode}
%    \begin{macrocode}
\def\bibitemContinue@Stop{%
 \@ifx@empty\@bibitemShut{\spacefactor\@mmm\space}{\@bibitemShut}%
}%
%    \end{macrocode}
%
% \changes{4.1k}{2009/11/11}{(AO, 561) remove dead code relating to \cs{bibitemContinue}} 
% We used to customize one of the productions of \classname{natbib}, but no longer.
%    \begin{verbatim}
%\let\bibitemContinue\bibitemContinue@rtx
%    \end{verbatim}
%
% Here ends the code to be executed at \cmd\rtx@require@packages\ time.
%    \begin{macrocode}
}%
%    \end{macrocode}
% \changes{4.1f}{2009/07/07}{(AO, 516) Merged references are separated with a semicolon}
% Redefine a macro of \classname{natbib} so that merged references are separated with a semicolon.
% \changes{4.1k}{2009/11/11}{(AO, 561) remove dead code relating to \cs{bibitemContinue}} 
%    \begin{verbatim}
% \def\bibitemContinue@rtx{;\spacefactor\@mmm\space}%
%    \end{verbatim}
%
% \begin{macro}{\onlinecite}
% \begin{macro}{\textcite}
% We extend \classname{natbib}'s syntax with two commands to set a citation
% on the baseline (as opposed to superscripted) and as text (rather than parenthetical), respectively.
% A journal substyle that makes citations be superscripted or parenthetical as the case may be, should ensure
% that the author has continued access to these two styles. 
%
% Note that the society or journal substyle override the meanings of \cmd\@onlinecite\ or
% \cmd\@textcite\ given here.
% 
%    \begin{macrocode}
\DeclareRobustCommand\onlinecite{\@onlinecite}%
\DeclareRobustCommand\textcite{\@textcite}%
%    \end{macrocode}
% \end{macro}
% \end{macro}
%
% \begin{macro}{\bibliography}
% \changes{4.1a}{2008/01/18}{(AO, 457) Endnotes to be sorted in with numerical citations.}%
% Provide a hook for supplying Bib\TeX\ a bibliographic database that may contain, say, 
% footnotes.
% 
% Note that Bib\TeX\ chokes if the argument of the \cmd\bibdata\ command has null fields,
% hence these tests.
%    \begin{macrocode}
\let\bibliography@latex\bibliography
\def\bibliography#1{%
 \auto@bib@empty
 \begingroup
  \let\auto@bib@innerbib\@empty
  \@ifx@empty{\pre@bibdata}{%
   \bibliography@latex{#1}%
  }{%
   \@if@empty{#1}{%
    \expandafter\bibliography@latex\expandafter{\pre@bibdata}%
   }{%
    \expandafter\bibliography@latex\expandafter{\pre@bibdata,#1}%
   }%
  }%
 \endgroup
}%
\let\pre@bibdata\@empty
%    \end{macrocode}
% \end{macro}
%
% \begin{environment}{rtx@thebibliography}%
% \begin{macro}{\present@bibnote}%
% \changes{4.1f}{2009/07/11}{(AO, 521) Lonely bibliography head}%
% \changes{4.1n}{2010/01/02}{(AO, 572) Independent footnote counter for title block. Abstract footnote counter shared with body.}%
% \changes{4.1o}{2010/02/02}{(AO, 575) Automatically incorporate the (Bib\TeX-generated) .bbl into an explicit \env{thebibliography}}%
% \changes{4.1q}{2010/04/01}{(AO, 586) When .bbl is pasted into the document, prevent automatic bibliography inclusion.}%
% We put a tail patch into \cmd\thebibliogrphy\ and
% a headpatch into \cmd\endthebibliography.
%
% Here we provide a default treatment for frontmatter notes 
% deferred to the bibliography; a journal substyle 
% might want to override the definition of \cmd\present@bibnote.
%
% We make provisions for the case where there are no \cmd\bibitem s 
% for the bibliography: we produce no bibliography head at all.
% 
%    \begin{macrocode}
\newenvironment{rtx@thebibliography}[1]{%
 \NAT@thebibliography{#1}%
 \let\@TBN@opr\present@bibnote
 \@FMN@list
}{%
%    \end{macrocode}
% The following line was commented out:
%    \begin{verbatim}
%\@endnotesinbib
%    \end{verbatim}
% The \cmd\auto@bib@innerbib\ directive has been moved from the begin processing to the end processing.
% This means that the content of the \env{thebibliography} environment can itself prevent the 
% automatic reading in of the .bbl file. This would be needed when the user has pasted in the 
% content of the .bbl file into the document itself, something required by APS and AIP editorial direction.
%    \begin{macrocode}
 \auto@bib@innerbib
 \edef\@currentlabel{\arabic{NAT@ctr}}%
 \label{LastBibItem}%
 \endNAT@thebibliography
 \aftergroup\auto@bib@empty
}%
\def\present@bibnote#1#2{%
 \item[%
  \textsuperscript{%
   \normalfont
   \Hy@raisedlink{\hyper@anchorstart{frontmatter.#1}\hyper@anchorend}%
   \begingroup
    \csname c@\@mpfn\endcsname#1\relax
    \frontmatter@thefootnote
   \endgroup
  }%
 ]#2\par
%    \end{macrocode}
% The following line was commented out:
%    \begin{verbatim}
%\global\let\NAT@bibitem@first@sw\@secondoftwo
%    \end{verbatim}
%    \begin{macrocode}
}%
%    \end{macrocode}
% \end{macro}
% \end{environment}
%
% \begin{macro}{write@bibliographystyle}
% \changes{4.1q}{2010/04/13}{(AO, 588) Only write \revtex-specific BibTeX .bib data if the .bst style is set by REVTeX.}%
% We wish to delay committing the \cmd\bibliographystyle\ until as late as possible.
% The journal substyle will define a default bibliography style, and 
% the document's explicit \cmd\bibliographystyle\ command, if any, will override that default.
%
% The \cmd\bibstyle\ command is allowed appear quite late in the \file{.aux} file. 
% We now delay the automatic writing of the \cmd\bibstyle\ command to the end of the job.
%
% The procedure \cmd\write@bibliographystyle\ tests whether a \cmd\bibliographystyle\ command has 
% already been given.
% If not, it effectively executes the needed \cmd\bibliographystyle\ command, then neutralizes itself
% (we only need to do this once per job).
%
% If the document lacks explicit \cmd\bibliographystyle\ markup, we execute
% \cmd\@bibdataout@rev, a hook for \revtex-aware processing. 
%    \begin{macrocode}
\def\write@bibliographystyle{%
 \@ifxundefined\@bibstyle{}{%
  \expandafter\bibliographystyle@latex\expandafter{\@bibstyle}%
  \bibliographystyle@sw{}{\@bibdataout@rev}%
 }%
 \global\let\write@bibliographystyle\relax
}%
\AtEndDocument{\write@bibliographystyle}%
%    \end{macrocode}
% \end{macro}
%
% \begin{macro}{\rtx@@citetp}
% \begin{macro}{\rtx@citex}
% \begin{macro}{\super@cite@let}
% \begin{macro}{\super@cite@end}
% \begin{macro}{\super@cite@swap}
% We wish to extend \classname{natbib} to move spaces and citations around
% a superscript-style citation, imitating Donald Arseneau's \classname{cite} package
% with the \classoption{super}.
%
% The \cmd\rtx@@citetp\ procedure is substituted for \cmd\NAT@@citetp;
% it then calls the \cmd\rtx@citex\ procedure and implements the features of the
% \classoption{citeautoscript} class option.
% In the end, \cmd\@citex\ is called with its customary parameters. 
%
% The document should be marked up as if citations were \emph{not} superscripted,
% and then if you select a journal substyle that has superscripted citations, \revtex\ will do its best
% to alter the formatting of the \cmd\cite s to accomodate superscript style.
%
% Only citations set as superscript are affected by this procedure,
% because we check \cmd\@cite\ against \cmd\NAT@citesuper. 
%
% Here's a subtle point: when is the argument of \cmd\super@cite@swap\ not the same as the token \cmd\@let@token?
% Answer: when the latter is \cmd\@sptoken! This case has to be handled separately.
% 
% Note that whether a punctuation is movable is determined by the definition
% of a particular control sequence name.
% A society or journal can alter things: to remove a character from the set,
% do, say, \cmd\expandafter \cmd\let \cmd\csname \texttt{rtx@automove;} \cmd\endcsname \cmd\relax.
% To add a character to the set, do, say, \cmd\expandafter \cmd\let \cmd\csname \texttt{rtx@automove;} \cmd\endcsname \cmd\@empty.
%
% Implementation note: due to a \TeX\ peculiarity, we have to check for the case
% where \cmd\@let@token\ is a space token \emph{before} we parse forward.
% At issue is the corner case where an end of file is at hand. If we 
% were to let \cmd\super@cite@swap\ parse forward, we would encounter a \TeX\ end-of-file
% error. Note that the test will be true in many distinct cases:
% the file ends, the next character is a line terminator, the next character is a space.
%    \begin{macrocode}
\def\rtx@@citetp[#1]{\@ifnextchar[{\rtx@citex[#1]}{\rtx@citex[][#1]}}%
\def\rtx@citex[#1][#2]#3{%
 \begingroup
  \def\@tempa{[#1][#2]{#3}}%
  \@ifx{\@cite\NAT@citesuper}{%
   \leavevmode
   \skip@\lastskip
   \unskip
   \super@cite@let
  }{%
   \super@cite@end
  }%
}%
\def\super@cite@let{%
 \futurelet\@let@token\super@cite@check
}%
\def\super@cite@end{%
 \aftergroup\@citex\expandafter\endgroup\@tempa
}%
\def\super@cite@check{%
 \@ifx{\@let@token\@sptoken}{%
  \super@cite@end
 }{%
  \super@cite@swap
 }%
}%
\long\def\super@cite@swap#1{%
 \expandafter\@ifx\expandafter{\csname rtx@automove#1\endcsname\@empty}{%
  #1%
  \super@cite@let
 }{%
  \super@cite@end
  #1%
 }%
}%
\expandafter\let\csname rtx@automove.\endcsname\@empty
\expandafter\let\csname rtx@automove,\endcsname\@empty
\expandafter\let\csname rtx@automove:\endcsname\@empty
\expandafter\let\csname rtx@automove;\endcsname\@empty
%    \end{macrocode}
% \end{macro}
% \end{macro}
% \end{macro}
% \end{macro}
% \end{macro}
%
% The following must execute only after \classname{natbib} is loaded and has set
% up its parameters (which it does at \cmd\AtBeginDocument\ time).
% If superscript citations have been selected, and 
% if the \classoption{citeautoscript} class option has been selected, 
% we patch into \classname{natbib}'s mechanism to migrate punctuation around the 
% citation, as in class \classname{cite} with the \classoption{superscript} option.
%    \begin{macrocode}
\appdef\class@documenthook{%
 \citeautoscript@sw{%
  \@ifx{\@cite\NAT@citesuper}{%
   \let\NAT@@citetp\rtx@@citetp
  }{}%
 }{}%
}%
%    \end{macrocode}
%
% Resolve an incompatability between \classname{natbib} and \classname{listings}.
% The latter package tests \cmd\chapter (which has now been \cmd\let\ to \cmd\relax\ as a side effect \classname{natbib}'s use of \LaTeX's \cmd\@ifundefined).
%
% We couch our fix in such terms that will not be disruptive if \cmd\chapter\ is actually defined at this point.
%
% \changes{4.1b}{2008/07/01}{(AO) No longer need to test \cs{chapter} as of \texttt{natbib} version 8.2}
%    \begin{verbatim}
%\@ifx{\chapter\relax}{\let\chapter\@undefined}{}%
%    \end{verbatim}
%
%
% \subsubsection{\cs{endnote}s and \cs{rtx@bibnote}s}
%
% \begin{macro}{\mini@note}
% \begin{macro}{\save@note}
% \changes{4.1n}{2010/01/02}{(AO, 572) Independent footnote counter for title block. Abstract footnote counter shared with body.}%
% QUERY: how do footnotes get thrown to the bibliography. \cmd\footinbib@sw\ appears to be irrelevant. 
%    \begin{macrocode}
\def\mini@note{\save@note\mini@notes}%Implicit #2
\def\save@note#1#2{%
  \stepcounter\@mpfn
  \protected@xdef\@thefnmark{\thempfn}%
  \@footnotemark
  \expandafter\g@addto@macro
  \expandafter#1%
  \expandafter{%
  \expandafter \@@footnotetext
  \expandafter {\@thefnmark}{#2}%
              }%
}%
\long\def\@@footnotetext#1{\def\@thefnmark{#1}\@footnotetext}%
\let\mini@notes\@empty
%    \end{macrocode}
% \end{macro}
% \end{macro}
%
% \begin{macro}{\endnote}
% A version of footnote that appears in the bibliography, or where \cmd\printendnotes\ appears.
%    \begin{verbatim}
%\def\@endnote{%
% \begingroup
%  \aftergroup\@footnotemark
%  \aftergroup\@endnotetext
%  \@ifnextchar[{%
%   \@xendnote
%  }{%
%   \stepcounter{footnote}%
%   \protected@xdef\@tempa{\thefootnote}%
%   \expandafter\@xendnote\expandafter[\the\c@footnote]%
% }%
%}%
%    \end{verbatim}
% \end{macro}
%
% \begin{macro}{\@xendnote}
% \changes{4.1a}{2008/01/18}{(AO, 457) Endnotes to be sorted in with numerical citations.}%
% \begin{verbatim}
%\def\unused@xendnote[#1]{%
%  \begingroup
%   \c@footnote#1\relax
%    \end{macrocode}
% New for 4.1
%    \begin{macrocode}
%   \unrestored@protected@xdef\@endnotelabel{Note\thefootnote}%
%   \authoryear@sw{%
%    \unrestored@protected@xdef\@thefnmark{\noexpand\ref{\@endnotelabel}}%
%   }{%
%    \unrestored@protected@xdef\@thefnmark{\@endnotelabel}%
%   }%
%    \end{macrocode}
% Was:
% \unrestored@protected@xdef\@thefnmark{endnote\thefootnote}%
% End 4.1 changes
%    \begin{macrocode}
%  \endgroup
% \endgroup
%}%
%\def\@endnotemark{%
% \expandafter\cite\expandafter{\@thefnmark}%
%}%
% \end{verbatim}
%    \begin{macrocode}
\def\rev@citemark#1{%
 \expandafter\cite\expandafter{\@thefnmark}%
}%
\def\rev@endtext#1{%
 \let\@endnotelabel\@thefnmark
 \@endnotetext
}%
%    \end{macrocode}
% \end{macro}
%
% \begin{macro}{\endnote@ext}
% \begin{macro}{\bibdata@app}
% \begin{macro}{\bibdata@ext}
% The macro \cmd\endnote@ext\ is the file extension for the auxiliary file holding footnotes.
% The \cmd\bibdata@app\ and \cmd\bibdata@ext\ macros are used to form the name of a
% Bib\TeX\ database file holding footnotes.
%    \begin{macrocode}
\def\endnote@ext{.end}%
\def\bibdata@app{Notes}%
\def\bibdata@ext{bib}%
%    \end{macrocode}
% \end{macro}
% \end{macro}
% \end{macro}
%
% \begin{macro}{\@endnotetext}
% \changes{4.1a}{2008/01/18}{(AO, 457) Endnotes to be sorted in with numerical citations.}%
% The procedure \cmd\@endnotetext\ writes a Bib\TeX\ .bib file for the purpose
% of insering a footnote into the (numbered, unsorted) bibliography.
%
% We need to define \cmd\pre@bibdata\ to be 
% \cmd\jobname\cmd\endnote@ext, and we probably should define \cmd\endnote@ext\ to 
% be something like ``Notes.bib''.
%
% In each case, the material to be written out requires robustification, provided by \cmd\endnote@relax.
% The commands \cmd\label, \cmd\index, and \cmd\glossary,
% which are robustified for \cmd\markright\ and \cmd\addcontentsline,
% are likewise robustified here.
%
% Procedure \cmd\@endnotetext@note\ is the alias for \cmd\@endnotetext\ when the endnotes are to be
% processed separately from the bibliography (generally true when citations are not sorted).
%    \begin{verbatim}
%\long\def\unused@endnotetext@note#1{%
%  \@ifxundefined\@endnoteout{%
%    \newwrite\@endnoteout
%    \gdef\endnote@stream{\jobname\endnote@ext}%
%    \immediate\openout\@endnoteout\endnote@stream\relax
%  }{}%
%  \begingroup
%    \endnote@relax
%    \immediate\write\@endnoteout{\string\@doendnote{\@endnotelabel}{#1}}%
%  \endgroup
%}%
%    \end{verbatim}
%
% \cmd\@doendnote\ is obsolete.
%    \begin{verbatim}
%\def\@doendnote#1#2{\bibitem{#1}#2}%
%    \end{verbatim}
%
% \changes{4.1o}{2010/02/05}{(AO, 579) Endnote shall comprise their own Bib\TeX\ entry type: @FOOTNOTE.} 
% Procedure \cmd\@endnotetext\ is the operative procedure\ when the endnotes are to be
% collated in with the other references, typically true when numerical citations are being used.
% The technique involves writing a .bib file (\cmd\@bibdataout) with each endnote typed as 
% a \texttt{@FOOTNOTE} entry. 
%
% Timing note: doing \cmd\openout\ should be deferred until the beginning of the document,
% as is done here. This allows one to make a format (\filename{.fmt}) file out of this class.
%    \begin{macrocode}
\long\def\@endnotetext#1{%
  \begingroup
    \endnote@relax
    \immediate\write\@bibdataout{%
     @FOOTNOTE{%
      \@endnotelabel,%
%    \end{macrocode}
% The \texttt{key} field is recommended in cases where there is no author (see \filename{btxdoc}).
%    \begin{macrocode}
      key="\@endnotelabel",%
%    \end{macrocode}
% The \texttt{note} field is simply the content of the footnote.
%    \begin{macrocode}
      note="#1"%
     }%
    }%
  \endgroup
}%
\newwrite\@bibdataout
%    \end{macrocode}
% \end{macro}
%
% \begin{macro}{\endnote@relax}
% At \cmd\AtBeginDocument\ time, we open the job's \filename{.bib} file.
%
% Procedure \cmd\endnote@relax\ robustifies commands that ought not to be expanded when
% the endnote is written out.
% Note the similarity between \cmd\endnote@relax\ and  \cmd\protected@write.
%    \begin{macrocode}
\def\endnote@relax{%
 \let\label\relax \let\index\relax \let\glossary\relax
 \let\cite \relax \let\ref  \relax \let\pageref \relax
 \let\(    \relax \let\)    \relax \let\\       \relax
 \let~\relax
%    \end{macrocode}
%    \begin{verbatim}
%\let\protect\noexpand
%    \end{verbatim}
%    \begin{macrocode}
 \let \protect \@unexpandable@protect
 \newlinechar`\^^M%
%    \end{macrocode}
%    \begin{verbatim}
%\newlinechar`\ %
%    \end{verbatim}
%    \begin{macrocode}
 \let\begin\relax \let\end\relax
}%
%    \end{macrocode}
% \end{macro}
%
% \begin{macro}{\@bibdataout@init}
% \begin{macro}{\@bibdataout@aps}
% \changes{4.1o}{2010/02/10}{(AO, 580) Provide a document class option to turn off production of eprint field in bibliography.} 
% \changes{4.1o}{2010/02/12}{(AO, 580) Control .bst at run time.}%
% \changes{4.1q}{2010/04/13}{(AO, 588) Only write \revtex-specific BibTeX .bib data if the .bst style is set by REVTeX.}%
% At \cmd\AtBeginDocument\ time, we open the job's \filename{.bib} file.
% The hook is available for use by a society to place its own @CONTROL record in the \cmd\@bibdataout\ stream.
%    \begin{macrocode}
\appdef\class@documenthook{\@bibdataout@init}%
\def\@bibdataout@init{%
 \immediate\openout\@bibdataout\pre@bibdata.\bibdata@ext\relax
}%
\def\@bibdataout@rev{%
 \immediate\write\@bibdataout{%
%    \end{macrocode}
% The entry that controls processing of the \filename{.bst} file has entry type \texttt{@CONTROL}. 
% The citation key (REVTEX41Control) is effectively a version number,
% which the \filename{.bst} can use to interpret the bib entry.
%    \begin{macrocode}
  @CONTROL{%
   REVTEX41Control%
%    \end{macrocode}
% Say if we want the \texttt{eprint} field disabled. Otherwise accept the default of the \filename{.bst}. 
%    \begin{macrocode}
   \eprint@enable@sw{}{,eprint=""}%
  }%
 }%
%    \end{macrocode}
% Place a \cmd\citation\ into the auxiliary file corresponding to this entry. 
%    \begin{macrocode}
 \if@filesw
  \immediate\write\@auxout{\string\citation{REVTEX41Control}}%
 \fi
}%
%    \end{macrocode}
% \end{macro}
% \end{macro}
%
% \begin{macro}{\printendnotes}
% \changes{4.1f}{2009/07/10}{(AO, 520) Automatically produce \cs{bibliography} command when needed}%
% We have removed the endnotes facility from REVTeX, so the \cmd\printendnotes\ command now does nothing.
%
% Moving footnotes to the bibliogrphy is now accomplished through 
% the automatic generation of a job BiB\TeX\ database (called \cmd\pre@bibdata) containing the footnotes.
%    \begin{macrocode}
\def\printendnotes{%
 \class@warn{The \string\printendnotes\space command no longer serves any function. Please remove it from your document.}%
}%
%    \end{macrocode}
% \end{macro}
%
% \begin{macro}{\@endnotesinbib}
% \begin{macro}{\@endnotesinbibliography}
% \changes{4.1a}{2008/01/18}{(AO, 457) Endnotes to be sorted in with numerical citations.}%
% We define a function \cmd\@endnotesinbib,
% and a variant \cmd\@endnotesinbibliography.
% The former is invoked at the start of 
% the end processing for \enve{thebibliography};
% the latter is a synonym.
%
% The procedure typesets the footnotes that are to 
% appear in the bibliography; the default is to
% simply arrange for the footnote counter to be
% reset at the start of the document.
%
% Note that this code make the assumption that 
% the counter used in \env{thebibliography} is \cmd\c@NAT@ctr.
% \changes{4.1f}{2009/07/10}{(AO, 520) Automatically produce \cs{bibliography} command when needed}%
%
% Here is the sole place where \cmd\footinbib@sw\ has an effect, other code simple assigning its value.
% If it is false, or \cmd\authoryear@sw\ is true, then footnotes are handled by the default mechanism.
%    \begin{macrocode}
\def\make@footnote@endnote{%
 \footinbib@sw{%
  \authoryear@sw{}{%
   \ltx@footnote@push
   \def\thempfn{Note\thefootnote}%
   \let\ltx@footmark\rev@citemark
   \let\ltx@foottext\rev@endtext
%    \end{macrocode}
% The endnotes facility has been removed. 
% Also, there is no need to queue up \cmd\auto@bib\ here, since it is always queued up elsewhere. 
% \changes{4.1f}{2009/07/10}{(AO, 520) Automatically produce \cs{bibliography} command when needed}%
%    \begin{verbatim}
%  \appdef\class@enddocumenthook{\auto@bib}%
%  \let\printendnotes\relax
%    \end{verbatim}
%    \begin{macrocode}
  }%
 }{}%
}%
%    \end{macrocode}
% 
%    \begin{macrocode}
\def\ltx@footnote@push{%
 \let\ltx@footmark@latex\ltx@footmark
 \let\ltx@foottext@latex\ltx@foottext
 \let\thempfn@latex\thempfn
 \def\ltx@footnote@pop{%
  \let\ltx@footmark\ltx@footmark@latex
  \let\ltx@foottext\ltx@foottext@latex
  \let\thempfn\thempfn@latex
 }%
}%
%    \end{macrocode}
%
% The switchover to setting footnotes in the bibliography
% changes the meaning of \cmd\footnote\ and
% substitutes the synonym for \cmd\@endnotesinbib. 
%
% We arrange for the procedure \cmd\make@footnote@endnote\ to be executed
% at \cmd\class@documenthook\ time (we mustn't do this earlier because
% the meaning of \cmd\@footnotemark\ must not be changed before then, 
% for the sake of \file{ltxutil.dtx}).
%    \begin{macrocode}
\appdef\class@documenthook{%
 \make@footnote@endnote
}%
%    \end{macrocode}
% \end{macro}
% \end{macro}
%
% \begin{macro}{\auto@bib}
% \begin{macro}{\auto@bib@empty}
% \begin{macro}{\test@bbl@sw}
% \begin{macro}{\bibitem@set}
% \begin{macro}{\auto@bib@innerbib}
% \begin{macro}{\thebibliography@nogroup}
% \changes{4.1f}{2009/07/10}{(AO, 520) Automatically produce \cs{bibliography} command when needed}%
% \changes{4.1n}{2010/01/02}{(AO, 572) Independent footnote counter for title block. Abstract footnote counter shared with body.}%
% \changes{4.1o}{2010/02/02}{(AO, 575) Automatically incorporate the (Bib\TeX-generated) .bbl into an explicit \env{thebibliography}}%
% Under some circumstances, we must typeset the bibliography automatically.
% If the document requires footnotes to be  set in the bibliography (effectively, class option \texttt{footinbib}), 
% or that frontmatter footnotes be set in the bibliography (effectively, class option \texttt{bibnotes}), 
% but contains no explicit \cmd\bibliography\ statement.
%
% Note that this facility is not able to work more than once per document.
% If multiple bibliographys are required (e.g., per article), it will be the responsibility of the journal style
% to restore \cmd\auto@bib\ to its original meaning so it can be re-invoked.
% 
% In procedure \cmd\auto@bib, we first test for the presence of frontmatter footnotes deferred to the bibliography.
% If none, we further test for the presence of \cmd\bibitem\ commands in the job's \filename{.bbl} file.
% If either condition is met, we ask for a bibliography. 
% We know that the document itself lacks a \cmd\bibliography\ statement, 
% so we know the argument of the \cmd\bibliography\ that we will issue.
%    \begin{macrocode}
\def\auto@bib{%
 \@ifx@empty\@FMN@list{%
  \footinbib@sw{%
   \@ifnum{\csname c@\@mpfn\endcsname>\z@}{%
    \true@sw
   }{%
    \test@bbl@sw
   }%
  }{%
   \test@bbl@sw
  }%
 }{%
  \true@sw
 }%
 {%
  \bibliography{}%
 }{}%
}%
\def\auto@bib@empty{%
 \let\auto@bib\@empty
}%
%    \end{macrocode}
% Testing the \filename{.bbl} file involves defanging all expected commands
% and processing that file inside a box register (that will be simply discarded).
% We provide a new meaning for the \cmd\bibitem\ command: it queues a Boolean.
%    \begin{macrocode}
\def\test@bbl@sw{%
 \setbox\z@\vbox\bgroup
  \let\providecommand\providecommand@j@nk
  \let\bibfield\@gobbletwo
  \let\bibinfo\@gobbletwo
  \let\translation\@gobble
  \let\BibitemOpen\@empty
  \let\bibitemStop\@empty
  \let\bibitemNoStop\@empty
  \let\EOS\@empty
  \let\BibitemShut\@gobble
  \let\bibAnnoteFile\@gobbletwo
  \let\bibAnnote\@gobblethree
  \let\textbf\@gobble
  \let\emph\@gobble
  \@booleanfalse\bibitem@sw
  \let\bibitem\bibitem@set
  \auto@bib@innerbib
  \bibitem@sw{\aftergroup\true@sw}{\aftergroup\false@sw}%
 \egroup
}%
%    \end{macrocode}
% The \cmd\bibitem@set\ is an alias for \cmd\bibitem\ for the purpose of detecting a non-trivial bibliography.
%    \begin{macrocode}
\newcommand\bibitem@set[1][]{%
 \bibitem@sw{}{%
  \@booleantrue\bibitem@sw
  \aftergroup\@booleantrue\aftergroup\bibitem@sw
 }%
}%
%    \end{macrocode}
% The \cmd\auto@bib@innerbib\ procedure reads in the \filename{.bbl} file (if it exists)
% within a context where its \env{thebibliography} environment does nothing, not even establishing a group. 
%    \begin{macrocode}
\def\auto@bib@innerbib{%
 \begingroup
  \let@environment{thebibliography}{thebibliography@nogroup}%
  \bibliography{}%
 \endgroup
}%
%    \end{macrocode}
% Environment \env{thebibliography@nogroup} is an alias of the \env{thebibliography} environment that cancels itself. 
% It assumes that it is called within a \env{thebibliography} environment. 
%    \begin{macrocode}
\def\thebibliography@nogroup#1{%
 \endgroup
 \def\@currenvir{thebibliography}%
}%
\def\endthebibliography@nogroup{\begingroup}%
%    \end{macrocode}
% The following should be part of \filename{ltxutil}.
%    \begin{macrocode}
\long\def \@gobblethree #1#2#3{}%
\def\providecommand@j@nk#1[#2]{%
 \@ifnum{#2=\z@}{\def\j@nk}{%
  \@ifnum{#2=\@ne}{\def\j@nk##1}{%
   \@ifnum{#2=\tw@}{\def\j@nk##1##2}{%
    \@ifnum{#2=\thr@@}{\def\j@nk##1##2##3}{%
    }%
   }%
  }%
 }%
}%
%    \end{macrocode}
% \end{macro}
% \end{macro}
% \end{macro}
% \end{macro}
% \end{macro}
% \end{macro}
%
% \section{Initial setup}
% 
% The standard LaTeX document classes execute certain commands 
% that are best deferred until \cmd\class@documenthook\ time.
% Here, we effectively split \cmd\pagenumbering\ into two halves,
% with a default definition for \cmd\thepage
% and an initialization of \cmd\c@page\ at \cmd\class@documenthook\ time.
%
% The meaning of \cmd\thepage\ can be overridden by society, journal, or
% anywhere within the document pramble, and the counter itself will be 
% preset at the beginning of the document. 
%    \begin{macrocode}
\def\thepage{\@arabic\c@page}%
%    \end{macrocode}
% \changes{4.0b}{1999/06/20}{Defer assignment until \cs{AtBeginDocument} time.}
% Note that this code is executed at \cmd\setup@hook\ time 
% to allow for the possibility of overrides by packages like \classname{geometry}.
%    \begin{macrocode}
\appdef\setup@hook{%
 \tabbingsep \labelsep
 \leftmargin\leftmargini
 \labelwidth\leftmargin\advance\labelwidth-\labelsep
 \let\@listi\@listI
 \@listi
}%
%    \begin{macrocode}
%
% We ensure that the ``environment'' component mark (implemented by \file{ltxgrid.dtx}) 
% is initialized properly (via a hook, itself defined via \file{ltxutil.dtx}). 
%    \begin{macrocode}
\appdef\class@documenthook{%
 \global\c@page\@ne
 \def\curr@envir{document}%
 \mark@envir{\curr@envir}%
}%
%    \end{macrocode}
%
% \begin{macro}{\open@onecolumn}%
% \begin{macro}{\open@column@two}%
% \changes{4.0c}{1999/11/13}{Grid changes with ltxgrid}
% \changes{4.0d}{2000/04/10}{Also alter how lists get indented.}
% When setting the column grid, we have to override the procedure
% for formatting lists. 
% Because \cmd\twocolumngrid\ requires rebalancing columns at some points, 
% typesetting must employ only the manipulation of 
% \cmd\leftskip\ and \cmd\rightskip, and must avoid the use of
% \cmd\moveleft, \cmd\moveright, and \cmd\parshape. 
%
% It is one of the stranger features of \TeX\ that these two separate
% mechanisms exist. The latter three have the effect of adding things to the 
% Main Vertical List that cannot be removed and later added back with all their
% properties intact.
%
% In detail, \cmd\moveleft, say, adds a box to the MVL with
% its reference point shifted horizontally by some amount relative to the 
% reference point of the enclosing list. If that box is removed from the 
% MVL (via a \cmd\lastbox\ operation in the output routine), and later
% thrown back to the MVL, the shift of the box will have been ``forgotten'' by \TeX.
% This is a bug, but not one ``acceptible to D. E. Knuth'', so it will never be fixed. 
%
%    \begin{macrocode}
\def\open@onecolumn{%
 \open@column@one\@ne
 \set@colht
 \@floatplacement
 \@dblfloatplacement
}%
\def\open@twocolumn{%
 \open@column@mlt\tw@
 \set@colht
 \@floatplacement
 \@dblfloatplacement
 \sloppy
 \let\set@listindent\set@listindent@
}%
%    \end{macrocode}
% \end{macro}
% \end{macro}
%
% \section{\cs{appendix}}
%
%    \begin{macrocode}
%\newif\ifappendixon
%    \end{macrocode}
% Note that, within appendices, 
% equations are numbered within sections (appendices).
%    \begin{macrocode}
\def\appendix{%
 \par
%\appendixontrue
 \setcounter{section}\z@
 \setcounter{subsection}\z@
 \setcounter{subsubsection}\z@
 \def\thesubsection{\arabic{subsection}}%
 \def\thesubsubsection{\alph{subsubsection}}%
 \@addtoreset{equation}{section}%
 \def\theequation@prefix{\thesection}%
 \addtocontents{toc}{\protect\appendix}%
 \@ifstar{%
  \def\thesection{\unskip}%
  \def\theequation@prefix{A.}%
 }{%
  \def\thesection{\Alph{section}}%
 }%
}%
%    \end{macrocode}
%
% \section{Changing the page grid}
% \changes{4.0b}{1999/06/20}{Define three separate environments, defer assignment to \cs{AtBeginDocument} time.}
%
% \subsection{Avoiding Grid Changes}%
%
% In preprint styles, ``wide text'' is a no-op, and the title page
% processing involves no grid change.
% 
% \begin{macro}{\title@column}%
% \begin{macro}{\close@column}%
% Provide default meanings for \cmd\title@column\ and \cmd\close@column,
% in case they were never defined. 
% Note that the society or journal substyle may define 
% \cmd\title@column\ or \cmd\close@column: this code will not override. 
%    \begin{macrocode}
\def\title@column#1{%
 \minipagefootnote@init
 #1%
 \minipagefootnote@foot
}%
\def\close@column{%
 \newpage
}%
%    \end{macrocode}
% \end{macro}
% \end{macro}
%
%
% \subsection{Galley Style: Margin Changes}%
% A variant of preprint processing. Emulate journal appearance somewhat.
%
% \begin{environment}{widetext@galley}
% DPC: We're in galley style so do a lob sided display environment.
%
% QUERY: How can we be sure that we are in galley style?
% ANSWER: as noted elsewhere, require that both \cmd\twocolumn@sw\ and \cmd\preprintsty@sw\ be false.
%    \begin{macrocode}
\def\galley@outdent{\rightmargin-\columnwidth\advance\rightmargin-\columnsep}%
\let\widetext@outdent\@empty
\newenvironment{widetext@galley}{%
  \list{}{%
    \topsep        \z@skip
    \listparindent \parindent
    \itemindent    \parindent
    \leftmargin    \z@
    \parsep        \z@\@plus\p@
    \widetext@outdent
    \relax
  }%
  \item\relax
}{
  \endlist
}%
%    \end{macrocode}
% \end{environment}
%
%
% \subsection{Grid Changing Via \classname{ltxgrid}}%
%
% In case \classoption{twocolumngrid} has been invoked,
% switch column grid using the column grid-changing commands.
% Supply stub definitions of those commands here.
% \changes{4.0c}{1999/11/13}{grid changes with ltxgrid}
%
% \begin{macro}{\title@column@grid}%
% \begin{macro}{\close@column@grid}%
% The title block always starts at the top of a new page.
% \changes{4.0c}{1999/11/13}{grid changes with ltxgrid}
%
% Note that, for the procedure \cmd\close@column@grid,
% we balance columns by switching to the one-column page grid. 
%    \begin{macrocode}
\def\title@column@grid#1{%
 \minipagefootnote@init
  \onecolumngrid
  \begingroup
   \let\@footnotetext\frontmatter@footnotetext
%<ignore>  \let\set@footnotewidth\set@footnotewidth@two
   \ltx@no@footnote
   #1%
  \endgroup
  \twocolumngrid
 \minipagefootnote@foot
}%
\def\close@column@grid{%
 \balancelastpage@sw{%
  \onecolumngrid
%<ignore>  \twocolumngrid
 }{}%
}%
%    \end{macrocode}
% \end{macro}
% \end{macro}
%
% \begin{environment}{widetext@grid}
% \changes{4.0c}{1999/11/13}{grid changes via ltxgrid procedures}
% \changes{4.0e}{2000/11/21}{adornments above and below.}
% \changes{4.1g}{2009/10/07}{(AO, 525) Remove phantom paragraph above display math that is given in vertical mode}%
% We slip into the one-column page grid
% within the scope of this environment.
%
% Note that we set adornments above and below the \env{widettext}.
% These are set as leaders, so they will disappear at a page break.
% \changes{4.1n}{2010/01/02}{(AO) fine-tune spacing above and below widetext}%
%    \begin{macrocode}
\newenvironment{widetext@grid}{%
  \par\ignorespaces
  \setbox\widetext@top\vbox{%
%<ignore>  \vskip15\p@
   \hb@xt@\hsize{%
    \leaders\hrule\hfil
    \vrule\@height6\p@
   }%
%<ignore>  \vskip6\p@
  }%
  \setbox\widetext@bot\hb@xt@\hsize{%
    \vrule\@depth6\p@
    \leaders\hrule\hfil
  }%
  \onecolumngrid
  \vskip10\p@
  \dimen@\ht\widetext@top\advance\dimen@\dp\widetext@top
  \cleaders\box\widetext@top\vskip\dimen@
%<ignore> \let\set@footnotewidth\set@footnotewidth@two
  \vskip6\p@
  \prep@math@patch
}{%
  \par
  \vskip6\p@
  \setbox\widetext@bot\vbox{%
   \hb@xt@\hsize{\hfil\box\widetext@bot}%
%<ignore>  \vskip14\p@
  }%
  \dimen@\ht\widetext@bot\advance\dimen@\dp\widetext@bot
  \cleaders\box\widetext@bot\vskip\dimen@
  \vskip8.5\p@
  \twocolumngrid\global\@ignoretrue
  \@endpetrue
}%
\newbox\widetext@top
\newbox\widetext@bot
%    \end{macrocode}
% \end{environment}
%
%
% Decide, finally, how the page grid is to be manipulated.
%    \begin{macrocode}
\def\set@page@grid{%
 \twocolumn@sw{%
%    \end{macrocode}
% The following two assignments determine what procedures are to be executed when 
% the footnote set width is calculated, and how footnotes are to be composed at the bottom of the page. 
% A society or journal wishing to do otherwise will override this code. 
%    \begin{macrocode}
  \let\set@footnotewidth\set@footnotewidth@two
  \let\compose@footnotes\compose@footnotes@two
  \let@environment{widetext}{widetext@grid}%
  \let\title@column\title@column@grid
  \let\close@column\close@column@grid
 }{%
  \let@environment{widetext}{widetext@galley}%
  \preprintsty@sw{%
%    \end{macrocode}
% Change the page grid not at all.
%    \begin{macrocode}
  }{%
%    \end{macrocode}
% If we are galley style, change the page margin only.
%    \begin{macrocode}
   \galley@sw{%
    \let\widetext@outdent\galley@outdent
   }{}%
  }%
 }%
}%
\appdef\setup@hook{\set@page@grid}%
%    \end{macrocode}
%
% \section{Old font commands}
%
%    \begin{macrocode}
\DeclareOldFontCommand{\rm}{\normalfont\rmfamily}{\mathrm}
\DeclareOldFontCommand{\sf}{\normalfont\sffamily}{\mathsf}
\DeclareOldFontCommand{\tt}{\normalfont\ttfamily}{\mathtt}
\DeclareOldFontCommand{\bf}{\normalfont\bfseries}{\mathbf}
\DeclareOldFontCommand{\it}{\normalfont\itshape}{\mathit}
\DeclareOldFontCommand{\sl}{\normalfont\slshape}{\@nomath\sl}
\DeclareOldFontCommand{\sc}{\normalfont\scshape}{\@nomath\sc}
\DeclareRobustCommand*\cal{\@fontswitch\relax\mathcal}
\DeclareRobustCommand*\mit{\@fontswitch\relax\mathnormal}
%    \end{macrocode}
%
% \section{English-Language Texts}
% As this class is just for English language journals, we
% could hardwire these texts,
% but to make it easier to use this as a
% basis for the code for similar journal styles, separate out all the
% fixed text strings into babel-style macros of the form 
% |\|\ldots|name|
%
% Note: for babel compatability, use version 1999/05/05 v3.6x or later.
%
% Some of these might need changing in the society-specific code.
%
% \begin{macro}{\today}
% Procedure \cmd\today\ is used in the article class, but not in 
% this document class.
%    \begin{macrocode}
\def\today{\ifcase\month\or
  January\or February\or March\or April\or May\or June\or
  July\or August\or September\or October\or November\or December\fi
  \space\number\day, \number\year}
%    \end{macrocode}
%    \end{macro}
%
% \begin{macro}{\notesname}
% Text entity \cmd\notesname\ had been used in \cmd\printendnotes.
% \changes{4.1f}{2009/07/10}{(AO, 520) Automatically produce \cs{bibliography} command when needed}%
% However, we have removed the endnotes facility from REVTeX.
% 
%    \begin{verbatim}
%\def\notesname{Notes}
%    \end{verbatim}
%    \end{macro}
%
% \begin{macro}{\partname}
% Text entity \cmd\partname\ is used in \cmd\@part.
%    \begin{macrocode}
\def\partname{Part}
%    \end{macrocode}
%    \end{macro}
%
% \begin{macro}{\tocname}
% Text entity \cmd\tocname\ is used in \cmd\tableofcontents,
% as defined in the standard \LaTeX\ book class.
%    \begin{macrocode}
\def\tocname{Contents}
%    \end{macrocode}
%    \end{macro}
%
% \begin{macro}{\lofname}
% Text entity \cmd\lofname\ is used in \cmd\listoffigures,
% as defined in the standard \LaTeX\ book class.
%    \begin{macrocode}
\def\lofname{List of Figures}
%    \end{macrocode}
%    \end{macro}
%
% \begin{macro}{\lotname}
% Text entity \cmd\lotname\ is used in \cmd\listoftables,
% as defined in the standard \LaTeX\ book class.
%    \begin{macrocode}
\def\lotname{List of Tables}
%    \end{macrocode}
%    \end{macro}
%
% \begin{macro}{\refname}
% Text entity \cmd\refname\ is used in \env{thebibliography}.
%    \begin{macrocode}
\def\refname{References}
%    \end{macrocode}
%    \end{macro}
%
% \begin{macro}{\indexname}
% Text entity \cmd\indexname\ is used in \env{theindex},
% as defined in the standard \LaTeX\ book class.
%    \begin{macrocode}
\def\indexname{Index}
%    \end{macrocode}
%    \end{macro}
%
% \begin{macro}{\figurename}
% \changes{4.0b}{1999/06/20}{Added localization of \cs{figuresname}}
% Text entity \cmd\figurename\ is used in \env{figure},
% \cmd\figuresname\ in \cmd\printfigures.
%    \begin{macrocode}
\def\figurename{FIG.}
\def\figuresname{Figures}%
%    \end{macrocode}
%    \end{macro}
%
% \begin{macro}{\tablename}
% \changes{4.0b}{1999/06/20}{Added localization of \cs{tablesname}}
% Text entity \cmd\tablename\ is used in \env{table},
% \cmd\tablesname\ in \cmd\printtables.
%    \begin{macrocode}
\def\tablename{TABLE}
\def\tablesname{Tables}%
%    \end{macrocode}
%    \end{macro}
%
% \begin{macro}{\abstractname}
% Text entity \cmd\abstractname\ is used in \env{abstract}.
%    \begin{macrocode}
\def\abstractname{Abstract}
%    \end{macrocode}
%    \end{macro}
%
% \begin{macro}{\appendixesname}
% \begin{macro}{\appendixname}
% Text entity \cmd\appendixesname\ is used in TOC.
%    \begin{macrocode}
\def\appendixesname{Appendixes}%
\def\appendixname{Appendix}%
%    \end{macrocode}
%    \end{macro}
%    \end{macro}
%
% \begin{macro}{\acknowledgmentsname}
% Text entity \cmd\acknowledgmentsname\ is used in \env{acknowledgments}.
%    \begin{macrocode}
\def\acknowledgmentsname{Acknowledgments}
%    \end{macrocode}
%    \end{macro}
%
% \begin{macro}{\journalname}
% This should be set by the society journal options, eg `pra'.
%    \begin{macrocode}
\def\journalname{??}
%    \end{macrocode}
%    \end{macro}
%
% \begin{macro}{\copyrightname}
% Default layout does not assign copyright, but a journal that wants
% to might use this.
%    \begin{macrocode}
\def\copyrightname{??}
%    \end{macrocode}
%    \end{macro}
%
% \begin{macro}{\andname}
% The text string ``and'' for use in author lists.
%    \begin{macrocode}
\def\andname{and}
%    \end{macrocode}
%    \end{macro}
%
% \begin{macro}{\@pacs@name}
% \begin{macro}{\@keys@name}
% The text string prepended to PACS numbers, resp. to keywords.
%    \begin{macrocode}
\def\@pacs@name{PACS numbers: }%
\def\@keys@name{Keywords: }%
%    \end{macrocode}
%    \end{macro}
%    \end{macro}
%
% \begin{macro}{\ppname}
% The text string ``pp'' for use in page ranges.
%    \begin{macrocode}
\def\ppname{pp}
%    \end{macrocode}
%    \end{macro}
%
% \begin{macro}{\numbername}
% The text string ``number'' for use in article reference.
%    \begin{macrocode}
\def\numbername{number}
%    \end{macrocode}
%    \end{macro}
%
% \begin{macro}{\volumename}
% The text string ``volume'' for use in article reference.
%    \begin{macrocode}
\def\volumename{volume}
%    \end{macrocode}
%    \end{macro}
%
% \begin{macro}{\Dated@name}
% \begin{macro}{\Received@name}
% \begin{macro}{\Revised@name}
% \begin{macro}{\Accepted@name}
% \begin{macro}{\Published@name}
% These texts are used in the \cmd\date, et al. commands.
%    \begin{macrocode}
\def\Dated@name{Dated: }%
\def\Received@name{Received }%
\def\Revised@name{Revised }%
\def\Accepted@name{Accepted }%
\def\Published@name{Published }%
%    \end{macrocode}
%    \end{macro}
%    \end{macro}
%    \end{macro}
%    \end{macro}
%    \end{macro}
%
% \section{Legacy Commands}
% We define some commands left over from version 3.1, or give default meanings.
% Some definitions can be overridden
% in the document preamble or in included packages.
%
% Note on the namespace: command names like \cmd\REV@\emph{name} are used here,
% because it is not clear that any of this code is generally useful.
%    \begin{macrocode}
\def\address{\replace@command\address\affiliation}%
\def\altaddress{\replace@command\altaddress\altaffiliation}%
\newenvironment{references}{%
 \class@warn@end{The references environment is not supported; use thebibliography instead.}
 \gdef\references{\thebibliography{}}\references
}{%
 \endthebibliography
}%
\def\draft{%
 \class@warn@end{Command \string\draft\space is obsolete;^^JInvoke option draft instead.}%
 \@booleantrue\draft@sw
}%
\def\tighten{%
 \class@warn@end{Command \string\tighten\space is obsolete;^^JInvoke option tightenlines instead.}%
 \@booleantrue\tightenlines@sw
}%
\def\tableline{%
 \noalign{%
  \class@warn@end{Command \string\tableline\space is obsolete;^^JUse \string\colrule\space instead.}%
  \global\let\tableline\colrule
 }%
 \tableline
}%
\def\case{\replace@command\case\frac}%
\def\slantfrac{\replace@command\slantfrac\frac}%
\def\tablenote{\replace@command\tablenote\footnote}%
\def\tablenotemark{\replace@command\tablenotemark\footnotemark}%
\def\tablenotetext{\replace@command\tablenotetext\footnotetext}%
% Lose the following definition:
\DeclareRobustCommand\REV@text[1]{%
 \relax
 \ifmmode
  \mathchoice
   {\hbox{{\everymath{\displaystyle     }#1}}}%
   {\hbox{{\everymath{\textstyle        }#1}}}%
   {\hbox{{\everymath{\scriptstyle      }\let\f@size\sf@size\selectfont#1}}}%
   {\hbox{{\everymath{\scriptscriptstyle}\let\f@size\ssf@size\selectfont#1}}}%
  \glb@settings
 \else
  \mbox{#1}%
 \fi
}%
% Lose the following definition:
\DeclareRobustCommand\REV@bbox[1]{%
 \relax
 \ifmmode
  \mathchoice
   {\hbox{{\everymath{\displaystyle     }\boldmath$#1$}}}%
   {\hbox{{\everymath{\textstyle        }\boldmath$#1$}}}%
   {\hbox{{\everymath{\scriptstyle      }\boldmath$#1$}}}%
   {\hbox{{\everymath{\scriptscriptstyle}\boldmath$#1$}}}%
  \glb@settings
 \else
  \mbox{#1}%
 \fi
}%
\DeclareRobustCommand\REV@bm[1]{%
 \class@warn@end{To use \string\bm, please load the bm package!}%
 \global\let\bm\relax
}%
\def\FL{\obsolete@command\FL}%
\def\FR{\obsolete@command\FR}%
\def\narrowtext{\obsolete@command\narrowtext}%
\def\mediumtext{\obsolete@command\mediumtext}%
\newenvironment{quasitable}{%
 \let@environment{tabular}{longtable}%
}{%
}%
%    \end{macrocode}
%
% \begin{macro}{\text}
% \begin{macro}{\bm}
% \begin{macro}{\bibinfo}
% \begin{macro}{\eprint}
% \begin{macro}{\url}
% 
% If not otherwise defined, give default meanings to certain commands.
% \changes{4.0d}{2000/04/10}{eprint takes an optional argument, syntactical only in this case.}
%FIXME: \cmd\bibinfo?
%    \begin{macrocode}
\let\text\REV@text
\let\bm\REV@bm
\appdef\setup@hook{%
 \providecommand\bibinfo[2]{#2}%
 \providecommand\eprint[2][]{#2}%
%\providecommand\url[1]{#1}%
}%
%    \end{macrocode}
% \end{macro}
% \end{macro}
% \end{macro}
% \end{macro}
% \end{macro}
%
% \begin{macro}{\bbox}
% 
%    \begin{macrocode}
\def\bbox#1{%
 \class@warn@end{\string\bbox\space is obsolete,^^Jload the bm package and use \string\bm\space instead.}%
 \global\let\bbox\relax
}%
%    \end{macrocode}
% \end{macro}
%
% \begin{macro}{\mathletters}
%
%    \begin{macrocode}
\newenvironment{mathletters}{%
 \class@warn@end{Environment {mathletters} is obsolete;^^Jload the amsmath package and use {subequations}!}%
 \global\let\mathletters\@empty
}{%
}%
%    \end{macrocode}
% \end{macro}
%
% \begin{macro}{\eqnum}
% 
%    \begin{macrocode}
\def\eqnum#1{%
 \class@warn@end{\string\eqnum\space is obsolete, load the amsmath package and use \string\tag!}%
 \global\let\eqnum\@gobble
}%
%    \end{macrocode}
% \end{macro}
%
% We read in the symbol definitions.
%    \begin{macrocode}
\appdef\rtx@require@packages{%
 \RequirePackage{revsymb4-1}%
}%
\appdef\class@documenthook{\revsymb@inithook}%
%    \end{macrocode}
%
% \section{Patches for lineno.sty}
%
% The \classname{lineno} package detects the case where the package has been loaded
% and the document invokes \cmd\linelabel, but the \cmd\linenumbers\ command has not been
% issued: it treats this case as an error. 
%
% It is wrong for validity of document syntax to be dependent upon package semantics: 
% we make the condition a warning rather than an error.  
% 
%    \begin{macrocode}
\def\@LN@LLerror@org{%
 \PackageError{lineno}{%
  \string\linelabel\space without \string\linenumbers
 }{% 
  Just see documentation. (New feature v4.11)%
 }%
 \@gobble
}%
\def\@LN@LLerror@ltx{%
 \PackageWarning{lineno}{%
  To make the \string\linelabel\space command work, you must issue the \string\linenumbers\ command 
 }%
 \@gobble
}%
%    \end{macrocode}
% 
% If appropriate, enable line numbering within the abstract. 
%
% This mechanism applies generally:
% Create the box in a context in which the meaning of \cmd\par\ has been patched by \classname{lineno},
% then \cmd\unvbox\ the box in a context where \cmd\set@linepenalties\ has been executed, and 
% follow up with \cmd\@linenumberpar, which forces a visit to the output routine just there.
% Note that here, we have to de-fang \cmd\@LN@parpgbrk, which would otherwise causes the appearance
% of a box with depth -1000 points. Go figure.
%    \begin{macrocode}
\appdef\class@documenthook{%
 \@ifx{\@LN@LLerror\@LN@LLerror@org}{%
  \class@info{Overriding \string\@LN@LLerror}%
  \let\@LN@LLerror\@LN@LLerror@ltx
 }{}%
 \@ifpackageloaded{lineno}{%
  \@ifxundefined{\set@linepenalties}{}{%
   \def\prep@absbox{\set@linepenalties}%
   \def\post@absbox{\let\@LN@parpgbrk\@empty\@linenumberpar}%
  }%
 }{}%
}%
%    \end{macrocode}
%
% One may well ask: how to obtain line numbering within an alignment in a float?
% This objective, along with line numbering within footnotes, would require extraordinary measures.
% The float would have to be thrown onto the MVL in order to acquire its line numbers, but
% that fragment of MVL would then have to be protected from being shipped out.
% The question of how to coordinate those lines' numbers with those of lines in the MVL would 
% also require dealing with.
%
% \section{Endgame for the Document Class}
%
% We provide for a ``job macro package'' that can override
% definitions and assignments made by the class or any other packages it loads.
%
% \subsection{Job Macro Package}
% You can create a ``job macro package'' for your document
% that will be read in automatically every time
% your document is processed.
% Thus, if your job is a file called \file{myarticle.tex},
% then the file \file{myarticle.rty} will be read in
% just the same as if you had placed a 
% \cmd\usepackage|{myarticle.rty}| statement
% immediately following your \cmd\documentclass\ statement.
%
% Within your \file{.rty} file,
% you can define and use control sequence names that use the |@| character
% and you can override any of the definitions or assignments made 
% by the \revtex\ document class or the selected journal substyle.
% That is, you have the power to really mess things up badly.
%
% If you choose to have a job macro package, you are well 
% advised to read the \LaTeX\ guide to document classes,
% \file{clsguide.tex} or read up on the subject 
% in a book like the \LaTeX\ Companion.
%
% The file \file{template.rty} contains a template for
% creating your own job macro package.
%
%    \begin{macrocode}
\appdef\rtx@require@packages{%
 \InputIfFileExists{\jobname.rty}{}{}%
}%
%    \end{macrocode}
%
% \subsection{Endgame Processing for the Document Class}
%
% The remaining steps in processing the document class involve
% determining the needed society, journal, and pointsize from 
% the document's class options and inputting the needed files or
% executing the indicated procedures. 
%
% Note that the society file is expected to declare options that will
% allow us to determine the journal involved, and the society and journal
% themselves determine the which pointsize options are declared, along with
% their meanings. 
% 
% Note also that required packages are read in only after the document options
% have been processed, because the latter can affect the former. 
% 
% Finally, the setup code is executed: this is code that depends on the 
% meanings of the switches we define and on the code within the packages we load. 
%
% Note that there are other hooks in use: \cmd\document@inithook, which is 
% executed right at the beginning of the document, and \cmd\class@documenthook, 
% which serves as a vehicle for any \cmd\AtBeginDocument\ code we might wish to
% execute. 
%
%FIXME: use \cmd\class@documenthook\ only for things that bear on the MVL; 
% use \cmd\document@inithook\ for all patches to procedures defined within the preamble. 
%
% Remember that \classname{natbib} changes its state at \cmd\AtBeginDocument\ time,
% so we have to install our own code at a later point in the processing.
%
% We determine the proper \cmd\@society\ by examining the document's class options. 
%    \begin{macrocode}
\@parse@class@options@society
%    \end{macrocode}
%
% Then, we input the society's substyle
% (which may in turn lead to loading a journal substyle or a pointsize substyle). 
% The substyle should not assume the value of any class option: 
% instead, it should install code into \cmd\setup@hook. 
%    \begin{macrocode}
\@process@society{aps}%
%    \end{macrocode}
%
% Now that the society has defined the class options relating to journals,
% and has defined \cmd\@journal@default, we can process the journal substyle. 
% We parse the options for one that sets \cmd\@journal.
%    \begin{macrocode}
\@parse@class@options@\@journal
%    \end{macrocode}
% And we process the journal. 
% Note that \emph{it is an error} for a society file to fail to define
% \cmd\@journal@default.
%
%    \begin{macrocode}
\expandafter\@process@journal\expandafter{\@journal@default}%
%    \end{macrocode}
%
% Now that the society and journal have finished defining any options relating to 
% point size, we process the class options for any that set \cmd\@pointsize.
%    \begin{macrocode}
\@parse@class@options@\@pointsize
%    \end{macrocode}
% And we process the pointsize.
% Note that it is an error for the society and journal to leave \cmd\@pointsize@default\ 
% undefined at this point, however, the journal may have overriden the assignment of the society. 
%
%    \begin{macrocode}
\expandafter\@process@pointsize\expandafter{\@pointsize@default}%
%    \end{macrocode}
%
% Next, we process the class options for once and all. 
% Doing so sets values for some of the Booleans that were introduced along with
% the \cmd\DeclareOption\ statements above. 
%
%CHANGE: We process the options in the order declared in the document; this 
% gives the document greater control. 
%
%    \begin{macrocode}
\@options
%    \end{macrocode}
% 
% Now that the class options have been processed, we can load all the 
% packages that we know need loading.
%    \begin{macrocode}
\rtx@require@packages
%    \end{macrocode}
%
% At this point, the society substyle, the journal substyle, and the 
% pointsize have all been processed, along with the document class options.
% Some of these have left things for later; we do these now.
%
% \begin{macro}{\setup@hook}
%
% This portion of the code for this class file
% \emph{must} appear at the very end:
% The procedure 
% \cmd\setup@hook\
% should be executed at the very end of the class file. 
% Any code that relies on the value of any of the \texttt{@sw} switches
% or will patch the code of one of the required packages
% should be executed here. 
%
%    \begin{macrocode}
\setup@hook
%    \end{macrocode}
% \end{macro}
%
%
% Warn if past maturation date. This code to be enabled only in beta software.
% \changes{4.1f}{2009/07/11}{(AO, 522) Warn if software is expired}%
% \changes{4.1h}{2009/10/09}{(AO) Remove expiry code in the release software}%
%    \begin{macrocode}
%<*ignore>
 \def\rtx@fin@year{2010}%
 \def\rtx@fin@month{01}%
 \def\rtx@fin@day{01}%
 \def\rtx@fin@warn{%
  \@ifnum{\rtx@fin@year>\the\year\relax}{\true@sw}{%
   \@ifnum{\rtx@fin@month>\the\month\relax}{\true@sw}{%
    \@ifnum{\rtx@fin@day>\the\day\relax}{\true@sw}{%
     \false@sw
    }%
   }%
  }%
  {%
   \class@info{Beta software expires \rtx@fin@year-\rtx@fin@month-\rtx@fin@day; updates available at http://publish.aps.org/revtex4/}%
  }{%
   \class@warn{Outdated software expired \rtx@fin@year-\rtx@fin@month-\rtx@fin@day; please retrieve an update at http://publish.aps.org/revtex4/}%
  }%
 }%
%</ignore>
%    \end{macrocode}
% In shipping (non-beta) software, the following line should be commented out.
% \changes{4.1f}{2009/07/11}{(AO, 522) Warn if software is expired}%
%    \begin{verbatim}
%\appdef\class@enddocumenthook{\rtx@fin@warn}%
%    \end{verbatim}
%
% End of the class file.
%    \begin{macrocode}
%</package>
%    \end{macrocode}
%
%
% \section{Symbols: the \texttt{revsymb} module}
% We immediately define a utility command: this module's warning.
%    \begin{macrocode}
%<*revsymb>
\def\REVSYMB@warn#1{\PackageWarningNoLine{revsymb}{#1}}%
%    \end{macrocode}
%
% \begin{macro}{\lambdabar}
%    \begin{macrocode}
\DeclareRobustCommand\lambdabar{%
  \bgroup
    \def\@tempa{%
      \hbox{%
        \raise.73\ht\z@
        \hb@xt@\z@{%
          \kern.25\wd\z@
          \vrule \@width.5\wd\z@\@height.1\p@\@depth.1\p@
          \hss
        }%
        \box\z@
      }%
    }%
    \mathchoice
      {\setbox\z@\hbox{$\displaystyle     \lambda$}\@tempa}%
      {\setbox\z@\hbox{$\textstyle        \lambda$}\@tempa}%
      {\setbox\z@\hbox{$\scriptstyle      \lambda$}\@tempa}%
      {\setbox\z@\hbox{$\scriptscriptstyle\lambda$}\@tempa}%
  \egroup
}%
%    \end{macrocode}
% \end{macro}
%
% \begin{macro}{\openone}
% DPC: Really should use a font that includes this glyph.
% Unfortunately not in AMS ones, but is in bbold, cmbb.
% (I think, must check),
% \changes{4.0a}{1998/01/31}{use font-dependent spacing}
% FIXME: check for bbold.
%    \begin{macrocode}
\DeclareRobustCommand\openone{\leavevmode\hbox{\small1\normalsize\kern-.33em1}}%
%    \end{macrocode}
% \end{macro}
%
% \begin{macro}{\corresponds}
% \begin{macro}{\overdots}
% \begin{macro}{\overcirc}
% J\"org Knappen suggests the replacements:
% replace \cmd\corresponds\ with \cmd\triangleq, source \classname{amssymb};
% replace \cmd\overcirc\ with \cmd\mathring, source \classname{latex2e};
% replace \cmd\overdots\ with \cmd\dddot, source \classname{amsmath}.
%
% Any use of any of these commands will result in a warning message at the end 
% of the log file. If the corresponding package is not loaded, a
% definition will quietly be provided.
%    \begin{macrocode}
\DeclareRobustCommand\corresponds{\replace@command\corresponds\triangleq}%
\DeclareRobustCommand\overcirc{\replace@command\overcirc\mathring}%
\DeclareRobustCommand\overdots{\replace@command\overdots\dddot}%
\DeclareRobustCommand\REV@triangleq{%
 {\lower.2ex\hbox{=}}{\kern-.75em^\triangle}%
}%
\DeclareRobustCommand\REV@dddot[1]{%
 \@ontopof{#1}{\cdots}{1.0}\mathord{\box2}%
}%
%    \end{macrocode}
% \end{macro}
% \end{macro}
% \end{macro}
%
% \begin{macro}{\succsim}
% \begin{macro}{\precsim}
% \begin{macro}{\lesssim}
% \begin{macro}{\gtrsim}
% \begin{macro}{\alt}
% \begin{macro}{\agt}
% These version 3.1 commands are always supplied, 
% but the definitions in \classname{amssymb} are preferred.
%    \begin{macrocode}
\DeclareRobustCommand\altsuccsim{\succ\kern-.9em_\sim\kern.3em}%
\DeclareRobustCommand\altprecsim{\prec\kern-1em_\sim\kern.3em}%
\let\REV@succsim\altsuccsim
\let\REV@precsim\altprecsim
\DeclareRobustCommand\REV@lesssim{\mathrel{\mathpalette\vereq{<}}}%
\DeclareRobustCommand\REV@gtrsim{\mathrel{\mathpalette\vereq{>}}}%
\DeclareRobustCommand\alt{\lesssim}
\DeclareRobustCommand\agt{\gtrsim}
\def\vereq#1#2{%
 \lower3\p@\vbox{%
  \baselineskip1.5\p@
  \lineskip1.5\p@
  \ialign{$\m@th#1\hfill##\hfil$\crcr#2\crcr\sim\crcr}%
 }%
}%
%    \end{macrocode}
% \end{macro}
% \end{macro}
% \end{macro}
% \end{macro}
% \end{macro}
% \end{macro}
%
% \begin{macro}{\tensor}
% \begin{macro}{\overstar}
% \begin{macro}{\loarrow}
% \begin{macro}{\roarrow}
% \changes{4.0b}{1999/06/20}{AO: remove duplicates}
%    \begin{macrocode}
\DeclareRobustCommand\tensor[1]{\@ontopof{#1}{\leftrightarrow}{1.15}\mathord{\box2}}
\DeclareRobustCommand\overstar[1]{\@ontopof{#1}{\ast}{1.15}\mathord{\box2}}
\DeclareRobustCommand\loarrow[1]{\@ontopof{#1}{\leftarrow}{1.15}\mathord{\box2}}
\DeclareRobustCommand\roarrow[1]{\@ontopof{#1}{\rightarrow}{1.15}\mathord{\box2}}
%    \end{macrocode}
% \end{macro}
% \end{macro}
% \end{macro}
% \end{macro}
%
% \begin{macro}{\@ontopof}
%    \begin{macrocode}
\def\@ontopof#1#2#3{%
 {%
  \mathchoice
    {\@@ontopof{#1}{#2}{#3}\displaystyle     \scriptstyle      }%
    {\@@ontopof{#1}{#2}{#3}\textstyle        \scriptstyle      }%
    {\@@ontopof{#1}{#2}{#3}\scriptstyle      \scriptscriptstyle}%
    {\@@ontopof{#1}{#2}{#3}\scriptscriptstyle\scriptscriptstyle}%
 }%
}%
%    \end{macrocode}
% \end{macro}
%
% \begin{macro}{\@@ontopof}
% Same as \revtex3, more or less.
%    \begin{macrocode}
\def\@@ontopof#1#2#3#4#5{%
  \setbox\z@\hbox{$#4#1$}%
  \setbox\f@ur\hbox{$#5#2$}%
  \setbox\tw@\null\ht\tw@\ht\z@ \dp\tw@\dp\z@
  \@ifdim{\wd\z@>\wd\f@ur}{%
    \setbox\f@ur\hb@xt@\wd\z@{\hss\box\f@ur\hss}%
    \mathord{\rlap{\raise#3\ht\z@\box\f@ur}\box\z@}%
  }{%
    \setbox\f@ur\hb@xt@.9\wd\f@ur{\hss\box\f@ur\hss}%
    \setbox\z@\hb@xt@\wd\f@ur{\hss$#4\relax#1$\hss}%
    \mathord{\rlap{\copy\z@}\raise#3\ht\z@\box\f@ur}%
  }%
}%
%    \end{macrocode}
% \end{macro}
%
% \begin{macro}{\frak}
% Deal with legacy \cmd\frak:
% if \classname{amsfonts} not loaded, defined in such a way as to ask for that package.
% Also, says to use \cmd\mathfrak\ instead.
%    \begin{macrocode}
\DeclareRobustCommand\frak{%
 \REVSYMB@warn{%
  Command \string\frak\space unsupported:^^J%
  please use \string\mathfrak\space instead.%
 }%
 \global\let\frak\mathfrak
 \frak
}%
\DeclareRobustCommand\REV@mathfrak{%
 \REVSYMB@warn{%
  Command \string\mathfrak\space undefined:^^J%
  please specify the amsfonts or amssymb option!%
 }%
 \global\let\mathfrak\@firstofone
 \mathfrak
}%
%    \end{macrocode}
% \end{macro}
%
% \begin{macro}{\Bbb}
% Deal with legacy \cmd\Bbb:
% if \classname{amsfonts} not loaded, defined in such a way as to ask for that package.
% Also, says to use \cmd\mathbb\ instead.
%    \begin{macrocode}
\DeclareRobustCommand\Bbb{%
 \REVSYMB@warn{%
  Command \string\Bbb\space unsupported:^^J%
  please use \string\mathbb\space instead.%
 }%
 \global\let\Bbb\mathbb
 \Bbb
}%
\DeclareRobustCommand\REV@mathfrak{%
 \REVSYMB@warn{%
  Command \string\mathbb\space undefined:^^J%
  please specify the amsfonts or amssymb option!%
 }%
 \global\let\mathbb\@firstofone
 \mathbb
}%
%    \end{macrocode}
% \end{macro}
%
% \begin{macro}{\Bigglb}
% Deal with legacy bold delimiters. 
% Each of the following takes an implicit argument consisting of
% the delimiter to be made big and bold.
% FIXME: \cmd\DeclareBoldMathCommand\ is not the right tool!
%    \begin{macrocode}
\def\Bigglb{\REV@boldopen \Bigg}%
\def\Biglb {\REV@boldopen \Big }%
\def\bigglb{\REV@boldopen \bigg}%
\def\biglb {\REV@boldopen \big }%
\def\Biggrb{\REV@boldclose\Bigg}%
\def\Bigrb {\REV@boldclose\Big }%
\def\biggrb{\REV@boldclose\bigg}%
\def\bigrb {\REV@boldclose\big }%
\def\REV@pmb#1{%
 \hbox{%
  \setbox\z@=\hbox{#1}%
  \kern-.02em\copy\z@\kern-\wd\z@
  \kern .04em\copy\z@\kern-\wd\z@
  \kern-.02em
  \raise.04em\copy\z@
 }%
}%
\def\REV@boldopen #1#2{\mathopen {\REV@pmb{$#1#2$}}}%
\def\REV@boldclose#1#2{\mathclose{\REV@pmb{$#1#2$}}}%
%    \end{macrocode}
% \end{macro}
%
% \begin{macro}{\revsymb@inithook}
% \changes{4.1b}{2008/07/01}{(AO) make settings at class time instead of deferring them to later.}
% Package dependencies are taken care of at \cmd\setup@hook\ time.
%    \begin{macrocode}
\def\revsymb@inithook{%
 \@ifxundefined\dddot{\let\dddot\REV@dddot}{}%
 \@ifxundefined\triangleq{\let\triangleq\REV@triangleq}{}%
 \@ifxundefined\succsim{\let\succsim\altsuccsim}{}%
 \@ifxundefined\precsim{\let\precsim\altprecsim}{}%
 \@ifxundefined\lesssim{\let\lesssim\REV@lesssim}{}%
 \@ifxundefined\gtrsim {\let\gtrsim \REV@gtrsim }{}%
 \@ifxundefined\mathfrak{\let\mathfrak\REV@mathfrak}{}%
 \@ifxundefined\mathbb{\let\mathbb\REV@mathbb}{}%
}%
%    \end{macrocode}
% \end{macro}
%
%    \begin{macrocode}
%</revsymb>
%    \end{macrocode}
%
% \section{The \classoption{10pt} class option: the \texttt{10pt} module}
% \changes{4.0b}{1999/06/20}{AO: all code for \protect\classoption{10pt} is in this module.}
% The file \file{aps10pt.rtx} is read in by the \classname{revtex4} document class
% if \cmd\@pointsize\ has the value 10.
%
%    \begin{macrocode}
%<*10pt>
%    \end{macrocode}
%
% \subsection{Defend Against Forseeable Errors}%
% Protect this file from being read in by anything but \revtex.
%    \begin{macrocode}
\ifx\undefined\substyle@ext
 \def\@tempa{%
  \endinput
  \GenericWarning{I must be read in by REVTeX! (Bailing out)}%
 }%
 \expandafter\else
  \def\@tempa{}%
 \expandafter\fi\@tempa
 \class@info{RevTeX pointsize 10pt selected}%
%    \end{macrocode}
%
%    \begin{macrocode}
\def\normalsize{%
   \@setfontsize\normalsize\@xpt{11.5}%
   \abovedisplayskip 10\p@ \@plus2\p@ \@minus5\p@
   \belowdisplayskip \abovedisplayskip
   \abovedisplayshortskip  \abovedisplayskip
   \belowdisplayshortskip \abovedisplayskip
   \let\@listi\@listI
}%
%    \end{macrocode}
%
%    \begin{macrocode}
\def\small{%
  \@setfontsize\small\@ixpt{10.5}%
  \abovedisplayskip 8.5\p@ \@plus3\p@ \@minus4\p@
  \belowdisplayskip \abovedisplayskip
  \abovedisplayshortskip \z@ \@plus2\p@
  \belowdisplayshortskip 4\p@ \@plus2\p@ \@minus2\p@
  \def\@listi{%
    \leftmargin\leftmargini
    \topsep 4\p@ \@plus2\p@ \@minus2\p@
    \parsep 2\p@ \@plus\p@ \@minus\p@
    \itemsep \parsep
  }%
}%
%    \end{macrocode}
%
%    \begin{macrocode}
\def\footnotesize{%
  \@setfontsize\footnotesize\@viiipt{9.5pt}%
  \abovedisplayskip 6\p@ \@plus2\p@ \@minus4\p@
  \belowdisplayskip \abovedisplayskip
  \abovedisplayshortskip \z@ \@plus\p@
  \belowdisplayshortskip 3\p@ \@plus\p@ \@minus2\p@
  \def\@listi{%
    \leftmargin\leftmargini
    \topsep 3\p@ \@plus\p@ \@minus\p@
    \parsep 2\p@ \@plus\p@ \@minus\p@
    \itemsep \parsep
  }%
}%
%    \end{macrocode}
%
%    \begin{macrocode}
\def\scriptsize{%
 \@setfontsize\scriptsize\@viipt\@viiipt
}%
\def\tiny{%
 \@setfontsize\tiny\@vpt\@vipt
}%
\def\large{%
 \@setfontsize\large\@xiipt{14pt}%
}%
\def\Large{%
 \@setfontsize\Large\@xivpt{18pt}%
}%
\def\LARGE{%
 \@setfontsize\LARGE\@xviipt{22pt}%
}%
\def\huge{%
 \@setfontsize\huge\@xxpt{25pt}%
}%
\def\Huge{%
 \@setfontsize\Huge\@xxvpt{30pt}%
}%
%    \end{macrocode}
%
% The values of these margin parameters are dependent upon 
% \cmd\twoside@sw; any society or journal that 
% has its own preferences should override these assignments
% by doing \cmd\appdef\cmd\setup@hook.
%    \begin{macrocode}
\appdef\setup@hook{%
 \twoside@sw{%
%   \oddsidemargin  -.1in
%   \evensidemargin -.4in
    \oddsidemargin  -20pt
    \evensidemargin -20pt
    \marginparwidth 107pt
 }{%
    \oddsidemargin  -.25in
    \evensidemargin -.25in
    \marginparwidth 30pt
 }%
}%
%    \end{macrocode}
%
%    \begin{macrocode}
\marginparsep 6pt
%    \end{macrocode}
%
%    \begin{macrocode}
\topmargin -61pt
%    \end{macrocode}
%
%    \begin{macrocode}
\headheight 25pt
\headsep 16pt
%    \end{macrocode}
%
% \changes{4.1n}{2010/01/02}{(AO, 571) class file must set \cs{splittopskip}; fine tune \cs{skip}\cs{footins}; \cs{footnoterule} defined in terms of \cs{skip}\cs{footins}}%
%    \begin{macrocode}
\topskip 10pt
\splittopskip\topskip
%    \end{macrocode}
%
%    \begin{macrocode}
\footskip 30pt
%    \end{macrocode}
% 
%    \begin{macrocode}
 \textheight = 56pc
%    \end{macrocode}
%
%    \begin{macrocode}
\textwidth42.5pc
%    \end{macrocode}
%
%    \begin{macrocode}
\columnsep 1.5pc
\columnseprule 0pt
%    \end{macrocode}
%
% \changes{4.1n}{2010/01/02}{(AO, 571) class file must set \cs{splittopskip}; fine tune \cs{skip}\cs{footins}; \cs{footnoterule} defined in terms of \cs{skip}\cs{footins}}%
%    \begin{macrocode}
\footnotesep 1pt
\skip\footins 39pt plus 4pt minus 12pt
\def\footnoterule{%
 \dimen@\skip\footins\divide\dimen@\tw@
 \kern-\dimen@\hrule width.5in\kern\dimen@
}%
%    \end{macrocode}
%
%    \begin{macrocode}
\floatsep 12pt plus 2pt minus 2pt
\textfloatsep 20pt plus 2pt minus 4pt
\intextsep 12pt plus 2pt minus 2pt
%    \end{macrocode}
%
%    \begin{macrocode}
\dblfloatsep 12pt plus 2pt minus 2pt
\dbltextfloatsep 20pt plus 2pt minus 4pt
%    \end{macrocode}
%
%    \begin{macrocode}
\@fptop 0pt plus 1fil
\@fpsep 8pt plus 2fil
\@fpbot 0pt plus 1fil
\@dblfptop 0pt plus 1fil
\@dblfpsep 8pt plus 2fil
\@dblfpbot 0pt plus 1fil
%    \end{macrocode}
%
%    \begin{macrocode}
\marginparpush 5pt
%    \end{macrocode}
%
%    \begin{macrocode}
\parskip 0pt plus 1pt
\parindent 10pt
\emergencystretch8\p@
%    \end{macrocode}
%
%    \begin{macrocode}
\partopsep 2pt plus 1pt minus 1pt
%    \end{macrocode}
%
%    \begin{macrocode}
\leftmargini 25pt
\leftmarginii 22pt
\leftmarginiii 18.7pt
\leftmarginiv 17pt
\leftmarginv 10pt
\leftmarginvi 10pt
%    \end{macrocode}
%
%    \begin{macrocode}
\def\@listI{%
  \leftmargin\leftmargini
  \parsep 4\p@ plus2\p@ minus\p@
  \topsep 8\p@ plus2\p@ minus4\p@
  \itemsep 4\p@ plus2\p@ minus\p@
}%
%    \end{macrocode}
%
%    \begin{macrocode}
\labelsep 4pt
%    \end{macrocode}
%
%    \begin{macrocode}
\def\@listii{%
  \leftmargin\leftmarginii
  \labelwidth\leftmarginii
  \advance\labelwidth-\labelsep
  \topsep 4\p@ plus2\p@ minus\p@
  \parsep 2\p@ plus\p@ minus\p@
  \itemsep \parsep
}%
%    \end{macrocode}
%
%    \begin{macrocode}
\def\@listiii{%
  \leftmargin\leftmarginiii
  \labelwidth\leftmarginiii
  \advance\labelwidth-\labelsep
  \topsep 2\p@ plus\p@ minus\p@
  \parsep \z@
  \partopsep \p@ plus\z@ minus\p@
  \itemsep \topsep
}%
%    \end{macrocode}
%
%    \begin{macrocode}
\def\@listiv{%
  \leftmargin\leftmarginiv
  \labelwidth\leftmarginiv
  \advance\labelwidth-\labelsep
}%
%    \end{macrocode}
%
%    \begin{macrocode}
\def\@listv{%
  \leftmargin\leftmarginv
  \labelwidth\leftmarginv
  \advance\labelwidth-\labelsep
}%
%    \end{macrocode}
%
%    \begin{macrocode}
\def\@listvi{%
  \leftmargin\leftmarginvi
  \labelwidth\leftmarginvi
  \advance\labelwidth-\labelsep
}%
%    \end{macrocode}
%
%    \begin{macrocode}
%</10pt>
%    \end{macrocode}
%
% \section{The \classoption{11pt} class option: the \texttt{11pt} module}
% \changes{4.0b}{1999/06/20}{AO: all code for \protect\classoption{11pt} is in this module.}
% The file \file{11pt.rtx} is read in by the \classname{revtex4} document class
% if \cmd\@pointsize\ has the value 11.
%
%    \begin{macrocode}
%<*11pt>
%    \end{macrocode}
%
% \subsection{Defend Against Forseeable Errors}%
% Protect this file from being read in by anything but \revtex.
%    \begin{macrocode}
\ifx\undefined\substyle@ext
 \def\@tempa{%
  \endinput
  \GenericWarning{I must be read in by REVTeX! (Bailing out)}%
 }%
 \expandafter\else
  \def\@tempa{}%
 \expandafter\fi\@tempa
 \class@info{RevTeX pointsize 11pt selected}%
%    \end{macrocode}
%
%    \begin{macrocode}
\def\normalsize{%
    \@setfontsize\normalsize\@xipt{13.6}%
    \abovedisplayskip 11\p@ \@plus3\p@ \@minus6\p@
    \belowdisplayskip \abovedisplayskip
    \abovedisplayshortskip  \abovedisplayskip
    \belowdisplayshortskip \abovedisplayskip
    \let\@listi\@listI
}%
%    \end{macrocode}
%
%    \begin{macrocode}
\def\small{%
   \@setfontsize\small\@xpt\@xiipt
   \abovedisplayskip 10\p@ \@plus2\p@ \@minus5\p@
   \abovedisplayshortskip \z@ \@plus3\p@
   \belowdisplayshortskip 6\p@ \@plus3\p@ \@minus3\p@
   \def\@listi{\leftmargin\leftmargini
               \topsep 6\p@ \@plus2\p@ \@minus2\p@
               \parsep 3\p@ \@plus2\p@ \@minus\p@
               \itemsep \parsep
   }%
   \belowdisplayskip \abovedisplayskip
}%
%    \end{macrocode}
%
%    \begin{macrocode}
\def\footnotesize{%
   \@setfontsize\footnotesize\@ixpt{11}%
   \abovedisplayskip 8\p@ \@plus2\p@ \@minus4\p@
   \abovedisplayshortskip \z@ \@plus\p@
   \belowdisplayshortskip 4\p@ \@plus2\p@ \@minus2\p@
   \def\@listi{\leftmargin\leftmargini
               \topsep 4\p@ \@plus2\p@ \@minus2\p@
               \parsep 2\p@ \@plus\p@ \@minus\p@
               \itemsep \parsep
   }%
   \belowdisplayskip \abovedisplayskip
}%
%    \end{macrocode}
%
%    \begin{macrocode}
\def\scriptsize{%
  \@setfontsize\scriptsize\@viiipt{9.5}%
}%
\def\tiny{%
  \@setfontsize\tiny\@vipt\@viipt
}%
\def\large{%
  \@setfontsize\large\@xiipt{14}%
}%
\def\Large{%
  \@setfontsize\Large\@xivpt{18}%
}%
\def\LARGE{%
  \@setfontsize\LARGE\@xviipt{22}%
}%
\def\huge{%
    \@setfontsize\huge\@xxpt{25pt}%
}%
\def\Huge{%
  \@setfontsize\Huge\@xxvpt{30pt}%
}%
%    \end{macrocode}
%
%    \begin{macrocode}
%</11pt>
%    \end{macrocode}
%
% \section{The \classoption{12pt} class option: the \texttt{12pt} module}
% \changes{4.0b}{1999/06/20}{AO: all code for \protect\classoption{12pt} is in this module.}
% The file \file{12pt.rtx} is read in by the \classname{revtex4} document class
% if \cmd\@pointsize\ has the value 12.
%
%    \begin{macrocode}
%<*12pt>
%    \end{macrocode}
%
% \subsection{Defend Against Forseeable Errors}%
% Protect this file from being read in by anything but \revtex.
%    \begin{macrocode}
\ifx\undefined\substyle@ext
 \def\@tempa{%
  \endinput
  \GenericWarning{I must be read in by REVTeX! (Bailing out)}%
 }%
 \expandafter\else
  \def\@tempa{}%
 \expandafter\fi\@tempa
 \class@info{RevTeX pointsize 12pt selected}%
%    \end{macrocode}
%
%    \begin{macrocode}
\def\normalsize{%
  \@setfontsize\normalsize\@xiipt{14pt}%
  \abovedisplayskip 12\p@ \@plus3\p@ \@minus7\p@
  \belowdisplayskip \abovedisplayskip
  \abovedisplayshortskip  \z@ plus3\p@
  \belowdisplayshortskip  6.5\p@ \@plus3.5\p@ \@minus3\p@
  \let\@listi\@listI
}%
%    \end{macrocode}
%
%    \begin{macrocode}
 \def\small{%
  \@setfontsize\small\@xipt{14.5pt}%
  \abovedisplayskip 8\p@ \@plus3\p@ \@minus6\p@
  \belowdisplayskip \abovedisplayskip
  \abovedisplayshortskip \z@ \@plus3\p@
  \belowdisplayshortskip 6.5\p@ \@plus3.5\p@ \@minus3\p@
  \def\@listi{%
    \leftmargin\leftmargini
    \topsep 9\p@ \@plus3\p@ \@minus5\p@
    \parsep 4.5\p@ \@plus2\p@ \@minus\p@
    \itemsep \parsep
  }%
}%
%    \end{macrocode}
%
% Same baselineskip as \cmd\small\ ?
%    \begin{macrocode}
 \def\footnotesize{%
  \@setfontsize\footnotesize\@xpt{14.5pt}%
  \abovedisplayskip 10\p@ \@plus2\p@ \@minus5\p@
  \belowdisplayskip \abovedisplayskip
  \abovedisplayshortskip \z@ \@plus3\p@
  \belowdisplayshortskip 6\p@ \@plus3\p@ \@minus3\p@
  \def\@listi{%
    \leftmargin\leftmargini
    \topsep 6\p@ \@plus2\p@ \@minus2\p@
    \parsep 3\p@ \@plus2\p@ \@minus\p@
    \itemsep \parsep
  }%
}%
%    \end{macrocode}
%
%    \begin{macrocode}
\def\scriptsize{%
  \@setfontsize\scriptsize\@viiipt{9.5pt}%
}%
\def\tiny{%
  \@setfontsize\tiny\@vipt{7pt}%
}%
\def\large{%
  \@setfontsize\large\@xivpt{18pt}%
}%
\def\Large{%
  \@setfontsize\Large\@xviipt{22pt}%
}%
\def\LARGE{%
  \@setfontsize\LARGE\@xxpt{25pt}%
}%
\def\huge{%
    \@setfontsize\huge\@xxvpt{30pt}%
}%
\let\Huge=\huge
%    \end{macrocode}
%
%    \begin{macrocode}
%</12pt>
%    \end{macrocode}
%
% \section{Page parameters}%
% This code is common to both \classoption{11pt} and \classoption{12pt}.
%
%    \begin{macrocode}
%<*11pt|12pt>
%    \end{macrocode}
%
%    \begin{macrocode}
\appdef\setup@hook{%
 \twoside@sw{%
  \oddsidemargin   0pt
  \evensidemargin  0pt
  \marginparwidth 60pt
 }{%
  \oddsidemargin 0pt
  \evensidemargin 0pt
  \marginparwidth 44pt
 }%
}%
%    \end{macrocode}
%
%    \begin{macrocode}
\marginparsep 10pt
%    \end{macrocode}
%
%    \begin{macrocode}
\topmargin -37pt
%    \end{macrocode}
%
%    \begin{macrocode}
\headheight 12pt
\headsep 25pt
%    \end{macrocode}
%
% \changes{4.1n}{2010/01/02}{(AO, 571) class file must set \cs{splittopskip}; fine tune \cs{skip}\cs{footins}; \cs{footnoterule} defined in terms of \cs{skip}\cs{footins}}%
%    \begin{macrocode}
\topskip 10pt
\splittopskip\topskip
%    \end{macrocode}
%
%    \begin{macrocode}
\footskip 30pt
%    \end{macrocode}
% 
%    \begin{macrocode}
\textheight=665.5\p@
%    \end{macrocode}
% 
%    \begin{macrocode}
\appdef\setup@hook{%
 \tightenlines@sw{%
  \def\baselinestretch{1}%
 }{%
  \def\baselinestretch{1.5}%
 }%
}%
%    \end{macrocode}
%
%    \begin{macrocode}
\textwidth 468pt
%    \end{macrocode}
%
%    \begin{macrocode}
\columnsep 10pt
\columnseprule 0pt
%    \end{macrocode}
%
% \changes{4.1n}{2010/01/02}{(AO, 571) class file must set \cs{splittopskip}; fine tune \cs{skip}\cs{footins}; \cs{footnoterule} defined in terms of \cs{skip}\cs{footins}}%
%    \begin{macrocode}
\footnotesep 1pt
\skip\footins 25.25pt plus 4pt minus 12pt
\def\footnoterule{%
 \dimen@\skip\footins\divide\dimen@\f@ur
 \kern-\dimen@\hrule width.5in\kern\dimen@
}%
%    \end{macrocode}
%
%    \begin{macrocode}
\floatsep        14pt plus 2pt minus 4pt
\textfloatsep    20pt plus 2pt minus 4pt
\intextsep       14pt plus 4pt minus 4pt
%    \end{macrocode}
%
%    \begin{macrocode}
\dblfloatsep     14pt plus 2pt minus 4pt
\dbltextfloatsep 20pt plus 2pt minus 4pt
%    \end{macrocode}
%
%    \begin{macrocode}
\@fptop 0pt plus 1fil
\@fpsep 10pt plus 2fil
\@fpbot 0pt plus 1fil
\@dblfptop 0pt plus 1fil
\@dblfpsep 10pt plus 2fil%
\@dblfpbot 0pt plus 1fil
%    \end{macrocode}
%
%    \begin{macrocode}
\marginparpush 7pt
%    \end{macrocode}
%
%    \begin{macrocode}
\parskip 0pt plus 1pt
\parindent 15pt
\emergencystretch8\p@
\partopsep 3pt plus 2pt minus 2pt
%    \end{macrocode}
%
%    \begin{macrocode}
\leftmargini   30pt
\leftmarginii  26pt
\leftmarginiii 22pt
\leftmarginiv  20pt
\leftmarginv   12pt
\leftmarginvi  12pt
%    \end{macrocode}
%
%    \begin{macrocode}
\def\@listI{\leftmargin\leftmargini \parsep 5\p@ plus2.5\p@ minus\p@
  \topsep 10\p@ plus4\p@ minus6\p@
  \itemsep 5\p@ plus2.5\p@ minus\p@
}%
%    \end{macrocode}
%
%    \begin{macrocode}
\labelsep 6pt
%    \end{macrocode}
%
%    \begin{macrocode}
\def\@listii{\leftmargin\leftmarginii
  \labelwidth\leftmarginii\advance\labelwidth-\labelsep
  \topsep 5\p@ plus2.5\p@ minus\p@
  \parsep 2.5\p@ plus\p@ minus\p@
  \itemsep \parsep
}%
%    \end{macrocode}
%
%    \begin{macrocode}
\def\@listiii{\leftmargin\leftmarginiii
  \labelwidth\leftmarginiii\advance\labelwidth-\labelsep
  \topsep 2.5\p@ plus\p@ minus\p@
  \parsep \z@ \partopsep \p@ plus\z@ minus\p@
  \itemsep \topsep
}%
%    \end{macrocode}
%
%    \begin{macrocode}
\def\@listiv{\leftmargin\leftmarginiv
  \labelwidth\leftmarginiv\advance\labelwidth-\labelsep
}%
%    \end{macrocode}
%
%    \begin{macrocode}
\def\@listv{\leftmargin\leftmarginv
  \labelwidth\leftmarginv\advance\labelwidth-\labelsep
}%
%    \end{macrocode}
%
%    \begin{macrocode}
\def\@listvi{\leftmargin\leftmarginvi
  \labelwidth\leftmarginvi\advance\labelwidth-\labelsep
}%
%    \end{macrocode}
%
%    \begin{macrocode}
%</11pt|12pt>
%    \end{macrocode}
%
% \section{The \classoption{aps} class extension: the \texttt{aps} module}
% \changes{4.0b}{1999/06/20}{AO: made aps.rtx part of revtex4.dtx}
% The file \file{aps.rtx} is read in by the \classname{revtex4} document class
% if \cmd\@society\ has the value \classoption{aps}.
%
% Here, code specific to APS journals is separated out from the \revtex\ document class.
% (Other societies can customize \revtex\ by supplying their own \file{.rtx} file.)
%
% This class extension file is a model for a class extension you might write yourself.
%
% First, incorporate a \cmd\ProvidesFile\ command with an optional argument giving
% the version information, e.g.,
% \begin{verbatim}
% \ProvidesFile{foo}[2001/09/11 v1.1 Docinfo]%
% \end{verbatim}
%
% Within the society substyle, there are two things we must do as well:
% define the default journal, 
%    \begin{verbatim}
% \def\@journal@default{pra}%
%    \end{verbatim}
% And do likewise for the point size:
%    \begin{verbatim}
% \def\@pointsize@default{10}%
%    \end{verbatim}
%
% We first define some text entities (amounting to journal abbreviations),
% then some APS-specific initialisations,
% then code for particular APS journals.
% In the latter case, the choice is 
% keyed off the macro \cmd\@journal.
%
%    \begin{macrocode}
%<*aps>
%    \end{macrocode}
%
% \subsection{Defend Against Forseeable Errors}%
% Protect this file from being read in by anything but \revtex.
%    \begin{macrocode}
\ifx\undefined\substyle@ext
 \def\@tempa{%
  \endinput
  \GenericWarning{I must be read in by REVTeX! (Bailing out)}%
 }%
 \expandafter\else
  \def\@tempa{}%
 \expandafter\fi\@tempa
 \class@info{RevTeX society APS selected}%
%    \end{macrocode}
%
% Here are the class options relating to the APS:
%    \begin{macrocode}
\DeclareOption{pra}{\change@journal{pra}}%
\DeclareOption{prb}{\change@journal{prb}}%
\DeclareOption{prc}{\change@journal{prc}}%
\DeclareOption{prd}{\change@journal{prd}}%
\DeclareOption{pre}{\change@journal{pre}}%
\DeclareOption{prl}{\change@journal{prl}}%
\DeclareOption{prstab}{\change@journal{prstab}}%
\DeclareOption{prstper}{\change@journal{prstper}}%
\DeclareOption{rmp}{\change@journal{rmp}}%
%    \end{macrocode}
%
% \subsection{Abbreviations}%
% \changes{4.1p}{2010/02/24}{(AO, 584) Per MD, remove trailing space character from each journal abbreviation: it had caused an extraneous space in the .bbl} 
% The following macros constitute typing shortcuts for
% certain journal names.
%    \begin{macrocode}
\def\ao{Appl.\  Opt.}%
\def\ap{Appl.\  Phys.}%
\def\apl{Appl.\ Phys.\ Lett.}%
\def\apj{Astrophys.\ J.}%
\def\bell{Bell Syst.\ Tech.\ J.}%
\def\jqe{IEEE J.\ Quantum Electron.}%
\def\assp{IEEE Trans.\ Acoust.\ Speech Signal Process.}%
\def\aprop{IEEE Trans.\ Antennas Propag.}%
\def\mtt{IEEE Trans.\ Microwave Theory Tech.}%
\def\iovs{Invest.\ Ophthalmol.\ Vis.\ Sci.}%
\def\jcp{J.\ Chem.\ Phys.}%
\def\jmo{J.\ Mod.\ Opt.}%
\def\josa{J.\ Opt.\ Soc.\ Am.}%
\def\josaa{J.\ Opt.\ Soc.\ Am.\ A}%
\def\josab{J.\ Opt.\ Soc.\ Am.\ B}%
\def\jpp{J.\ Phys.\ (Paris)}%
\def\nat{Nature (London)}%
\def\oc{Opt.\ Commun.}%
\def\ol{Opt.\ Lett.}%
\def\pl{Phys.\ Lett.}%
\def\pra{Phys.\ Rev.\ A}%
\def\prb{Phys.\ Rev.\ B}%
\def\prc{Phys.\ Rev.\ C}%
\def\prd{Phys.\ Rev.\ D}%
\def\pre{Phys.\ Rev.\ E}%
\def\prl{Phys.\ Rev.\ Lett.}%
\def\rmp{Rev.\ Mod.\ Phys.}%
\def\pspie{Proc.\ Soc.\ Photo-Opt.\ Instrum.\ Eng.}%
\def\sjqe{Sov.\ J.\ Quantum Electron.}%
\def\vr{Vision Res.}%
%    \end{macrocode}
%
% \subsection{APS Setup}%
% Here we define the default procedures for APS journals.
% Individual APS journals may override these definitions.
%
% \subsubsection{Title block}%
%
% The specifics of the title block.
% Apply to all APS journals; individual journals may override these settings.
%
% \begin{macro}{\@fnsymbol}
% The \LaTeX\ kernel definition of \cmd\@fnsymbol\ is overridden.
% The definition in \filename{fixltx2e.sty} serves as a guide to the new way to symbol,
% working in both text- and math modes. 
%
% \filename{fixltx2e.sty} duplicates some features of \filename{ltxgrid} and \filename{ltxutil},
% however, so it may be incompatible with \revtex. 
% In case it is not loaded, we must provide a meaning for \cmd\TextOrMath,
% which that package makes robust. 
% I believe that it is \cmd\@fnsymbol\ itself that ought to be robustified.
% e\TeX\ further complicates matters; we do not especially accomodate it. 
%
% \changes{4.1d}{2009/03/27}{Definition of \cs{	@fnsymbol} follows fixltx2e.sty}
%
% Not! \cmd\TextOrMath\ must be made robust in any case (Bug 530). I return things to follow
% core \LaTeXe\ (\filename{latex.ltx}). 
% \changes{4.1f}{2009/10/05}{(AO, 530) \cs{@fnsymbol}: Failed to import fixltx2e.sty technology. Return to LaTeX core.}
%    \begin{macrocode}
\def\@fnsymbol#1{%
 \ensuremath{%
  \ifcase#1\or 
   *\or
   \dagger\or
   \ddagger\or
   \mathsection\or
   \mathparagraph\or
%  \|\or
   **\or
   \dagger\dagger\or
   \ddagger\ddagger\else
   \mathsection\mathsection\or
   \mathparagraph\mathparagraph\or
   ***\or
   \dagger\dagger\dagger\or
   \ddagger\ddagger\ddagger\else
   \mathsection\mathsection\mathsection\or
   \mathparagraph\mathparagraph\mathparagraph\or
%
   \@ctrerr 
  \fi
 }%
}%
\appdef\document@inithook{%
 \@ifxundefined\TextOrMath{%
  \DeclareRobustCommand\TextOrMath{\@ifmmode{\false@sw}{\true@sw}}%
 }{}%
}%
\let\thefootnote@latex\thefootnote
%    \end{macrocode}
% \end{macro}
%
% We assign the default titlepage style for APS; a journal or document instance 
% may override by invoking one of the other \cmd\clo@... procedures defined in \revtex.
%    \begin{macrocode}
\clo@groupedaddress
%    \end{macrocode}
%
% \begin{macro}{\titlepage}
%    \begin{macrocode}
\renewenvironment{titlepage}{%
  \let\wastwocol@sw\twocolumn@sw
  \onecolumngrid
  \newpage
  \thispagestyle{titlepage}%
  \c@page\z@
%    \end{macrocode}
% A comment: ``article.cls sets this to one not zero?''
%    \begin{macrocode}
}{%
  \wastwocol@sw{\twocolumngrid}{\newpage}%
}%
%    \end{macrocode}
% \end{macro}
%
% \begin{macro}{\frontmatter@abstractheading}
% APS Journals all set the abstract head the same way, with no head.
% However, if the user has specified the \classoption{preprint} class option,
% then the abstract will have a head.
%    \begin{macrocode}
\def\frontmatter@abstractheading{%
 \preprintsty@sw{%
  \begingroup
   \centering\large
   \abstractname
   \par
  \endgroup
 }{}%
}%
%    \end{macrocode}
% \end{macro}
%
% \begin{macro}{\frontmatter@abstractwidth}
% All APS journals set the abstract to the same width.
%    \begin{macrocode}
\def\frontmatter@abstractwidth{400\p@}%
%    \end{macrocode}
% \end{macro}
%
% \begin{macro}{\frontmatter@abstractfont}
% All APS journals set the abstract body the same way.
% \changes{4.0c}{1999/11/13}{(AO, 123) add parskip to the abstract.}
% \changes{4.1e}{2008/06/29}{(AO, 455) be nice to a list within the abstract}
%    \begin{macrocode}
\def\frontmatter@abstractfont{%
 \small
 \parindent1em\relax
 \adjust@abstractwidth
}%
\def\adjust@abstractwidth{%
 \dimen@\textwidth\advance\dimen@-\frontmatter@abstractwidth
 \divide\dimen@\tw@
 \galley@sw{%
  \advance\rightskip\tw@\dimen@
 }{%
  \advance\leftskip\dimen@
  \advance\rightskip\dimen@
 }%
 \@totalleftmargin\leftskip
}%
%    \end{macrocode}
% \end{macro}
%
% All APS journal preprints use separate titlepage and full-width abstract.
% 
% In effect, we establish a society default value for \cmd\preprintsty@sw,
% and for \cmd\titlepage@sw.
%    \begin{macrocode}
\@booleanfalse\preprintsty@sw
\@booleantrue\titlepage@sw
%    \end{macrocode}
%
% We choose the page style for all APS journals. 
% The journal may override by inserting its own code in \cmd\setup@hook. 
% Users wishing to customize their documents will be able to invoke a \cmd\pagestyle\
% command anywhere in the preamble; it will override the assignments here. 
%
% Here is the big switch for APS preprints. Note that \cmd\preprintsty@sw
% is also consulted in various procedures, but we assume its value does 
% not change after \cmd\setup@hook\ time. 
%    \begin{macrocode}
\appdef\setup@hook{%
  \preprintsty@sw{%
   \ps@preprint
   \def\frontmatter@abstractwidth{\textwidth}%
   \def\frontmatter@affiliationfont{\it}%
   \let\section\section@preprintsty
%    \end{macrocode}
% The following line of code had been commented out at this point.
%    \begin{verbatim}
%  \let\@hangfrom@section\@hangfrom@section@preprintsty
%    \end{verbatim}
%    \begin{macrocode}
   \let\subsection\subsection@preprintsty
   \let\subsubsection\subsubsection@preprintsty
  }{%
   \ps@article
  }%
}%
%    \end{macrocode}
%
% \begin{macro}{\frontmatter@authorformat}
% All APS journals set the author list the same.
% The leading is 11.5 points, and
% there is 11.5 points of extra space above the first author line
% (which amounts to the same thing as 11.5 points extra below the title)
% for a total of 23 points base-to-base.
%
%    \begin{macrocode}
\def\frontmatter@authorformat{%
 \skip@\@flushglue
 \@flushglue\z@ plus.3\hsize\relax
 \centering
 \advance\baselineskip\p@
 \parskip11.5\p@\relax
 \@flushglue\skip@
%    \end{macrocode}
% The following line of code had been commented out at this point.
%    \begin{verbatim}
%\preprintsty@sw{}{%
% \addvspace{0\p@}%
%}%
%    \end{verbatim}
%    \begin{macrocode}
}%
%    \end{macrocode}
% \end{macro}
%
% \begin{macro}{\frontmatter@above@affilgroup}
% The default amount of space above affiliation.
% APS Journals have 24 points b-b above an affiliation group.
%    \begin{macrocode}
\def\frontmatter@above@affilgroup{%
%    \end{macrocode}
% The following line of code had been commented out at this point.
%    \begin{verbatim}
%\preprintsty@sw{}{%
% \addvspace{11\p@}%
%}%
%    \end{verbatim}
%    \begin{macrocode}
}%
%    \end{macrocode}
% \end{macro}
%
% \begin{macro}{\frontmatter@above@affiliation}
% \begin{macro}{\frontmatter@above@affiliation@script}
% The default amount of space above affiliation.
% APS Journals have no extra space between author group down to common affiliation.
%    \begin{macrocode}
\def\frontmatter@above@affiliation@script{%
 \skip@\@flushglue
 \@flushglue\z@ plus.3\hsize\relax
 \centering
 \@flushglue\skip@
 \addvspace{3.5\p@}%
}%
\def\frontmatter@above@affiliation{%
 \preprintsty@sw{}{%
%    \end{macrocode}
% The following line of code had been commented out at this point.
%    \begin{verbatim}
% \addvspace{12\p@}%
%    \end{verbatim}
%    \begin{macrocode}
 }%
}%
%    \end{macrocode}
% \end{macro}
% \end{macro}
%
% \begin{macro}{\frontmatter@affiliationfont}
% All APS journals set the affiliation the same.
%    \begin{macrocode}
\def\frontmatter@affiliationfont{%
 \small\it
}%
%    \end{macrocode}
% \end{macro}
%
% \begin{macro}{\frontmatter@collaboration@above}
% PRL: 1.5 points extra: 13 points base-to-base above.
%    \begin{macrocode}
\def\frontmatter@collaboration@above{%
 \preprintsty@sw{%
 }{%
  \parskip1.5\p@\relax
 }%
}%
%    \end{macrocode}
% \end{macro}
%
% \begin{macro}{\frontmatter@setup}
% All APS journals set the title page using the same font and size.
% However, justification varies for the title block elements, so
% we assert none here.
%    \begin{macrocode}
\def\frontmatter@setup{%
 \normalfont
}%
%    \end{macrocode}
% \end{macro}
%
% \begin{macro}{\frontmatter@title@above}
% \begin{macro}{\frontmatter@title@format}
% \begin{macro}{\frontmatter@title@below}
% All APS journals set the article title the same.
%
% Note: Spacing from title to author is 23 points base-to-base.
%    \begin{macrocode}
\def\frontmatter@title@above{\addvspace{6\p@}}%
\def\frontmatter@title@format{\large\bfseries\centering\parskip\z@skip}%
\def\frontmatter@title@below{}%
%    \end{macrocode}
% \end{macro}
% \end{macro}
% \end{macro}
%
% \begin{macro}{\frontmatter@makefnmark}
% All APS journals share this procedure for setting the titlepage footnote text.
%    \begin{macrocode}
\def\@author@parskip{3\p@}%
\def\frontmatter@makefnmark{%
 \@textsuperscript{%
  \normalfont\@thefnmark
 }%
}%
\def\frontmatter@authorbelow{%
 \addvspace{3\p@}%
}%
%    \end{macrocode}
% \end{macro}
%
% \begin{macro}{\frontmatter@RRAP@format}
% All APS journals use the same format for the ``Received, Revised, etc.'' block on the title page.
%
% Change note: 11.5 points b-b from author/affiliation down to date.
%    \begin{macrocode}
\def\frontmatter@RRAP@format{%
  \small
  \centering
%    \end{macrocode}
% The following line of code had been commented out at this point.
%    \begin{verbatim}
% \preprintsty@sw{}{\parskip.5ex\relax}%
%    \end{verbatim}
%    \begin{macrocode}
  \everypar{\hbox\bgroup(\@gobble@leavemode@uppercase}%
  \def\par{\@ifvmode{}{\unskip)\egroup\@@par}}%
}%
\def\punct@RRAP{;\egroup\ \hbox\bgroup}%
\def\@gobble@leavemode@uppercase#1#2{\expandafter\MakeTextUppercase}%
%    \end{macrocode}
% \end{macro}
%
% \begin{macro}{\frontmatter@PACS@format}
% 
%    \begin{macrocode}
\def\frontmatter@PACS@format{%
   \addvspace{11\p@}%
   \footnotesize
   \adjust@abstractwidth
   \parindent\z@
   \parskip\z@skip
   \samepage
}%
%    \end{macrocode}
% \end{macro}
%
% \begin{macro}{\frontmatter@keys@format}
% 
%    \begin{macrocode}
\def\frontmatter@keys@format{%
   \footnotesize
   \adjust@abstractwidth
   \parindent\z@
   \samepage
}%
%    \end{macrocode}
% \end{macro}
%
% \begin{macro}{\ps@titlepage}
% \changes{4.0a}{1998/06/10}{multiple preprint commands}
% \changes{4.0b}{1999/06/20}{Stack \cs{preprint} args flush right at right margin.}
% Title page style. Currently empty except for preprint header,
% which consists of all the \cmd\preprint\ arguments,
% stacked flush right at the right margin.
% \changes{4.0c}{1999/11/13}{Insert procedure \cs{checkindate}}
%    \begin{macrocode}
\def\ps@titlepage{%
  \def\@oddhead{%
   \hfill
   \preprint@sw{%
    \expandafter\produce@preprints\expandafter{\@preprint}%
   }{}%
  }%
  \let\@evenhead\@oddhead
  \def\@oddfoot{%
   \hb@xt@\z@{\byrevtex\hss}%
   \hfil
   \preprintsty@sw{\thepage}{}%
   \quad\checkindate
   \hfil
  }%
  \let\@evenfoot\@oddfoot
}%
\def\byrevtex{\byrevtex@sw{Typeset by REV\TeX}{}}%
%    \end{macrocode}
% \end{macro}
%
% \begin{macro}{\produce@preprints}
% \changes{4.0c}{1999/11/13}{(AO, 115) If three or more preprints specified, set on single line, with commas.}
%    \begin{macrocode}
\def\produce@preprints#1{%
 \vtop to \z@{%
  \def\baselinestretch{1}%
  \small
  \let\preprint\preprint@count
  \count@\z@
  #1%
  \@ifnum{\count@>\tw@}{%
   \hbox{%
    \let\preprint\preprint@hlist
    #1%
    \setbox\z@\lastbox
   }%
  }{%
   \let\preprint\preprint@cr
   \halign{\hfil##\cr#1\crcr}%
   \par
   \vss
  }%
 }%
}%
\def\preprint@cr#1{#1\cr}%
\def\preprint@count#1{\advance\count@\@ne}%
\def\preprint@hlist#1{#1\hbox{, }}%
%    \end{macrocode}
% \end{macro}
%
% \subsubsection{Stacked Heads}%
% All APS journals put a period (.), followed by quad space, after the section number.
% Also, no hanging section number.
%    \begin{macrocode}
\def\@seccntformat#1{\csname the#1\endcsname.\quad}%
\def\@hang@from#1#2#3{#1#2#3}%
%    \end{macrocode}
%
% Note that in the following, we wish to set the section head uppercase, so we use
% David Carlisle's \cmd\MakeTextUppercase. However, because this procedure effectively 
% parses its argument (looking for things to \emph{not} translate), it has to be invoked
% in such a way that the argument of the \cmd\section\ command is passed to it as its
% own argument. 
%
% To accomplish this, we use the \cmd\@hangfrom@\ hook, which was developed for this 
% purpose.
%
%    \begin{macrocode}
\def\section{%
  \@startsection
    {section}%
    {1}%
    {\z@}%
    {0.8cm \@plus1ex \@minus .2ex}%
    {0.5cm}%
    {%
      \normalfont\small\bfseries
      \centering
    }%
}%
\def\@hangfrom@section#1#2#3{\@hangfrom{#1#2}\MakeTextUppercase{#3}}%
\def\@hangfroms@section#1#2{#1\MakeTextUppercase{#2}}%
%    \end{macrocode}
%
%    \begin{macrocode}
\def\subsection{%
  \@startsection
    {subsection}%
    {2}%
    {\z@}%
    {.8cm \@plus1ex \@minus .2ex}%
    {.5cm}%
    {%
     \normalfont\small\bfseries
     \centering
    }%
}%
%    \end{macrocode}
%
%    \begin{macrocode}
\def\subsubsection{%
  \@startsection
    {subsubsection}%
    {3}%
    {\z@}%
    {.8cm \@plus1ex \@minus .2ex}%
    {.5cm}%
    {%
     \normalfont\small\itshape
     \centering
    }%
}%
%    \end{macrocode}
%
% \subsubsection{Runin Heads}%
%    \begin{macrocode}
\def\paragraph{%
  \@startsection
    {paragraph}%
    {4}%
    {\parindent}%
    {\z@}%
    {-1em}%
    {\normalfont\normalsize\itshape}%
}%
%    \end{macrocode}
%
%    \begin{macrocode}
\def\subparagraph{%
  \@startsection
    {subparagraph}%
    {5}%
    {\parindent}%
    {3.25ex \@plus1ex \@minus .2ex}%
    {-1em}%
    {\normalfont\normalsize\bfseries}%
}%
%    \end{macrocode}
%
%
%
% \begin{macro}{\section@preprintsty}
% \begin{macro}{\subsection@preprintsty}
% \begin{macro}{\subsubsection@preprintsty}
% Here are the formatting procedures specific to the preprint style;
% the only difference is that the heads are flush left instead of centered.
%
%    \begin{macrocode}
\def\section@preprintsty{%
  \@startsection
    {section}%
    {1}%
    {\z@}%
    {0.8cm \@plus1ex \@minus .2ex}%
    {0.5cm}%
    {%
      \normalfont\small\bfseries
%     \centering
    }%
}%
%\def\@hangfrom@section@preprintsty#1#2#3{\@hangfrom{#1#2}\MakeTextUppercase{#3}}%
%    \end{macrocode}
%    \begin{macrocode}
\def\subsection@preprintsty{%
  \@startsection
    {subsection}%
    {2}%
    {\z@}%
    {.8cm \@plus1ex \@minus .2ex}%
    {.5cm}%
    {%
     \normalfont\small\bfseries
%    \centering
    }%
}%
%    \end{macrocode}
%    \begin{macrocode}
\def\subsubsection@preprintsty{%
  \@startsection
    {subsubsection}%
    {3}%
    {\z@}%
    {.8cm \@plus1ex \@minus .2ex}%
    {.5cm}%
    {%
     \normalfont\small\itshape
%    \centering
    }%
}%
%    \end{macrocode}
% \end{macro}
% \end{macro}
% \end{macro}
%
% By default, APS journals set titlepage notes as footnotes.
%    \begin{verbatim}
\let\frontmatter@footnote@produce\frontmatter@footnote@produce@footnote
%    \end{verbatim}
%
% \subsubsection{Table of Contents}%
% The toc will itself make an entry in the toc, 
% but we temporarily turn off toc formatting for the duration.
% \changes{4.1a}{2008/01/19}{(AO, 461) Change the csname revtex uses from @dotsep to ltxu@dotsep. The former is understood in mu. (What we wanted was a dimension.)}%
%    \begin{macrocode}
\def\@pnumwidth{1.55em}%
\def\@tocrmarg {2.55em}%
\def\@dotsep{2}%
\def\ltxu@dotsep{4.5pt}%
\setcounter{tocdepth}{3}%
%    \end{macrocode}
%
%    \begin{macrocode}
\def\tableofcontents{%
 \addtocontents{toc}{\string\tocdepth@munge}%
 \print@toc{toc}%
 \addtocontents{toc}{\string\tocdepth@restore}%
}%
%    \end{macrocode}
%
%    \begin{macrocode}
\def\tocdepth@munge{%
  \let\l@section@saved\l@section
  \let\l@section\@gobble@tw@
}%
\def\@gobble@tw@#1#2{}%
%    \end{macrocode}
%
%    \begin{macrocode}
\def\tocdepth@restore{%
  \let\l@section\l@section@saved
}%
%    \end{macrocode}
%
% The following definition of \cmd\l@part\ is a variant on
% the definition of \cmd\l@@sections\ in \file{ltxutil.dtx}.
%    \begin{macrocode}
\def\l@part#1#2{\addpenalty{\@secpenalty}%
 \begingroup
  \set@tocdim@pagenum\@tempboxa{#2}%
% \@tempdima 3em %
  \parindent \z@
  \rightskip\tocleft@pagenum plus 1fil\relax
  \skip@\parfillskip\parfillskip\z@
  \addvspace{2.25em plus\p@}%
  \large \bf %
  \leavevmode\ignorespaces#1\unskip\nobreak\hskip\skip@
  \hb@xt@\rightskip{\hfil\unhbox\@tempboxa}\hskip-\rightskip\hskip\z@skip
  \par
  \nobreak %
 \endgroup
}%
%    \end{macrocode}
%
% \begin{macro}{\l@section}
% Determine which TOC elements are automatically indented.
%
% We set the TOC to the standard of RMP. 
% If APS has its own specification, we will code it, and RMP must override.
%    \begin{macrocode}
\def\tocleft@{\z@}%
\def\tocdim@min{5\p@}%
\def\l@section{%
 \l@@sections{}{section}% Implicit #3#4
}%
\def\l@f@section{%
 \addpenalty{\@secpenalty}%
 \addvspace{1.0em plus\p@}%
 %\bf
}%
\def\l@subsection{%
 \l@@sections{section}{subsection}% Implicit #3#4
}%
\def\l@subsubsection{%
 \l@@sections{subsection}{subsubsection}% Implicit #3#4
}%
\def\l@paragraph#1#2{}%
\def\l@subparagraph#1#2{}%
%    \end{macrocode}
% \end{macro}
%
% Activate the auto TOC processing.
%    \begin{macrocode}
\let\toc@pre\toc@pre@auto
\let\toc@post\toc@post@auto
%    \end{macrocode}
%
% \subsubsection{Default column bottom}%
% All APS journal styles have flush bottoms.
%\changes{4.1n}{2009/11/30}{(AO, 566) restore 4.0 behavior: flush column bottoms}
%    \begin{macrocode}
\@booleanfalse\raggedcolumn@sw
%    \end{macrocode}
%
%
% \subsubsection{Table alignment style}%
%
% \begin{macro}{\tableft@skip@float}
% \begin{macro}{\tabmid@skip@float}
% \begin{macro}{\tabright@skip@float}
% \begin{macro}{\array@row@pre@float}
% \begin{macro}{\array@row@pst@float}
% All APS publications have the same table specification:
% Scotch rules above and below, centered in column.
%    \begin{macrocode}
\def\tableft@skip@float{\z@ plus\hsize}%
\def\tabmid@skip@float{\@flushglue}%
\def\tabright@skip@float{\z@ plus\hsize}%
\def\array@row@pre@float{\hline\hline\noalign{\vskip\doublerulesep}}%
\def\array@row@pst@float{\noalign{\vskip\doublerulesep}\hline\hline}%
%    \end{macrocode}
% \end{macro}
% \end{macro}
% \end{macro}
% \end{macro}
% \end{macro}
%
%
% \subsubsection{Footnote formatting}%
% \changes{4.1n}{2010/01/02}{(AO, 572) \cs{@makefntext} and \cs{frontmatter@makefntext} must be defined harmoniously}%
% We customize the formatting of footnotes for all APS journals.
% \begin{macro}{\@makefntext}
%    \begin{macrocode}
\long\def\@makefntext#1{%
 \def\baselinestretch{1}%
 \leftskip1em%
 \parindent1em%
 \noindent
 \nobreak\hskip-\leftskip
 \hb@xt@\leftskip{%
  \hss\@makefnmark\ %
 }%
 #1%
 \par
}%
%    \end{macrocode}
% \end{macro}
% \begin{macro}{\frontmatter@makefntext}
% We ensure that frontmatter footnotes format similarly to body footnotes. 
% But we provide our own hypertext anchor, otherwise not provided.
% \changes{4.1n}{2010/01/02}{(AO, 572) Independent footnote counter for title block. Abstract footnote counter shared with body.}%
%    \begin{macrocode}
\long\def\frontmatter@makefntext#1{%
 \def\baselinestretch{1}%
 \leftskip1em%
 \parindent1em%
 \noindent
 \nobreak\hskip-\leftskip
 \Hy@raisedlink{\hyper@anchorstart{frontmatter.\expandafter\the\csname c@\@mpfn\endcsname}\hyper@anchorend}%
 \hb@xt@\leftskip{%
  \hss\@makefnmark\ %
 }%
 #1%
 \par
}%
%    \end{macrocode}
% \end{macro}
%
%
% \subsubsection{Appendix}%
% \begin{macro}{\appendix}
% \begin{macro}{\@hangfrom@appendix}
% \begin{macro}{\@hangfroms@appendix}
% \begin{macro}{\@appendixcntformat}
% \changes{4.0c}{1999/11/13}{(AO, 129) section* within appendix was producing appendixname}
%    \begin{macrocode}
\prepdef\appendix{%
 \par
 \let\@hangfrom@section\@hangfrom@appendix
%\let\@hangfroms@section\@hangfroms@appendix
 \let\@sectioncntformat\@appendixcntformat
}%
\def\@hangfrom@appendix#1#2#3{%
 #1%
 \@if@empty{#2}{%
  #3%
 }{%
  #2\@if@empty{#3}{}{:\ #3}%
 }%
}%
\def\@hangfroms@appendix#1#2{%
 #1#2%
}%
\def\@appendixcntformat#1{\appendixname\ \csname the#1\endcsname}%
%    \end{macrocode}
% \end{macro}
% \end{macro}
% \end{macro}
% \end{macro}
%
% \subsubsection{Bibliography}%
% \changes{4.1n}{2010/01/04}{(AO, 576) In .bst files, remove support for the annote field}%
%
% Customize \revtex\ for the journal substyle;
% this task requires three components:
% the \BibTeX\ \file{apsrev.bst} and \file{apsrmp.bst} style files, and
% customizations of the \env{thebibliography} environment.
%
% \begin{macro}{\@bibstyle}
% Define the argument of the \cmd\bibliographystyle\ command
% (if the document does not do so).
% The user must have installed a \file{.bst} file of the corresponding name.
% This file will then be used by \BibTeX\ when compiling the document's \file{.bbl} file.
%
%  To generate \file{apsrev.bst}, use \classname{custom-bib} version 4.21 or later.
%  Run the \file{.bst} generator, \file{makebst.tex}, and
%  accept all defaults, with the following exceptions:
%
% \def\QandA#1#2#3{\item#1: \textbf{#2}: \texttt{#3}---}%
% \def\CandE#1{\item\relax\texttt{#1}---}%
% \begin{enumerate}
% \QandA{LANGUAGE FIELD}{l}{lang}Use language field to switch hyphenation patterns for title
% \QandA{PRESENTATIONS}{b}{pres,pres-bf}Presentation, speaker bold face 
% \QandA{ORDERING OF REFERENCES}{c}{seq-no}Citation order (unsorted, like unsrt.bst)
% \QandA{ORDER ON VON PART}{x}{vonx}Sort without von part (de la Maire after Mahone)
% \QandA{AUTHOR NAMES}{i}{nm-init,ed-au}Initials + surname (J. F. Smith)
% \QandA{POSITION OF JUNIOR}{*}{jnrlst}Junior comes last as Smith, John, Jr.
% \QandA{TYPEFACE FOR AUTHORS IN LIST OF REFERENCES}{u}{nmft,nmft-def}User defined author font (\cmd\bibnamefont)
% \QandA{FONT FOR FIRST NAMES}{u}{fnm-def}First names in user defined font (\cmd\bibfnamefont)
% \QandA{EDITOR NAMES IN INCOLLECTION ETC}{a}{nmfted}Editors incollection like authors font
% \QandA{FONT FOR `AND' IN LIST}{r}{nmand-rm}`And' in normal font (JONES and JAMES)
% \QandA{FONT OF CITATION LABELS IN TEXT}{u}{lab,lab-def}User defined citation font (\cmd\citenamefont)
% \QandA{FONT FOR `AND' IN CITATIONS}{r}{and-rm}Cited `and' in normal font 
% \QandA{DATE FORMAT}{*}{yr-par}Date in parentheses as (May 1993)
% \QandA{DATE EMPTY}{-}{date-nil-x}If date is empty, then do not produce the surrounding punctuation (parens, brackets, colon, comma)
% \QandA{TITLE OF ARTICLE}{d}{tit-qq}Title and punctuation in double quotes (``Title,'' ..)
% \QandA{INPROCEEDINGS CHAPTER AND PAGES, LIKE INBOOK}{-}{inproceedings-chapter}produce pages after chapter, just as in InBook
% \QandA{TITLE PRESENT IN ARTICLE, INCOLLECTION, AND INPROCEEDINGS}{x}{jtit-x}Title is ignored
% \QandA{INPROCEEDINGS CHAPTER AND PAGES, LIKE INBOOK}{y}{inproceedings-chapter}produce pages after chapter, just as in InBook
% \QandA{ARTICLE BOOKTITLE PRESENT}{}{article-booktitle}format booktitle 
% \QandA{ARTICLE SERIES PRESENT}{}{article-series}article can has series
% \QandA{JOURNAL NAME FONT}{r}{jttl-rm}Journal name normal font
% \QandA{JOURNAL NAME WITH ADDRESS}{y}{journal-address}Include address field (in parentheses) along with journal name
% \QandA{BOOK TITLE FIELDS}{y}{book-bt}Field `booktitle', or if absent field `title', is book title 
% \QandA{THESIS TITLE OPTIONAL}{}{thesis-title-o}Title is optional: no warning issued if empty
% \QandA{TECHNICAL REPORT TITLE}{b}{trtit-b}Tech. report title like books 
% \QandA{TECHNICAL REPORT INSTITUTION}{}{techreport-institution-par}format tech report institution like book publisher
% \QandA{JOURNAL VOLUME}{b}{vol-bf}Volume bold as {\bf vol}(num)
% \QandA{JOURNAL VOL AND NUMBER}{x}{vnum-x}Journal vol, without number as 34
% \QandA{VOLUME PUNCTUATION}{c}{volp-com}Volume with comma as vol(num), ppp
% \QandA{PAGE NUMBERS}{f}{jpg-1}Only start page number 
% \QandA{BOOK EDITOR W/O AUTHOR}{}{book-editor-booktitle}Book permits empty author, produces title before editor in this case
% \QandA{INBOOK PERMITS TITLE, BOOKTITLE, AUTHOR, EDITOR}{a}{inbook-editor-booktitle}Allow using both title/booktitle, both author/editor
% \QandA{CONFERENCE ADDRESS FOR BOOK, INBOOK, INCOLLECTION, INPROCEEDINGS, PROCEEDINGS}{a}{bookaddress}Italic booktitle followed by bookaddress in roman
% \QandA{NUMBER AND SERIES FOR BOOK, INBOOK, INCOLLECTION, INPROCEEDINGS, PROCEEDINGS}{*}{num-xser}Allows number without series and suppresses word "number"
% \QandA{WORD NUMBER CAPITALIZED FOR NUMBER AND SERIES}{c}{number-cap}Capitalize word `number' as: "Number 123"
% \QandA{WORD CHAPTER CAPITALIZED}{c}{chapter-cap}Capitalize word `chapter' as: `Chapter 42'
% \QandA{COMBINING NUMBER AND SERIES}{x}{series-number}Series number as: `Springer Lecture Notes No. 125'
% \QandA{POSITION OF NUMBER AND SERIES}{b}{numser-booktitle}After book title and conference address, and before editors 
% \QandA{VOLUME AND SERIES FOR BOOKS/COLLECTIONS}{s}{ser-vol}Series, vol. 23
% \QandA{VOLUME AND SERIES FORMATTING}{y}{ser-rm}format series roman , even when used with volume
% \QandA{WORD VOLUME CAPITALIZED FOR VOLUME AND SERIES}{y}{volume-cap}Capitalize word `volume', as: `Volume 7 in Lecture Series'
% \QandA{POSITION OF VOLUME AND SERIES FOR INCOLLECTION, INBOOK, AND INPROCEEDINGS}{e}{ser-ed}Series and volume after booktitle and before editors
% \QandA{JOURNAL NAME PUNCTUATION}{x}{jnm-x}Space after journal name
% \QandA{PAGES IN BOOK}{*}{pg-bk,book-chapter-pages}As chapter and page: chapter 42, page 345
% \QandA{PUBLISHER IN PARENTHESES}{d}{pub-date}Publisher with address and date in parentheses (Oxford, 1994)
% \QandA{EMPTY PUBLISHER PARENTHESES}{y}{ay-empty-pub-parens-x}eliminate parentheses altogether if nothing inside
% \QandA{PUBLISHER POSITION}{}{pre-pub}Publisher before volume, chapter, pages 
% \QandA{}{}{pre-edn}Edition before publisher
% \QandA{}{p}{pre-pub,pre-edn}Edition, publisher, volume, chapter, pages
% \QandA{ISBN NUMBER}{*}{isbn}Include ISBN for books, booklets, etc.
% \QandA{ISSN NUMBER}{*}{issn}Include ISSN for periodicals
% \QandA{DOI NUMBER}{a}{doi-link,doi}Doi forms a link to the publication, anchored to the volume or title
% \QandA{EDITOR IN COLLECTIONS}{b}{edby}In booktitle, edited by ..  (where .. is names)
% \QandA{PUNCTUATION BETWEEN SECTIONS (BLOCKS)}{c}{blk-com}Comma between blocks 
% \QandA{FINAL PUNCTUATION}{c}{fin-endbibitem}Command at end instead of period 
% \QandA{ABBREVIATE WORD `PAGES'}{a}{pp}`Page' abbreviated as p. or pp.
% \QandA{ABBREVIATE WORD `EDITORS'}{a}{ed}`Editor' abbreviated as ed. or eds.
% \QandA{OTHER ABBREVIATIONS}{a}{abr}Abbreviations of such words
% \QandA{ABBREVIATION FOR `EDITION'}{a}{ednx}`Edition' abbreviated as `ed' 
% \QandA{EDITION NUMBERS}{n}{ord}Numerical editions as 1st, 2nd, 3rd, etc
% \QandA{STORED JOURNAL NAMES}{a}{jabr}Abbreviated journal names 
% \QandA{FONT OF `ET AL'}{i}{etal-it}Italic et al 
% \QandA{ADDITIONAL REVTeX DATA FIELDS}{r}{revdata,eprint,url,url-blk,translation}Include REVTeX data fields collaboration, eid, eprint, archive, url, translation
% \QandA{SLACcitation FIELD}{}{SLACcitation}Produce SLACcitation field
% \QandA{NUMPAGES DATA FIELD}{*}{numpages-x}Do not include numpages field 
% \QandA{URL ADDRESS}{*}{url,url-prefix-x}URL without prefix (default: `URL ')
% \QandA{REFERENCE COMPONENT TAGS}{b}{bibinfo}Reference component tags like \cmd\bibinfo in the content of \cmd\bibitem
% \QandA{ELEMENT TAGS}{b}{bibfield}Element tags like \cmd\bibfield in the content of \cmd\bibitem
% \QandA{COMPATIBILITY WITH PLAIN TEX}{*}{nfss}Use LaTeX commands which may not work with Plain TeX
% \end{enumerate}
%
% A file \file{apsrev.dbj} file equivalent to the following should result:
% \begin{verbatim}
%\input docstrip
%\preamble
%----------------------------------------
%*** REVTeX-compatible Phys Rev 2010-02-12 ***
%\endpreamble
%\postamble
%End of customized bst file
%\endpostamble
%\keepsilent
%\askforoverwritefalse
%\def\MBopts{\from{merlin.mbs}{%
%  head,\MBopta}
%\from{physjour.mbs}{\MBopta}
%\from{geojour.mbs}{\MBopta}
%\from{photjour.mbs}{\MBopta}
%\from{merlin.mbs}{tail,\MBopta}}
%\def\MBopta{%
%  ay,%: Author-year with some non-standard interface
%  nat,%: Natbib for use with natbib v5.3 or later
%  lang,%: Use language field to switch hyphenation patterns for title
%  pres,pres-bf,%: Presentation, speaker bold face 
%  seq-no,%: Citation order (unsorted, only meaningful for numericals)
%  vonx,%: Sort without von part (de la Maire after Mahone)
%  nm-init,ed-au,%: Initials + surname (J. F. Smith)
%  jnrlst,%: Junior comes last as Smith, John, Jr.
%  nmft,nmft-def,%: User defined author font (\bibnamefont)
%  fnm-def,%: First names in user defined font (\bibfnamefont)
%  nmfted,%: Editors incollection like authors font
%  nmand-rm,%: `And' in normal font (JONES and JAMES)
%  lab,lab-def,%: User defined citation font (\citenamefont)
%  and-rm,%: Cited `and' in normal font 
%  keyxyr,%: Year blank when KEY replaces missing author (for natbib 7.0)
%  blkyear,%: Missing date left blank 
%  yr-par,%: Year in parentheses as (1993)
%  dtrev,%: Date as year month 
%  date-nil-x,%: If date is empty, then do not produce the surrounding punctuation (parens, brackets, colon, comma)
%  tit-qq,%: Title and punctuation in double quotes (``Title,'' ..)
%  inproceedings-chapter,%: produce pages after chapter, just as in InBook
%  jtit-x,%: Title is ignored
%  inproceedings-chapter,%: produce pages after chapter just as in InBook
%  article-booktitle,%: format booktitle 
%  article-series,%: article can has series
%  jttl-rm,%: Journal name normal font
%  journal-address,%: Include address field (in parentheses) along with journal name
%  book-bt,%: Field `booktitle', or if absent field `title', is book title 
%  thesis-title-o,%: Title is optional: no warning issued if empty
%  trtit-b,%: Tech. report title like books 
%  techreport-institution-par,%: format tech report institution like book publisher
%  vol-bf,%: Volume bold as {\bf vol}(num)
%  vnum-x,%: Journal vol, without number as 34
%  volp-com,%: Volume with comma as vol(num), ppp
%  jpg-1,%: Only start page number 
%  book-editor-booktitle,%: Book permits empty author, produces title before editor in this case
%  inbook-editor-booktitle,%: Allow using both title/booktitle, both author/editor
%  bookaddress,%: Italic booktitle followed by bookaddress in roman
%  num-xser,%: Allows number without series and suppresses word "number"
%  number-cap,%: Capitalize word `number' as: "Number 123"
%  chapter-cap,%: Capitalize word `chapter' as: `Chapter 42'
%  series-number,%: Series number as: `Springer Lecture Notes No. 125'
%  numser-booktitle,%: After book title and conference address, and before editors 
%  ser-vol,%: Series, vol. 23
%  ser-rm,%: format series roman , even when used with volume
%  volume-cap,%: Capitalize word `volume', as: `Volume 7 in Lecture Series'
%  ser-ed,%: Series and volume after booktitle and before editors
%  jnm-x,%: Space after journal name
%  pg-bk,book-chapter-pages,%: As chapter and page: chapter 42, page 345
%  pub-date,%: Publisher with address and date in parentheses (Oxford, 1994)
%  ay-empty-pub-parens-x,%: eliminate parentheses altogether if nothing inside
%  pre-pub,pre-edn,%: Edition, publisher, volume, chapter, pages
%  isbn,%: Include ISBN for books, booklets, etc.
%  issn,%: Include ISSN for periodicals
%  doi-link,doi,%: Doi forms a link to the publication, anchored to the volume or title
%  edby,%: In booktitle, edited by ..  (where .. is names)
%  blk-com,%: Comma between blocks 
%  fin-endbibitem,%: Command at end instead of period 
%  pp,%: `Page' abbreviated as p. or pp.
%  ed,%: `Editor' abbreviated as ed. or eds.
%  abr,%: Abbreviations of such words
%  ednx,%: `Edition' abbreviated as `ed' 
%  ord,%: Numerical editions as 1st, 2nd, 3rd, etc
%  jabr,%: Abbreviated journal names 
%  etal-it,%: Italic et al 
%  revdata,eprint,url,url-blk,translation,%: Include REVTeX data fields collaboration, eid, eprint, archive, url, translation
%  SLACcitation,%: Produce SLACcitation field
%  numpages-x,%: Do not include numpages field 
%  url,url-prefix-x,%: URL without prefix (default: `URL ')
%  bibinfo,%: Reference component tags like \bibinfo in the content of \bibitem
%  bibfield,%: Element tags like \bibfield in the content of \bibitem
%  nfss,%: Use LaTeX commands which may not work with Plain TeX
%,{%
%  }}
%\generate{\file{apsrev4-1.bst}{\MBopts}}
%\endbatchfile
% \end{verbatim}
%
% \subsubsection{Comparing apsrev.bst and apsrmp.bst}\label{sec:CompRevRmp}
% These two bibliographic styles differ as follows:
% \file{apsrev.dbj} has the following guard codes, which \file{apsrmp.dbj} does not:
% \begin{itemize}
% \CandE{seq-no}%
%    Citation order (unsorted, like unsrt.bst)
% \CandE{nm-init,ed-au}%
%    Initials + surname (J. F. Smith)
% \CandE{blkyear}%
%    Missing date left blank 
% \CandE{date-nil-x}%
%    If date is empty, then do not produce the surrounding punctuation (parens, brackets, colon, comma)
% \CandE{inproceedings-chapter}%
%    produce pages after chapter, just as in InBook
% \CandE{techreport-institution-par}%
%    format tech report institution like book publisher
% \CandE{vnum-x}%
%    Journal vol, without number as `34'
% \CandE{pub-date}%
%    Publisher with address and date in parentheses (Oxford, 1994)
% \CandE{pre-pub}%
%    Edition, publisher, volume, chapter, pages. Note that both use guard code pre-edn.
% \end{itemize}
%
% \file{apsrmp.dbj} has the following guard codes, which \file{apsrev.dbj} does not:
% \begin{itemize}
% \CandE{nm-rev1}%
%    Only first name reversed, initials (AGU style: Smith, J. F., H. K. Jones)
% \CandE{dt-beg}%
%    Date after authors 
% \CandE{vnum-sp}%
%    Journal vol (num) as `34 (2)'
% \CandE{pp-last}%
%    Pages at end, but before any notes
% \CandE{pub-par}%
%    Publisher in parentheses
% \CandE{school-par}%
%    School/address in parens: `(school, address)' 
% \CandE{bkedcap}%
%    `Name Editor,' as above, editor upper case
% \CandE{and-com}%
%    Comma even with 2 authors as `Tom, and Harry'
%    
% \end{itemize}
%
% We ensure that the journal substyle has the first word
% in the matter by installing the (default) APS code 
% later on (see Section~\ref{sec:aps-defaults}).
% \end{macro}
%
% \begin{macro}{\authoryear@sw}
% \changes{4.1a}{2008/01/18}{(AO, 457) Endnotes to be sorted in with numerical citations.}%
% Numerical citations: default value of \cmd\authoryear@sw\ is false.
%    \begin{macrocode}
\@booleanfalse\authoryear@sw
%    \end{macrocode}
% \end{macro}
%
% \begin{macro}{\bibpunct}
% The following commands effectively establish the style in which \cmd\cite\ commands are
% formatted.
% You can think of them as the second needed component for the bibliography.
% 
% Set up for APS numerical citations (once the packages are loaded). 
% The journal substyle can override these choices.
%
% Note that, prior to \classname{natbib} version 8.21, changing \cmd\NAT@sort\ at this late hour would not be totally effective;
% you would have to give \classname{natbib} the relevant options at load time.
% From version 8.21 on, \cmd\NAT@sort\ and \cmd\NAT@cmprs\ are not bound at all. 
% \changes{4.1a}{2008/01/19}{For natbib versions before 8.21, \cs{NAT@sort} was consulted only as natbib was being read in. Now it is fully dynamic.}
%    \begin{macrocode}
\appdef\setup@hook{%
 \bibpunct{[}{]}{,}{n}{}{,}%
}%
%    \end{macrocode}
% \end{macro}
%
% \begin{macro}{\pre@bibdata}
% 
% Set up to write endnotes to a .bib file; its data will be incorporated into the bibliography.
%    \begin{macrocode}
 \def\pre@bibdata{\jobname\bibdata@app}%
%    \end{macrocode}
% \end{macro}
%
% \begin{macro}{\bibsection}
% We define the sectioning command to use when starting the bibliography.
%
% \changes{4.0rc1}{2001/06/18}{grid changes with push and pop}
%    \begin{macrocode}
\appdef\setup@hook{%
 \def\bibsection{%
  \par
  \onecolumngrid@push
  \begingroup
   \baselineskip26\p@
   \bib@device{\textwidth}{245.5\p@}%
  \endgroup
  \nobreak\@nobreaktrue
  \addvspace{19\p@}%
  \par
  \onecolumngrid@pop
 }%
}%
%    \end{macrocode}
% \end{macro}
%
% \begin{macro}{\bib@device}
% \begin{macro}{\bibpreamble}
% \begin{macro}{\bibsep}
% \begin{macro}{\newblock}
% We define the sectioning command to use when starting the bibliography.
%    \begin{macrocode}
\def\bib@device#1#2{%
 \hb@xt@\z@{%
  \hb@xt@#1{%
   \hfil
   \phantomsection
   \addcontentsline {toc}{section}{\protect\numberline{}\refname}%
%  \hyper@anchorstart {\@currentHref }%
   \hb@xt@#2{%
    \skip@\z@\@plus-1fil\relax
                           \leaders\hrule height.25 \p@ depth.25 \p@ \hskip\z@\@plus1fil
    \hskip\skip@
    \hskip\z@\@plus0.125fil\leaders\hrule height.375\p@ depth.375\p@ \hskip\z@\@plus0.75fil \hskip\z@\@plus0.125fil
    \hskip\skip@
    \hskip\z@\@plus0.25 fil\leaders\hrule height.5  \p@ depth.5  \p@ \hskip\z@\@plus0.5 fil \hskip\z@\@plus0.25 fil
    \hskip\skip@
    \hskip\z@\@plus0.375fil\leaders\hrule height.625\p@ depth.625\p@ \hskip\z@\@plus0.25fil \hskip\z@\@plus0.375fil
 %  \hskip\skip@
 %  \hfil
   }%
%  \hyper@anchorend
   \hfil
  }%
  \hss
 }%
}%
\appdef\setup@hook{%
 \let\bibpreamble\@empty
 \bibsep\z@\relax
 \def\newblock{\ }%
}%
%    \end{macrocode}
% \end{macro}
% \end{macro}
% \end{macro}
% \end{macro}
%
% \begin{macro}{\bibfont}
% We define the font switch that applies to the body of the bibliography.
%
%    \begin{macrocode}
\appdef\setup@hook{%
 \def\bibfont{%
  \small
  \@clubpenalty\clubpenalty
 }%
}%
%    \end{macrocode}
% \end{macro}
%
%
% \subsubsection{Index}%
% FIXME: the following call to \cmd\twocolumn\ appears wrong if we were in two-column grid.
% \changes{4.0c}{1999/11/13}{grid changes with ltxgrid}
% \changes{4.0rc1}{2001/06/18}{grid changes with push and pop}
%    \begin{macrocode}
\newenvironment{theindex}{%
 \columnseprule \z@
 \columnsep 35\p@
 \c@secnumdepth-\maxdimen
 \onecolumngrid@push
 \section{\indexname}%
 \thispagestyle{plain}%
 \parindent\z@
 \parskip\z@ plus.3\p@\relax
 \let\item\@idxitem
 \onecolumngrid@pop
}{%
%\onecolumngrid@pop
}%
%
\def\@idxitem{\par\hangindent 40\p@}%
%
\def\subitem{\par\hangindent 40\p@ \hspace*{20\p@}}%
%
\def\subsubitem{\par\hangindent 40\p@ \hspace*{30\p@}}%
%
\def\indexspace{\par \vskip 10\p@ plus5\p@ minus3\p@\relax}%
%    \end{macrocode}
%
%
% \subsection{Journal- and Pointsize-Specific Code}%
%
% After this substyle is read in, we will execute the code specific to the selected 
% journal: execute the society/journal \file{.rtx} file if it exists, or
% execute the society/journal macro (if the latter is not defined, it will \cmd\relax\ out). 
% Here we define the default journal.
%    \begin{macrocode}
\def\@journal@default{pra}%
%    \end{macrocode}
%
% \subsection{Typesize-Specific Code}%
%
% After this society file is read in, we will process the \cmd\@pointsize-specific
% code. Here we define the default.
%    \begin{macrocode}
\def\@pointsize@default{10}%
%    \end{macrocode}
%
% Note: the convention in \revtex\ and its substyles is that 
% the substyle must not override any explicit class options
% declared by the document.
% This means that the various Booleans of Section~\ref{sec:options}
% may be assigned here only if they are still undefined at this point.
%
% For the APS, we supply code specific to journals PRA, PRB, PRC, PRD, PRE, PRL, and RMP.
% At present, they are identical,
% with the exception of PRB and RMP.
%
% For most all of the APS journals, the journal-dependent 
% code is relatively meager and is therefore embedded in this file.
% However, the RMP code is sufficiently extensive that 
% splitting it out into a separate file is more convenient.
%
% \subsubsection{\classoption{pra}}%
% There is no code specific to \classoption{pra}.
%    \begin{macrocode}
\def\rtx@apspra{%
 \class@info{APS journal PRA selected}%
}%
%    \end{macrocode}
%
% \subsubsection{\classoption{prb}}%
% We define a \cmd\citealp-like procedure that uses a numerical citation; it is for use in 
% the PRB journal.
%    \begin{macrocode}
\def\rtx@apsprb{%
 \class@info{APS journal PRB selected}%
%    \end{macrocode}
% PRB requires superscript citations.
% We use Patrick Daly's \classname{natbib} package,
% and hyperref and other packages are already set up to cope with this.
%
% \begin{macro}{\bibpunct}
% By default, PRB selects the \classoption{bibnotes} option.
% We also set up for numerical citations.
%    \begin{macrocode}
 \let\frontmatter@footnote@produce\frontmatter@footnote@produce@endnote
 \appdef\setup@hook{%
  \bibpunct{}{}{,}{s}{}{\textsuperscript{,}}%
  \let\onlinecite\rev@citealpnum
 }%
%    \end{macrocode}
% \end{macro}
%
% \begin{macro}{\pre@bibdata}
% \begin{macro}{\@endnotetext}
% \changes{4.1a}{2008/01/18}{(AO, 457) Endnotes to be sorted in with numerical citations.}%
% 
% Set up to write endnotes to a .bib file; its data will be incorporated into the bibliography.
%    \begin{macrocode}
 \def\pre@bibdata{\jobname\bibdata@app}%
%    \end{macrocode}
% \end{macro}
% \end{macro}
%
% Invoke \classoption{superbib} option. 
% If the document makes a selection of its own, it will override this assignment.
%    \begin{macrocode}
 \let\place@bibnumber\place@bibnumber@sup
%    \end{macrocode}
%
% End of \classoption{prb} code.
%    \begin{macrocode}
}%
%    \end{macrocode}
%
% \subsubsection{\classoption{prc}}%
% There is no code specific to \classoption{prc}. 
%    \begin{macrocode}
\def\rtx@apsprc{%
 \class@info{APS journal PRC selected}%
}%
%    \end{macrocode}
%
% \subsubsection{\classoption{prd}}%
% There is no code specific to \classoption{prd}. 
%    \begin{macrocode}
\def\rtx@apsprd{%
 \class@info{APS journal PRD selected}%
}%
%    \end{macrocode}
%
% \subsubsection{\classoption{pre}}%
% There is no code specific to \classoption{pre}. 
%    \begin{macrocode}
\def\rtx@apspre{%
 \class@info{APS journal PRE selected}%
}%
%    \end{macrocode}
%
% \subsubsection{\classoption{prl}}%
%    \begin{macrocode}
\def\rtx@apsprl{%
 \class@info{APS journal PRL selected}%
%    \end{macrocode}
% In PRL, the default is the \classoption{bibnotes} option,
% and the Acknowledgments section has no head.
%
% The References head is a device that may be described as a lozenge centered on the baseline,
% 71 points wide by 2 points thick, with the ends tapering to a half point in thickness.
% Space above 26 points base to base, below 31 base to base.
%FIXME: this code may confound \classname{geometry}
%    \begin{macrocode}
 \let\frontmatter@footnote@produce\frontmatter@footnote@produce@endnote
 \@booleanfalse\acknowledgments@sw
 \appdef\setup@hook{%
  \def\bibsection{%
   \par
   \begingroup
    \baselineskip26\p@
    \bib@device{\hsize}{72\p@}%
   \endgroup
   \nobreak\@nobreaktrue
   \addvspace{19\p@}%
  }%
 }%
%    \end{macrocode}
% Implement length checking.
% Use the \classname{times} and \classname{mathtime} packages,
% plus whatever other processing is required
% to make the formatted output be true
% to the metrics of the journal.
%
%    \begin{macrocode}
\appdef\setup@hook{%
  \lengthcheck@sw{%
   \RequirePackage{times}%
%    \end{macrocode}
% Wait. Do not use mathtime after all. APS has their own way of doing math pi, involving
% Adobe Mathematical Pi and other fonts.
%    \begin{verbatim}
%  \RequirePackage{mathtime}%
%    \end{verbatim}
%    \begin{macrocode}
  }{}%
}%
%    \end{macrocode}
% A PRL does not have numbered sections.
%    \begin{macrocode}
 \c@secnumdepth=-\maxdimen
%    \end{macrocode}
% Note: we defer this code until after type size file is read in.
%    \begin{macrocode}
 \appdef\setup@hook{%
  \@ifnum{\@pointsize=10\relax}{%
   \lengthcheck@sw{%
    \def\large{%
     \@setfontsize\large{12.5}{14\p@}%
    }%
    \def\normalsize{%
     \@setfontsize\normalsize{10.5}\@xiipt
     \abovedisplayskip 6\p@ \@plus6\p@ \@minus5\p@
     \belowdisplayskip \abovedisplayskip
     \abovedisplayshortskip  \abovedisplayskip
     \belowdisplayshortskip \abovedisplayskip
     \let\@listi\@listI
    }%
    \def\small{%
     \@setfontsize\small{9.5}\@xipt
     \abovedisplayskip 5\p@ \@plus5\p@ \@minus4\p@
     \belowdisplayskip \abovedisplayskip
     \abovedisplayshortskip  \abovedisplayskip
     \belowdisplayshortskip \abovedisplayskip
     \let\@listi\@listI
    }%
    \DeclareMathSizes{12.5}{12.5}{9}{6}%
    \DeclareMathSizes{10.5}{10.5}{7.5}{5}%
    \DeclareMathSizes{9.5}{9.5}{7.0}{5}%
   }{%
    \def\normalsize{%
     \@setfontsize\normalsize\@xpt\@xiipt
     \abovedisplayskip 10\p@ \@plus2\p@ \@minus5\p@
     \belowdisplayskip \abovedisplayskip
     \abovedisplayshortskip  \abovedisplayskip
     \belowdisplayshortskip \abovedisplayskip
     \let\@listi\@listI
    }%
   }%
  }{}%
 }%
 \textheight = 694.0\p@
%    \end{macrocode}
%
% End of \classoption{prl} code.
%    \begin{macrocode}
}%
%    \end{macrocode}
%
% \subsubsection{\classoption{per}}%
% Here is code specific to \classoption{per}. 
%    \begin{macrocode}
\def\rtx@apsprstper{%
 \class@info{APS journal PRST-PER selected}%
%    \end{macrocode}
% To first order, PER is the same as PRB, with superscript citations.
%
%    \begin{macrocode}
 \rtx@apsprb
%    \end{macrocode}
% Per Mark Doyle (2009-12-22), the default is longbibliography.
% \changes{4.1n}{2010/01/04}{(AO, 575) the default for journal prstper is longbibliography}%
%    \begin{macrocode}
 \@booleantrue\longbibliography@sw
%    \end{macrocode}
%
% ``Section heads are bold, centered (Roman numeral, Latin letter)''
% 
% Article titles do not have surrounding double-quotes!
%
% End of \classoption{prb} code.
%    \begin{macrocode}
}%
%    \end{macrocode}
%
% \subsubsection{\classoption{prstab}}%
% There is no code specific to \classoption{prstab}. 
%    \begin{macrocode}
\def\rtx@apsprstab{%
 \class@info{APS journal PRSTAB selected}%
}%
%    \end{macrocode}
%
% \subsubsection{\classoption{rmp}}%
% If this option has been selected, we will read in the needed code from the file \file{apsrmp.rtx}. 
%
% \subsection{Establish APS Defaults}\label{sec:aps-defaults}
%
% \begin{macro}{\footinbib@sw}
% All APS journals except RMP effectively invoke the \classoption{footinbib} option.
% We rely on the RMP-specific code to override this assignment. 
%    \begin{macrocode}
\@booleantrue\footinbib@sw
%    \end{macrocode}
% \end{macro}
%
% \begin{macro}{\@bibdataout@init}
% \begin{macro}{\@bibdataout@aps}
% \changes{4.1o}{2010/02/12}{(AO, 580) Control .bst at run time.}%
% Procedure \cmd\@bibdataout@aps\ has the job of writing the control record into the
% job's \cmd\jobname\filename{.bib} file, where it will adjust the options to \filename{.bst} processing. 
% It is installed into the initialization procedure \cmd\@bibdataout@init, and its meaning 
% is set by the society (APS) and journal. 
% For all but RMP, we select the Physical Review style.
% For the latter case, we change the meaning, per the code in \file{apsrmp.rtx}.
% 
%    \begin{macrocode}
\appdef\@bibdataout@rev{\@bibdataout@aps}%
\def\@bibdataout@aps{%
 \immediate\write\@bibdataout{%
%    \end{macrocode}
% An entry that controls processing of the \filename{.bst} file has entry type \texttt{@CONTROL}. 
%    \begin{macrocode}
  @CONTROL{%
   apsrev41Control%
%    \end{macrocode}
% Say whether we want long bibliography style (the default), or the abbreviated style. 
%    \begin{macrocode}
   \longbibliography@sw{%
    ,author="00",editor="1",pages="1",title="0",year="0"%
   }{%
    ,author="08",editor="1",pages="0",title="",year="1"%
   }%
  }%
 }%
%    \end{macrocode}
% Place a \cmd\citation\ into the auxiliary file corresponding to this entry. 
%    \begin{macrocode}
 \if@filesw
  \immediate\write\@auxout{\string\citation{apsrev41Control}}%
 \fi
}%
%    \end{macrocode}
% \end{macro}
% \end{macro}
%
% \begin{macro}{\place@bibnumber}
% \begin{macro}{\@bibstyle}
% We install code that will select the presentation for \cmd\bibitem s
% and govern the \BibTeX\ processing.
%    \begin{macrocode}
\let\place@bibnumber\place@bibnumber@inl
\def\@bibstyle{apsrev\substyle@post}%
%    \end{macrocode}
% \begin{verbatim}
%\appdef\setup@hook{%
% \longbibliography@sw{%
% \appdef\@bibstyle{long}%
% }{}%
%}%
% \end{verbatim}
% \end{macro}
% \end{macro}
%
% \subsection{APS Sanity Checking}\label{sec:aps-sanity}
% \changes{4.1f}{2009/07/20}{(AO, 524) Makes no sense if citations are superscript numbers and so are footnotes}
%
% Rule: if \cmd\place@bibnumber\ is \cmd\place@bibnumber@sup\ 
% (citations are numbered and set superscript),
% then it makes no sense for \cmd\footinbib@sw\ to be \cmd\false@sw\ 
% (footnotes set in the bibliography, as endnotes). 
% If both conditions prevailed, then the document would have footnotes
% and citations both as superscript arabic numbers, but independently numbered, 
% which would be confusing. 
%
% Any society that provides for both superscript numbered citations as well as for 
% numbered footnotes should check for this same condition, and deal with it. 
%
% Note: an alternative would be for footnotes to use the same sequence of footnote devices
% that are used by the frontmatter footnotes 
% (\cmd\frontmatter@thefootnote\ instead of arabic numbers). 
%
% In this case, we would want to refrain from resetting \cmd\c@footnote\ at the end of 
% the title page formatting.
% We would also want to treat body footnotes identically to frontmatter footnotes: 
% the assignments in \cmd\titleblock@produce\ would persist throughout the document.
%
% But APS do not choose to go that route. 
%    \begin{macrocode}
\appdef\setup@hook{%
 \@ifx{\place@bibnumber\place@bibnumber@sup}{%
  \footinbib@sw{}{%
   \class@warn{Citations are superscript numbers: footnotes must be endnotes; changing to that configuration}%
   \@booleantrue\footinbib@sw
  }%
 }{}%
}%
%    \end{macrocode}
% 
% Here ends the substyle for society APS.
%    \begin{macrocode}
%</aps>
%    \end{macrocode}
%
%
% \section{The \classoption{rmp} journal substyle: the \texttt{rmp} module}
%
% The file \file{apsrmp.rtx} is read in by the \classname{revtex4} document class
% if \cmd\@society\ has the value \classoption{aps}
% and \cmd\@journal\ has the value \classoption{rmp}.
%
% It is read at the end of the \file{aps.rtx}, so all definitions
% and assignments in that file are operative unless overridden here.
%
%
%    \begin{macrocode}
%<*rmp>
%    \end{macrocode}
%
% Protect this file from being read in by anything but \revtex.
%    \begin{macrocode}
\ifx\undefined\substyle@ext
 \def\@tempa{%
  \endinput
  \GenericWarning{I must be read in by REVTeX! (Bailing out)}%
 }%
 \expandafter\else
  \def\@tempa{}%
 \expandafter\fi\@tempa
%    \end{macrocode}
%
% Protect this file from being read in as a society instead of a journal.
% In such a case, \cmd\@journal\ will be undefined. 
%    \begin{macrocode}
\@ifxundefined\@journal{%
%    \end{macrocode}
% One alternative: abort the document.
% Another alternative: try to recover: force load the aps society file
%    \begin{macrocode}
 \class@warn{Please specify the REVTeX options [aps,rmp]!}%
 \@@end
}{}%
%    \end{macrocode}
% Log the journal substyle.
%    \begin{macrocode}
 \class@info{APS journal RMP selected}%
%    \end{macrocode}
%
% \subsection{Frontmatter}%
%
% We assign the titlepage style for RMP; a document instance 
% may override by invoking one of the class options of \revtex.
%    \begin{macrocode}
\clo@groupedaddress
%    \end{macrocode}
%
% \begin{macro}{\frontmatter@setup}
% 
%    \begin{macrocode}
\def\frontmatter@setup{%
 \normalfont\sffamily\raggedright
}%
%    \end{macrocode}
% \end{macro}
%
% \begin{macro}{\PACS@warn}
% Per Mark Doyle, RMP  never displays the PACS, so they don't want the `use showpacs' warning spit out.
%    \begin{macrocode}
\def\PACS@warn{RMP documents do not display PACS. Your \string\pacs\space will be ignored}%
%    \end{macrocode}
% \end{macro}
%
% \begin{macro}{\frontmatter@title@above}
% \begin{macro}{\frontmatter@title@format}
% \begin{macro}{\frontmatter@title@below}
% 
%    \begin{macrocode}
\def\frontmatter@title@above{}%
\def\frontmatter@title@format{\Large\bfseries\raggedright}% HelveticaNeue-Medium(Italic) 14pt.
\def\frontmatter@title@below{\addvspace{12\p@}}% 24pt b-b down to first author
%    \end{macrocode}
% \end{macro}
% \end{macro}
% \end{macro}
%
% \begin{macro}{\frontmatter@authorformat}
% \changes{4.1b}{2008/08/04}{Rag the right even more: .8\cs{hsize}. Also, assign \cs{@totalleftmargin}.}%
% \changes{4.1b}{2008/06/29}{(AO, 455) Be nice to a list within the abstract (assign \cs{@totalleftmargin}).}
% Set the rag to a milder value, because we want to do true ragged right
% typesetting, as opposed to the \LaTeX\ default, which gives very poor results.
%
% Note: author font is 9.8bp. 19.2bp/14.3bp above/below.
%    \begin{macrocode}
\def\frontmatter@authorformat{%
 \preprintsty@sw{\vskip0.5pc\relax}{}%
 \@tempskipa\@flushglue
 \@flushglue\z@ plus.8\hsize
 \raggedright\advance\leftskip.5in\relax
 \@flushglue\@tempskipa
 \parskip\z@skip
 \@totalleftmargin\leftskip
}%
%    \end{macrocode}
% \end{macro}
%
% \begin{macro}{\frontmatter@affiliationfont}
% \changes{4.1b}{2008/08/04}{As with author formatting, rag the right more, and assign \cs{@totalleftmargin}. Also neutralize \cs{def@after@address}.}%
% \changes{4.1b}{2008/06/29}{(AO, 455) Be nice to a list within the abstract (assign \cs{@totalleftmargin}).}
% The hook \cmd\frontmatter@affiliationfont\ controls the formatting of affiliations and affiliation groups.
% The hook \cmd\frontmatter@above@affilgroup\ is invoked just before proceeding with author/affiliation processing.
% The \cmd\frontmatter@above@affiliation\ is the amount of space above affiliations in the \classoption{groupedaddress} style,
% and \cmd\frontmatter@above@affiliation@script\ is that for \classoption{superscriptaddress}.
%
% Note: affiliation font is 9.03/10.4bp, 14.3bp/19.2bp b-b above/below
%    \begin{macrocode}
\def\frontmatter@affiliationfont{% Helvetica 9/10.2
 \small\slshape\selectfont\baselineskip10.5\p@\relax
 \@tempskipa\@flushglue
 \@flushglue\z@ plus.8\hsize
 \raggedright\advance\leftskip.5in\relax
 \@flushglue\@tempskipa
 \@totalleftmargin\leftskip
 \let\def@after@address\def@after@address@empty
}%
\def\frontmatter@above@affilgroup{\addvspace{7.2\p@}}% additional leading above an author
\def\frontmatter@above@affiliation{\addvspace{5.3\p@}}%
\def\frontmatter@above@affiliation@script{}%
%    \end{macrocode}
% \end{macro}
%
% Set up the default RMP style for title block authors and affiliations.
% We assign the titlepage style for RMP; a document instance 
% may override by invoking one of the class options of \revtex.
% \changes{4.1b}{2008/08/04}{The \texttt{rmp} journal substyle selects \texttt{groupedaddress} by default.}%
% \changes{4.1b}{2008/07/01}{(AO) make settings at class time instead of deferring them to later.}
%
% This command should override the effect of the 
% corresponding command in the society substyle,
% and any document class option bearing on same will in turn override. 
%    \begin{macrocode}
\clo@groupedaddress
%    \end{macrocode}
%
% \begin{macro}{\frontmatter@RRAP@format}
% \changes{4.1b}{2008/06/29}{(AO, 455) Be nice to a list within the abstract (assign \cs{@totalleftmargin}).}
% Note: in RMP, if we are not in preprint mode, the date will not be produced.
%
% Note: Helvetica C/lc, 8.98bp, space above: 16.3bp b-b.
%    \begin{macrocode}
\def\frontmatter@RRAP@format{%
  \addvspace{7.3\p@}%
  \small
  \raggedright\advance\leftskip.5in\relax
 \@totalleftmargin\leftskip
}%
\def\produce@RRAP#1{%
  \@if@empty{#1}{}{%
   \@ifvmode{\leavevmode}{}%
   \unskip(\ignorespaces#1\unskip)\quad
  }%
}%
%    \end{macrocode}
% \end{macro}
%
% \begin{macro}{\frontmatter@abstractheading}
% Space above 21.8bp b-b.
%    \begin{macrocode}
\def\frontmatter@abstractheading{%
 \preprintsty@sw{%
  \begingroup
   \centering\large
   \abstractname
   \par
  \endgroup
  \vspace{.5pc}%
 }{}%
}%
%    \end{macrocode}
% \end{macro}
%
% \begin{macro}{\frontmatter@abstractfont}
% TimesTen 8.93bp/9.6bp X 360bp, indented 36bp, with 21.9/37.6bp b-b above/below
% \changes{4.1b}{2008/06/29}{(AO, 455) Be nice to a list within the abstract (assign \cs{@totalleftmargin}).}
%    \begin{macrocode}
\def\frontmatter@abstractfont{%
 \footnotesize
 \hsize360\p@
 \leftskip=0.5in
 \parindent\z@
 \@totalleftmargin\leftskip
}%
%    \end{macrocode}
% \end{macro}
%
% \begin{macro}{\frontmatter@preabstractspace}
% \begin{macro}{\frontmatter@postabstractspace}
% Space above and space below abstract in title block.
% Should be 22/36 points base-to-base.
%    \begin{macrocode}
\def\frontmatter@preabstractspace{7.7\p@}%
\def\frontmatter@postabstractspace{24.6\p@}%
%    \end{macrocode}
% \end{macro}
% \end{macro}
%
% FIXME: Not done: PACS.
% FIXME: TOC: Head is same as \section: HelveticaNeue 8.98. 32/22bp b-b above/below,
%             Body: TimesTen 8/10.5.
%
% \subsection{General Text}%
%
% \changes{4.1b}{2008/07/01}{(AO) make settings at class time instead of deferring them to later.}
% If not in preprint mode, RMP sets the type size to 10/12 point. Note: s/b 11.6bp leading
%FIXME: define \cmd\normalsize\ only if nobody else has done so. 
%    \begin{macrocode}
 \appdef\setup@hook{%
  \preprintsty@sw{}{%
   \def\normalsize{%
    \@setsize\normalsize{12pt}\xpt\@xpt
    \abovedisplayskip 10\p@ plus2\p@ minus5\p@
    \belowdisplayskip \abovedisplayskip
    \abovedisplayshortskip  \abovedisplayskip
    \belowdisplayshortskip \abovedisplayskip
    \let\@listi\@listI
   }%
  }%
 }%
%    \end{macrocode}
%
% Footnote mods:
% \changes{4.1n}{2010/01/02}{(AO, 571) class file must set \cs{splittopskip}; fine tune \cs{skip}\cs{footins}; \cs{footnoterule} defined in terms of \cs{skip}\cs{footins}}%
%    \begin{macrocode}
 \footnotesep 9.25pt
 \skip\footins 36pt plus 4pt minus 12pt
 \def\footnoterule{%
  \dimen@\skip\footins\divide\dimen@\thr@@
  \kern-\dimen@\hrule width.5in\kern\dimen@
 }%
%    \end{macrocode}
%
%
% \subsection{Sectioning}
% \changes{4.1b}{2008/08/04}{Use \cs{setup@hook} to initialize all.}
% \changes{4.1b}{2008/08/01}{Section numbering via procedures \cs{secnums@rtx} and \cs{secnums@arabic}.}
%
% We override the meaning of \cmd\secnums@rtx. 
% The class option \classoption{secnumarabic} will continue to work.
%    \begin{macrocode}
\def\secnums@rtx{%
 \@ifxundefined\thepart{%
  \def\thepart{\Roman{part}}%
 }{}%
 \@ifxundefined\thesection{%
  \def\thesection       {\Roman{section}}%
  \def\p@section        {}%
 }{}%
 \@ifxundefined\thesubsection{%
  \def\thesubsection    {\Alph{subsection}}%
  \def\p@subsection     {\thesection.}%
 }{}%
 \@ifxundefined\thesubsubsection{%
  \def\thesubsubsection {\arabic{subsubsection}}%
  \def\p@subsubsection  {\thesection.\thesubsection.}%
 }{}%
 \@ifxundefined\theparagraph{%
  \def\theparagraph     {\alph{paragraph}}%
  \def\p@paragraph      {\thesection.\thesubsection.\thesubsubsection.}%
 }{}%
 \@ifxundefined\thesubparagraph{%
  \def\thesubparagraph  {\arabic{subparagraph}}%
  \def\p@subparagraph   {\thesection.\thesubsection.\thesubsubsection.\theparagraph.}%
 }{}%
}%
%    \end{macrocode}
%
% In RMP, put a period (.), followed by `nut space', after the section number.
% Also, hang the section number (the \LaTeX\ default).
%    \begin{macrocode}
\def\@seccntformat#1{\csname the#1\endcsname.\hskip0.5em\relax}%
%    \end{macrocode}
%
% Note that we wish to set the section head uppercase, so we use
% David Carlisle's \cmd\MakeTextUppercase. However, because this procedure effectively 
% parses its argument (looking for things to \emph{not} translate), it has to be invoked
% in such a way that the argument of the \cmd\section\ command is passed to it as its
% own argument. 
%
% To accomplish this, we use the \cmd\@hangfrom@\ hook, which was developed for this 
% purpose.
% \changes{4.1b}{2008/07/01}{Add \cs{@hangfroms@section}}
%    \begin{macrocode}
 \def\section{%
  \@startsection{section}{1}{\z@}{0.8cm plus1ex minus.2ex}{0.4cm}%
  {%
   \small\sffamily\bfseries\selectfont
   \raggedright
   \parindent\z@
  }%
 }%
 \def\@hangfrom@section#1#2#3{\@hangfrom{#1#2}\MakeTextUppercase{#3}}%
 \def\@hangfroms@section#1#2{#1\MakeTextUppercase{#2}}%
 \def\subsection{%
  \@startsection{subsection}{2}{\z@}{0.8cm plus1ex minus.2ex}{0.4cm}%
  {%
   \small\sffamily\bfseries
   \raggedright
   \parindent\z@
  }%
 }%
 \def\subsubsection{%
  \@startsection{subsubsection}{3}{\z@}{.8cm plus1ex minus.2ex}{0.4cm}%
  {%
   \small\sffamily\selectfont
   \raggedright
   \parindent\z@
  }%
 }%
 \def\paragraph{%
  \@startsection{paragraph}{4}{\z@}{.8cm plus1ex minus.2ex}{-1em}%
  {%
   \small\slshape\selectfont
   \raggedright
   \parindent\z@
  }%
 }%
 \def\subparagraph{%
  \@startsection{subparagraph}{4}{\parindent}{3.25ex plus1ex minus.2ex}{-1em}%
  {\normalsize\bfseries\selectfont}%
 }%
%
 \setcounter{tocdepth}{4}% FIXME: has no effect
%    \end{macrocode}
%
% \begin{macro}{\appendix}
% \begin{macro}{\@hangfrom@appendix}
% \begin{macro}{\@hangfroms@appendix}
% \begin{macro}{\@appendixcntformat}
% \changes{4.1b}{2008/07/01}{Add \cs{@hangfroms@section}}
%    \begin{macrocode}
\appdef\appendix{%
 \let\@hangfrom@section\@hangfrom@appendix
 \let\@sectioncntformat\@appendixcntformat
}%
\def\@hangfrom@appendix#1#2#3{%
 #1%
 \@if@empty{#2}{%
  #3%
 }{%
  #2\@if@empty{#3}{}{:\ #3}%
 }%
}%
\def\@hangfroms@appendix#1#2{%
 #1\appendixname\@if@empty{#2}{}{:\ #2}%
}%
\def\@appendixcntformat#1{\appendixname\ \csname the#1\endcsname}%
%    \end{macrocode}
% \end{macro}
% \end{macro}
% \end{macro}
% \end{macro}
%
%
% \subsection{Figure and Table Caption Formatting}
% \begin{macro}{\@makecaption}
% \changes{4.1b}{2008/07/01}{Break out \cs{@caption@fignum@sep}}
%
%    \begin{macrocode}
\setlength\belowcaptionskip{2\p@}
\long\def\@makecaption#1#2{%
  \vskip\abovecaptionskip
  \vbox{%
   \flushing
   \small\rmfamily
   \noindent
   #1\@caption@fignum@sep#2\par
  }%
  \vskip\belowcaptionskip
}%
\def\@caption@fignum@sep{\nobreak\hskip.5em plus.2em\ignorespaces}%
%    \end{macrocode}
% \end{macro}
%
% \subsection{Citations and Bibliography}%
%
% Customize \revtex\ for the journal substyle;
% this task requires three components:
% a \BibTeX\ \file{.bst} style file,
% customizing code for \file{natbib}, and
% customizations of the \env{thebibliography} environment.
%
% \begin{macro}{\@bibstyle}
% Define the argument of the \cmd\bibliographystyle\ command
% (if the document does not do so).
%
% The user must have installed a \file{.bst} file of the corresponding name.
% This file will then be used by \BibTeX\ when compiling the document's \file{.bbl} file.
%
%  To generate \file{apsrmp.bst}, use \classname{custom-bib} version 3.89d1 or later.
%  Run the \file{.bst} generator, \file{makebst.tex}, with the following options:
%
% \def\QandA#1#2#3{\item#1: \textbf{#2}: \texttt{#3}---}%
% \def\CandE#1{\item\relax\texttt{#1}---}%
% \begin{enumerate}
% \QandA{STYLE OF CITATIONS}{a}{ay}Author-year with some non-standard interface
% \QandA{AUTHOR}{*}{nat}Natbib for use with natbib v5.3 or later
% \QandA{LANGUAGE FIELD}{l}{lang}Use language field to switch hyphenation patterns for title
% \QandA{PRESENTATIONS}{b}{pres,pres-bf}Presentation, speaker bold face 
% \QandA{ORDER ON VON PART }{x}{vonx}Sort without von part (de la Maire after Mahone)
% \QandA{AUTHOR NAMES}{a}{nm-rev1}Only first name reversed, initials (AGU style: Smith, J. F., H. K. Jones)
% \QandA{POSITION OF JUNIOR }{*}{jnrlst}Junior comes last as Smith, John, Jr.
% \QandA{TYPEFACE FOR AUTHORS IN LIST OF REFERENCES}{u}{nmft,nmft-def}User defined author font (\cmd\bibnamefont)
% \QandA{FONT FOR FIRST NAMES }{u}{fnm-def}First names in user defined font (\cmd\bibfnamefont)
% \QandA{EDITOR NAMES IN INCOLLECTION ETC}{a}{nmfted}Editors incollection like authors font
% \QandA{FONT FOR `AND' IN LIST}{r}{nmand-rm}`And' in normal font (JONES and JAMES)
% \QandA{FONT OF CITATION LABELS IN TEXT }{u}{lab,lab-def}User defined citation font (\cmd\citenamefont)
% \QandA{FONT FOR `AND' IN CITATIONS }{r}{and-rm}Cited `and' in normal font 
% \QandA{LABEL WHEN AUTHORS MISSING }{*}{keyxyr}Year blank when KEY replaces missing author (for natbib 7.0)
% \QandA{DATE POSITION}{b}{dt-beg}Date after authors 
% \QandA{DATE FORMAT }{m}{yr-com}Date preceded by comma as `, 1993'
% \QandA{INCLUDE MONTHS}{m}{aymth}Include month in date 
% \QandA{REVERSED DATE }{r}{dtrev}Date as year month 
% \QandA{TRUNCATE YEAR }{*}{note-yr}Year text full as 1990--1993 or `in press'
% \QandA{TITLE OF ARTICLE}{d}{tit-qq}Title and punctuation in double quotes (``Title,'' ..)
% \QandA{TITLE PRESENT IN ARTICLE, INCOLLECTION, AND INPROCEEDINGS}{x}{jtit-x}Title is ignored
% \QandA{INPROCEEDINGS CHAPTER AND PAGES, LIKE INBOOK}{y}{inproceedings-chapter}produce pages after chapter just as in InBook
% \QandA{ARTICLE BOOKTITLE PRESENT}{?}{article-booktitle}format booktitle 
% \QandA{ARTICLE SERIES PRESENT}{?}{article-series}article can has series
% \QandA{JOURNAL NAME FONT}{r}{jttl-rm}Journal name normal font
% \QandA{JOURNAL NAME WITH ADDRESS}{y}{journal-address}Include address field (in parentheses) along with journal name
% \QandA{BOOK TITLE FIELDS}{y}{book-bt}Field `booktitle', or if absent field `title', is book title 
% \QandA{THESIS TITLE OPTIONAL}{?}{thesis-title-o}Title is optional: no warning issued if empty
% \QandA{TECHNICAL REPORT TITLE}{b}{trtit-b}Tech. report title like books 
% \QandA{JOURNAL VOLUME}{b}{vol-bf}Volume bold as {\bf vol}(num)
% \QandA{JOURNAL VOL AND NUMBER}{s}{vnum-sp}Journal vol (num) as 34 (2)
% \QandA{VOLUME PUNCTUATION}{c}{volp-com}Volume with comma as vol(num), ppp
% \QandA{PAGE NUMBERS}{f}{jpg-1}Only start page number 
% \QandA{POSITION OF PAGES}{e}{pp-last}Pages at end but before any notes
% \QandA{BOOK EDITOR W/O AUTHOR}{}{book-editor-booktitle}Book permits empty author, produces title before editor in this case
% \QandA{INBOOK PERMITS TITLE, BOOKTITLE, AUTHOR, EDITOR}{a}{inbook-editor-booktitle}Allow using both title/booktitle, both author/editor
% \QandA{CONFERENCE ADDRESS FOR BOOK, INBOOK, INCOLLECTION, INPROCEEDINGS, PROCEEDINGS}{a}{bookaddress}Italic booktitle followed by bookaddress in roman
% \QandA{NUMBER AND SERIES FOR BOOK, INBOOK, INCOLLECTION, INPROCEEDINGS, PROCEEDINGS}{*}{num-xser}Allows number without series and suppresses word "number"
% \QandA{WORD NUMBER CAPITALIZED FOR NUMBER AND SERIES}{c}{number-cap}Capitalize word `number' as: "Number 123"
% \QandA{WORD CHAPTER CAPITALIZED}{c}{chapter-cap}Capitalize word `chapter' as: `Chapter 42'
% \QandA{COMBINING NUMBER AND SERIES}{x}{series-number}Series number as: `Springer Lecture Notes No. 125'
% \QandA{POSITION OF NUMBER AND SERIES}{b}{numser-booktitle}After book title and conference address, and before editors 
% \QandA{VOLUME AND SERIES FOR BOOKS}{s}{ser-vol}Series, vol. 23
% \QandA{VOLUME AND SERIES FORMATTING}{y}{ser-rm}format series roman , even when used with volume
% \QandA{WORD VOLUME CAPITALIZED FOR VOLUME AND SERIES}{y}{volume-cap}Capitalize word `volume', as: `Volume 7 in Lecture Series'
% \QandA{POSITION OF VOLUME AND SERIES FOR INCOLLECTION, INBOOK, AND INPROCEEDINGS}{e}{ser-ed}Series and volume after booktitle and before editors
% \QandA{JOURNAL NAME PUNCTUATION}{x}{jnm-x}Space after journal name
% \QandA{PAGES IN BOOK}{*}{pg-bk,book-chapter-pages}As chapter and page: chapter 42, page 345
% \QandA{PUBLISHER IN PARENTHESES}{p}{pub-par}Publisher in parentheses 
% \QandA{EMPTY PUBLISHER PARENTHESES}{y}{ay-empty-pub-parens-x}eliminate parentheses altogether if nothing inside
% \QandA{PUBLISHER POSITION}{e}{pre-edn}Edition before publisher
% \QandA{SCHOOL}{p}{school-par}school/address in parens: `(school, address)' 
% \QandA{ISBN NUMBER}{*}{isbn}Include ISBN for books, booklets, etc.
% \QandA{ISSN NUMBER}{*}{issn}Include ISSN for periodicals
% \QandA{DOI NUMBER}{a}{doi-link,doi}Doi forms a link to the publication, anchored to the volume or title
% \QandA{`EDITOR' AFTER NAMES}{a}{bkedcap}`Name Editor,' as above, editor upper case
% \QandA{EDITOR IN COLLECTIONS}{b}{edby}In booktitle, edited by ..  (where .. is names)
% \QandA{PUNCTUATION BETWEEN SECTIONS }{c}{blk-com}Comma between blocks 
% \QandA{FINAL PUNCTUATION}{c}{fin-endbibitem}Command at end instead of period 
% \QandA{ABBREVIATE WORD `PAGES' }{a}{pp}`Page' abbreviated as p. or pp.
% \QandA{ABBREVIATE WORD `EDITORS'}{a}{ed}`Editor' abbreviated as ed. or eds.
% \QandA{OTHER ABBREVIATIONS}{a}{abr}Abbreviations of such words
% \QandA{ABBREVIATION FOR `EDITION' }{a}{ednx}`Edition' abbreviated as `ed' 
% \QandA{EDITION NUMBERS}{n}{ord}Numerical editions as 1st, 2nd, 3rd, etc
% \QandA{STORED JOURNAL NAMES}{a}{jabr}Abbreviated journal names 
% \QandA{COMMA BEFORE `AND'}{c}{and-com}Comma even with 2 authors as `Tom, and Harry'
% \QandA{FONT OF `ET AL'}{i}{etal-it}Italic et al 
% \QandA{ADDITIONAL REVTeX DATA FIELDS}{r}{revdata,eprint,url,url-blk,translation}Include REVTeX data fields collaboration, eid, eprint, archive, url, translation
% \QandA{SLACcitation FIELD}{?}{SLACcitation}Produce SLACcitation field
% \QandA{NUMPAGES DATA FIELD}{*}{numpages-x}Do not include numpages field 
% \QandA{REFERENCE COMPONENT TAGS}{b}{bibinfo}Reference component tags like \cmd\bibinfo\ in the content of \cmd\bibitem
% \QandA{ELEMENT TAGS}{b}{bibfield}Element tags like \cmd\bibfield\ in the content of \cmd\bibitem
% \QandA{COMPATIBILITY WITH PLAIN TEX}{*}{nfss}Use LaTeX commands which may not work with Plain TeX
% \end{enumerate}
%
% A file \file{apsrmp.dbj} file equivalent to the following should result:
% \begin{verbatim}
%\input docstrip
%\preamble
%----------------------------------------
%*** REVTeX-compatible RMP 2010-02-12 ***
%\endpreamble
%\postamble
%End of customized bst file
%\endpostamble
%\keepsilent
%\askforoverwritefalse
%\def\MBopts{\from{merlin.mbs}{%
%  head,\MBopta}
%\from{physjour.mbs}{\MBopta}
%\from{geojour.mbs}{\MBopta}
%\from{photjour.mbs}{\MBopta}
%\from{merlin.mbs}{tail,\MBopta}}
%\def\MBopta{%
%  ay,%: Author-year with some non-standard interface
%  nat,%: Natbib for use with natbib v5.3 or later
%  lang,%: Use language field to switch hyphenation patterns for title
%  pres,pres-bf,%: Presentation, speaker bold face 
%  vonx,%: Sort without von part (de la Maire after Mahone)
%  nm-rev1,%: Only first name reversed, initials (AGU style: Smith, J. F., H. K. Jones)
%  jnrlst,%: Junior comes last as Smith, John, Jr.
%  nmft,nmft-def,%: User defined author font (\bibnamefont)
%  fnm-def,%: First names in user defined font (\bibfnamefont)
%  nmfted,%: Editors incollection like authors font
%  nmand-rm,%: `And' in normal font (JONES and JAMES)
%  lab,lab-def,%: User defined citation font (\citenamefont)
%  and-rm,%: Cited `and' in normal font 
%  keyxyr,%: Year blank when KEY replaces missing author (for natbib 7.0)
%  dt-beg,%: Date after authors 
%  yr-par,%: Year in parentheses as (1993)
%  dtrev,%: Date as year month 
%  date-nil-x,%: If date is empty, then do not produce the surrounding punctuation (parens, brackets, colon, comma)
%  tit-qq,%: Title and punctuation in double quotes (``Title,'' ..)
%  inproceedings-chapter,%: produce pages after chapter, just as in InBook
%  jtit-x,%: Title is ignored
%  inproceedings-chapter,%: produce pages after chapter just as in InBook
%  article-booktitle,%: format booktitle 
%  article-series,%: article can has series
%  jttl-rm,%: Journal name normal font
%  journal-address,%: Include address field (in parentheses) along with journal name
%  book-bt,%: Field `booktitle', or if absent field `title', is book title 
%  thesis-title-o,%: Title is optional: no warning issued if empty
%  trtit-b,%: Tech. report title like books 
%  techreport-institution-par,%: format tech report institution like book publisher
%  vol-bf,%: Volume bold as {\bf vol}(num)
%  vnum-sp,%: Journal vol (num) as 34 (2)
%  volp-com,%: Volume with comma as vol(num), ppp
%  jpg-1,%: Only start page number 
%  pp-last,%: Pages at end but before any notes
%  book-editor-booktitle,%: Book permits empty author, produces title before editor in this case
%  inbook-editor-booktitle,%: Allow using both title/booktitle, both author/editor
%  bookaddress,%: Italic booktitle followed by bookaddress in roman
%  num-xser,%: Allows number without series and suppresses word "number"
%  number-cap,%: Capitalize word `number' as: "Number 123"
%  chapter-cap,%: Capitalize word `chapter' as: `Chapter 42'
%  series-number,%: Series number as: `Springer Lecture Notes No. 125'
%  numser-booktitle,%: After book title and conference address, and before editors 
%  ser-vol,%: Series, vol. 23
%  ser-rm,%: format series roman , even when used with volume
%  volume-cap,%: Capitalize word `volume', as: `Volume 7 in Lecture Series'
%  ser-ed,%: Series and volume after booktitle and before editors
%  jnm-x,%: Space after journal name
%  pg-bk,book-chapter-pages,%: As chapter and page: chapter 42, page 345
%  pub-par,%: Publisher in parentheses 
%  ay-empty-pub-parens-x,%: eliminate parentheses altogether if nothing inside
%  pre-edn,%: Edition before publisher
%  school-par,%: school/address in parens: `(school, address)' 
%  isbn,%: Include ISBN for books, booklets, etc.
%  issn,%: Include ISSN for periodicals
%  doi-link,doi,%: Doi forms a link to the publication, anchored to the volume or title
%  bkedcap,%: `Name Editor,' as above, editor upper case
%  edby,%: In booktitle, edited by ..  (where .. is names)
%  blk-com,%: Comma between blocks 
%  fin-endbibitem,%: Command at end instead of period 
%  pp,%: `Page' abbreviated as p. or pp.
%  ed,%: `Editor' abbreviated as ed. or eds.
%  abr,%: Abbreviations of such words
%  ednx,%: `Edition' abbreviated as `ed' 
%  ord,%: Numerical editions as 1st, 2nd, 3rd, etc
%  jabr,%: Abbreviated journal names 
%  and-com,%: Comma even with 2 authors as `Tom, and Harry'
%  etal-it,%: Italic et al 
%  revdata,eprint,url,url-blk,translation,%: Include REVTeX data fields collaboration, eid, eprint, archive, url, translation
%  SLACcitation,%: Produce SLACcitation field
%  numpages-x,%: Do not include numpages field 
%  url,url-prefix-x,%: URL without prefix (default: `URL ')
%  bibinfo,%: Reference component tags like \bibinfo in the content of \bibitem
%  bibfield,%: Element tags like \bibfield in the content of \bibitem
%  nfss,%: Use LaTeX commands which may not work with Plain TeX
%,{%
%  }}
%\generate{\file{apsrmp4-1.bst}{\MBopts}}
%\endbatchfile
% \end{verbatim}
%
% For a comparison between apsrmp.bst and apsrev.bst, see Section~\ref{sec:CompRevRmp}. 
%
% \changes{4.1c}{2008/08/15}{Document class option longbibliography via \cs{substyle@post}}
%    \begin{macrocode}
\def\@bibstyle{apsrmp\substyle@post}%
%    \end{macrocode}
% \end{macro}
%
% \begin{macro}{\authoryear@sw}
% \changes{4.1a}{2008/01/18}{(AO, 457) Endnotes to be sorted in with numerical citations.}%
% Author-year citations: default value of \cmd\authoryear@sw\ is true.
%    \begin{macrocode}
\@booleantrue\authoryear@sw
%    \end{macrocode}
% \end{macro}
%
% \begin{macro}{\@bibdataout@rmp}
% \changes{4.1o}{2010/02/12}{(AO, 580) Control .bst at run time.}%
% When the journal is RMP, the meaning of the procedure \cmd\@bibdataout@aps\ needs to be different 
% because of the way the author names are formatted. In other respects, it is the same. 
%    \begin{macrocode}
\def\@bibdataout@aps{%
 \immediate\write\@bibdataout{%
%    \end{macrocode}
% An entry that controls processing of the \filename{.bst} file has entry type \texttt{@CONTROL}. 
% This entry's cite key is \texttt{apsrmp41Control}, which serves as a version number.
%    \begin{macrocode}
  @CONTROL{%
   apsrmp41Control%
%    \end{macrocode}
% Say whether we want long bibliography style (the default), or the abbreviated style. 
%    \begin{macrocode}
   \longbibliography@sw{%
    ,author="03",editor="0",pages="1",title="0",year="0"%
   }{%
    ,author="0B",editor="0",pages="0",title="",year="1"%
   }%
  }%
 }%
%    \end{macrocode}
% Place a \cmd\citation\ into the auxiliary file corresponding to this entry. 
%    \begin{macrocode}
 \if@filesw
  \immediate\write\@auxout{\string\citation{apsrmp41Control}}%
 \fi
}%
%    \end{macrocode}
% \end{macro}
%
% \begin{macro}{\bibpunct}
% \begin{macro}{\bibsection}
% \begin{macro}{\bibpreamble}
% \begin{macro}{\newblock}
% \begin{macro}{\bibhang}
% \begin{macro}{\bibsep}
% \begin{macro}{\cite}
% The following commands effectively establish the style in which \cmd\cite\ commands are
% formatted.
% You can think of them as the second needed component for the bibliography.
% 
% Set up for author-year citations:
% when \cmd\NAT@set@cites\ executes (at \envb{document} time), the \cmd\@biblabel\ will
% be set to \cmd\NAT@biblabel.
%
% Per Karie Friedman (friedman@phys.washington.edu), multiple citations are separated by semicolons, e.g., 
% (Jones, 1999; Abbott and Smith, 2000; Wortley, 2001a), 
% and 
% multiple citations by the same author by commas, e.g., 
% Abela et al. (1995, 1997a, 1997b).
% The third argument of \cmd\bibpunct\ handles the former.
%
% The fifth argument puts a comma after the author when the year is not in parens:
% (Lee et al., 1996).
%
% Incidently, this \cmd\bibpunct\ command specifies the \classname{natbib} default values. 
%
% We define the sectioning command to use when starting the bibliography.
%
% We change \classname{natbib}'s \cmd\NAT@def@citea\ procedure to effect more elaborate
% punctuation for RMP: see item 473: 
% \cmd\cite\ order punctuation: ``If possible, \cmd\textcite should put the word 'and' between two citations and before the last citation in a list of 3 or more.''
%
%    \begin{macrocode}
\appdef\setup@hook{%
%    \end{macrocode}
%
% We define the punctuation to use in the \cmd\cite\ command's production.
%
%    \begin{macrocode}
 \bibpunct{(%)
 }{%(
 )}{;}{a}{,}{,}%
%    \end{macrocode}
%
% We define the sectioning command to use when starting the bibliography.
%
%    \begin{macrocode}
 \def\bibsection{%
  \expandafter\section\expandafter*\expandafter{\refname}%
  \@nobreaktrue
 }%
 \let\bibpreamble\@empty
 \def\newblock{\ }%
 \bibhang10\p@
 \bibsep\z@
%    \end{macrocode}
%
% Per Mark Doyle, \cmd\cite\ is mapped to \cmd\citep\ in RMP.
% \changes{4.1i}{2009/10/23}{(AO, 541) Defer assignment of \cs{cite} until after natbib loads}
%    \begin{macrocode}
 \let\cite\citep
%    \end{macrocode}
%
% End of code to be delayed until after \classname{natbib} loads.
%    \begin{macrocode}
}%
%    \end{macrocode}
% \end{macro}
% \end{macro}
% \end{macro}
% \end{macro}
% \end{macro}
% \end{macro}
% \end{macro}
%
% \begin{macro}{\footinbib@sw}
% Footnotes in bibliography are consistent only with numbered citations,
% and are particularly nasty under \classname{natbib}:
% the packcage will automatically change to numbered references if
% any \cmd\bibitem\ commands lack the optional argument.
% Therefore, we must uninvoke it now, even if invoked by the document.
% The same is quietly done with 
% \classname{natbib}'s \classoption{mcite} and \classoption{compress} options. 
% 
% \changes{4.1f}{2009/07/15}{(AO, 523) Add class option nomerge, to turn off new natbib 8.3 syntax}
% (AO 523) I changed the code that alters \cmd\NAT@merge\ so that it will not override
% when \cmd\NAT@merge\ has been set to \cmd\z@.
%    \begin{macrocode}
\@booleanfalse\footinbib@sw
\appdef\setup@hook{%
 \footinbib@sw{%
  \class@warn{%
   Footnotes in bibliography are incompatible with RMP.^^J%
   Undoing the footinbib option.
  }%
  \@booleanfalse\footinbib@sw
 }{}%
 \@ifnum{\NAT@merge>\@ne}{\let\NAT@merge\@ne}{}%
 \def\NAT@cmprs{\z@}%
}%
%    \end{macrocode}
% \end{macro}
%
% \begin{macro}{\eprint}
% RMP requires the \cmd\eprint\ field in the bib entry to be set off with the word ``eprint''.
%    \begin{macrocode}
\def\eprint#1{eprint #1}%
%    \end{macrocode}
% \end{macro}
%
% \subsection{Table of Contents}
%
% We set up for auto-sizing of certain TOC elements.
% 
% To do this, we override the definitions for 
% the default TOC font (\cmd\toc@@font),
% and define
% formatting for the needed elements (\cmd\l@\dots).
% Finally, we activate the autosizing by 
% assigning \cmd\toc@pre\ and \cmd\toc@post.
%
% \begin{macro}{\toc@@font}
% \changes{4.1a}{2008/01/19}{(AO, 461) Change the csname revtex uses from @dotsep to ltxu@dotsep. The former is understood in mu. (What we wanted was a dimension.)}%
% Set the formatting characteristics of the auto-indenting part
%  of the TOC.
%    \begin{macrocode}
\def\toc@@font{%
 \footnotesize\rmfamily
 \def\\{\space\ignorespaces}%
}%
\def\ltxu@dotsep{5.5pt}%
%    \end{macrocode}
% \end{macro}
%
% \begin{macro}{\l@section}
% Determine which TOC elements are automatically indented.
%    \begin{macrocode}
\def\tocleft@{\z@}%
\def\tocdim@min{5\p@}%
\def\l@section{%
 \l@@sections{}{section}% Implicit #3#4
}%
\def\l@subsection{%
 \l@@sections{section}{subsection}% Implicit #3#4
}%
\def\l@subsubsection{%
 \l@@sections{subsection}{subsubsection}% Implicit #3#4
}%
%\def\l@subsubsection#1#2{}%
\def\l@paragraph#1#2{}%
\def\l@subparagraph#1#2{}%
%    \end{macrocode}
% \end{macro}
%
% Activate the TOC processing.
%    \begin{macrocode}
\let\toc@pre\toc@pre@auto
\let\toc@post\toc@post@auto
%    \end{macrocode}
%
%
%    \begin{macrocode}
%</rmp>
%    \end{macrocode}
%
%
% \Finale
% %Here ends the programmer's documentation.
% \endinput
%
\endinput
