\documentclass[UTF8]{article}

\author{H.~Partl}
\title{Minimalism}
\usepackage{amsfonts}
\usepackage{amsmath}
\usepackage{latexsym}
\usepackage{amssymb}
\usepackage{makeidx}

\usepackage{amsmath}
\usepackage{graphicx}

\usepackage{graphicx,graphics,color}
\usepackage{multirow}


\usepackage[compat=1.1.0]{tikz-feynman}

\begin{document}

the first diagram:

\feynmandiagram [horizontal=a to b] {
i1 -- [fermion] a -- [fermion] i2,
a -- [photon] b,
f1 -- [fermion] b -- [fermion] f2,
};


the second diagram:

\feynmandiagram [large, vertical=e to f] {
a -- [fermion] b -- [photon, momentum=\(k\)] c -- [fermion] d,
b -- [fermion, momentum'=\(p_{1}\)] e -- [fermion, momentum'=\(p_{2}\)] c,
e -- [gluon] f,
h -- [fermion] f -- [fermion] i;
};

the 3 diagram:

\feynmandiagram [horizontal=a to b] {
i1 [particle=\(e^{-}\)] -- [fermion] a -- [fermion] i2 [particle=\(e^{+}\)],
a -- [photon, edge label=\(\gamma\), momentum'=\(k\)] b,
f1 [particle=\(\mu^{+}\)] -- [fermion] b -- [fermion] f2 [particle=\(\mu^{-}\)],
};

the 4 diagram:

\feynmandiagram [horizontal=a to b] {
i1 [particle=\(e^{-}\)] -- [fermion, very thick] a -- [fermion, opacity=0.2] i2 [particle=\(e^{+}\)],
a -- [red, photon, edge label=\(\gamma\), momentum'={[arrow style=red]\(k\)}] b,
f1 [particle=\(\mu^{+}\)] -- [fermion, opacity=0.2] b -- [fermion, very thick] f2 [particle=\(\mu^{-}\)],
};

the 5 diagram:

\feynmandiagram [horizontal=a to b] {
i1 [particle=\(\tilde W\)] -- [plain, boson] a -- [anti fermion] i2 [particle=\(q\)],
a -- [charged scalar, edge label=\(\tilde q\)] b,
f1 [particle=\(\tilde g\)] -- [plain, gluon] b -- [fermion] [particle=\(q\)],
};

the 6 diagram:

% No invisible edge to keep the two photons together
\feynmandiagram [small, horizontal=a to t1] {
a [particle=\(\pi^{0}\)] -- [scalar] t1 -- t2 -- t3 -- t1,
t2 -- [photon] p1 [particle=\(\gamma\)],
t3 -- [photon] p2 [particle=\(\gamma\)],
};

the 7 diagram:

% Invisible edge ensures photons are parallel
\feynmandiagram [small, horizontal=a to t1] {
a [particle=\(\pi^{0}\)] -- [scalar] t1 -- t2 -- t3 -- t1,
t2 -- [photon] p1 [particle=\(\gamma\)],
t3 -- [photon] p2 [particle=\(\gamma\)],
p1 -- [opacity=0.2] p2,
};

the 8 diagram:

% Using the default spring layout
\feynmandiagram [horizontal=a to b] {
a [particle=\(\mu^{-}\)] -- [fermion] b -- [fermion] f1 [particle=\(\nu_{\mu}\)],
b -- [boson, edge label=\(W^{-}\)] c,
f2 [particle=\(\overline \nu_{e}\)] -- [fermion] c -- [fermion] f3 [particle=\(e^{-}\)],
};

the 9 diagram:

% Using the layered layout
\feynmandiagram [layered layout, horizontal=a to b] {
a [particle=\(\mu^{-}\)] -- [fermion] b -- [fermion] f1 [particle=\(\nu_{\mu}\)],
b -- [boson, edge label'=\(W^{-}\)] c,
c -- [anti fermion] f2 [particle=\(\overline \nu_{e}\)],
c -- [fermion] f3 [particle=\(e^{-}\)],
};

%the diagram 10:

% Using the layered layout f2--c--f3
% \feynmandiagram [layered layout, horizontal=a to b] {
% a [particle=\(\mu^{-}\)] -- [fermion] b -- [fermion] f1 [particle=\(\nu_{\mu}\)],
% b -- [boson, edge label'=\(W^{-}\)] c,
% f2 -- [anti fermion] c [particle=\(\overline \nu_{e}\)] -- 
% [fermion] f3 [particle=\(e^{-}\)],
% };

the diagram 11:

% mannuly decide the drawing process
\begin{tikzpicture}
    \begin{feynman}
        \vertex (a) {\(\mu^{-}\)};
        \vertex [right=of a] (b);
        \vertex [above right=of b] (f1) {\(\nu_{\mu}\)};
        \vertex [below right=of b] (c);
        \vertex [above right=of c] (f2) {\(\overline \nu_{e}\)};
        \vertex [below right=of c] (f3) {\(e^{-}\)};
        \diagram* {
        (a) -- [fermion] (b) -- [fermion] (f1),
        (b) -- [boson, edge label'=\(W^{-}\)] (c),
        (c) -- [anti fermion] (f2),
        (c) -- [fermion] (f3),
        };
    \end{feynman}
\end{tikzpicture}

the diagram 12:
\feynmandiagram [nodes=circle, horizontal=a1 to b3] {
a1 -- {b1, b2, b3 -- {c1, c2 -- d1}}
};




\end{document}