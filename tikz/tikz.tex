% Circumscribed Parallelepiped
% Author: Axel Pavillet
\documentclass[tikz,border=10pt]{standalone}
%%%<
\usepackage{verbatim}
\usepackage{tikz}
\usepackage{tikz-feynman}
%%%>
\begin{comment}
:Title: Circumscribed Parallelepiped
:Tags: 3D;Geometry;Mathematics
:Author: Axel Pavillet
:Slug: parallelepiped

This is a drawing of a tetrahedron inscibed in a parallelepiped. 
See the following reference p. 58-63 \S 189 to 202

  @BOOK{altshiller1935modern,
    title     = {Modern pure solid geometry},
    publisher = {The Macmillan company},
    year      = {1935},
    author    = {Altshiller-Court, N.},
    address   = {New York},
    edition   = {first},
    lccn      = {35024297},
    url       = {http://books.google.ca/books?id=DDYGAQAAIAAJ}
  }
\end{comment}

\begin{document}
\begin{tikzpicture}[font=\LARGE] 

\coordinate [label=left:a](a) at (0,0);
\node[draw,circle,label=right:b] (b) at (6,4){};
\draw (a) to (b);
\draw (a) to [out=90, in=90] (b);
\draw (a) to [bend right=60] (b);

\end{tikzpicture}
\end{document}

\coordinate [label=left:{$a$}](a) at (0,0);
\draw (a) circle (0.5);
\node[inner color=white, outer color=orange,inner sep=0.5cm] (b) at (5,2){$b$};
\draw (a)--(b);
\draw (a) controls (1,3) and (5,5)--(b);
\draw (a) -| (b);
\draw (a)|- (b);

\coordinate [label=left:a](a) at (0,0);
\node[draw,circle,label=right:b] (b) at (6,4);
(a)to(b),
(a)to [out=90, in=90]b,
(a) to [bend right=60]b,

