% Circumscribed Parallelepiped
% Author: Axel Pavillet
\documentclass[tikz,border=10pt]{standalone}
%%%<
\usepackage{verbatim}
\usepackage{tikz}
\usepackage{tikz-feynman}
%%%>
\begin{comment}
:Title: Circumscribed Parallelepiped
:Tags: 3D;Geometry;Mathematics
:Author: Axel Pavillet
:Slug: parallelepiped

This is a drawing of a tetrahedron inscibed in a parallelepiped. 
See the following reference p. 58-63 \S 189 to 202

  @BOOK{altshiller1935modern,
    title     = {Modern pure solid geometry},
    publisher = {The Macmillan company},
    year      = {1935},
    author    = {Altshiller-Court, N.},
    address   = {New York},
    edition   = {first},
    lccn      = {35024297},
    url       = {http://books.google.ca/books?id=DDYGAQAAIAAJ}
  }
\end{comment}
\begin{document}
\begin{tikzpicture}[font=\LARGE] 

    \begin{feynman}

        \vertex (a1) {\(p\)};
        \vertex[right=2cm of a1] (a2);
        \vertex[right=4cm of a2] (a3);
        \vertex[right=2cm of a3] (a4) {\(p\)};

        \diagram* {
        {
            [edges=fermion]
        (a1) -- (a2) -- (a3) -- (a4),
        },
        (a2) -- [ edge label=\(\pi^+\),half left,charged scalar] (a3),
        };
    \end{feynman}

\end{tikzpicture}
\end{document}
