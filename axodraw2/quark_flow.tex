 %automatica generated by python script
    \documentclass[a4paper]{article}
        \usepackage{axodraw2}
        \usepackage{pstricks}
        \usepackage{color}
        \begin{document}
        
    \begin{center}
    \begin{axopicture}(100,200) 
    
        % (起点),(终点)
        % 水平的费米子线1
        \Line[arrow,arrowpos=0.5,arrowlength=6,arrowwidth=3,arrowinset=0.1](10,70)(90.0,70.0)
        % 圆弧
        \Arc[arrow,dash,clockwise](50.0,70.0)(30,180,360)
        % 文字部分
        \Text(10, 80)(0){$p$}
        \Text(90.0, 80.0)(0){$p$}
        \Text(50.0, 80.0)(0){$n$}
        \Text(50.0, 110.0)(0){$\pi^{+}$}
        
        % (起点),(终点)
        % 水平的费米子线1
        \Line[arrow,arrowpos=0.5,arrowlength=6,arrowwidth=3,arrowinset=0.1](10,10)(50.0,10)
        % 水平的费米子线2
        \Line[arrow,arrowpos=0.5,arrowlength=6,arrowwidth=3,arrowinset=0.1](50.0,10)(90,10)
        % 倾斜的介子线
        \Line[arrow,arrowpos=0.5,arrowlength=6,arrowwidth=3,arrowinset=0.1,dash](10.0,50.0)(50.0,10)
        %倾斜的光子线
        \Photon(90.0,50.0)(50.0,10){3}{7}
        % 额外相互作用的点
        \Vertex(50.0,10.0){3}
        % 文字部分
        \Text(10, 20)(0){$\Sigma^{+}$}
        \Text(90, 20)(0){$\bar{\Xi^0}$}
        \Text(20.0, 50.0)(0){$K^{-}$}
        \Text(80.0, 50.0)(0){$\gamma$}
        
        \end{axopicture}
        \end{center}
        \end{document}
        