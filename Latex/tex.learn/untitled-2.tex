\documentclass[UTF8]{ctexart}

\setlength{\parindent}{0pt}
\setlength{\parskip}{1ex plus 0.5ex minus 0.2ex}

% define the title
\author{H.~Partl}
\title{Minimalism}
\usepackage{amsfonts}
\usepackage{amsmath}
\usepackage{latexsym}
\usepackage{amssymb}
\usepackage{makeidx}
\makeindex

\newcommand{\txsit}[1]
{This is the \emph{#1} Short
	Introduction to \LaTeXe}


% preamble
\newtheorem{law}{Law}
\newtheorem{jury}[law]{Jury}

\newtheorem{mur}{Murphy}[subsection]

\begin{document}
% generates the title
\maketitle
% insert the table of contents
\tableofcontents
\section{Start}
Well, and here begins my lovely article.

\section{End}
\ldots{} and here it ends.

\ldots when Einstein introduced his formula
\begin{equation}
e = m \cdot c^2 \; ,
\end{equation}
which is at the same time the most widely known
and the least well understood physical formula.
% Example 2
\ldots from which follows Kirchoff’s current law:
\begin{equation}
\sum_{k=1}^{n} I_k = 0 \; .
\end{equation}
Kirchhoff’s voltage law can be derived \ldots
% Example 3
\ldots which has several advantages.
\begin{equation}
I_D = I_F - I_R
\end{equation}
is the core of a very different transistor model. \ldots

\section{sfs}

不是%Not
shelfful\\
而是%but
shelf\mbox{}ful

\section{2.4.7}

Stra\ss e

\i



\u o

\TU

\j

\section{sfsaf}

Mr.~Smith was happy to see her\\
cf.~Fig.~5\\
I like BASIC\@. What about you?

\section{safsdaf}

A reference to this subsection
\label{sec:this} looks like:
‘‘see section~\ref{sec:this} on
page~\pageref{sec:this}.’’


Footnotes\footnote{This is
a footnote.} are often used
by people using \LaTeX.

\section{saofjsd}

\textit{You can also
\emph{emphasize} text if
it is set in italics,}
\textsf{in a
\emph{sans-serif} font,}
\texttt{or in
\emph{typewriter} style.}

\section{asfd}

\flushleft
\begin{enumerate}
\item You can mix the list
environments to your taste:
\begin{itemize}
\item[o] But it might start to
look silly.
\item[-] With a dash.
\end{itemize}
\item Therefore remember:
\begin{description}
\item[Stupid] things will not
become smart because they are
in a list.
\item[Smart] things, though, can be
presented beautifully in a list.
\end{description}
\end{enumerate}

\section{saf}

\begin{flushleft}
	This text is\\ left-aligned.
	\LaTeX{} is not trying to make
	each line the same length.
\end{flushleft}

\section{saf}
\begin{flushright}
	This text is right-\\aligned.
	\LaTeX{} is not trying to make
	each line the same length.
\end{flushright}

\begin{center}
	At the centre\\of the earth
\end{center}


\section{2.11.3 Quote, Quotation, and Verse}

A typographical rule of thumb
for the line length is:
\begin{quote}
	On average, no line should
	be longer than 66 characters.
\end{quote}
This is why \LaTeX{} pages have
such large borders by default and
also why multicolumn print is
used in newspapers.


A typographical rule of thumb
for the line length is:
\begin{quote}
	On average, no line should
	be longer than 66 characters.
\end{quote}
This is why \LaTeX{} pages have
such large borders by default and
also why multicolumn print is
used in newspapers.



I know only one English poem by
heart. It is about Humpty Dumpty.
\begin{flushleft}
	\begin{verse}
		Humpty Dumpty sat on a wall:\\
		Humpty Dumpty had a great fall.\\
		All the King’s horses and all
		the King’s men\\
		Couldn’t put Humpty together
		again.
	\end{verse}
\end{flushleft}


\section{sfsf}
The \verb|\ldots| command \ldots
\begin{verbatim}
10 PRINT "HELLO WORLD ";
20 GOTO 10
\end{verbatim}


\begin{verbatim*}
the starred version of
the verbatim
environment emphasizes
the spaces in the text
\end{verbatim*}

\section{tabular}
\begin{tabular}{|r|l|}
	\hline
	7C0 & hexadecimal \\
	3700 & octal \\
	\cline{2-2}
	11111000000 & binary \\
	\hline \hline
	1984 & decimal \\
	\hline
\end{tabular}



\begin{tabular}{|p{4.7cm}|}
	\hline
	Welcome to Boxy’s paragraph.
	We sincerely hope you’ll
	all enjoy the show.\\
	\hline
\end{tabular}


\begin{tabular}{@{} l @{}}
	\hline
	no leading space\\
	\hline
\end{tabular}


\begin{tabular}{l}
	\hline
	leading space left and right\\
	\hline
\end{tabular}
\section{tabular}
\begin{tabular}{c r @{.} l}
	Pi expression &
	\multicolumn{2}{c}{Value} \\
	\hline
	$\pi$ & 3&1416 \\
	$\pi^{\pi}$ & 36&46 \\
	$(\pi^{\pi})^{\pi}$ & 80662&7 \\
\end{tabular}

\subsection{tabular}
\begin{tabular}{|c|c|}
	\hline
	\multicolumn{2}{|c|}{Ene} \\
	\hline
	Mene & Muh! \\
	\hline
\end{tabular}

\subsection{square}

Figure~\ref{white} is an example of Pop-Art.
\begin{figure}[!hbp]
	\makebox[\textwidth]{\framebox[5cm]{\rule{0pt}{5cm}}}
	\caption{Five by Five in Centimetres.} \label{white}
\end{figure}


\section{I am considerate
	\protect\footnote{and protect my footnotes}}

\subsection{title}
Add $a$ squared and $b$ squared
to get $c$ squared. Or, using
a more mathematical approach:
$c^{2}=a^{2}+b^{2}$


\TeX{} is pronounced as
$\tau\epsilon\chi$.\\[20pt]
100~m$^{3}$ of water\\[6pt]
This comes from my $\heartsuit$

\subsection{title}
Add $a$ squared and $b$ squared
to get $c$ squared. Or, using
a more mathematical approach:

\[
c^{2}=a^{2}+b^{2}
\]
And just one more line.

\begin{equation} \label{eq:eps}
\epsilon > 0
\end{equation}
From (\ref{eq:eps}), we gather
\ldots

\subsection{title}
$\lim_{n \to \infty}
\sum_{k=1}^n \frac{1}{k^2}
= \frac{\pi^2}{6}
$

\begin{displaymath}
\lim_{n \to \infty}
\sum_{k=1}^n \frac{1}{k^2}
= \frac{\pi^2}{6}
\end{displaymath}

\subsection{title}
a\,b

a\qquad b

a\quad 
\subsection{title}
\begin{equation}
\forall x \in \mathbf{R}:
\qquad x^{2} \geq 0
\end{equation}

\begin{equation}
x^{2} \geq 0\qquad
\textrm{for all }x\in\mathbf{R}
\end{equation}

\begin{displaymath}
x^{2} \geq 0\qquad
\textrm{for all }x\in\mathbb{R}
\end{displaymath}

\begin{equation}
a^x+y \neq a^{x+y}
\end{equation}

\subsection{title}
$a_{1}$ \qquad $x^{2}$ \qquad\\[15pt]
$e^{-\alpha t}$ \qquad\\[15pt]
$a^{3}_{ij}$\\[15pt]
$e^{x^2} \neq {e^x}^2$\\[15pt]\index{exponential}
$\overline{m+n}$ \qquad
$\underline{m+n}$\\[15pt]
$\underbrace{ a+b+\cdots+z }_{26}$\\[15pt]
\begin{displaymath}
y=x^{2}\qquad y’=2x\qquad y’’=2
\end{displaymath}\\[15pt]
\begin{displaymath}
\vec a\quad\overrightarrow{AB}
\end{displaymath}\\[15pt]

\begin{displaymath}
\vec a\quad\overleftarrow{AB}
\end{displaymath} 

\begin{displaymath}
v = {\sigma}_1 \cdot {\sigma}_2
{\tau}_1 \cdot {\tau}_2
\end{displaymath}

$1/2$


\[\lim_{x \rightarrow 0}
\frac{\sin x}{x}=1\]

$a \pmod b $

$1\frac{1}{2}$~hours
\begin{displaymath}
\frac{ x^{2} }{ k+1 }\qquad
x^{ \frac{2}{k+1} }\qquad
x^{ 1/2 }
\end{displaymath}

\begin{displaymath}
{n \choose k}\qquad {x \atop y+2}
\end{displaymath}

\begin{displaymath}
\int f_N(x) \stackrel{!}{=} 1
\end{displaymath}

\begin{displaymath}
\sum_{i=1}^{n} \qquad
\int_{0}^{\frac{\pi}{2}} \qquad
\prod_\epsilon
\end{displaymath}
\{\}

\begin{displaymath}
{a,b,c}\neq\{a,b,c\}
\end{displaymath}

\begin{displaymath}
1 + \left( \frac{1}{ 1-x^{2} }
\right) ^3
\end{displaymath}
\subsection{title}
$\Big( (x+1) (x-1) \Big) ^{2}$\\
$\big(\Big(\bigg(\Bigg($\quad
$\big\}\Big\}\bigg\}\Bigg\}$\quad
$\big\|\Big\|\bigg\|\Bigg\|$
\subsection{title}
\begin{displaymath}
x_{1},\ldots,x_{n} \qquad
x_{1}+\cdots+x_{n} \qquad
x_{1}+\ddots+x_{n} \qquad
x_{1}+\vdots+x_{n} \qquad
\end{displaymath}
\subsection{title}
\newcommand{\ud}{\mathrm{d}}
\begin{displaymath}
\int\!\!\!\int_{D} g(x,y)
\, \ud x\, \ud y
\end{displaymath}
instead of
\begin{displaymath}
\int\int_{D} g(x,y)\ud x \ud y
\end{displaymath}


\begin{displaymath}
\iint_{D} \,  \ud x \, \ud y
\end{displaymath}
\subsection{title}

\begin{displaymath}
\mathbf{X} =
\left( \begin{array}{ccc}
x_{11} & x_{12} & \ldots \\
x_{21} & x_{22} & \ldots \\
\vdots & \vdots & \ddots
\end{array} \right)
\end{displaymath}

\begin{displaymath}
y = \left\{ \begin{array}{ll}
a & \textrm{if $d>c$}\\
b+x & \textrm{in the morning}\\
l & \textrm{all day long}
\end{array} \right.
\end{displaymath}

\subsection{title}
\begin{displaymath}
\left(\begin{array}{c|c}
1 & 2 \\
\hline
3 & 4
\end{array}\right)
\end{displaymath}

\subsection{eqnarray}
\begin{eqnarray}
f(x) & = & \cos x \\
f’(x) & = & -\sin x \\
\int_{0}^{x} f(y)dy &
= & \sin x
\end{eqnarray}
{\setlength\arraycolsep{2pt}
	\begin{eqnarray}
	\sin x & = & x -\frac{x^{3}}{3!}
	+\frac{x^{5}}{5!}-{}
	\nonumber\\
	& & {}-\frac{x^{7}}{7!}+{}\cdots
	\end{eqnarray}}
\begin{eqnarray}
\lefteqn{ \cos x = 1
	-\frac{x^{2}}{2!} +{} }
\nonumber\\
& & {}+\frac{x^{4}}{4!}
-\frac{x^{6}}{6!}+{}\cdots
\end{eqnarray}
\subsection{title}
\begin{displaymath}
{}^{12}_{\phantom{1}6}\textrm{C}
\qquad \textrm{versus} \qquad
{}^{12}_{6}\textrm{C}
\end{displaymath}

\begin{displaymath}
\Gamma_{ij}^{\phantom{ij}k}
\qquad \textrm{versus} \qquad
\Gamma_{ij}^{k}
\end{displaymath}

\subsection{title}
\begin{equation}
2^{\textrm{nd}} \quad
2^{\mathrm{nd}}
\end{equation}

\[\displaystyle(123)\]

\subsection{title}
\begin{displaymath}
\mathop{\mathrm{corr}}(X,Y)=
\frac{\displaystyle
	\sum_{i=1}^n(x_i-\overline x)
	(y_i-\overline y)}
{\displaystyle\biggl[
	\sum_{i=1}^n(x_i-\overline x)^2
	\sum_{i=1}^n(y_i-\overline y)^2
	\biggr]^{1/2}}
\end{displaymath}


\subsection{title}
\begin{displaymath}
\mathop{\mathrm{corr}}(X,Y)=
\frac{\displaystyle
	\sum_{i=1}^n(x_i-\overline x)
	(y_i-\overline y)}
{\displaystyle\left[
	\sum_{i=1}^n(x_i-\overline x)^2
	\sum_{i=1}^n(y_i-\overline y)^2
	\right]^{1/2}}
\end{displaymath}

\begin{law} \label{law:box}
	Don’t hide in the witness box
\end{law}
\begin{jury}[The Twelve]
	It could be you! So beware and
	see law~\ref{law:box}\end{jury}
\begin{law}No, No, No\end{law}
\subsection{title}

\begin{mur}
	If there are two or more\index{asdfa}
	ways to do something, and
	one of those ways can result
	in a catastrophe, then
	someone will do it.\footnote{what the fuck}
\end{mur}

\begin{displaymath}
\mu, M \qquad \mathbf{M} \qquad
\mbox{\boldmath $\mu, M$}
\end{displaymath}

\subsection{title}
\begin{displaymath}
\mu, M \qquad
\boldsymbol{\mu}, \boldsymbol{M}
\end{displaymath}

Partl~\cite{pa} has
proposed that \ldots

\printindex

\subsection{title}
% in the document body:
\begin{itemize}
	\item \txsit{not so}
	\item \txsit{very}
\end{itemize}

\begin{thebibliography}{99}
	\bibitem{pa} H.~Partl:
	\emph{German \TeX},
	TUGboat Volume~9, Issue~1 (1988)
\end{thebibliography}

x\hspace{\stretch{1}}
x\hspace{\stretch{3}}x

这是\hspace{1.5cm}一段长为
1.5 厘米的空白。

Some text \ldots

\vspace{\stretch{1}}
这一行将出现在页的最后。\pagebreak

fuck me 
\bigskip
from front side

\makebox[\textwidth]{%
	c e n t r a l}\par
\makebox[\textwidth][s]{%
	s p r e a d}\par
\framebox[1.1\width]{Guess I’m
	framed now!} \par
\framebox[0.8\width][r]{Bummer,
	I am too wide} \par
\framebox[1cm][l]{never
	mind, so am I}
Can you read this?

\settowidth{\parindent}{slslsls}

\makebox[0pt][r]{slslsls}

\subsection{title}
\flushleft
\newenvironment{vardesc}[1]{%
	\settowidth{\parindent}{#1:\ }
	\makebox[0pt][r]{#1:\ }}{}
\begin{displaymath}
a^2+b^2=c^2
\end{displaymath}
\flushright
\begin{vardesc}{Where}$a$,
	$b$ -- are adjunct to the right
	angle of a right-angled triangle.
	$c$ -- is the hypotenuse of
\flushleft
	the triangle and feels lonely.
	$d$ -- finally does not show up
	here at all. Isn’t that puzzling?
\end{vardesc}

$\vec{e_r}$















\end{document}