\documentclass[./main.tex]{subfiles}
\setcounter{chapter}{3}

\chapter{引言-深度非弹散射}

深度非弹性散射解面参数化为:

\begin{equation}\begin{aligned}
        \label{eq.1.1}
        %%%%%+++++++++++++++++++++++---------------------
        \frac{d^2 \sigma}{d E^\prime d \Omega}=
        \frac{a \alpha^2}{q^4}
        [
            \frac{F_2}{\nu} \cos^2 \frac{\theta^2}{2}+
            \frac{F_1}{M} \sin^2 \frac{\theta^2}{2}
        ]
        %%%%%+++++++++++++++++++++++
    \end{aligned}\end{equation}

In which, $Q^2=-q^2,~\alpha^2\equiv \frac{e^2}{4 \pi},~\nu=E^\prime -E$, and,
重新标度的变量:$x=Q^2/2M\nu$。

$F_1$和$F_2$是核子的电磁结构函数,它们是$x$和$Q^2$的函数,

在深度非弹性散射中,如果$\nu$和$Q^2$非常大,结构函数$F_1$和$F_2$不再同时是$\nu$和$Q^2$的函数,
而是$x$的函数,这时结构函数可参数化为

\begin{equation}\begin{aligned}
        \label{eq.1.2}
        %%%%%+++++++++++++++++++++++---------------------
        F_2(x)=\sum e_q^2[q(x)+\bar{q}(x)],\quad F_1(x)=2x F_2(x)
        %%%%%+++++++++++++++++++++++
    \end{aligned}\end{equation}

In which, $e_q$ is the charge of partons, $q(x)$ is the unpolarized distribution function of partons.

for nucleons:

\begin{equation}\begin{aligned}
        \label{eq.1.3}
        %%%%%+++++++++++++++++++++++---------------------
        F_2^p(x) & {}
        =\frac{4}{9}[u_p(x)+\bar{u}_p(x)]
        +\frac{1}{9}[d_p(x)+\bar{d}_p(x)]
        +\frac{1}{9}[s_p(x)+\bar{s}_p(x)]+\ldots \\
        %%%%%%%%%%
        F_2^n(x) & {}
        =\frac{4}{9}[u_n(x)+\bar{u}_n(x)]
        +\frac{1}{9}[d_n(x)+\bar{d}_n(x)]
        +\frac{1}{9}[s_n(x)+\bar{s}_n(x)]
        %%%%%+++++++++++++++++++++++
    \end{aligned}\end{equation}

in which, $q(x)$ and $\bar{q}(x)$分别代表正反夸克分布函数。

因为$u,d,s$夸克质量与非弹性散射能量相比小得多,
假设胶子分裂成$u,\bar{u}$夸克和$d,\bar{d}$夸克的概率相等。
如果核子海夸克满足同位旋对称性即$\bar{d}_p(x)=\bar{u}_p(x)$,??
质子和中子的结构函数,$F_2^p(x)$和$_2^n(x)$应满足$GottFried$求和规则:

\begin{equation}\begin{aligned}
        \label{eq.1.3}
        %%%%%+++++++++++++++++++++++---------------------
        S_G=\int_0^1\frac{dx}{x}
        [
            F^2_p(x)-F^2_n(x)=\frac{1}{3}
        ]
        %%%%%+++++++++++++++++++++++
    \end{aligned}\end{equation}

in which, 我们假设核子奇异海夸克分布满足$s_n(x)=s_p(x)$.\\
但是NMC实验返现$S_G=0235\pm 0.026(Q^2=4 \gevs )$
这说明核子夸克分布对称性也许被破坏$\int_0^1 dx [\bar{d}_p-\bar{u}_p=0] \neq 0$ ??

另一个实验依据是 Paschos-Wolfenstein 关系

\begin{equation}\begin{aligned}
        \label{eq.1.3}
        %%%%%+++++++++++++++++++++++---------------------
        R_N^-
        = \frac{
        \sigma_{NC}^{\nu N}-\sigma_{NC}^{\bar{\nu} N}
        }{
        \sigma_{CC}^{\nu N}-\sigma_{CC}^{\bar{\nu} N}
        }
        =R^- - \delta R_S^-
        %%%%%+++++++++++++++++++++++
    \end{aligned}\end{equation}

in which, $R^-=\frac{1}{2}-\sin^2(\theta_W)$是原来的奇异夸克部分,
$\delta R_S^-=[1-\frac{7}{3}\sin^2(\theta_W)]\frac{S^-}{S^- + Q_\nu}$是奇异夸克对称性破缺部分。$S^-=\int dx x[s-\bar{s}],\quad Q_v=\int dx x[u_v+d_v]$
in which, the subscript $v$ refers to ``valence''

根据他们的实验结果,核子夸克海里面的奇异夸克对核子结构有一定的贡献。\\
随着实验数据的丰富,人们发现了更多的核子海中对称性破缺现象,
比如极化的味道对称性破缺$\Delta \bar{d}-\Delta \bar{u}$和粲夸克对称性破缺。
虽然这种效应的贡献不大,但它体现出关于核子海夸克分布的更多信息。

可以肯定的是,这种效应在QCD框架下无法得到解释,核子海中的味道对称性破缺跟QCD微扰演化无关。
核子海夸克可以看成来自于胶子分裂成正反海夸克对,此过程满足CP对称性,因此不会出现味道对称性和奇异夸克对称性破缺现象。\\
那么最有可能的是,味道对称性破缺和奇异夸克对称性破缺跟核子束缚态的非微扰机制有关。随着能量的增大,胶子分裂成正反夸克对的过程仍然满足味道对称性,但最初的对称性破缺得以保留。

如泡利(Pauli)阻碍模型。根据夸克模型,质子包含两个$u_v$ 和一个$d_v$,而且受泡利不相容原
理的限制,在QCD演化过程中产生的海夸克对$\bar{u}u$的个数少于海夸克对$\bar{d}d$的个数,
因而体现出味道对称性破缺。
但是泡利阻碍模型给出的核子味道对称性
$\int_0^1 [\bar{d}_p(x)-\bar{u}_p(x)]$无法解释全部实验数据。


另一方面,介子云(Meson cloud)在深度非弹性散射过程中的作用不可忽略,
尤其在核子的长距离结构中,它的作用更为重要。
介子云模型的主要观点是,核子外部结构由赝标量介子云组成,因而对散射过程有贡献。
介子云模型成功地解释了核子味道对称性破缺,
介子云模型还表明,手征对称性破缺应该也可以在其他种类的夸克分布中发现。
其中一个重要的问题是如何建立夸克分布的Mellin矩与QCD的关系。

正如[Phys.Rev.Lett.85,2892(2000).]所发现,可以从手征微扰理论得到非单态夸克分布的领头阶非解析项(LNA),它跟QCD具有相同的手征对称性 [22–24]。
格点QCD证明了,当π介子质量很大的时候,夸克分布的Mellin矩和其他观测量都能拟合到物理点。
此外,格点QCD还证实了QCD红外端的行为直接导致不为零的$\bar{d}-\bar{u}$。

在不同的模型中,各种费曼图对领头阶非解析项的贡献独立存在,而且依赖于红外端行为。
但是,在计算完全振幅的时候,需要选取具体的正规化方案来处理紫外发散。
各文献采用了不同的正规化方案,比如横向动量锐截断、
Pauli-Villars正规化和维数正规化以及有效形状因子方法。
有效形状因子方法考虑强子的有限大小和非定域性,其他的方法把强子看成点粒子。

在非协变正规化中,如果正规化因子是三维动量的函数,则电荷是守恒的,
因为电荷守恒与守恒流的时间分量有关。

另一方面,协变计算中,采用协变(相对论性)的正规化因子,将导致电荷守恒明显破缺。
理论上这些问题促使我们,在定域规范不变的拉氏量中引入非定域相互作用,
来解决电荷守恒被破坏的问题。

构造非定域拉氏量的过程见于[J.Terning,Phys.Rev.D44,887(1991).]。\\
$\sigma$介子的特性,由规范不变的相对论性非定域夸克模型体现。
非定域拉氏量中规范链接算符的出现,保证了相互作用的规范不变性,具体表现在产生了额外的费曼图。
考虑这些额外费曼图时,介子圈图对质子与中子电荷的总贡献完全抵消掉。额外费曼图使得核子奇异数为零。

非定域模型的构造过程更加自洽,并满足电荷守恒与Ward恒等式。
与Pauli-Villars正规化不同的是,非定域正规化跟定域手征拉氏量及其对称性有关,因此需要回顾手征对称性和手征拉氏量。

我们的计算包括SU(3)八重态和十重态重子贡献,
通过选取一个满足洛伦兹不变性和规范不变的正规化因子,使得圈图贡献收敛。
除此之外,为了更直观的解释不同正规化方法之间的区别,及其所限制的物理量之间的差别,
我们还会讨论用Pauli-Villars正规化得到的的SU(3)八重态分裂函数,
及其对核子奇异夸克对称性破缺的贡献。

本文结构如下:

\begin{itemize}
    \item[第二章] 手征对称性和手征微扰理论的构造过程,非定域拉氏量的基本思想和构造过程
    \item[第三章] 非定域和Pauli-Villars正规化中,质子$\to$重子$+$赝标量介子,的分裂函数(包括八重态和十重态)
    \item[第四章] 推导中间态介子和重子夸克分布,对核子奇异夸克分布进行数值计算
    \item[第五章] 总结理论计算结果
  \end{itemize}