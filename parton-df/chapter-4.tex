\documentclass[./main.tex]{subfiles}
\setcounter{chapter}{3}


\chapter{引言-深度非弹散射}

深度非弹性散射解面参数化为:

\begin{equation}\begin{aligned}
        \label{eq.1.1}
        %%%%%+++++++++++++++++++++++---------------------
        \frac{d^2 \sigma}{d E^\prime d \Omega}=
        \frac{a \alpha^2}{q^4}
        [
            \frac{F_2}{\nu} \Cos^2 \frac{\theta^2}{2}+
            \frac{F_1}{M} \Sin^2 \frac{\theta^2}{2}
        ]
        %%%%%+++++++++++++++++++++++
    \end{aligned}\end{equation}

In which, $Q^2=-q^2,~\alpha^2\equiv \frac{e^2}{4 \pi},~\nu=E^\prime -E$, and,
重新标度的变量:$x=Q^2/2M\nu$。

$F_1$和$F_2$是核子的电磁结构函数,它们是$x$和$Q^2$的函数,

在深度非弹性散射中,如果$\nu$和$Q^2$非常大,结构函数$F_1$和$F_2$不再同时是$\nu$和$Q^2$的函数,
而是$x$的函数,这时结构函数可参数化为

\begin{equation}\begin{aligned}
        \label{eq.1.2}
        %%%%%+++++++++++++++++++++++---------------------
        F_2(x)=\sum e_q^2[q(x)+\bar{q}(x)],\quad F_1(x)=2x F_2(x)
        %%%%%+++++++++++++++++++++++
    \end{aligned}\end{equation}

In which, 1 is the charge of partons, 1 is the unpolarized distribution function of partons