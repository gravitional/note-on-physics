\documentclass[./main.tex]{subfiles}
\setcounter{chapter}{0}


\chapter{留数定理}

对于多项式单宗量复变函数积分,总结起来:

\begin{equation}\begin{aligned}
        %%\label{eq.6.1.2}
        %%%%%+++++++++++++++++++++++---------------------
         & {} \frac{1}{2 \pi i } \oint  \frac{dz}{z-\alpha}=
        \begin{cases}
            0,~\alpha \mbox{ isn't included in l} \\
            %%%%%+++++++++++++++++++++++
            1,~\alpha \mbox{is included in l}
        \end{cases}                                     \\
         & {} \frac{1}{2 \pi i } \oint (z-\alpha)^n dz = 0 (n\neq -1)
        %%%%%+++++++++++++++++++++++
    \end{aligned}\end{equation}

因为原函数一个是多值函数$\ln (z-\alpha)$ ,一个是单值函数$(z-\alpha)^{n+1}/(n+1)$

柯西定理(2.2.1)指出,如被积函数$f(z)$在回路$l$所围闭区域上是解析的,则回路积分$\oint f(z) dz$等于$0$. \\
现考虑回路$l$包围$f(z)$的奇点的情形。

先设$l$只包围着$f(z)$的一个孤立奇点$z_0$,在以$z_0$为圆心而半径为零的圆环域上将$f(z)$展为洛朗级数

\begin{equation}\begin{aligned}
        %%\label{eq.6.1.2}
        %%%%%+++++++++++++++++++++++---------------------
        f(z)=\sum\limits_{k=-\infty}^{\infty} a_k (z-z-0)^k
        %%%%%+++++++++++++++++++++++
    \end{aligned}\end{equation}

由柯西定理回路任意易形,洛朗级数除去$k=-1$的一项之外全为$0$,而$k=-1$的一项的积分等于$2 \pi i$. 于是

\begin{equation}\begin{aligned}
        %%\label{eq.6.1.2}
        %%%%%+++++++++++++++++++++++---------------------
        \oint_l f(z) dz =2 \pi i a_{-1}
        %%%%%+++++++++++++++++++++++
    \end{aligned}\end{equation}

洛朗级数的$(z-z_0)^{-1}$项的系数因而具有特别重要的地位,专门起了名字,称为函数$f(z)$在点$z_0$的\textbf{留数}(或残数), 通常记作$\res\,f(z_0)$,这样,

\begin{equation}\begin{aligned}
        %%\label{eq.6.1.2}
        %%%%%+++++++++++++++++++++++---------------------
        \oint_l f(z) dz =2 \pi i \res \,f(z_0)
        %%%%%+++++++++++++++++++++++
    \end{aligned}\end{equation}

如果$l$包围着$f(z)$的$n$个孤立奇点$b_1,b_2,\cdots,b_n$的情形. 做回路$l_1,l_2,\cdots,l_n$分别包围$b_1,b_2,\cdots,b_n$. 并使每个回路只包围一个奇点, 按照柯西定理,

\begin{equation}\begin{aligned}
        %%\label{eq.6.1.2}
        %%%%%+++++++++++++++++++++++---------------------
        \oint_l f(z) dz =
        \oint_{l1} f(z) dz
        +\oint_{l2} f(z) dz
        + \cdots +\oint_{ln} f(z) dz
        %%%%%+++++++++++++++++++++++
    \end{aligned}\end{equation}

于是得到,

\begin{equation}\begin{aligned}
        \label{eq.4.1.4}
        %%%%%+++++++++++++++++++++++---------------------
        \oint_l f(z) dz =
        2 \pi i
            [
                \res f(b_1) + \res f(b_2) + \cdots + \res f(b_n)
            ]
        %%%%%+++++++++++++++++++++++
    \end{aligned}\end{equation}

\begin{theorem}{留数定理}{}%%\ref{thm:label}
    设函数$f(z)$在回路$l$所围区域$B$上除有限个孤立奇点$b_1,b_2,\cdots,b_n$外解析,在闭区域$\bar{B}$上除$b_1,b_2,\cdots,b_n$外连续,则
    \begin{equation}\begin{aligned}
            %%\label{eq.6.1.2}
            \oint f(z)dz=2 \pi i \sum\limits_{j=1}^n \res f(b_j)
        \end{aligned}\end{equation}
\end{theorem}

留数定理将回路积分归结为被积函数在回路所围区域上各奇点的留数之和.

函数$f(z)$在全平面上所有各点的留数之和为零.这里说的所有各点包括无限远点和有限远的奇点.

\section{留数定理-prepare}

一般地, 在以奇点为圆心的圆环域上将函数展开为洛朗级数, 取它的负一次幂项的系数就行了.
对于极点, 可以不用做洛朗展开而直接求出留数.

设$z_0$是$f(z)$的单极点. 洛朗展开应是

\begin{equation}\begin{aligned}
        \label{eq.4.1.4}
        %%%%%+++++++++++++++++++++++---------------------
        f(z)=\frac{a_{-1}}{z-z_0}+a_0+a_1 (z-z_0)+a_2(z-z_0)^2+\cdots
        %%%%%+++++++++++++++++++++++
    \end{aligned}\end{equation}

这样,

\begin{equation}\begin{aligned}
        \label{eq.4.1.8}
        %%%%%+++++++++++++++++++++++---------------------
        \lim_{z \to z_0}[(z-z_0)f(z)] = \res f(z_0)
        %%%%%+++++++++++++++++++++++
    \end{aligned}\end{equation}

\eqref{eq.4.1.8}既可用来计算函数$f(z)$在单极点$z_0$的留数, 也可用来判断$z_0$是否为函数$f(z)$
的单极点.

若$f(z)$可以表示为$P(z)/Q(z)$的特殊形式,
其中$P(z)$和$Q(z)$都在$z_0$点解析,
$z_0$是$Q(z)$的一阶零点.
$P(z_0)\neq 0$, 从而$z_0$是$f(z)$的一阶极点, 则

\begin{equation}\begin{aligned}
        \label{eq.4.1.9}
        %%%%%+++++++++++++++++++++++---------------------
        \res f(z_0) = \lim_{z \to z_0} \frac{P(z)}{Q(z)} =  \frac{P(z_0)}{Q'(z_0)}
        %%%%%+++++++++++++++++++++++
    \end{aligned}\end{equation}

上式最后一步应用了洛必达法则.

同理, 若$z_0$是$f(z)$的$m$阶极点. 洛朗展开应是

\begin{equation}\begin{aligned}
        \label{eq.4.1.10}
        %%%%%+++++++++++++++++++++++---------------------
        f(z)= \frac{a_{-m}}{(z-z_0)^m}
        +\frac{a_{-(m-1)}}{(z-z_0)^{m-1}}
        +\cdots
        +\frac{a_{-1}}{z-z_0}
        +a_0
        +a_1(z-z_0)
        +a_2(z-z_0)^2
        +\cdots
        %%%%%+++++++++++++++++++++++
    \end{aligned}\end{equation}

故有, 判断极点:

\begin{equation}\begin{aligned}
        \label{eq.4.1.12}
        %%%%%+++++++++++++++++++++++---------------------
        \lim_{z\to z_0}[(z - z_0)^m f(z)]= \text{非零有限值}
        %%%%%+++++++++++++++++++++++
    \end{aligned}\end{equation}

求留数:

\begin{equation}\begin{aligned}
        \label{eq.4.1.12}
        %%%%%+++++++++++++++++++++++---------------------
        \res f(z_0) = \lim_{z\to z_0} \frac{1}{(m-1)!}
        {
            \frac{d^{m-1}}{d z^{m-1}}
            [
                (z-z_0)^m f(z)
            ]
        }
        %%%%%+++++++++++++++++++++++
    \end{aligned}\end{equation}

用以上公式, 可以判断函数$f(z)$的{极点阶数},并求出$f(z)$在\textbf{极点的留数}

\section{应用留数定理计算实变函数定积分}

首先将实变函数定积分跟复变函数回路积分联系起来. 其要点如下:\\
定积分$\int_a^b f(x)dx$的积分区间$[a,b]$可以看作是复数平面上实轴上的一段$l_1$,于是,
或者利用自变数的变换把$l_1$变换为某个新的复数平面上的回路,这样就可以应用留数定理了;\\
或者另外补上一段曲线$l_2$,使$l_1$和$l_2$合成回路$l$,$l$包围着区域$B$。将$f(x)$解析延拓到闭区域$B$(这个延拓往往只是把$f(x)$改为$f(z)$而已),并把它沿着$l$积分,

\begin{equation}\begin{aligned}
        \label{eq.4.1.12}
        %%%%%+++++++++++++++++++++++---------------------
        \oint f(z) dz = \int_{l_1} f(x)dx + \int_{l_2} f(z) dz
        %%%%%+++++++++++++++++++++++
    \end{aligned}\end{equation}

上式左边可以应用留数定理,右边第一个积分就是所求的定积分。如果右边第二个积分较易算出(往往证明为零)或可用第一个积分表出,问题就算解决了。

下面介绍几种类型的实变定积分。

\begin{enumerate}
    \item[类型一]
          $\int_0^{2\pi} R(\cos x, \sin x)d x$. 被积函数是三角函数的有理式;积分区间是$[0,2\pi]$. \\
          作自变数代换
          \begin{equation}\begin{aligned}
                  \label{eq.4.2.1}
                  %%%%%+++++++++++++++++++++++---------------------
                  z=e^{ix}
                  %%%%%+++++++++++++++++++++++
              \end{aligned}\end{equation}
          当实变数$x$从$0$变到$2\pi$时,复变数$z=e^{ix}$从$z=1$出发沿单位圆$|z|=1$逆时针走一圈又回到$z=1$,实变定积分化为复变回路积分,就可以应用留数定理了。
          至于实变定积分里的$\cos x$和$\sin x$,则按\eqref{eq.4.2.1}而做如下变换
          \begin{equation}\begin{aligned}
                  \label{eq.4.2.2}
                  %%%%%+++++++++++++++++++++++---------------------
                  \cos x=\frac{1}{2}(z+z^{-1}),\quad \sin x=\frac{1}{2 i}(z-z^{-1}),\quad
                  dx=\frac{1}{iz}dz
                  %%%%%+++++++++++++++++++++++
              \end{aligned}\end{equation}
          于是,原积分化为
          \begin{equation}\begin{aligned}
                  \label{eq.4.2.3}
                  %%%%%+++++++++++++++++++++++---------------------
                  I= \oint_{|z|=1} R (\frac{z+z^{-1}}{2},\frac{z-z^{-1}}{2i}) \frac{dz}{iz}
                  %%%%%+++++++++++++++++++++++
              \end{aligned}\end{equation}

    \item[类型二]
          $\int_{-\infty}^{\infty}f(x)dx$. 积分区间是$(-\infty,+\infty)$;\\
          复变函数$f(z)$在实轴上没有奇点,在上半平面除有限个奇点外是解析的;
          当$z$在上半平面及实轴上$\to \infty$时,$zf(z)$一致地$\to 0$。

          如果$f(x)$是有理分式$\phi(x)/\psi(x)$,上述条件意味着$\psi(x)$没有实的零点,
          $\psi(x)$的次数至少高于$\phi(x)$两次

          这一积分通常理解为下列极限:
          \begin{equation}\begin{aligned}
                  \label{eq.4.2.4}
                  %%%%%+++++++++++++++++++++++---------------------
                  I=\lim_{R_1\to\infty,R2\to\infty}
                  \int_{-R_1}^{R^2} f(x)dx
                  %%%%%+++++++++++++++++++++++
              \end{aligned}\end{equation}
          若极限存在的话,这一极限便称为反常积分$\int_{-\infty}^{\infty}f(x)dx$的值。
          而当$ R_1=R_2\to \infty $时极限存在的话,该极限便称为积分$\int_{-\infty}^{\infty}f(x)dx$的\textbf{主值},记作
          \begin{equation}\begin{aligned}
                  \label{eq.4.2.5}
                  %%%%%+++++++++++++++++++++++---------------------
                  \mathcal{P}\int_{-\infty}^{\infty} f(x)dx=
                  \lim_{R\to\infty}\int_{-R}^{R}f(x)dx
                  %%%%%+++++++++++++++++++++++
              \end{aligned}\end{equation}
          本类型积分要计算的是积分主值。考虑半圆形回路$l$
          \begin{equation}\begin{aligned}
                  \label{eq.4.2.5}
                  %%%%%+++++++++++++++++++++++---------------------
                  \oint f(z)dz=\int_{-R}^{R} f(x)dx + \int_{C_R}f(z)dz
                  %%%%%+++++++++++++++++++++++
              \end{aligned}\end{equation}
          根据留数定理,上式即

          \begin{equation}\begin{aligned}
                  % \label{eq.4.2.5}
                  %%%%%+++++++++++++++++++++++---------------------
                  2\pi i{f(z)\text{在}l
                          \text{所围半圆内各奇点的留数之和}}=
                  \int_{-R}^{R} f(x)dx+
                  \int_{C_R}f(z)dz
                  %%%%%+++++++++++++++++++++++
              \end{aligned}\end{equation}

          令$R \to \infty$。上式左边趋于$2\pi i{f(z)\text{在}l
                      \text{所围半圆内各奇点的留数之和}}$,
          右边第一个积分趋于所求的定积分$\int_{-\infty}^{\infty}f(x)dx$,
          第二个定积分可证明是趋于零的:
          \begin{equation}\begin{aligned}
                  % \label{eq.4.2.5}
                  %%%%%+++++++++++++++++++++++---------------------
                   & {}\abs{\int_{C_R}f(z)dz}=
                  \abs{\int_{C_R} z f(z) \frac{dz}{z} }
                  \leq
                  \int_{C_R} \abs{z f(z)} \frac{\abs{dz}}{\abs{z}}
                   & {}\leq
                  \max \abs{z f(z)} \frac{\pi R}{R}
                  =\pi \cdot \max
                  \abs{z f(z)} \to 0
                  %%%%%+++++++++++++++++++++++
              \end{aligned}\end{equation}
          式中$\max \abs{z f(z)}$是$\abs{z f(z)}$在$C_R$上的最大值。于是得到结果
          \begin{equation}\begin{aligned}
                  \label{eq.4.2.6}
                  %%%%%+++++++++++++++++++++++---------------------
                  \int_{-\infty}^{\infty} f(x)dx =
                  2\pi i{
                          f(z)\text{在}
                          \text{上半平面所有奇点的留数之和}
                      }
                  %%%%%+++++++++++++++++++++++
              \end{aligned}\end{equation}

    \item[类型三]
          $\int_0^\infty F(x) \cos mx dx$, $\int_0^\infty G(x) \sin mx dx$.
          积分区间是$[0,+\infty]$;偶函数$F(z)$和奇函数$G(z)$在实轴上没有奇点,在上半平面除有限个奇点外是解析的;\\
          当$z$在上半平面或实轴上$\to \infty$时,$F(z)$及$G(z)$一致地$\to 0$

          首先,将所求积分的形式变换一下,

          \begin{equation}\begin{aligned}
                  % \label{eq.4.2.5}
                  %%%%%+++++++++++++++++++++++---------------------
                  \int_0^\infty F(x) \cos mx dx & {}=
                  \int_0^\infty F(x) \frac{1}{2} (e^{imx}+e^{-imx})dx \\
                                                & {}=
                  \frac{1}{2} \int_0^\infty F(x) e^{imx} dx
                  +\frac{1}{2} \int_0^\infty F(x) e^{-imx} dx
                  %%%%%+++++++++++++++++++++++
              \end{aligned}\end{equation}

          在右边第二个积分中做代换$x=-y$,并考虑到$F(x)$是偶函数,得
          \begin{equation}\begin{aligned}
                  % \label{eq.4.2.5}
                  %%%%%+++++++++++++++++++++++---------------------
                  \int_0^\infty F(x) \cos mx dx =
                  \frac{1}{2} \int_0^\infty F(x) e^{imx} dx
                  -\frac{1}{2}\int_0^{- \infty} F(y) e^{imy} dy
                  %%%%%+++++++++++++++++++++++
              \end{aligned}\end{equation}

          将右边第二项积分的积分变数再改成$x$,则
          \begin{equation}\begin{aligned}
                  \label{eq.4.2.7}
                  %%%%%+++++++++++++++++++++++---------------------
                  \int_0^\infty F(x) \cos mx dx =
                  \frac{1}{2} \int_0^\infty F(x) e^{imx} dx
                  +\frac{1}{2}\int_{- \infty}^{0} F(x) e^{imx} dx
                  =\frac{1}{2} \int_{- \infty}^{\infty} e^{imx} dx
                  %%%%%+++++++++++++++++++++++
              \end{aligned}\end{equation}

          同理,
          \begin{equation}\begin{aligned}
                  \label{eq.4.2.8}
                  %%%%%+++++++++++++++++++++++---------------------
                  \int_0^\infty G(x) \sin mx dx =
                  \frac{1}{2i} \int_{- \infty}^\infty G(x) e^{imx} dx
                  %%%%%+++++++++++++++++++++++
              \end{aligned}\end{equation}

          \eqref{eq.4.2.7}and \eqref{eq.4.2.8}右边积分的计算,需要用到约当引理。

          \subsection{约当引理}

          如$m$为证书,$C_R$是以原点为圆心而位于上半平面的半圆周,又设当$z$在上半平面及实轴上$\to \infty$时$F(z)$一致地$\to 0$,则
          \begin{equation}\begin{aligned}
                  % \label{eq.4.2.8}
                  %%%%%+++++++++++++++++++++++---------------------
                  \lim_{R\to infty} \int_{C_R} F(z) e^{imz} dz =0
                  %%%%%+++++++++++++++++++++++
              \end{aligned}\end{equation}

          证明:
          \begin{equation}\begin{aligned}
                  % \label{eq.4.2.8}
                  %%%%%+++++++++++++++++++++++---------------------
                  \abs{ \int_{C_R} F(z) e^{imz} dz } =
                  \abs{ \int_{C_R} F(z) e^{imx-my} dz } \\
                   & {}=\abs{
                      \int_{0}^{\pi} F(R e^{i\phi})
                      e^{-mR \sin \phi} e^{im R \cos \phi}
                      R e^{i \phi} i d \phi
                  }                                     \\
                   & {}\leq \max \abs{F(z)}\cdot
                  \int_{0}^{\pi} e^{-m R \sin \phi} R d \phi
                  %%%%%+++++++++++++++++++++++
              \end{aligned}\end{equation}

          当$z$在上半平面或实轴上$\to \infty$时,$F(z)$一致地$\to 0$,所以$\max \abs{F(z)} \to 0$,
          从而只需证明
          \begin{equation}\begin{aligned}
                  % \label{eq.4.2.8}
                  %%%%%+++++++++++++++++++++++---------------------
                  \lim_{R \to \infty} \int_0^{\pi} e^{-m R \sin \phi} R d \phi \\
                  i.e.:~ 2 \lim_{R\to \infty} \int_0^{\pi} e^{-m R \sin \phi} R d\phi
                  %%%%%+++++++++++++++++++++++
              \end{aligned}\end{equation}

          是有界的。

          在$0\leq \phi \leq \frac{\pi}{2}$ 范围内,$0\leq 2 \phi / \pi \leq \sin \phi$,
          \begin{equation}\begin{aligned}
                  % \label{eq.4.2.8}
                  %%%%%+++++++++++++++++++++++---------------------
                  \int_0^{\pi/2} e^{-m R \sin \phi} R d\phi
                  \leq \int_0^{\pi/2} e^{-2m R \phi / \pi} R d\phi
                  = \frac{\pi}{2m}(1-e^{-mR})
                  %%%%%+++++++++++++++++++++++
              \end{aligned}\end{equation}

          于$R\to \infty$,上式趋于有限值,这就证明了约当引理。

          如果$m$是负数,则约当引理应为
          \begin{equation}\begin{aligned}
                  % \label{eq.4.2.8}
                  %%%%%+++++++++++++++++++++++---------------------
                  \lim_{R \to \infty} \int_{C^\prime_{R}} F(z)e^{imz}dz=0
                  %%%%%+++++++++++++++++++++++
              \end{aligned}\end{equation}

          $C^\prime_{R}$是$C_R$对于实轴的映像。

          对于类型二,计算\eqref{eq.4.2.7}and \eqref{eq.4.2.8}右边积分的主值,利用留数定理及约当引理得到,

          \begin{equation}\begin{aligned}
                  % \label{eq.4.2.8}
                  %%%%%+++++++++++++++++++++++---------------------
                  \int_0^\infty F(x) \cos mxdx =
                  \pi i\text{ all} \res \text{of } F(z)e^{imz} \text{ in the above half plane} \\
                  \int_0^\infty G(x) \sin mxdx =
                  \pi\text{ all} \res \text{of } G(z)e^{imz} \text{ in the above half plane}
                  %%%%%+++++++++++++++++++++++
              \end{aligned}\end{equation}

    \item[类型四]

          \textbf{实轴上有单极点的情形} \\

          考虑积分$\int_{-\infty}^{\infty} f(x)dx $,被积函数$f(x)$在实轴上有单极点$z=\alpha$,除此之外,$f(x)$满足类型二或类型三($f(z)$应被理解为$f(z) e^{imz} $或 $G(z) e^{imz} $ )的条件。
          由于存在这个奇点,我们以$z=\alpha$为圆心,以充分小的正数$\epsilon$为半径作半圆弧绕过奇点$\alpha$构成积分回路。于是,
          \begin{equation}\begin{aligned}
                  \label{eq.4.2.11}
                  %%%%%+++++++++++++++++++++++---------------------
                  \oint_{l} f(z)dz & {}=
                  \int_{-R}^{\alpha-\epsilon} f(x)x+
                  \int_{a+\epsilon}^{R} f(x)dx
                                   & {} \int_{C_R} f(z)dz + \int_{C_\epsilon} f(z) dz
                  %%%%%+++++++++++++++++++++++
              \end{aligned}\end{equation}

          取极限$R\to \infty,~\epsilon \to 0 $,
          \eqref{eq.4.2.11}左边积分值等于$2\pi i \sum_\text{above half}$.
          右边第一、第二项之和即为所求积分。按类型二或类型三的条件,第三项为零。
          对于第四项,计算如下:将$f(z)$在$z=\alpha$的邻域展开为洛朗级数,
          由于$z=\alpha$是$f(z)$的单极点,于是,
          \begin{equation}\begin{aligned}
                %   \label{eq.4.2.11}
                  %%%%%+++++++++++++++++++++++---------------------
                  f(z)=\frac{a_{-1}}{z-\alpha}+ P(z-\alpha)
                  %%%%%+++++++++++++++++++++++
              \end{aligned}\end{equation}

          其中$P(z-\alpha)$为级数的解析部分,它在$C_\epsilon$上连续且有界,因此,
          \begin{equation}\begin{aligned}
                %   \label{eq.4.2.11}
                  %%%%%+++++++++++++++++++++++---------------------
                  \abs{\int_{C_\epsilon} P(z-\alpha) dz }
                  \leq \max \abs{P(z-\alpha)} \abs{\int_{C_\epsilon} } \abs{dz}
                  = \pi \epsilon \cdot \max \abs{P(z-\alpha)},
                  %%%%%+++++++++++++++++++++++
              \end{aligned}\end{equation}

          所以,
          \begin{equation}\begin{aligned}
                %   \label{eq.4.2.11}
                  %%%%%+++++++++++++++++++++++---------------------
                  \lim_{\epsilon \to 0} \int_{C_\epsilon} P(z-\alpha) dz=0
                  %%%%%+++++++++++++++++++++++
              \end{aligned}\end{equation}

          而
          \begin{equation}\begin{aligned}
                %   \label{eq.4.2.11}
                  %%%%%+++++++++++++++++++++++---------------------
                  \int_{C_\epsilon} \frac{a_{-1}}{z-\alpha}dz
                   & {}=
                  \int_{C_\epsilon}\frac{a_{-1}}{z-\alpha}d(z-\alpha) \\
                   & {}=\int_{\pi}^0 \frac{a_{-1}}{\epsilon e^{i\phi}} 
                   \epsilon e^{i\phi} i d\phi
                  =- \pi i a_{-1} = - \pi i \res f(\alpha)
                  %%%%%+++++++++++++++++++++++
              \end{aligned}\end{equation}

          于是,由\eqref{eq.4.2.11}取极限$R\to \infty, \epsilon \to 0$,得
          \begin{equation}\begin{aligned}
            \label{eq.4.2.12}
            %%%%%+++++++++++++++++++++++---------------------
            \int_{-\infty}^{\infty} f(x) dx =
            2 \pi i \sum_\text{the above half} \res f(z) 
            + \pi i \res f(\alpha)
            %%%%%+++++++++++++++++++++++
        \end{aligned}\end{equation}

        若实轴上存在有限个单极点,则类似地可以得到

        \begin{equation}\begin{aligned}
            \label{eq.4.2.13}
            %%%%%+++++++++++++++++++++++---------------------
            \int_{-\infty}^{\infty} f(x) dx =
            2 \pi i \sum_\text{the above half}\res f(z) 
            +\pi i \sum_\text{in the real axis} \res f(z)
            %%%%%+++++++++++++++++++++++
        \end{aligned}\end{equation}

        从以上计算看到,实轴上有奇点时,仍归结为留数的计算,但需注意以下两点:
        \begin{enumerate}
            \item $C_\epsilon$ 不是闭合曲线,
            $f(z)$ 洛朗展开的解析部分的积分值只是由于$\epsilon\to 0$才趋近于零
            \item 实轴上的奇点只能是单极点,不能是二阶或者二阶以上的极点,
            更不能是本性奇点,否则当$\epsilon \to 0$时,
            积分$\int_{C_\epsilon}$之值将趋于$\infty$(极点情形)或不存在(本性奇点情形)。
        \end{enumerate}

\end{enumerate}

\section{色散关系}

对于在上半平面处处解析的函数$f(z)$,如果当$z\to \infty$时,
$f(z)$一致地$\to 0~(0\leq Arg~z \leq \pi )$,$\alpha$为一实数,则按照\eqref{eq.4.2.12},

\begin{equation}\begin{aligned}
    \label{eq.4.2.15}
    %%%%%+++++++++++++++++++++++---------------------
    \mathcal{P} \int_{-\infty}^{\infty} \frac{f(x)}{x-\alpha} dx =
    \pi i \cdot \res \text{ of } \frac{f(z)}{z-\alpha}
    = \pi i~f(\alpha)
    %%%%%+++++++++++++++++++++++
\end{aligned}\end{equation}

分别写出实部和虚部
\begin{equation}\begin{aligned}
    \label{eq.4.2.16}
    %%%%%+++++++++++++++++++++++---------------------
    \re{f(\alpha)} &{}= \frac{1}{\pi} \mathcal{P}
    \int_{-\infty}^{\infty} \frac{\im f(x)}{x-\alpha} dx \\
    %%%%%%%%%%%%%%%%%%%%%%%%%%%%%%%
    \im{f(\alpha)} &{}= -\frac{1}{\pi} \mathcal{P}
    \int_{-\infty}^{\infty} \frac{\re f(x)}{x-\alpha} dx 
    %%%%%+++++++++++++++++++++++
\end{aligned}\end{equation}

这一对关系在数学上称之为希尔伯特变换,在物理上称为色散关系。