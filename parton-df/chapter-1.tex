\documentclass[./main.tex]{subfiles}

\setcounter{chapter}{0}

\chapter{Light Cone relevant}

\begin{definition}{光锥动量}{}%%\ref{def:label}
    在光锥坐标中,动量的各分量定义为:
    \begin{equation}\begin{aligned}
            %%\label{eq.6.1.2}
            k^{\pm}=k_0 \pm k_3, \quad k_\perp^2 = k_1^2+k_2^2    \\
            k^2=k_0^2 - k_1^2 - k_2^2-k_3^2 = k^+ k^- -k_\perp^2, \\
            p^+p^-=M^2,\quad p_\perp^2=0.
        \end{aligned}\end{equation}
\end{definition}
%per.pen.dicu.lar 垂直的;成直角的

$k_\perp$ :垂直方向的动量,在垂直面上可以分解成两个方向的投影,把这个矢量记作 $\vec{k_\perp}$,\\
其模方为$k_\perp^2$

上式最后一行中我们假设核子在$z$方向快速运动,因此$p_\perp^2 = 0$

\begin{definition}{光锥动量}{}%%\ref{def:label}
    符号约定:介子和重子传播子因子$D_\phi(k)$和$D_B(k)$定义为:
    \begin{equation}\begin{aligned}
            %%\label{eq.6.1.2}
            D_\phi(k)=k^2 - {m_\phi}^2+i \epsilon
            =k^+ k^- - k_\perp^2 - {m_\phi}^2 +i \epsilon
            =k^+ k^- - \Omega_\phi + i \epsilon,                \\
            %%%%%%%%%%%%%%%%%%%%%%%%%%%%%%%%%%%%%
            D_B(p-k)=(p-k)^2 - {m_B}^2 + i \epsilon
            =(p-k)^+ (p-k)^- - k_\perp^2 - {m_B}^2 + i \epsilon \\ \notag
            =(p-k)^+ (p-k)^- - \Omega_B + i \epsilon            \\
        \end{aligned}\end{equation}
\end{definition}


\begin{definition}{光锥动量}{}%%\ref{def:label}
    光锥积分:如(3.33)式所示,被积函数化简后变为$\frac{1}{D_\phi^n D_B^m}$($n$和$m$为自然数) 类型的积分。此类积分可通过计算它们的留数求解:
    \begin{equation}\begin{aligned}
            %%\label{eq.6.1.2}
            \int d k^- \frac{1}{D_\phi D_B} & {}= - \frac{2 \pi i  p^+ \bar{y}}{D_{\phi B}}                          \\
            %%%%%%%%%%%%%%%%%%%%%%%%%%%%%%%%%%
            \int d k^- \frac{1}{D_\phi^2 D_B}
                                            & {}= \frac{\partial}{\partial m_\phi^2} \int d k^- \frac{1}{D_\phi D_B}
            = \frac{-2 \pi i}{D_{\phi B}^2 p^+ \bar{y}}                                                              \\
            %%%%%%%%%%%%%%%%%%%%%%%%%%%%%%%%%%
            \int d k^- \frac{1}{D_\phi D_B^2}
                                            & {} = \frac{\partial}{\partial M_B^2} \int d k^- \frac{1}{D_\phi D_B}
            = \frac{-2 \pi i y}{D_{\phi B}^2 p^+ \bar{y}^2}                                                          \\
        \end{aligned}\end{equation}
\end{definition}

作为一个例子,演示第一个积分的计算过程,根据留数定理:

\begin{equation}\begin{aligned}
        %%\label{eq.6.1.2}
        %%%%%+++++++++++++++++++++++---------------------
         & {} \int d k^- \frac{1}{D_\phi D_B}
        = -2 \pi i \mathrm{Res}\left[\frac{1}{D_\phi D_B}\right]_{\mathrm{lower~half~plane}} \\ \notag
         & {}= -2 \pi i \lim\limits_{k^- \to \frac{\Omega_\phi}{k^+}}
        \frac{k^- - \frac{\Omega_\phi}{k^+}}
        {\left(k^+ k^- -\Omega_\phi+ i \epsilon\right)
            \left[\left(p^+ - k^+\right)\left(p^- - k^-\right) - \Omega_B +i\epsilon  \right]
        }                                                                                    \\ \notag
         & {}= -2 \pi i \lim\limits_{k^- \to \frac{\Omega_\phi}{k^+}}
        \frac{k^- - \frac{\Omega_\phi}{k^+}}
        {
            \left(p^+ - k^+\right) k^ +
            \left(
            k^- -\frac{\Omega_\phi}{k^+} + \frac{i \epsilon }{k^+}
            \right)
            \left[
                \left(p^- - k^-\right) - \frac{\Omega_B}{p^+ - k^+}  + \frac{i\epsilon}{p^+ - k^+}
                \right]
        }                                                                                    \\ \notag
         & {}=\frac{-2 \pi i}
        {
            \left(p^+ - k^+ \right) k^ +
            \left[
                \left(
                p^- - \frac{\Omega_\phi}{k^+}
                \right) - \frac{\Omega_B}{p^+ - k^+}
                \right]
        }                                                                                    \\ \notag
         & {}=\frac{ 2 \pi i}
        {
            p^+
            \left[
                k_\perp^2+(1-y)m_\phi^2+y(1-y)M^2+y M_B^2
                \right]
        }                                                                                    \\ \notag
         & {}= \frac{-2\pi i}{ p^+ D_{\phi B} \bar{y} }                                      \\ \notag
        %%%%%+++++++++++++++++++++++
    \end{aligned}\end{equation}

$\delta$函数项的积分:$\frac{1}{D_\phi^n}$和$\frac{K^+ k^-}{D_\phi^n}$($n$为自然数)类型的积分由下式给出:

\begin{equation}\begin{aligned}
        %%\label{eq.6.1.2}
        %%%%%+++++++++++++++++++++++---------------------
        \int d k^-\frac{1}{D_\phi}          & {}=
        2\pi i \log \left(\frac{\Omega_\phi}{\mu^2}\right)
        \frac{\delta\left(y\right)}{p^+},                    \\ \notag
        %%%%%%sssssssssssssssssssssssssssssssssssssssssss
        \int d k^- \frac{1}{D_\phi^2}       & {}=
        \frac{\partial}{\partial \Omega_\phi} \int d k^- \frac{1}{D_\phi}
        =\frac{2\pi i}{p^+ \Omega_\phi} \delta\left(y\right) \\ \notag
        %%%%%%sssssssssssssssssssssssssssssssssssssssssss
        \int d k^- \frac{k^+ k^-}{D_\phi}   & {}=
        2 \pi i \, \delta \left(y\right) \frac{\Omega_\phi}{p^+}
        \left[
            \log \left( \frac{\Omega_\phi}{\mu^2} \right)-1
            \right]
        \\ \notag
        %%%%%%sssssssssssssssssssssssssssssssssssssssssss
        \int d k^- \frac{k^+ k^-}{D_\phi^2} & {}=
        \frac{\partial}{\partial \Omega_\phi}
        \int d k^- \frac{k^+ k^-}{D_\phi}=
        2 \pi i \, \log \left( \frac{\Omega_\phi}{\mu^2} \right)
        \frac{\delta\left(y\right)}{p^+}                     \\ \notag
        %%%%%+++++++++++++++++++++++
    \end{aligned}\end{equation}


\chapter{留数定理}

\begin{equation}\begin{aligned}
        %%\label{eq.6.1.2}
        %%%%%+++++++++++++++++++++++---------------------
        \frac{1}{2 \pi i } \oint  \frac{dz}{z-\alpha}=
        \begin{cases}
            0, \mathrm{\alpha isn't included in l} \\
            %%%%%+++++++++++++++++++++++
            1,\mathrm{\alpha is included in l}
        \end{cases}
        %%%%%+++++++++++++++++++++++
    \end{aligned}\end{equation}

因为原函数一个是多值函数$\ln (z-\alpha)$ ,一个是单值函数$(z-\alpha)^{n+1}/(n+1)$

柯西定理(2.2.1)指出,如被积函数$f(z)$在回路$l$所围比区域上是解析的,则回路积分$\oint f(z) dz$等于$0$. \\
现考虑回路$l$包围$f(z)$的奇点的情形。

先设$l$只包围着$f(z)$的一个孤立奇点$z_0$,在以$z_0$为圆心而半径为零的圆环域上将$f(z)$展为洛朗级数

\begin{equation}\begin{aligned}
        %%\label{eq.6.1.2}
        %%%%%+++++++++++++++++++++++---------------------
        f(z)=\sum\limits_{k=-\inf}^{\inf} a_k (z-z-0)^k
        %%%%%+++++++++++++++++++++++
    \end{aligned}\end{equation}

由柯西定理回路任意易形,洛朗级数除去$k=-1$的一项之外全为$0$,而$k=-1$的一项的积分等于$2 \pi i$. 于是

\begin{equation}\begin{aligned}
        %%\label{eq.6.1.2}
        %%%%%+++++++++++++++++++++++---------------------
        \oint_l f(z) dz =2 \pi i a_{-1}
        %%%%%+++++++++++++++++++++++
    \end{aligned}\end{equation}

洛朗级数的$(z-z_0)^{-1}$项的系数因而具有特别重要的地位,专门起了名字,称为函数$f(z)$在点$z_0$的\textbf{留数}(或残数), 通常记作$\res\,f(z_0)$,这样,

\begin{equation}\begin{aligned}
        %%\label{eq.6.1.2}
        %%%%%+++++++++++++++++++++++---------------------
        \oint_l f(z) dz =2 \pi i \res \,f(z_0)
        %%%%%+++++++++++++++++++++++
    \end{aligned}\end{equation}

如果$l$包围着$f(z)$的$n$个孤立奇点$b_1,b_2,\cdots,b_n$的情形. 做回路$l_1,l_2,\cdots,l_n$分别包围$b_1,b_2,\cdots,b_n$. 并使每个回路只包围一个奇点, 按照柯西定理,

\begin{equation}\begin{aligned}
    %%\label{eq.6.1.2}
    %%%%%+++++++++++++++++++++++---------------------
    \oint_l f(z) dz =
    \oint_{l1} f(z) dz 
    +\oint_{l2} f(z) dz
    + \cdots +\oint_{ln} f(z) dz
    %%%%%+++++++++++++++++++++++
\end{aligned}\end{equation}

于是得到,

\begin{equation}\begin{aligned}
    \label{eq.4.1.4}
    %%%%%+++++++++++++++++++++++---------------------
    \oint_l f(z) dz =
    2 \pi i 
    [
\res f(b_1) + \res f(b_2) + \cdots + \res f(b_n)
    ] 
    %%%%%+++++++++++++++++++++++
\end{aligned}\end{equation}

\begin{theorem}{留数定理}{}%%\ref{thm:label}
    设函数$f(z)$在回路$l$所围区域$B$上除有限个孤立奇点$b_1,b_2,\cdots,b_n$外解析,在闭区域$\bar{B}$上除$b_1,b_2,\cdots,b_n$外连续,则
    \begin{equation}\begin{aligned}
    %%\label{eq.6.1.2}
    \oint f(z)dz=2 \pi i \sum\limits_{j=1}^n \res f(b_j)
    \end{aligned}\end{equation}
    \end{theorem}

    留数定理将回路积分归结为被积函数在回路所围区域上各奇点的留数之和.

    函数$f(z)$在全平面上所有各点的留数之和为零.这里说的所有各点包括无限远点和有限远的奇点.

\subchapter{留数定理}  
    
一般地, 在以奇点为圆心的圆环域上将函数展开为洛朗级数, 取它的负一次幂项的系数就行了. 
对于极点, 可以不用做洛朗展开而直接求出留数.

设$z_0$是$f(z)$的单极点. 洛朗展开应是

\begin{equation}\begin{aligned}
    \label{eq.4.1.4}
    %%%%%+++++++++++++++++++++++---------------------
    f(z)=\frac{a_{-1}}{z-z_0}+a_0+a_1 (z-z_0)+a_2(z-z_0)^2+\cdots
    %%%%%+++++++++++++++++++++++
\end{aligned}\end{equation}

这样,

\begin{equation}\begin{aligned}
    \label{eq.4.1.8}
    %%%%%+++++++++++++++++++++++---------------------
    \lim_{z \to z_0}[(z-z_0)f(z)] = \res f(z_0)
    %%%%%+++++++++++++++++++++++
\end{aligned}\end{equation}

\ref{eq.4.1.8}既可用来计算函数$f(z)$在单极点$z_0$的留数, 也可用来判断$z_0$是否为函数$f(z)$
的单极点.

若$f(z)$可以表示为$P(z)/Q(z)$的特殊形式, 
其中$P(z)$和$Q(z)$都在$z_0$点解析, 
$z_0$是$Q(z)$的一阶零点. 
$P(z_0)\neq 0$, 从而$z_0$是$f(z)$的一阶极点, 则

\begin{equation}\begin{aligned}
    \label{eq.4.1.9}
    %%%%%+++++++++++++++++++++++---------------------
    \res f(z_0) = \lim_{z \to z_0} \frac{P(z)}{Q(z)} =  \frac{P(z_0)}{Q'(z_0)}
    %%%%%+++++++++++++++++++++++
\end{aligned}\end{equation}

上式最后一步应用了洛必达法则.

同理, 若1是1的1阶极点. 洛朗展开应是