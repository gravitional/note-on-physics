\documentclass[./main.tex]{subfiles}
\setcounter{chapter}{2}

\chapter{单举与遍举}

\href{https://physics.stackexchange.com/questions/1217/whats-the-difference-between-inclusive-and-exclusive-decays#}{reference: What's the difference between inclusive and exclusive decays?}

\subsection{An experimental take}

\textbf{Exclusive} implies that you have measured the energy and momenta of all the products (well, with an exception I'll discuss below).

\textbf{Inclusive} means that you may have left some of the products unmeasured.

This applies to scattering processes as well as decays.

Some things to note:

Exclusive measurements allow you to nail down(确定,明确;用钉钉住) one, well defined physics process,\\
while inclusive measurements may tell you about a collection of processes.

It is generally difficult to measure neutral particles, 
If there are more than a couple of products, 
it begins to require a lot of instrumentation to reliably collect them all and (crucially) to know how well you have done so.

In the process you are asking about, the neutrino is necessarily unobserved, rendering the measurement inclusive, \\
further an \textbf{X} in the final state is often used to indicate unmeasured and unspecified stuff (i.e. it means the measurement is inclusive by design). \\
Here unspecified includes case, in high acceptance instruments, where you consider all events with the specified products: those for which we know \textbf{X} is empty, those for which \textbf{X} is non-empty and well characterized, and those for which \textbf{X} is ill-characterized.

\subsection{Theoretical view}

I'm less sure of how theorist use these terms, but I believe there is a parallel. Something like:\\
\textbf{exclusive} means one and only one process, while \textbf{inclusive} means all processes that include the specified products.

\subsection{Convergence of theory and experiment}

Of course, we haven't really learned anything until we get theory and experiment together, which is sometimes traumatic(痛苦的;极不愉快的) for both communities.\\
Still exclusive measurements and calculations are clearly talking about the same thing, and inclusivity can be made to agree, with some care in building the experiment and assembling the theoretical results.

\subsection{Experimenters cheating on exclusivity}

Sometimes in nuclear physics we talk about scattering measurements as exclusive when there is an unmeasured, heavy, recoiling nucleus involved.\\
The assumption being that that it carries a small fraction of the total energy and momentum involved and can be neglected, though there is some risk from this if the nucleus is left in a highly excited state.

In particular my dissertation(专题论文;学位论文) project was on $A(e, e^\prime p)$
reaction \\
(elastic scattering of protons out of a stationary nuclear target where the beam was characterized and both the proton and outgoing electron were observed),\\
and we assumed that the remnant(残余部份;剩余部份) nucleus was left largely undisturbed and recoiling with a momentum opposite the Fermi motion of the target proton.