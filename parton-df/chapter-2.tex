\documentclass[./main.tex]{subfiles}
\setcounter{chapter}{1}


\chapter{Light Cone relevant}

\begin{definition}{光锥动量}{}%%\ref{def:label}
    在光锥坐标中,动量的各分量定义为:
    \begin{equation}\begin{aligned}
            %%\label{eq.6.1.2}
            k^{\pm}=k_0 \pm k_3, \quad k_\perp^2 = k_1^2+k_2^2    \\
            k^2=k_0^2 - k_1^2 - k_2^2-k_3^2 = k^+ k^- -k_\perp^2, \\
            p^+p^-=M^2,\quad p_\perp^2=0.
        \end{aligned}\end{equation}
\end{definition}
%per.pen.dicu.lar 垂直的;成直角的

$k_\perp$ :垂直方向的动量,在垂直面上可以分解成两个方向的投影,把这个矢量记作 $\vec{k_\perp}$,\\
其模方为$k_\perp^2$

上式最后一行中我们假设核子在$z$方向快速运动,因此$p_\perp^2 = 0$

\begin{definition}{光锥动量}{}%%\ref{def:label}
    符号约定:介子和重子传播子因子$D_\phi(k)$和$D_B(k)$定义为:
    \begin{equation}\begin{aligned}
            %%\label{eq.6.1.2}
            D_\phi(k)=k^2 - {m_\phi}^2+i \epsilon
            =k^+ k^- - k_\perp^2 - {m_\phi}^2 +i \epsilon
            =k^+ k^- - \Omega_\phi + i \epsilon,                \\
            %%%%%%%%%%%%%%%%%%%%%%%%%%%%%%%%%%%%%
            D_B(p-k)=(p-k)^2 - {m_B}^2 + i \epsilon
            =(p-k)^+ (p-k)^- - k_\perp^2 - {m_B}^2 + i \epsilon \\ \notag
            =(p-k)^+ (p-k)^- - \Omega_B + i \epsilon            \\
        \end{aligned}\end{equation}
\end{definition}


\begin{definition}{光锥动量}{}%%\ref{def:label}
    光锥积分:如(3.33)式所示,被积函数化简后变为$\frac{1}{D_\phi^n D_B^m}$($n$和$m$为自然数) 类型的积分。此类积分可通过计算它们的留数求解:
    \begin{equation}\begin{aligned}
            %%\label{eq.6.1.2}
            \int d k^- \frac{1}{D_\phi D_B} & {}= - \frac{2 \pi i  p^+ \bar{y}}{D_{\phi B}}                          \\
            %%%%%%%%%%%%%%%%%%%%%%%%%%%%%%%%%%
            \int d k^- \frac{1}{D_\phi^2 D_B}
                                            & {}= \frac{\partial}{\partial m_\phi^2} \int d k^- \frac{1}{D_\phi D_B}
            = \frac{-2 \pi i}{D_{\phi B}^2 p^+ \bar{y}}                                                              \\
            %%%%%%%%%%%%%%%%%%%%%%%%%%%%%%%%%%
            \int d k^- \frac{1}{D_\phi D_B^2}
                                            & {} = \frac{\partial}{\partial M_B^2} \int d k^- \frac{1}{D_\phi D_B}
            = \frac{-2 \pi i y}{D_{\phi B}^2 p^+ \bar{y}^2}                                                          \\
        \end{aligned}\end{equation}
\end{definition}

作为一个例子,演示第一个积分的计算过程,根据留数定理:

\begin{equation}\begin{aligned}
        %%\label{eq.6.1.2}
        %%%%%+++++++++++++++++++++++---------------------
         & {} \int d k^- \frac{1}{D_\phi D_B}
        = -2 \pi i \mathrm{Res}\left[\frac{1}{D_\phi D_B}\right]_{\mathrm{lower~half~plane}} \\ \notag
         & {}= -2 \pi i \lim\limits_{k^- \to \frac{\Omega_\phi}{k^+}}
        \frac{k^- - \frac{\Omega_\phi}{k^+}}
        {\left(k^+ k^- -\Omega_\phi+ i \epsilon\right)
            \left[\left(p^+ - k^+\right)\left(p^- - k^-\right) - \Omega_B +i\epsilon  \right]
        }                                                                                    \\ \notag
         & {}= -2 \pi i \lim\limits_{k^- \to \frac{\Omega_\phi}{k^+}}
        \frac{k^- - \frac{\Omega_\phi}{k^+}}
        {
            \left(p^+ - k^+\right) k^ +
            \left(
            k^- -\frac{\Omega_\phi}{k^+} + \frac{i \epsilon }{k^+}
            \right)
            \left[
                \left(p^- - k^-\right) - \frac{\Omega_B}{p^+ - k^+}  + \frac{i\epsilon}{p^+ - k^+}
                \right]
        }                                                                                    \\ \notag
         & {}=\frac{-2 \pi i}
        {
            \left(p^+ - k^+ \right) k^ +
            \left[
                \left(
                p^- - \frac{\Omega_\phi}{k^+}
                \right) - \frac{\Omega_B}{p^+ - k^+}
                \right]
        }                                                                                    \\ \notag
         & {}=\frac{ 2 \pi i}
        {
            p^+
            \left[
                k_\perp^2+(1-y)m_\phi^2+y(1-y)M^2+y M_B^2
                \right]
        }                                                                                    \\ \notag
         & {}= \frac{-2\pi i}{ p^+ D_{\phi B} \bar{y} }                                      \\ \notag
        %%%%%+++++++++++++++++++++++
    \end{aligned}\end{equation}

$\delta$函数项的积分:$\frac{1}{D_\phi^n}$和$\frac{K^+ k^-}{D_\phi^n}$($n$为自然数)类型的积分由下式给出:

\begin{equation}\begin{aligned}
        %%\label{eq.6.1.2}
        %%%%%+++++++++++++++++++++++---------------------
        \int d k^-\frac{1}{D_\phi}          & {}=
        2\pi i \log \left(\frac{\Omega_\phi}{\mu^2}\right)
        \frac{\delta\left(y\right)}{p^+},                    \\ \notag
        %%%%%%sssssssssssssssssssssssssssssssssssssssssss
        \int d k^- \frac{1}{D_\phi^2}       & {}=
        \frac{\partial}{\partial \Omega_\phi} \int d k^- \frac{1}{D_\phi}
        =\frac{2\pi i}{p^+ \Omega_\phi} \delta\left(y\right) \\ \notag
        %%%%%%sssssssssssssssssssssssssssssssssssssssssss
        \int d k^- \frac{k^+ k^-}{D_\phi}   & {}=
        2 \pi i \, \delta \left(y\right) \frac{\Omega_\phi}{p^+}
        \left[
            \log \left( \frac{\Omega_\phi}{\mu^2} \right)-1
            \right]
        \\ \notag
        %%%%%%sssssssssssssssssssssssssssssssssssssssssss
        \int d k^- \frac{k^+ k^-}{D_\phi^2} & {}=
        \frac{\partial}{\partial \Omega_\phi}
        \int d k^- \frac{k^+ k^-}{D_\phi}=
        2 \pi i \, \log \left( \frac{\Omega_\phi}{\mu^2} \right)
        \frac{\delta\left(y\right)}{p^+}                     \\ \notag
        %%%%%+++++++++++++++++++++++
    \end{aligned}\end{equation}
